\documentclass[10pt,oneside]{uhthesis}
% \usepackage[showframe]{geometry}
\usepackage{subfigure}
\usepackage[ruled,lined,linesnumbered,titlenumbered,algochapter,spanish,onelanguage]{algorithm2e}
\usepackage{amsmath}
\usepackage{amssymb}
\usepackage{amsbsy}
\usepackage{caption,booktabs}
\captionsetup{
	justification = centering
}
\usepackage[subfigure]{tocloft}
%\usepackage{mathpazo}
\usepackage{float}
% \usepackage{todonotes}
\usepackage{adjustbox}
\usepackage{qtree}
\usepackage{tikz}
\usetikzlibrary{tikzmark,calc,positioning,babel}
\usepackage{fontawesome}


\floatstyle{plaintop}
\restylefloat{table}

\renewcommand{\tablename}{Tabla}
\renewcommand{\listalgorithmcfname}{Índice de Algoritmos}%
%\dontprintsemicolon
\SetAlgoNoEnd

% list of definitions
\newcommand{\listdefinitions}{Índice de definiciones}
\newlistof{importantdefinition}{mcf}{\listdefinitions}

% Reset section numbering between unnumbered chapters
% https://tex.stackexchange.com/questions/71162/reset-section-numbering-between
\newcommand{\importantdefinition}[1]
{%
	\refstepcounter{importantdefinition}
	\addcontentsline{mcf}{importantdefinition}
	{\protect #1}\par
%	{\protect\numberline{\theimportantdefinition}#1}\par
}

% end of list of definitions


% list of Codes
\newcommand{\listcodes}{Índice de códigos implementados en Matlab18a y ejecutados en laptop MSI, i5-10500h, 24 gb ram}
\newlistof{importantcodes}{mcfc}{\listcodes}

% Reset section numbering between unnumbered chapters
% https://tex.stackexchange.com/questions/71162/reset-section-numbering-between
\newcommand{\importantcodes}[1]
{%
	\refstepcounter{importantcodes}
	\addcontentsline{mcfc}{importantcodes}
	{\protect #1}\par
	%	{\protect\numberline{\theimportantdefinition}#1}\par
}

% end of list of definitions


\title{Métodos de Linealización Local de Orden Superior para problemas de valor inicial de dimensiones no pequeñas}
\author{Lic. Frank Sadan Naranjo Noda}
\advisor{Dr. Juan Carlos Jiménez Sobrino}
\advisorsingle
\degree{Doctor en Ciencias Matemáticas}
\faculty{Facultad de Matemática y Computación}
% \date{}
\logo{Graphics/uhlogo}
\makenomenclature

\newcommand{\nonl}{\renewcommand{\nl}{\let\nl\oldnl}}
\renewcommand{\vec}[1]{\boldsymbol{#1}}
\newcommand{\diff}[1]{\ensuremath{\mathrm{d}#1}}
\newcommand{\me}[1]{\mathrm{e}^{#1}}
\newcommand{\pf}{\mathfrak{p}}
\newcommand{\qf}{\mathfrak{q}}
%\newcommand{\kf}{\mathfrak{k}}
\newcommand{\kt}{\mathtt{k}}
\newcommand{\mf}{\mathfrak{m}}
\newcommand{\hf}{\mathfrak{h}}
\newcommand{\fac}{\mathrm{fac}}
\newcommand{\maxx}[1]{\max\left\{ #1 \right\} }
\newcommand{\minn}[1]{\min\left\{ #1 \right\} }
\newcommand{\lldpcf}{1.25}
\newcommand{\nnorm}[1]{\left\lvert #1 \right\rvert }
\newcommand{\nnnorm}[1]{\left\lVert #1 \right\rVert }
\newcommand{\norme}[1]{\nnorm{\nnorm{ #1 }}_2}
\newcommand{\scal}[2]{\left< #1, #2 \right>}
\newtheorem{example}{Ejemplo}
\newcommand{\redmark}[1]{\textcolor{black}{#1}}
\newcommand{\bluemark}[1]{\textcolor{black}{#1}}
\newcommand{\mmark}[1]{\textcolor{black}{#1}}
\newcommand{\lmatrix}{\mathtt{L}}
\newcommand{\rvector}{\mathtt{r}}

% marker tikz in qtree
\makeatletter
\newcommand{\myLines}[1]{% Three-way only
	\begin{picture}(1,1)
	%	\put(0,0){\line(2,1){2}}
	%	\put(1,0){\vector(0,1){0.7}}
	%	\put(4,0){\line(-2,1){2}}
\end{picture}}
\let\qdrawReal=\qdraw@branches
\newcommand\brOverride{\let\qdraw@branches=\myLines}

\makeatletter
\newcommand{\myLiness}[1]{% Three-way only
	\begin{picture}(4,1)
		\put(0,0){\line(2,1){2}}
		%\put(1,0){\vector(0,1){0.7}}
		\put(4,0){\line(-2,1){2}}
\end{picture}}
\let\qdrawReal=\qdraw@branches
\newcommand\brOverridedd{\let\qdraw@branches=\myLiness}

\newcommand\brRestore{\let\qdraw@branches=\qdrawReal}
\makeatother


\begin{document}

\frontmatter
\maketitle

%\include{FrontMatter/Dedication}
% \begin{acknowledgements}

Gracias a las personas que me apoyaron a lo largo del proceso de investigación y escritura de esta tesis. Especialmente a mis padres, mi hermano y su mujer que
  siempre me alentaron a continuar y superarme, a mi tutor que siempre apoyó en todo momento e instó a perfeccionar los resultados y ser minucioso y transparente
  durante el proceso de investigación y al Prof. Carlos que siempre me ha brindado su sabiduría y consejos a lo largo del tiempo. A la organización BIOCUBAFARMA por permitir al ICIMAF utilizar el cluster para ejecutar los experimentos numéricos.

\end{acknowledgements}
% \include{FrontMatter/SupervisorOpinion}
\begin{abstract}
	df
\end{abstract}

\newpage

\begin{center}
	{\large \textbf{Hipótesis}}
\end{center}
¿Es factible aplicar los métodos de Linealización Local de Orden Superior para resolver problemas de valor inicial de grandes dimensiones?

\qquad

\begin{center}
	{\large \textbf{Objetivos de la tesis}}
\end{center}
\textbf{Objetivo general}

Desarrollo de métodos de Linealización Local de Orden Superior para resolver problemas de valor inicial de grandes dimensiones

\qquad\\
\textbf{Objetivos específicos}

\begin{enumerate}
	\item Desarrollo de aproximaciones Krylov-Padé para la solución de ecuaciones lineales de dimensiones no pequeñas.
	\item Construcción de nuevas fórmulas embebidas de Dormand y Prince localmente linealizadas para problemas de valor inicial de dimensiones no pequeñas.
	\item Desarrollo de métodos de Linealización Local de Orden Superior Libres de Jacobiano
\end{enumerate}

\tableofcontents
\newpage
\listofimportantdefinition
\newpage
\listoffigures
\newpage
\listoftables
\newpage
\listofalgorithms
\newpage
\listofimportantcodes



\mainmatter

%===================================================================================
% Chapter: Introduction
%===================================================================================
\chapter*{Introducción}\label{chapter:introduction}
\addcontentsline{toc}{chapter}{Introducción}
%===================================================================================

Las Ecuaciones en Derivadas Parciales(EDP) se utilizan para modelar de manera eficiente la evolución temporal de complejos fenómenos físicos, biofísicos y físico-químicos. En general no es posible resolver dichas ecuaciones de manera analítica, por lo que se recurre a métodos numéricos para aproximar sus soluciones. Una vía de solución es utilizar una discretización espacio-temporal para transformar la EDO en un sistema algebraico, el cual es mal condicionado y se necesita del uso de precondicionadores para cada problema en específico. Otra vía de solución es utilizar el método de línea para convertir la EDP en un sistema EDO que suele ser de tipo \emph{stiff}.

Debido a las características de las EDO tipo \emph{stiff} es necesario utilizar integradores implícitos. Estos integradores implícitos también necesitan resolver in sistema algebraico en cada paso de integración. Para resolver de manera eficiente estas EDOs se diseñaron los integradores de tipo exponencial. El desarrollo de esta clase de métodos ha sido estimulado por su capacidad para preservar numerosas propiedades dinámicas de las EDOs con tamaños de pasos mucho mayores que los integradores explícitos y con menor costo computacional que los implícitos.

Inicialmente, los integradores exponenciales se centraron principalmente en la integración de sistemas de Ecuaciones Diferenciales Ordinarias(EDO) semilineales resultantes de la discretización espacial de las mencionadas EDP. Los integradores derivados de este enfoque llamados Integradores Exponenciales Globalmente Linealizados~(IEGL), son, por ejemplo, el método de Lawson y Lawson generalizado, la diferenciación temporal exponencial y los métodos exponenciales de Runge-Kutta(ver \cite{Berland07} para una revisión). En general, estos métodos implican el cálculo de las llamadas funciones phi en el paso de integración inicial y funcionan bien cuando el término no lineal en las EDO semilineales mencionadas es pequeño o está acotado en términos del término lineal. Sin embargo, cuando el término no lineal es relevante para la dinámica de las EDO semilineales, la efectividad de estos métodos se pierde.

Para solventar la deficiencia que tenían IEGL y extenderlos a EDOs en general se introdujo un nuevo tipo de integradores exponenciales. Estos nuevos integradores, conocidos como esquemas de Linealización Local~(LL), se derivan de un principio común: la linealización local de la EDO en cada paso de integración. Resultados teóricos y de simulaciones muestran que los métodos de linealización local poseen un gran número de propiedades deseables entre las que se encuentran la A-estabilidad, ausencia de puntos de equilibrio espurios bajo condiciones bastante generales y la preservación del comportamiento dinámico de la solución alrededor de puntos de equilibrio hiperbólicos y órbitas periódicas ~\cite{Jimenez02AMC,delaCruz06,delaCruz07,Jimenez13}. La estabilidad y las propiedades dinámicas antes mencionadas son de vital importancia para la integración de EDOs, no obstante su bajo orden de convergencia constituía una limitante en algunas circunstancias. Para solucionar el bajo orden de convergencia de los LL se desarrollaron los métodos localmente linealizados de orden superior los cuales retienen las propiedades antes mencionadas, pero con orden de convergencia mayor. Estos nuevos métodos se obtienen de agregar a los LL un término adicional de orden superior que resulta de aproximar la parte no lineal de la EDO linealizada. Los integradores localmente linealizados son, por ejemplo, el método de propagación iterativa exponencial~\cite{tokman2006efficient}, el método exponencial-Rosenbrock~\cite{tranquilli2014rosenbrock} y el método exponencial-Adams~\cite{hochbruck2011exponential}, los esquemas de Linealización Local de orden 2~\cite{pope1963exponential,hochbruck1997krylov,hochbruck1998exponential} y de Linealización Local de Orden Superior~~(LLOS)\cite{delaCruz06,delaCruz07,Jimenez13,Jimenez14AMC}.

Los integradores exponenciales localmente linealizados requieren el cálculo de productos de diferentes funciones phi por vectores en cada paso de integración, los cual tiene el mayor costo computacional. Por lo tanto, un punto crítico en la eficiencia computacional de estos integradores exponenciales es la aproximación efectiva de las funciones phi en cada paso de integración. Para EDOs de dimensiones pequeñas estas funciones phi son aproximadas mediante expansiones de Taylor, Padé o Chebychev. Estas expansiones son efectivas para calcular exponenciales de matrices relativamente pequeñas y densas; pero no así para matrices
de dimensiones moderadas o grandes. Con el propósito de extender el cálculo de exponenciales matriciales a matrices grandes, se han propuesto varias aproximaciones utilizando subespacios de Krylov con diferentes estrategias para controlar sus errores, y varios enfoques para estimar las dimensiones óptimas de Krylov \cite{hochbruck1998exponential,sidje1998expokit,niesen2012algorithm,gaudreault2018kiops}. Algunos integradores exponenciales han sido extendidos a grades dimensiones al combinarlos dichas aproximaciones de Krylov, obteniéndose así esquemas numéricos con dimensión variable de Krylov pero con tamaño de paso fijo \cite{tokman2006efficient,tranquilli2014rosenbrock,hochbruck2011exponential,kloeden2011exponential}.

La adaptabilidad de los integradores exponenciales localmente linealizados, mediante fórmulas embebidos de alto orden y estrategias adaptativas para la selección de tamaño de paso variable, es clave en la mejorar su desempeño en la integración de EDOs. La gran mayoría de los integradores exponenciales son de paso fijo con excepciones notables como los esquemas de tamaño de paso variable propuestos en \cite{hochbruck1998exponential,caliari2009implementation,tokman2012new,luan2014exponential} con estrategias adaptativas basadas en fórmulas embebidas. Sin embargo, las fórmulas integradas de alto orden de estos esquemas requieren el cálculo de varias funciones phi por vectores en el mismo paso de integración, lo que implica un alto costo computacional. Otro aspecto importante a tener en cuenta es el caso en que no es posible calcular o almacenar el Jacobiano del EDO que se está resolviendo. Para estos casos son necesarios métodos libres de Jacobiano. Libre de Jacobiano quiere decir métodos que nunca calculan o almacenan el Jacobiano parcial o completamente.

En general los integradores exponenciales representan una alternativa a los métodos implícitos para resolver ecuaciones diferenciales \emph{stiff} o altamente oscilatorias. Sin embargo, hasta el momento, estos integradores o son de paso fijo o poseen una estrategia adaptativa pero tienen un alto costo computacional ya que requieren del cálculo de varias funciones phi por diferentes vectores. Además existen pocos casos en que se estudia como afectan las aproximaciones libres de Jacobiano al orden de convergencia. Estas razones justifican la construcción de códigos de Linealización Local eficientes que permitan la integración de las EDOs que usualmente aparecen en situaciones prácticas. Trabajos en esta dirección han sido \cite{Jimenez13,Jimenez14AMC} donde se han desarrollado esquemas adaptativos Runge-Kutta Localmente Linealizados, pero para pequeñas dimensiones. Por esto el objetivo de este trabajo es analizar la factibilidad de aplicar los métodos de Linealización Local de Orden Superior para resolver problemas de valor inicial de grandes dimensiones. Para ello se desarrollaran las aproximaciones Krylov-Padé con y sin la evaluación del Jacobiano. Estas aproximaciones se utilizarán para la construcción de métodos de Linealización Local de Orden Superior para resolver problemas de valor inicial de grandes dimensiones y se estudiará su desempeño en problemas de prueba.

El documento está organizado de la siguiente forma. En el primer capítulo se presenta una breve revisión de los integradores exponenciales y los métodos de Linealización Local necesarios para la comprensión de este trabajo. En el segundo capítulo se construirán las aproximaciones Krylov-Padé con y sin la evaluación del Jacobiano. Estas aproximaciones se utilizarán en la integración de EDOs lineales. Además, se hallarán las cotas de dichas aproximaciones y se medirá su desempeño en un conjunto de ecuaciones de prueba. En el tercer capítulo, los métodos Runge-Kutta de Dormand y Prince Localmente Linealizado, se construirán. En el cuarto capítulo se extenderán los resultad de los métodos de linealización local de orden superior a métodos de linealización local de orden superior libres de Jacobiano; se construirán esquemas basados en estos nuevos métodos libres de jacobiano y se medirá el desempeño de estos nuevos esquemas en ecuaciones de prueba.



\chapter{Integradores exponenciales y método de Linealización Local}\label{chapter:exp-int-and-ll-methods}

En este capítulo se presenta una revisión de la clase general de  integradores numéricos de tipo exponencial y de los Métodos de Linealización Local en particular, así como resultados básicos sobre el cálculo de la exponencial matricial.

En general, se considerarán los problemas de valor inicial definidos por las ecuaciones diferenciales no autónomas
 \begin{equation}
 \frac{dx}{dt}=f(t,x), \;\; \label{ODE-SYST}
 \end{equation}
 \begin{equation*}
 x(t_0)=x_0,
 \end{equation*}
 donde $x_0\in \mathfrak{D}$ es un valor inicial dado, $f: [t_0,T] \times \mathfrak{D}\longrightarrow \mathbb{R}^{d}$ es una función diferenciable y $\mathfrak{D}\subset\mathbb{R}^{d}$ un conjunto abierto. Se asumen condiciones de suavidad y condición de Lipschit en $f$ para asegurar la unicidad de la solución en $\mathfrak{D}$.

Consideremos, ademas, una partición $(t)_{h}={t_{n}:n=0,1,\ldots,N}$ del
intervalo $[t_{0},T]$ tal que $t_{0}<t_{1}<\ldots<t_{N}=T$, y $%
h_{n}=t_{n+1}-t_{n}\leq h$; para $n=0,\ldots,N-1$.

\section{Integradores exponenciales}

En general, se denomina integradores exponenciales a la familia
de esquemas numéricos para EDOs que requieren del cálculo de algún término exponencial. Inicialmente, los integradores de este tipo comenzaron a desarrollarse en los años 60 del siglo pasado~\cite{Berland07} con el propósito de resolver sistemas de ecuaciones semilineales de la forma
\begin{equation}
\frac{dx}{dt} = Ax(t) + F(x(t))  \;\;  \label{ODE Lines Method}
\end{equation}
que resultan de la discretización espacial de una ecuación diferencial en derivadas parciales, donde $A$ es una
matriz cuadrada y $F$ una función no lineal de $x(t)\in\mathbb{R}^d$ y $t\geq t_0 \in \mathbb{R}$. Más recientemente, en los años 90, esta clase de integradores fue extendida a EDOs de la forma general (\ref{ODE-SYST}).

Los integradores exponenciales tanto para ecuaciones semilineales como para ecuaciones generales provienen de la aproximación de la integral
que resulta de aplicar la fórmula de variación de la constante a estas ecuaciones. Al aplicar la fórmula de variación
de la constante a la ecuación (\ref{ODE Lines Method}) se obtiene
\begin{equation}
x(t_n+h)=\me{Ah}\left(x(t_n) +\int\limits_{0}^{h} \me{-As}F(x(t_n+s)) \,ds\right)\,. \label{VCF2}
\end{equation}
Los integradores exponenciales derivados de esta aproximación son, por ejemplo, los conocidos métodos de Lawson, Lawson Generalizados, Exponential Time
Differencing and the Exponential Runge-Kutta (ver \cite{Berland07} para una revisión). 

Por otra parte, después de la linealización local de la ecuación (\ref{ODE-SYST}) en $t_{n}$,  la fórmula de variación
de la constante queda
\begin{eqnarray}
x(t_{n}+h) &= & x(t_{n})+\int\limits_{0}^{h}e^{A	_{n}(h-s)}(f(t_n,x(t_{n}))+sf_t(t_n,x(t_{n})))ds \nonumber \\
 & & +\int\limits_{0}^{h}e^{A
	_{n}(h-s)}g_{n}(t_{n}+s,x(t_{n}+s))ds, \label{VCF}
\end{eqnarray}
donde $A_{n}$ es la matriz Jacobiana de $f$ evaluada en $(t_n,x(t_{n}))$ y $g_{n}(s,u)=f(s,u)-A_{n}(u-x(t_n))+f_t(s,u)(s-t_n)$. 
Ejemplos de integradores exponenciales derivados de esta aproximación son el método clásico de Pope~\cite{pope1963exponential} de orden 2, los métodos Runge Kutta Localmente 
Linealizados~\cite{delaCruz06,Jimenez13,Jimenez14AMC}, los métodos Exponenciales de Propagación 
Iterativa~\cite{tokman2013comparative}, los métodos de Taylor Localmente Linealizados~\cite{delaCruz07}, los
métodos Exponenciales tipo Rosenbrok~\cite{hochbruck2009exponential} y los métodos Exponenciales tipo Adams~\cite{hochbruck2011exponential}, todos de orden superior a 2.

Los integradores exponenciales también son aplicados a otras áreas además de las ecuaciones diferenciales. Dichos integradores fueron introducidos primeramente en la solución de Ecuaciones Diferenciales Estocásticas~\cite{ozaki19852,ozaki1985statistical}, problemas de filtrado continuo-discreto~\cite{ozaki1993local} y estimación de procesos de difusión~\cite{ozaki19852,ozaki1985statistical,ozaki1994local}. Luego fueron extendidos para la integración de Ecuaciones diferenciales aleatorias~\cite{carbonell2005local} y con retardo~\cite{jimenez2006local} y también para Ecuaciones Diferenciales
Parciales Estocásticas semilineales~\cite{jentzen2009overcoming,kloeden2011exponential}.

En general los integradores exponenciales representan una alternativa a los métodos implícitos para resolver ecuaciones diferenciales \textit{stiff} o altamente oscilatorias. Teniendo en cuenta que una parte significativa de la estabilidad y las propiedades dinámicas del sistema debe estar contenida en la linealización dela ecuación tratada. Estos integradores por lo general son explícitos y están formados por una solución tan exacta como se requiera de la aproximación lineal y una aproximación numérica de la parte no lineal. Por lo tanto, construir un integrador exponencial requiere de dos tareas básicas: (\textbf{I}) calcular una aproximación para integrales de la forma 
genérica $\int_{0}^{h} \me{-As}F(x(t_n+s)) \,ds$, y (\textbf{II}) evaluar productos de funciones exponenciales matriciales con vectores.

Típicamente, una aproximación polinomial del término no lineal $F$ en (\ref{VCF2}) (o $g_n$ en (\ref{VCF})) resulta en un esquema
exponencial que aproxima la solución de la EDO como una combinación lineal de productos de tipo $\phi_k(\gamma hA)v_k$, con 
$v_k\in \mathbb{R}^{d}$, donde la función $\phi_k$ está definida como
\begin{eqnarray}
    \phi_k(z) & = &\int\limits_{0}^{1}\me{z(1-s)}\frac{s^{k-1}}{(k-1)!}\,ds \nonumber \\
    \phi_k(z) & = &\sum\limits_{j=0}^{\infty}\frac{z^{j}}{(k+j)!}\label{DEF-PHI}\\
    \phi_0(z) &=& \me{z} \nonumber\\
    \phi_{k+1}(z) &=& \frac{\phi_k(z)-\phi_k(0)}{z}\,,\,k\geq0. \nonumber
\end{eqnarray}

Entre la familia de funciones $\phi_k$ y la exponencial matricial existe la siguiente importante relación.

\begin{theorem}\label{exp-phi}
    \cite{sidje1998expokit} Sea $c\in\mathbb{C}^{m}$, $H_m\in\mathbb{C}^{m\times m}$ y 
    \begin{equation}
    \overline{H}_{m+p} = \left[\begin{array}{ccccc}
    H_m & c & 0 & \cdots & 0 \\
          & 0 & 1 & \ddots & \vdots \\
          &   & 0 & \ddots & 0 \\
          &   &   & \ddots & 1 \\
       0  &   &   &        & 0
    \end{array}\right]\in \mathbb{C}^{(m+p)\times(m+p)}
    \end{equation}
    entonces
     \begin{equation}
    \me{\tau\overline{H}_{m+p}} = \left[\begin{array}{ccccc}
    \me{\tau H_m} & \tau\phi_1(\tau H_m)c & \tau^{2}\phi_2(\tau H_m)c & \cdots & \tau^{p}\phi_p(\tau H_m)c \\
      & 1 & \frac{\tau}{1!} & \cdots & \frac{\tau^{p-1}}{(p-1)!} \\
      &   & 1 & \ddots & \vdots \\
      &   &   & \ddots & \frac{\tau}{1!} \\
    0 &   &   &        & 1
    \end{array}\right]  \;.
    \end{equation}
\end{theorem} 


\section{Método de Linealización Local}
En esta sección se presentan los resultados básicos los Métodos de Linealización Local que se requieren para esta tesis.

\subsection{Aproximación Lineal Local}

El método clásico de Linealización Local de orden 2 de Pope~\cite{pope1963exponential}
 consiste en aproximar en cada paso el campo vectorial de (%
\ref{ODE-SYST}) mediante la expansión de Taylor de primer orden, para
luego resolver exactamente la ecuación lineal resultante. Más
precisamente, si $A_{n}y(t)+a_{n}(t)$ denota la aproximación de Taylor
de primer orden de $f$ en una vecindad de $(t_n,y_{n})$, donde \mbox{$
	A_{n}=f_{x}(t_n,y_{n})$} y $a_{n}(t)=f(t_n,y_{n})-A_{n}y_{n}+f_t(t_n,y_n)(t-t_n)$, entonces la
solución de la ecuación 
\begin{eqnarray*}
\frac{dy}{dt} & = & A_{n}y_n+a_{n}(t) \\ %\label{ODE-SYST-LINEAL-1} \\
y(t_{n})& = & y_{n}  \nonumber
\end{eqnarray*}
para todo $t\in[t_{n},t_{n+1}]$ es una aproximación de la solución
de (\ref{ODE-SYST}) con condición inicial \mbox{$x(t_{n})=y_{n}$}. Mediante la fórmula de variación de la constante obtenemos 
\begin{equation}
y(t)=y_{n}+\varphi(t_{n},y_{n},t-t_{n})  \label{ODE-SYST-FORM-LL}
\end{equation}
donde 
\begin{equation*}
\varphi(t_{n},y_{n};t-t_{n})=\int\limits^{t-t_{n}}_{0} e^{{A_{n}(t-t_{n}-u)}}
(A_{n}y_{n}+a_{n}(t_{n}+u))\,du.  %\label{REV-PHI-DEF-2}
\end{equation*}

Aplicando recursivamente la expresión anterior obtenemos la siguiente
definición.

\begin{definition}
	\label{definition LLD} Dada una partición $(t)_{h}$, la discretización Lineal Local de la solución de (\ref{ODE-SYST}) está dada por la expresión recursiva
	\begin{equation}
	y_{n+1}=y_{n}+\varphi \left( t_{n},y_{n},h_{n}\right) ,  \label{ODE-LLA-4}
	\end{equation}%
	con $y_{0}=x_{0}$ y $n=0,1,\ldots,N-1$.
\end{definition}

Para todo $t\in[t_{0},T]$ se define la siguiente aproximación.
\begin{definition}
	\label{definition LLA} Dada una partición $(t)_{h}$, la aproximación
	Lineal Local de la solución de (\ref{ODE-SYST}) está dada por 
	\begin{equation}
	y(t)=y_{n_{t}}+\varphi(t_{n_{t}},y_{n_{t}},t-t_{n_{t}})
	\label{ODE-REV-FORM-LLA}
	\end{equation}
	para todo $t\in[t_{0},T]$, donde $y_{0}=x_{0}$, $y_{n_{t}}$ es la discretización Lineal Local (\ref{ODE-LLA-4}) y ${n_{t}=\maxx{n:t_{n}\leq t}}$.
\end{definition}
%Otros  integradores que provienen de esta aproximación son~\cite{Jimenez02AMC,Jimenez05AMC}.

 La convergencia, la estabilidad lineal y varias propiedades dinámicas de  la discretización Lineal Local (\ref{ODE-LLA-4}) se pueden encontrar en \cite{Jimenez02AMC}.

\subsection{Aproximación Lineal Local de Orden Superior}

La estabilidad y las propiedades dinámicas que la discretización Lineal Local (\ref{ODE-LLA-4}) posee son de vital importancia para la integración de EDOs, no obstante su bajo orden de convergencia constituye una limitante en algunas circunstancias. Para solucionar esa dificultad se construyen los métodos de Linealización Local de Orden Superior los cuales retienen las propiedades mencionadas, pero con orden de convergencia mayor. 

Estos nuevos métodos se obtienen de agregar a la aproximación (\ref%
{ODE-REV-FORM-LLA}) un término adicional $\xi$ de orden superior a 2 
que resulta de aproximar la segunda integral de la expresión (\ref{VCF}) por algún método de cuadratura o por la solución numérica de una ecuación diferencial auxiliar \cite{delaCruz06}. Usando diferentes tipos de cuadraturas se obtienen los méodos Exponenciales de Propagación 
Iterativa~\cite{tokman2006efficient}, los méodos de Taylor Localmente Linealizados~\cite{delaCruz07}, los
méodos Exponenciales tipo Rosenbrok~\cite{hochbruck2009exponential} y los méodos Exponenciales tipo Adams~\cite{hochbruck2011exponential}. Por otra parte, usando esquemas Runge Kutta (RK) para resolver la mencionada ecuación auxiliar se obtienen los métodos Runge Kutta Localmente Linealizados~\cite{delaCruz06,Jimenez13,Jimenez14AMC}.

Por su relevancia en esta tesis, a continuación se expondrá con mas detalle los métodos Runge Kutta Localmente Linealizados. 

Dada una partición $(t)_{h}$, una aproximación Lineal Local de orden $%
\gamma$ de la solución de (\ref{ODE-SYST}) se define por la expresión 
\begin{equation}
y(t)=y_{n_{t}}+\varphi(t_{n_{t}},y_{n_{t}},t-t_{n_{t}})+\xi(t,t_{n_{t}},y_{n_{t}}),
\label{ODE-REV-HOLL}
\end{equation}
donde $\xi(t,t_{n_{t}},y_{n_{t}})$ es la solución numérica de orden $\gamma >2$ de la ecuación auxiliar
\begin{eqnarray}
\frac{du}{dt} & = & q(t_{n},y_{n},t,u(t))  \label{ODE r} \\
u(t_{n}) & = & 0 \nonumber
\end{eqnarray}
para todo $t\in [t_{n},t_{n+1})$, con campo vectorial 
\begin{eqnarray*}
q(t_n,y_n,s,\varsigma)&=&f(s,y_n+\varphi(t_n,y_n,s-t_n)+\varsigma)-f_x(t_n,y_n)\varphi(t_n,y_n,s-t_n)-f(t_n,y_n)%\\
% & &-f_t(t_n,y_n)(s-t_n) -f(t_n,y_n) ,
%\label{AUX-ODE}
\end{eqnarray*}
con $s\in [t_{n},t_{n+1})$ y $\varsigma \in \mathbb{R}^d $. 

Cuando $\xi$ es la aproximación obtenida mediante un esquema Runge Kutta de orden $\gamma$ se llega a las definiciones siguientes.
\begin{definition}
	\label{definition HLLD} \cite{Jimenez13} Dada una partición $(t)_{h}$, la discretización Lineal Local-Runge Kutta 
    de orden $\gamma >2$ se define por la expresión recursiva:
	\begin{equation*}
	y_{n+1}=y_{n}+\varphi \left( t_{n},y_{n},h_{n}\right) +h_{n}\sum_{j=1}^{s}b_{j}\kt_{j}
	%\label{LLRK_Discretizat}
	\end{equation*}%
	con $y_{0}=x_{0}$, donde las constantes $b_{j}$ y las funciones $k_{j}$ están definidas por el método Runge Kutta de $s$ estados aplicado a la ecuación~(\ref{ODE r}).
\end{definition}

\begin{definition}
	\label{definition HOLLA} \cite{Jimenez13} Dada una partición $(t)_{h}$, la aproximación Lineal Local-Runge Kutta de orden $\gamma$ se define por 
	\begin{equation*}
	y(t)=y_{n_{t}}+\varphi(t_{n_{t}},y_{n_{t}},t-t_{n_{t}})+(t-t_{n_{t}})%
	\sum_{j=1}^{s}b_{j}\kt_{j}(t_{n_{t}},y_{n_{t}},t-t_{n_{t}}) %\label{LLRK_Approx}
	\end{equation*}
	para todo $t\in[t_{0},T]$, donde $y_{n_{t}}$ es la discretización Lineal Local-Runge Kutta de orden $\gamma$, y $n_{t}=\maxx{n:t_{n}\leq t}$.
\end{definition}
\todo[inline]{mejorar o cambiar siguiente oración}
A continuación se construirán dos discretizaciones Lineales Locales-Runge Kutta de orden 4 y 4-5 respectivamente.

Si se utiliza la fórmula Runge-Kutta clásica de order 4~\cite{hairer1993solving} para integrar la ecuación auxiliar (\ref{ODE r}), entonces se obtiene la siguiente discretización Lineal Local-Runge Kutta de orden 4~\cite{Jimenez13}:

\begin{equation}
    y_{n+1}\,=\,y_n+\varphi(t_n,y_n,h_n)+\frac{h_n}{6}(2\kt+2\kt_3+\kt_4)
    \label{LLDP scheme}
    \end{equation}
    \[ \kt_j = f\left( t_n+c_jh_n\, , \, y_n+\varphi(t_n,y_n,c_jh_n)+c_jh_n\kt_{j-1} \right)
- f_x(t_n\, , \, y_n)\varphi(t_n,y_n,c_jh_n) - f(t_n\, ,\, y_n) ,\]

donde $\kt_1=0$ y $c = \left[ 0 \; \frac{1}{2} \; \frac{1}{2} \; 1  \right]$.

Si se utilizan las fórmulas embebidas Runge-Kutta de Dormand y Prince para integrar la ecuación auxiliar (\ref{ODE r}), entonces se obtiene la siguiente discretización Lineal Local-Runge Kutta de orden 5-4~\cite{Jimenez14AMC}:

\begin{equation}
y_{n+1}\,=\,y_n+\varphi_s+h_n \sum_{j=1}^{s}b_j \kt_j \quad \text{y} \quad
\widehat{y}_{n+1}\,=\, y_n+\varphi_s+h_n \sum_{j=1}^{s}b_j \kt_j
\label{LLDP scheme}
\end{equation}
donde $s = 7$ es el número de estados,
\[ \kt_j = f\left( t_n+c_jh_n\, , \, y_n+\varphi_j+h_n \sum_{i=1}^{j-1}a_{j,i}\kt_i \right)  
- f_x(t_n\, , \, y_n)\varphi_j - f(t_n\, ,\, y_n),\]
\[ \varphi _j = \varphi \left( t_{n},y_{n},c_jh_{n}\right), \]
y coeficientes Runge-Kutta $a_{j,i}$, $b_j$, $\hat{b}_j$ and $c_j$ definidos en~\cite{hairer1993solving} Tabla 5.2.

\subsection{Esquemas de Linealización Local}

Como puede notarse de sus respetivas definiciones, las
discretizaciones Lineales Locales no pueden ser implementadas directamente por
estar expresadas en término de integrales que en general no pueden ser
calculadas analíticamente. Dependiendo de la forma de calcular  numéricamente esas integrales,
 diferentes esquemas numéricos pueden ser obtenidos \cite{Jimenez05AMC,Jimenez13}.
 Más precisamente tenemos la siguiente definición.
\begin{definition}
	\label{definition LLS} Dada la discretización Lineal Local 
	  $y_n=y_{n}+\digamma(t_{n},y_{n},t-t_{n})$
	 de orden $\gamma $, toda fórmula recursiva de la forma 
	\begin{equation*}
	 \widetilde{y}_{n+1}=\widetilde{y}_{n}+\widetilde{\digamma}(t_{n},\widetilde{y}_{n},t-t_{n})
	 , \text{ \ \ \ \
		\ con }\widetilde{y}_{0}=y_{0},
	\end{equation*}%
	donde $\widetilde{\digamma}$ denota una aproximación
	de $\digamma$ mediante un algoritmo numérico, es
	llamada genéricamente esquema de Linealización Local (LL).
\end{definition}

Por ejemplo, cuando la fórmula de Padé con escalamiento y potenciación para exponenciales matriciales \cite{moler2003nineteen} es utilizada para
aproximar $\varphi$ en la ecuación~(\ref{ODE-SYST-FORM-LL}) se obtiene el siguiente esquema LL de orden $2$ \cite{Jimenez02AMC}:
\begin{equation} 
\widetilde{y}_{n+1}=\widetilde{y}_{n}+\widetilde{\varphi}\left( t_{n},\widetilde{y}_{n},h_{n}\right) \label{LL-scheme}
\end{equation} 
donde $\widetilde{\varphi }\left( t_{n},\widetilde{y}
_{n},h_{n}\right) =L\,(P_{\pf,\qf}(2^{-k_{n}}\widetilde{M}_{n}h_{n}))^{2^{k_{n}}}\,r$, 
$(P_{\pf,\qf}(2^{-k_{n}}\widetilde{M}_{n}h_{n}))^{2^{k_{n}}}$ es la aproximación
de Padé de orden $(\pf,\qf)$ de  $\me{\widetilde{M}_{n}h_{n}}$, donde
$k$ es el menor número natural tal que $\left\Vert 2^{-k_{n}}\widetilde{M}_{n}h_{n}\right\Vert \leq \frac{1}{2}$, 
y las matrices $\widetilde{M}_{n}$, $L$, $r$ están definidas como

\begin{equation*}
\widetilde{M}_{n}=\left[ 
\begin{array}{ccc}
f_{x}(t_{n},\widetilde{y}_{n}) &f%
_{t}(t_{n},\widetilde{y}_{n}) & f(t_{n},\widetilde{
		y}_{n}) \\ 
0 & 0 & 1 \\ 
0 & 0 & 0%
\end{array}%
\right] \in \mathbb{R}^{(d+2)\times (d+2)},
\end{equation*}%
$L=\left[ 
\begin{array}{ll}
I_{d} & 0_{d\times 2}%
\end{array}%
\right] $ y $r^{\intercal }=\left[ 
\begin{array}{ll}
\mathbf{0}_{1\times (d+1)} & 1%
\end{array}%
\right] $ para EDOs no autónomas; y 
\begin{equation*}
\widetilde{M}_{n}=\left[ 
\begin{array}{cc}
f_{x}(t_{n},\widetilde{y}_{n}) & f(t_{n},%
\widetilde{y}_{n}) \\ 
0 & 0%
\end{array}%
\right] \in \mathbb{R}^{(d+1)\times (d+1)},
\end{equation*}%
$L=\left[ 
\begin{array}{ll}
I_{d} & 0_{d\times 1}%
\end{array}%
\right] $ y $r^{\intercal }=\left[ 
\begin{array}{ll}
\mathbf{0}_{1\times d} & 1%
\end{array}%
\right] $ para ecuaciones autónomas.

Si se utiliza la fórmula de Padé para aproximar $\varphi_j$ que aparecen en las discretizaciones (\ref{LL-scheme}) y (\ref{LLDP scheme}) se obtienen los esquemas propuestos en \cite{Jimenez13,Jimenez14AMC} que están orientados a resolver problemas de valor inicial de dimensiones pequeñas. 

\section{Cálculo de la exponencial de una matriz}

\begin{definition}
    \label{EXPM}\cite{golub2013matrix} Sea $A$ una matriz de $n\times n$ real o compleja. La exponencial de $A$ denotada por
    $ \me{A} $ o $\mathrm{exp}(A)$ es una matriz de $n\times n$ dada por la serie de potencias
    \[\me{A}=\sum_{k=0}^{\infty}\frac{1}{k!}A^{k}.\]
\end{definition}
En muchas aplicaciones se utiliza $\me{tA}$ donde $t$ es un escalar, por tanto 
\begin{equation}
\me{tA}=I+tA+\frac{t^{2}A^{2}}{2!}+\cdots\label{EXPM-SERIE}
\end{equation}
\begin{theorem}\cite{IntroMatrix}
    La serie de matrices definida en~(\ref{EXPM-SERIE}) existe para toda matriz $A$ para $t$ fijo y
    para todo $t$ para $A$ fija. La serie converge uniformemente en cualquier región de t del plano complejo.
\end{theorem}

En \cite{golub2013matrix} se presenta una revisión de los méodos numéricos
para aproximar
la exponencial de matrices. Los méodos más utilizados son la aproximación de Padé con escalamiento
y potenciación para matrices pequeñas y densas, y el de los subespacios de Krylov para matices grandes y esparcidas.


\subsection{Aproximación de Padé}

\begin{definition}
    \cite{golub2013matrix}~La aproximación racional de Padé $P_{\pf,\qf}(z)$
    de $\me{z}$ est\'{a} dada por 
    \begin{equation*}
    P_{\pf,\qf}(z)=D_{\pf,\qf}(z)^{-1}N_{\pf,\qf}(z),  \label{FORM PADE}
    \end{equation*}%
    donde 
    \[
    N_{\pf,\qf}(z)=\sum\limits_{k=0}^p\frac{(p+q-k)!p!}{(p+q)!k!(p-k)!}z^k
    \]%
    y 
    \[
    D_{\pf,\qf}(z)=\sum\limits_{k=0}^q\frac{(p+q-k)!q!}{(p+q)!k!(q-k)!}(-z)^k
    \]
\end{definition}

Cuando la norma de $A$ es grande, el costo computacional  y los errores de redondeo aumentan, por lo que
es necesaria la alternativa de escalamiento y potenciación.

\begin{definition}\cite{golub2013matrix}~
    Sea $A$ una matriz cuadrada, y sea $P_{\pf,\qf}(2^{-k}A%
    )$ la aproximación de Padé de orden $(\pf,\qf)\ $de $\me{2^{-k}A%
    }$, donde $k$ es el menor número natural tal que $\left\vert 2^{-k}A%
    \right\vert \leq \frac{1}{2}$. \ Se define la aproximación de  Padé-$(\pf,\qf)$ con escalamiento y potenciación como 
    \begin{equation}
    F_{k}^{\pf,\qf}(A)=(P_{\pf,\qf}(2^{-k}A))^{2^{k}}.
    \label{P-MA-2}
    \end{equation}
\end{definition}

Resultados sobre el error y la estabilidad de las aproximaciones de Padé se presentan a continuación.

\begin{theorem}
    \label{Conv. Pade}\cite{jimenez2012convergence} El error
    relativo y absoluto de la aproximación de Padé con escalamiento y
    potenciación (\ref{P-MA-2}) de $\me{A}$ est\'{a} dado por: 
    \[
    \frac{\left\vert e^{A}-F_{k}^{\pf,\qf}(A)\right\vert 
    }{\left\vert e^{A}\right\vert }\leq \epsilon _{p,q}\left\vert 
    A\right\vert e^{\epsilon _{p,q}\left\vert A\right\vert }
    \]%
    y 
    \begin{equation}
    \left\vert \me{A}-F_{k}^{\pf,\qf}(A)\right\vert \leq
    c_{p,q}(k,\left\vert A\right\vert )\left\vert A\right\vert
    ^{p+q+1}, \label{errpade}
    \end{equation}
    donde $\epsilon _{p,q}=\alpha (\frac{1}{2})^{p+q-3}$, $%
    c_{p,q}(k,\left\vert A\right\vert )=\alpha
    2^{-k(p+q)+3}\me{(1+\epsilon _{p,q})\left\vert A\right\vert }$ y $%
    \alpha =\frac{p!q!}{(p+q)!(p+q+1)!}$.
\end{theorem}

\begin{theorem}\label{Stab. Pade}[Teorema 4.12 en \cite{wanner1996solving}] 
    La aproximación de Padé $P_{\pf,\qf}(z)$  es \emph{A-estable} si y solo si $\pf\leq \qf\leq \pf+2$. 
    Todos los ceros y todos los polos son simples.
\end{theorem}

\subsection{Aproximación de Krylov}

Los estudios muestran que la aproximación de Padé con escalamiento y potenciación
es efectiva para calcular exponenciales de matrices relativamente peque\~nas y densas; pero no as\'i para matrices de dimensiones moderadas o grandes.

Es conocido del \'Algebra Lineal Numérica que para problemas de
grandes dimensiones los métodos iterativos son preferidos por sobre los métodos directos.
En particular,
los métodos iterativos de los subespacios de Krylov han cobrado auge en la resolución de
problemas de \'Algebra Lineal que involucran el producto de una función matricial y un vector,
tales como la resolución de grandes sistemas lineales y en el c\'alculo de valores
propios~\cite{golub2013matrix}.

Es importante destacar que en la mayor\'ia de los integradores exponenciales
 no se necesita el c\'alculo de la matriz $\me{A}$ completa, sino
el producto $\me{A}b$ donde $b$ es un vector determinado. Por ejemplo, la solución del problema de valor inicial homogéneo $\dot{x}(t)=Ax(t),\;x(0)=x_0$, es $x(t)=\me{At}x_0$ en forma de producto exponencial de matriz y vector. Esta es la razón por la que
los métodos de los subespacios de Krylov han sido ampliamente utilizados para evaluar productos de esta forma.

\begin{definition}
        \cite{Saad92} Para $\mf\geq 1$, el $\mf$-ésimo subespacio de Krylov de la matriz $A\in\mathbb{C}^{N\times N}$
    y el vector $b\in\mathbb{C}^{N}$ se define como
    \[ \mathcal{K}_\mf(A,b)=\mathrm{span}\left\{ b,Ab,A^{2},b,\ldots,A^{\mf-1}b \right\}. \]
\end{definition}

Antes de definir la aproximación por subespacios de Krylov del producto de una función matricial y un vector es
necesario construir una base ortonormal. Para construir dicha base se utiliza el algoritmo de
Arnoldi~\ref{alg:Arnoldi} para matrices generales y el algoritmo de 
Lanczos para matrices hermíticas~\cite{arnoldi,saad2003iterative}.

\begin{algorithm}
    \caption{Algoritmo de Arnoldi para construir una base ortonormal del subespacio de Krylov $\mathcal{K}_\mf(\tau A,b)$}
    \label{alg:Arnoldi}
    \KwIn{Matriz $A\in \mathbb{R}^{d\times d}$, vector $b\in \mathbb{R}^{d}$ y constante $\tau$}
    \KwOut{Base ortonormal $\{v_1,v_2,\ldots,v_\mf,v_{\mf+1}\}$ de $K_\mf(\tau A,b)$ y matriz de Hessenberg $H_\mf=V_\mf^{\intercal}\tau A V_\mf $,
        donde $V_{\mf+1}=[v_1\,v_2\,\cdots \,v_\mf\,v_{\mf+1}]\in \mathbb{R}^{d\times \mf+1}$, $breakdown$ }
    $breakdown=false$\\
    $v_1=b/\lVert b \rVert_2$\\
    \For{ $j=1,2,\ldots,\mf$ }{
        $w_j=\tau A v_j$\\
        \For{ $i=1,\ldots,j$}{ 
            $\hf_{ij}=\langle w_j,v_i \rangle$\\
            $w_j=w_j-\hf_{ij}v_i$       
        }
        $\hf_{j+1,i}=\lVert w_j \rVert_2$\\
        \eIf(\tcp*[h] \emph{Break Down}\label{alg:breakdown}){$\hf_{j+1,i}<2\epsilon_{mach}$}{
            $\mf_{cut}=j$\\
             $breakdown=true$\\
            Stop
        }{
            $v_{j+1}=w_j/\hf_{j+1,i}$
        }
    }
\end{algorithm}


\begin{definition}
    \cite{Saad92} Sea $V_\mf$ una base ortonormal de $\mathcal{K}_\mf(A,b)$ y sea $f$ una función matricial definida
    en el espectro de $A$. La aproximación de $f(A)b$ por el  $\mf$-ésimo subespacio de Krylov se define
    \begin{equation}
        f(A)b\approx ||b||_2 V_\mf f(H_\mf)e_1, \label{MFUNC-APROX}
    \end{equation}
    donde $H_\mf=V_\mf^{\intercal}AV_\mf$ y $e_1$ es el primer vector canónico de $\mathbb{R}^\mf$.
    
\end{definition}

Cuando la función $f$ en~(\ref{MFUNC-APROX}) es la exponencial matricial se obtiene el siguiente resultado.
\begin{theorem}\label{exp-bound}
	\cite{Saad92} Sea $A\in\mathbb{C}^{N\times N}$ una matriz. El error de aproximación de $\me{A}b$ por el $\mf$-ésimo subespacio
	de Krylov es
	\begin{equation*}
	\left\lvert\left\lvert \me{A}b - \lvert\lvert b \rvert\rvert_2 V_\mf\me{H_\mf}e_1 \right\rvert\right\rvert_2 
	\leq \frac{2\vert\lvert b \rvert\rvert_2 \left( \sigma(A) \right)^{\mf}\me{\sigma(A)}}{\mf!},
	\end{equation*}
	donde $\sigma(A)$ es el radio espectral de $A$.
\end{theorem}

En general, para la familia de funciones $\phi_k$ se establece la expansión en serie \cite{Saad92,Sidje98}
\begin{equation}
\phi_k(\tau A)b=||b||_2\tau^{k} V_\mf \phi_k(\tau H_\mf)e_1 + ||b||_2\hf_{\mf+1,\mf}
\sum_{j=k+1}^{\infty}\tau^{j} e_\mf^T\phi_j(\tau H_\mf)e_1A^{j-k-1}v_{m+1} \label{PHI-EXPANSION}
\end{equation}
donde $\tau$ es un número positivo y $e_\mf$ es el  $\mf$-ésimo vector canónico de $\mathbb{R}^\mf$.
\chapter{Solución de ecuaciones lineales de dimensiones no pequeñas}\label{chapter:solve-non-smal-lineal-eq}

En el capítulo anterior, se presentó la discretización Lineal Local~(\ref{definition LLS}), así como su implementación basada en Padé~(\ref{LL-scheme}) para pequeños PVI. Para PVI de medianas y grandes dimensiones la aproximación de Padé(\ref{section:pade-approx}) para la exponencial ya no es computacionalmente eficiente y necesita ser reemplazada por alguna aproximación basada en subespacios de Krylov o alguna técnica proyectiva similar. Con el propósito de aplicar aproximación basadas en subespacios de Krylov primero se reescribirá la solución de la ecuación lineal. Nótese que dada la ecuación lineal
\begin{equation*}
	\frac{dx(t)}{dt}=Ax(t)\;,\;\; x(t_0)=x_0,\;\;\; A\in\mathbb{R}^{d\times d}
\end{equation*}
además de la forma integral, la solución puede ser escrita de las siguientes formados
\begin{eqnarray}
	x(t) & = & \me{tA}x_0 \label{eqaux:lineal-solution-ex} \\
	x(t) & = & x_0 + L\me{tM}r \label{eqaux:lineal-solution-xex} \\
	x(t) & = & x_0 + t\varphi_1(tA)Ax_0 \label{eqaux:lineal-solution-xphix}
\end{eqnarray}
donde $\varphi_1$ está definida en (\ref{phi-definition}) y
\begin{equation*}
	L=\left[ \begin{array}{ll} I_{d} & 0_{d\times 1}\end{array} \right] \;,\;\;
	r^{\intercal }=\left[ \begin{array}{ll} \mathbf{0}_{1\times d} & 1\end{array}\right] \;,\;\;
	M=\left[
	\begin{array}{cc}
	A & Ax_0 \\
	0 & 0
	\end{array}
	\right]\;.
\end{equation*}
\todo[inline]{mejorar oración}
En trabajos como~\cite{Saad92,jimenez2012convergence} se utilizan aproximaciones de Krylov para las soluciones de las forma~(\ref{eqaux:lineal-solution-ex},\ref{eqaux:lineal-solution-xex}), dichas aproximaciones de orden $\mf$, donde $\mf$ es la dimensión del subespacio utilizado. Las soluciones de la forma~(\ref{eqaux:lineal-solution-xphix}) se han aproximado con orden $\mf+1$ en trabajos como~\cite{sidje1998expokit,hochbruck1998exponential}. Tomando soluciones de la forma~(\ref{eqaux:lineal-solution-xphix}) es necesario tener buenas aproximaciones de $\varphi_1$, por lo que en este capítulo se introducirán las aproximaciones Krylov-Padé para $\varphi_1$, así como dichas aproximaciones sin evaluar el Jacobiano. También se derivarán cotas para las aproximaciones propuestas y se propondrá una estrategia para la selección de la dimensión de los subespacios.

\section{Aproximaciones Krylov-Padé}\label{section:krylov-pade-approx}

 La familia de operadores $\varphi$ tiene un papel fundamental en las aproximaciones Localmente Linealizadas, por esto es necesario poder evaluar estos operadores de forma rápida y precisa. Además, como estos operadores aparecen actuando sobre un vector, entonces se aproxima la acción sobre el vector. La expansión en serie~(\ref{PHI-EXPANSION}) permite relacionar la acción de un operador sobre un vector por un subespacio de Krylov. Truncando la serie~(\ref{PHI-EXPANSION}) diferentes aproximaciones son obtenidas. Típicamente~\cite{niesen2012algorithm,sidje1998expokit,tokman2006efficient} el primer término de la serie es tomado y el segundo término es utilizado como medida del error; pero, en este trabajo, similar a~\cite{Saad92} se tomarán los dos primeros términos de la serie (\ref{PHI-EXPANSION})
 \begin{equation}\label{exact-two-terms-km}
    K_\mf(\tau,A,b) = \norme{b}\tau V_\mf\varphi_1(\tau H_\mf)e_1 + \norme{b}\tau^2\hf_{\mf+1,\mf}e^T_\mf\varphi_2(\tau H_\mf)e_1v_{\mf+1}
 \end{equation}
para aproximar $\tau\varphi_1(\tau A)b$ y el tercer término como medida del error.
-
Al aplicar el algoritmo de Arnoldi (\ref{alg:Arnoldi}), la propiedad de invarianza ante escalado de la base ortonormal del subespacio y el Teorema \ref{theorem:exp-phi} podemos calcular las $\phi_j$ necesarias realizando una exponencial; del Algoritmo (\ref{alg:Arnoldi}) se obtiene la matriz de Hessenberg $H^*\mf$ para $\mathcal{K}_\mf(h_n f_x,f)$. Escalando esta matriz se obtiene $H_\mf=H^*\mf/h_n$ para $\mathcal{K}_\mf(f_x,f)$ y al realizar la exponencial de la matriz particionada
\begin{equation*}
    \overline{H} = \left[\begin{array}{cccc}
    H_\mf & e_1 & 0_{\mf\times 1} & 0_{\mf\times 1}\\
    0_{1\times\mf} & 0 & 1 & 0\\
    0_{1\times\mf} & 0 & 0 & 1\\
    0_{1\times\mf} & 0 & 0 & 0
    \end{array}\right] \label{hhat_hessenberg_matrix}
    \end{equation*}
se obtiene de aplicar el Teorema \ref{theorem:exp-phi}
\begin{equation}
    \me{\tau\overline{H}} = \left[\begin{array}{cccc}
    \me{\tau H_m} & \tau\varphi_1(\tau H_m)e_1 & \tau^{2}\varphi_2(\tau H_m)e_1 &
    \tau^{3}\varphi_3(\tau H_m)e_1 \\
    & 1 & \tau & \frac{\tau^{2}}{2}\\
    &  & 1 & \tau \\
    &   &   & 1 \\
    \end{array}\right] \,. \label{phi_hhat_exponential}
\end{equation}

Se denotará por $E_\tau = \me{\tau \overline{H} }$ y $[E_\tau]_{ij}$ a las particiones de la matriz $E_\tau$. Al aplicar (\ref{phi_hhat_exponential}) en (\ref{exact-two-terms-km}) se obtiene
\begin{equation*}
    K_\mf(\tau,A,b) = \norme{b}V_\mf [E_\tau]_{12} + \norme{b}\hf_{\mf+1,\mf}e^T_\mf[E_\tau]_{13}v_{\mf+1} \, .
 \end{equation*}

La exponencial (\ref{phi_hhat_exponential}) no puede ser calculada de forma exacta en el caso general, por lo que se utilizará los aproximantes de Padé para aproximarla. La exponencial matricial es aproximada por
\begin{equation*}
    \widetilde{E}_{\tau} = F_k^{\pf,\qf}\left(\tau\overline{H}\right),
\end{equation*}
donde $F_k^{\pf,\qf}$~(\ref{P-MA-2}) denota la aproximación de Padé con estrategia escalamiento y potenciación de la función exponencial. Consecuentemente la aproximación (\ref{exact-two-terms-km}) quedaría
\begin{equation} \label{eq:kp_aprox}
    K_{\mf,k}^{\pf,\qf}(\tau,A,b) = \norme{b}V_\mf [\widetilde{E}_\tau]_{12} + \norme{b}\hf_{\mf+1,\mf}e^T_\mf[\widetilde{E}_\tau]_{13}v_{\mf+1} \, .
 \end{equation}

 \subsection{Cotas para aproximaciones Krylov-Padé}
 En está sección se enunciará un teorema para acotar el error de la aproximación (\ref{eq:kp_aprox}) en función del tamaño de paso $h$, la dimensión del subespacio de Krylov $\mf$ y los órdenes de Padé $\pf$ y $\qf$. Para poder enunciar dicho teorema dos lemas adicionales son necesarios. El primer lema extiende para $\tau \varphi_1(\tau A)b$ los resultados del Teorema 4.7 en~\cite{Saad92} para aproximaciones de $\varphi_0(A)b$ mediante subespacios de Krylov. El segundo lema ofrece una cota para la matriz $\overline{H}$.

 \begin{lemma}\cite{naranjo2021locally}\label{lemma:CORRECTED-ERROR}
	Sea $A\in\mathbb{C}^{d\times d}$ una matriz, $b\in\mathbb{C}^{d}$ un vector, $\tau$ un número positivo, y \[{K_{\mf}\left(\tau,A , b \right) = \tau ||b||_2 V_\mf \varphi_1(\tau H_\mf)e_1}+\tau^2 ||b||_2 \hf_{\mf+1,\mf}e_\mf^T\varphi_2\left(\tau H_\mf\right) e_1 v_{\mf+1} \] la aproximación a $\tau \varphi_1(\tau A)b$ tomando los dos primeros términos de la serie (\ref{PHI-EXPANSION}). Entonces,
	\begin{gather*}
	\left\lvert\left\lvert \tau\varphi_1(\tau A)b - K_{\mf}\left(\tau,A , b \right) \right\rvert\right\rvert_2%\nonumber\\
	\leq \frac{2\vert\lvert b \rvert\rvert_2\tau^{\mf+2}\rho^{\mf+1}\me{\tau \rho}}{(\mf+2)!},
	\end{gather*}
	donde $\rho=\nnorm{\nnorm{A}}_2$.
\end{lemma}

\emph{Demostración}
Siguiendo la definición (\ref{phi-definition}), $\tau \varphi_1(\tau A)b$ puede ser reescrito como
\begin{eqnarray*}
	\tau \varphi_1(\tau A)b &=& \tau b+\tau^{2}A\varphi_{2}(\tau A)b \\
	&=& \tau b+\tau^{2}A \,\ (\,\ \nnorm{\nnorm{b}}_2V_\mf\varphi_{2}(\tau H_\mf)e_1 +s_\mf(\tau A) \,\ ), %\label{eq:tauaphi1tauab}
\end{eqnarray*}
donde \[ s_\mf (\tau A)= \varphi_{2}(\tau A)b -\nnorm{\nnorm{b}}_2V_\mf\varphi_{2}(\tau H_\mf)e_1 .\]
De la reescritura de $\phi$, utilizando $AV_\mf=V_\mf H_\mf + \hf_{\mf+1,\mf}v_{\mf+1}e^{T}_\mf$ y  $\nnorm{\nnorm{b}}_2V_\mf e_1=b$ se obtiene
\begin{equation}
\tau\varphi_1(\tau A)b = \tau\nnorm{\nnorm{b}}_2V_\mf\varphi_1(\tau H_\mf)e_1 + \tau^{2}\nnorm{\nnorm{b}}_2\hf_{\mf+1,\mf}e^T_\mf\varphi_{2}(\tau H_\mf)e_1v_{\mf+1}+\tau^{2}As_\mf (\tau A), %\label{eq:tauaphi1tauabapprox}
\end{equation}
y se tiene que
\begin{equation}
\tau\varphi_1(\tau A)b - K_{\mf}\left(\tau,A , b \right) = \tau^{2}As_\mf(\tau A), \label{eq:phidiffasm}
\end{equation}
donde \[{K_{\mf}\left(\tau,A , b \right) = \tau ||b||_2 V_\mf \varphi_1(\tau H_\mf)e_1}+\tau^2 ||b||_2 \hf_{\mf+1,\mf}e_\mf^T\varphi_2\left(\tau H_\mf\right) e_1 v_{\mf+1} \] es una aproximación de $\tau \varphi_1(\tau A)b$.

Utilizando el Lema 4.1 en \cite{Saad92} se tiene
\begin{equation}
s_\mf(\tau A)=\nnorm{\nnorm{b}}_2 ( r_\mf(\tau A)v_1-V_\mf r_\mf(\tau H_\mf)e_1) ,\label{eq:smeq}
\end{equation}
donde $r_\mf(z) = \varphi_{2}(z)-p_{2,\mf-1}(z)$, y $p_{2,\mf-1}$ es un polinomio de grado $\mf-1$. Tomando
\begin{equation*}
p_{2,\mf-1}(z)\equiv \frac{p_{1,\mf}(z)-1}{z},
\end{equation*}
donde $p_{1,\mf}$ es la expansión de Taylor de $\phi_1$ hasta el orden  $\mf$
definida por
\[ p_{1,\mf}(z)= \frac{p_{0,\mf+1}(z)-1}{z}, \]
donde $p_{0,\mf+1}(z)$  es la expansión de Taylor de $\me{z}$ hasta el orden $\mf+1$ dada por
\[ p_{0,m+1}(z)=\sum\limits^{\mf+1}_{j=0}\frac{z^j}{j!}. \]

Con estos polinomios y (\ref{phi-definition}) se tiene
\begin{equation*}
r_\mf(z) = \frac{\varphi_1(z)-1}{z}-\frac{p_{1,\mf}(z)-1}{z}=\frac{\me{z}-1}{z^2}-\frac{p_{0,m+1}(z)-1}{z^2}
=\frac{\me{z}-p_{0,m+1}(z)}{z^2}.
\end{equation*}

El Lema 4.2 en \cite{Saad92} enuncia que
\[ \nnorm{\me{z}-p_{0,m+1}(z)} \leq \frac{z^{\mf+2}\me{z}}{(\mf+2)!}\;\;, \]
de esto se obtiene
\begin{equation}
\nnorm{\nnorm{r_\mf(\tau A)v_1}}_2\leq \frac{\rho^{\mf}\me{\rho}}{(\mf+2)!}\;\;\;\;\; and \;\;\;\;\; \nnorm{\nnorm{r_\mf(\tau H_\mf)v_1}}_2\leq \frac{\widehat{\rho}^{\mf}\me{\widehat{\rho}}}{(\mf+2)!}, \label{polinom}
\end{equation}
donde $\rho=\nnorm{\nnorm{\tau A}}_2 $ y  $\widehat{\rho}=\nnorm{\nnorm{\tau H_\mf}}_2$. Ya que $\nnorm{\nnorm{H_\mf}}\leq\nnorm{\nnorm{A}}$, de (\ref{eq:smeq}) y (\ref{polinom}) se obtiene
\begin{equation*}
\nnorm{\nnorm{\tau As_\mf(\tau A)}}_2  \leq \frac{2 \nnorm{\nnorm{b}}_2\rho^{\mf+1}\me{\rho}}{(\mf+2)!}.
\end{equation*}
Finalmente, de estas desigualdades y (\ref{eq:phidiffasm}) se obtiene
\begin{equation*}
\nnorm{\nnorm{\tau\varphi_1(\tau A)b - K_{\mf}\left(\tau,A , b \right)}}_2  \leq
\frac{2 \nnorm{\nnorm{b}}_2\tau^{\mf+2}\nnorm{\nnorm{A}}_2^{\mf+1}\me{\tau \nnorm{\nnorm{A}}_2}}{(\mf+2)!},
\end{equation*}
lo cual completa la prueba. $\Box$

\begin{lemma}\cite{naranjo2021locally}\label{H-bound}
	Sea $\overline{H}$ la matriz definida en in~(\ref{hhat_hessenberg_matrix}), sea  $H_\mf=V^{T}_\mf AV_\mf$ la matriz de Hessenberg asociada al subespacio de Krylov $\mathcal{K}_\mf(A,b)$ con base ortonormal $V_m$. Entonces
	\[ \lVert\overline{H}\rVert_2 \leq 1 +  \lVert A\rVert_2. \]
\end{lemma}
\emph{Demostración}
De (\ref{hhat_hessenberg_matrix}) se tiene
\begin{eqnarray*}
	\overline{H}&=&L H_\mf L^T + B
\end{eqnarray*}
donde $ L^T=[I_\mf \;\; 0_{\mf\times 3}] $ y
\[B=\left[\begin{array}{ccc}
0_{\mf\times\mf} & e_1 & 0_{\mf \times 2}\\
0_{2\times\mf} & 0_{2 \times 1} & I_2\\
0_{1\times\mf} & 0 & 0_{1\times 2}
\end{array}\right].\]
Ya que $\lVert B\rVert_2 = \lVert L \rVert_2 = 1$, y $\lVert H_m\rVert_2 \le \lVert A \rVert_2$, se obtiene que
\begin{eqnarray*}
	\lVert\overline{H}\rVert_2 &\leq& \lVert L \rVert_2 \lVert H_\mf \rVert_2 \lVert L^T \rVert_2 + \lVert B \rVert_2\\
	&\leq& 1 + \lVert A\rVert_2
\end{eqnarray*}
$\Box$\\ \\


\begin{theorem}\cite{naranjo2021locally}\label{theorem:Krylov-bound}
	Sea 
	\begin{equation} \label{eq:gen_kp_aprox}
	K_{\mf,k}^{\pf,\qf}\left(\tau, A , b \right)=\nnorm{\nnorm{b}}_2 V_{\mf}\;[\widetilde{E}_{\tau}]_{12} + \nnorm{\nnorm{b}}_2 \hf_{\mf+1,\mf}e_\mf^T\;[\widetilde{E}_{\tau}]_{13} v_{\mf+1}
	\end{equation}
	la aproximación $(\mf , \pf ,\qf , k)$-Krylov-Padé de $\tau \varphi_1(\tau A)b$, donde las matrices $V_{\mf}\in\mathbb{R}^{d\times \mf}$ y $H_{\mf}\in\mathbb{R}^{{\mf} \times {\mf}}$, el vector $v_{\mf+1}$, y el número $\hf_{\mf+1,\mf}$ son obtenidos del algoritmo de Arnoldi para el $\mf$-ésimo subespacio de Krylov $\mathcal{K}_\mf(A,b)$;  $\widetilde{E}_{\tau}=F_k^{\pf,\qf}\left(\tau\overline{H}\right)$ es la aproximación $(\pf,\qf)$-Padé con escalamiento $k$ para la exponencial matricial (\ref{phi_hhat_exponential}), y $e_m$ el $m$-ésimo vector canónico de $\mathbb{R}^\mf$.
	Entonces,
	\begin{equation}
	\left\lvert\left\lvert  \tau \varphi_1(\tau A)b -
	K_{\mf,k}^{\pf,\qf}\left( \tau, A , b \right)\right\rvert\right\rvert_2%\nonumber\\
	\leq C_{\mf,k}^{\pf,\qf}\left(\lvert\lvert A \rvert\rvert_2\right) \;
	\nnorm{\nnorm{b}}_2 \; \tau^{\mathrm{min}\left\{ \mf+2,\pf+\qf+1 \right\}}
	\end{equation}
	donde $C_{\mf,k}^{\pf,\qf}(\varLambda)= \frac{2 \varLambda^{m+1} \me{\varLambda}}{(\mf+2)!}+
	\alpha (1+\hf_{\mf+1,\mf}) (1+\varLambda^{p+q+1}) 2^{-k(\pf+\qf)+3}\me{(1+\alpha(\frac{1}{2})^{\pf+\qf-3})(1+\varLambda)} $,
	con $\alpha=\frac{\pf!\qf!}{(\pf+\qf)!(\pf+\qf+1)!}$ y $\tau \in [0,1]$.
\end{theorem}
\textbf{Demostración}. Por desigualdad triangular
\begin{eqnarray} \label{importatdeq}
\left\lvert\left\lvert  \tau \varphi_1(\tau A)b -
K_{\mf,k}^{\pf,\qf}\left( \tau , A , b \right)\right\rvert\right\rvert_2
& \leq & \left\lvert\left\lvert \tau \phi_1(\tau A)b -  K_{\mf}\left( \tau , A , b \right) \right\rvert\right\rvert_2 \\
& & + \left\lvert\left\lvert  K_{\mf}\left( \tau , A , b \right) -
K_{\mf,k}^{\pf,\qf}\left( \tau, A , b \right)\right\rvert\right\rvert_2, \nonumber
\end{eqnarray}
donde
\begin{equation*}
K_{\mf}\left(\tau,A , b \right)=\nnorm{\nnorm{b}}_2 \tau V_{\mf}\varphi_1\left( \tau H_{\mf} \right) e_1 + \nnorm{\nnorm{b}}_2 \tau^{2}\hf_{\mf+1,\mf}e_\mf^T\varphi_2\left(\tau H_\mf\right) e_1 v_{\mf+1}
\end{equation*}
es la aproximación de Krylov de $\tau \varphi_1(\tau A)b$ definida en~(\ref{eq:kp_aprox}), y
\begin{equation*}
K_{\mf,k}^{\pf,\qf}\left(\tau, A , b \right)=\nnorm{\nnorm{b}}_2 V_{\mf}\;[\widetilde{E}_{\tau}]_{12} + \nnorm{\nnorm{b}}_2 \hf_{\mf+1,\mf}e_\mf^T\;[\widetilde{E}_{\tau}]_{13} v_{\mf+1}
\end{equation*}
la aproximación Krylov-Padé de $\tau \varphi_1(\tau A)b$ definida en~(\ref{eq:gen_kp_aprox}).

De (\ref{phi_hhat_exponential}) se tiene que
$\tau^j \varphi_j\left( \tau H_{\mf} \right) e_1 = L \me{\tau \overline{H}} r_j$
y
$[\widetilde{E}_{\tau}]_{1,j+1} = L F_k^{\pf,\qf}(\tau \overline{H}) r_j$,
con $j=1,2$, donde $F_k^{\pf,\qf}(\tau \overline{H})$ es la aproximación $(\pf,\qf)$-Padé de $\me{\tau \overline{H}}$ con estrategia escalamiento y potenciación,
$ L=[I_\mf \;\; 0_{\mf\times 3}] $, $r_1=[0_{1\times \mf}\;\; 1 \;\;0 \;\;0]^T$ y $r_2=[0_{1\times \mf}\;\; 0 \;\;1 \;\;0]^{T}$. De la desigualdad triangular
\begin{equation}\label{proffktheouneq}
%\begin{array}{ccc}
\left\lvert\left\lvert   K_{\mf}\left( \tau , A , b \right) -
K_{\mf,k}^{\pf,\qf}\left( \tau, A , b \right)\right\rvert\right\rvert_2
\leq \left\lvert\left\lvert T_1 -
\widetilde{T}_1\right\rvert\right\rvert_2 \\
+\left\lvert\left\lvert T_2 - \widetilde{T}_2
\right\rvert\right\rvert_2,
%\end{array}
\end{equation}
donde
\begin{eqnarray*}
	T_1&=&\nnorm{\nnorm{b}}_2 V_{\mf} L \me{\tau \overline{H}} r_1\\
	\widetilde{T}_1&=&\nnorm{\nnorm{b}}_2 V_{\mf} L F_k^{\pf,\qf}(\tau \overline{H}) r_1\\
	T_2&=&\nnorm{\nnorm{b}}_2 \hf_{\mf+1,\mf}e_\mf^T\;L \me{\tau \overline{H}} r_2 v_{\mf+1}\\
	\widetilde{T}_2&=&\nnorm{\nnorm{b}}_2 \hf_{\mf+1,\mf}e_\mf^T\;L F_k^{\pf,\qf}(\tau \overline{H}) r_2 v_{\mf+1}.
\end{eqnarray*}

Del Lema 4.1 en \cite{jimenez2012convergence} y el Lema~\ref{H-bound}, se tiene que
\begin{eqnarray}
\left\lvert\left\lvert T_1 - \widetilde{T}_1\right\rvert\right\rvert_2
&\leq& \nnorm{\nnorm{b}}_2 \left\lvert\left\lvert  V_{\mf} \right\rvert\right\rvert_2 \left\lvert\left\lvert  L \right\rvert\right\rvert_2 \left\lvert\left\lvert \me{\tau  \overline{H}}-F_k^{\pf,\qf}(\tau \overline{H}) \right\rvert\right\rvert_2
\left\lvert\left\lvert  r_1 \right\rvert\right\rvert_2 \nonumber\\
&\leq& \nnorm{\nnorm{b}}_2 \;  c_{\pf,\qf}(k,\lvert\lvert\tau\overline{H}\rvert\rvert_2) \;
\lvert\lvert\tau \overline{H}\rvert\rvert_2^{\pf+\qf+1}\nonumber\\
&\leq& \nnorm{\nnorm{b}}_2 \; c_{\pf,\qf}(k,\tau (1+\lvert\lvert A \rvert\rvert_2))
\; (1 + \lvert\lvert A \rvert\rvert_2^{\pf+\qf+1}) \; \tau^{\pf+\qf+1}
\label{err1}
\end{eqnarray}
y
\begin{eqnarray}
\left\lvert\left\lvert T_2 - \widetilde{T}_2 \right\rvert\right\rvert_2
&\leq& \nnorm{\nnorm{b}}_2 \lvert \hf_{\mf+1,\mf}\rvert \left\lvert\left\lvert  e_\mf^T \right\rvert\right\rvert_2 \left\lvert\left\lvert  L \right\rvert\right\rvert_2 \left\lvert\left\lvert \me{\tau \overline{H}}-
F_k^{\pf,\qf}(\tau \overline{H}) \right\rvert\right\rvert_2  \left\lvert\left\lvert  r_2 \right\rvert\right\rvert_2 \left\lvert\left\lvert v_{\mf+1}\right\rvert\right\rvert_2 \nonumber\\
&\leq&\nnorm{\nnorm{b}}_2 \lvert \hf_{\mf+1,\mf}\rvert \; c_{\pf,\qf}
(k,\lvert\lvert\tau \overline{H}\rvert\rvert_2) \;
\lvert\lvert\tau \overline{H}\rvert\rvert_2^{\pf+\qf+1}\nonumber\\
&\leq& \nnorm{\nnorm{b}}_2 \lvert \hf_{\mf+1,\mf}\rvert \; c_{\pf,\qf}(k,\tau (1+\lvert\lvert A \rvert\rvert_2))
\; (1 + \lvert\lvert A \rvert\rvert_2^{\pf+\qf+1}) \; \tau^{\pf+\qf+1}\label{err2},
\end{eqnarray}
donde $c_{\pf,\qf}(k,\vartheta )=\alpha
2^{-k(\pf+\qf)+3}\me{(1+\alpha(\frac{1}{2})^{\pf+\qf-3})\vartheta }$ y $%
\alpha =\frac{\pf!\qf!}{(\pf+\qf)!(\pf+\qf+1)!}$. sustituyendo (\ref{err1}) y (\ref{err2}) en (\ref{proffktheouneq}), se obtiene que
%\begin{eqnarray}
\begin{equation}
\left\lvert\left\lvert   K_{\mf}\left( \tau , A , b \right) -
K_{\mf,k}^{\pf,\qf}\left( \tau, A , b \right) \right\rvert\right\rvert_2
\leq \nnorm{\nnorm{b}}_2 (1+\lvert\hf_{\mf+1,\mf}\rvert) \; c_{\pf,\qf}(k,\tau (1+\lvert\lvert A \rvert\rvert_2))
\; (1 + \lvert\lvert A \rvert\rvert_2^{\pf+\qf+1}) \; \tau^{\pf+\qf+1}. \label{errp1}
\end{equation}
%\end{eqnarray}

Finalmente, de las desigualdades~(\ref{importatdeq}) y (\ref{errp1}), el Lema \ref{lemma:CORRECTED-ERROR}, y la condición $\tau \in [0,1]$ se obtiene
\begin{eqnarray*}
	\left\lvert\left\lvert  \tau \varphi_1(\tau A)b-  K_{\mf,k}^{\pf,\qf}\left( \tau,  A , b \right)\right\rvert\right\rvert_2
	&\leq& \frac{2 \nnorm{\nnorm{b}}_2 \tau^{\mf+2}  \vert\lvert A\rvert\rvert^{\mf+1}_2
		\me{\tau \lvert\lvert A\rvert\rvert_2}}{(\mf+2)!}\\
	&+&
	\nnorm{\nnorm{b}}_2 (1+\lvert\hf_{\mf+1,\mf}\rvert) \; c_{\pf,\qf}(k,\tau (1+\lvert\lvert A \rvert\rvert_2))
	\; (1 + \lvert\lvert A \rvert\rvert_2^{\pf+\qf+1}) \; \tau^{\pf+\qf+1}\\
	& \leq & \nnorm{\nnorm{b}}_2 C_{\mf,k}^{\pf,\qf}\left(\lvert\lvert A \rvert\rvert_2\right)\tau^{\mathrm{min}\left\{ \mf+2,\pf+\qf+1 \right\}},
\end{eqnarray*}
lo cual concluye la prueba.
$\Box$

\section{Aproximaciones Krylov-Padé libre de Jacobiano}\label{section:fj-krylov-pade-approx}
En general para PVI de grandes dimensiones puede no ser posible la evaluación de la matriz jacobiana $f_x$ del campo vectorial $f$, ya sea por no ser eficiente en términos de memoria u operaciones flotantes. En estos casos algunas aproximaciones libres de Jacobiano~\cite{al2009complex,knoll2004jacobian} son necesarias. Por construcción, la discretización Lineal Local~(\ref{definition LLS}), contiene productos de la matriz jacobiana por un vector  $f_x(y)b$ que pueden ser aproximadas por diferencias finitas de orden 1 or 2. A modo ilustrativo se tomará la diferencia finita hacia adelante
\begin{equation*}
	\frac{F(u+\delta b)-F(u)}{\delta} =  \left(\begin{array}{c}
		\frac{F_1(u_1+\delta b_1,u_2+\delta b_2)-F_1(u_1,u_2)}{\delta}\\
		\frac{F_2(u_1+\delta b_1,u_2+\delta b_2)-F_2(u_1,u_2)}{\delta}\\
	\end{array}\right)
\end{equation*}
para aproximar el producto del jacobiano por un vector \begin{equation*}
	J(u)b = \left[\begin{array}{c}
		b_1\frac{\partial F_1(u_1,u_2)}{\partial u_1} + b_2\frac{\partial F_1(u_1,u_2)}{\partial u_2}\\
		b_1\frac{\partial F_2(u_1,u_2)}{\partial u_1} + b_2\frac{\partial F_2(u_1,u_2)}{\partial u_2}\\
	\end{array}\right],
\end{equation*}
donde $J(u)$ es la matriz jacobiana de la función vectorial suave 2-dimensional $F(u_1,u_2)$, y $b=[b_1,b_2]$ es un vector 2-dimensional. Aproximando $F(u+\delta b)$ con la expansión en seria de Taylor de primer orden $u$, se tiene
\begin{equation*}
	\frac{F(u+\delta b)-F(u)}{\delta} =  \left(\begin{array}{c}
		\frac{F_1(u_1,u_2) + \delta b_1 \frac{\partial F_1}{\partial u_1} + \delta b_2 \frac{\partial F_1}{\partial u_2} -F_1(u_1,u_2)}{\delta}\\
		\frac{F_2(u_1,u_2) + \delta b_1 \frac{\partial F_2}{\partial u_1} + \delta b_2 \frac{\partial F_2}{\partial u_2} -F_2(u_1,u_2)}{\delta}\\
	\end{array}\right)  + O(\delta\nnnorm{b}^2),
\end{equation*}
y por tanto
\begin{equation*}
	\frac{F(u+\delta b)-F(u)}{\delta} =  J(u)b + O(\delta\nnnorm{b}^2).
\end{equation*}

En lo que resta del documento se asumirá q existe una función $(\eta+1)$ continuamente diferenciable $g: \mathbb{R}^{d}\times \mathbb{R}^{d} \times \mathbb{R}_+ \to \mathbb{R}^{d}$ que aproxima al producto $f_x(y)b$ con orden $\eta$ para la cual la cota
\begin{equation} \label{eq:g_bound}
	\nnnorm{g(y,b;\delta)-f_x(y)b} \leq \mathfrak{L}\nnnorm{b}^{\eta+1}\delta^{\eta}
\end{equation}
se cumple, donde $f_x(y)$ es la matriz jacobiana del campo vectorial $f$ en el punto $y$, $y,b$ son vectores $d$-dimensionales y $\mathfrak{L}$ es una constante positiva que depende solamente de la norma de las derivadas de $f_x$.

Al utilizar el algoritmo de Arnoldi~(\ref{alg:Arnoldi}), la aproximación $K_{\mf,k}^{\pf,\qf}(\tau,A,b)$~(\ref{eq:gen_kp_aprox}) requiere de la evaluación explícita de la matriz jacobiana. Al reemplazar el algoritmo de Arnoldi~(\ref{alg:Arnoldi}) utilizado en~(\ref{eq:gen_kp_aprox}) por el algoritmo de Arnoldi sin matriz~(\ref{alg:iArnoldi}) se obtiene la siguiente aproximación libre de jacobiano.

\begin{definition}\label{def:gen_kp_aprox_fj}
	Sean las matrices $\widehat{V}_{\mf}\in\mathbb{R}^{d\times \mf}$ y $\widehat{H}^*_{\mf}\in\mathbb{R}^{{\mf} \times {\mf}}$, el vector $\widehat{v}_{\mf+1}\in\mathbb{R}^d$, y el número  $ \widehat{\hf}^*_{\mf+1,\mf}$ salidas del algoritmo de Arnoldi sin matriz~(\ref{alg:iArnoldi}) para el $\mf$-ésimo subespacio de Krylov $\widehat{\mathcal{K}}_\mf(\tau^\beta f_x(y),b;\delta)$, donde $f_x$ es la matriz jacobiana del campo vectorial $f$, $b,y\in\mathbb{R}^d$ vectores, $\tau,\delta>0$ y $\beta \ge 0$. Sea $\eta$ el orden de la aproximación $g(.;\delta)$ en el algoritmo~\ref{alg:iArnoldi}. Con $A=f_x(y)$, la aproximación $(\mf , \pf ,\qf , k)$-Krylov-Padé libre de jacobiano de $\tau \varphi_1(\tau A)b$ se define como 
	\begin{equation} \label{eq:gen_kp_aprox_fj}
		\widehat{K}_{\mf,k}^{\pf,\qf}\left(\tau, A , b ; \eta, \delta, \beta \right)=\nnorm{\nnorm{b}}_2 \widehat{V}_{\mf}\;[\widetilde{P}_{\tau}]_{12} + \nnorm{\nnorm{b}}_2 \widehat{\hf}_{\mf+1,\mf}e_\mf^T\;[\widetilde{P}_{\tau}]_{13} \widehat{v}_{\mf+1},
	\end{equation}
	donde $\widetilde{P}_{\tau}$ denota la aproximación $(\pf,\qf)$-Padé con escalamiento y potenciación $k$ para la exponencial matricial $\me{\tau\overline{H}}$, 
	\begin{equation}
		\overline{H} = \left[\begin{array}{cccc}
			\widehat{H}_\mf & e_1 & 0_{\mf\times 1} & 0_{\mf\times 1}\\
			0_{1\times\mf} & 0 & 1 & 0\\
			0_{1\times\mf} & 0 & 0 & 1\\
			0_{1\times\mf} & 0 & 0 & 0
		\end{array}\right] \label{hhat},
	\end{equation}
	$\widehat{H}_\mf=\widehat{H}^*_\mf/\tau^\beta$, $\widehat{\hf}_{\mf+1,\mf}=\widehat{\hf}^*_{\mf+1,\mf}/\tau^\beta$, y $e_i$ el $i$-ésimo vector canónico de $\mathbb{R}^\mf$.
\end{definition}


{\SetAlgoNoLine
	\begin{algorithm}
		\caption{\cite{brown1987local} Algoritmo de Arnoldi sin matriz para construir una base ortonormal $\{\widehat{v}_1,\ldots,\widehat{v}_\mf \}$ 
			del $\mf$-ésimo subespacio de Krylov $\widehat{\mathcal{K}}_\mf(\tau f_x(y),b;\delta)$}
		\label{alg:iArnoldi}
		\KwIn{función $g: \mathbb{R}^{d}\times \mathbb{R}^{d}\times \mathbb{R}_+ \to \mathbb{R}^{d}$ definida en (\ref{eq:g_bound}), $y,b \in \mathbb{R}^{d}$, $\tau,\delta>0$, y $\mf$ la dimensión del subespacio de Krylov}
		\KwOut{$\widehat{V}_{\mf}=[\widehat{v_1}\,\cdots \,\widehat{v}_\mf]\in \mathbb{R}^{d\times \mf}$,
			upper Hessenberg matrix $\widehat{H}^*_\mf=(\widehat{\hf}^*_{ij})\in \mathbb{R}^{\mf\times \mf} $,
			$\widehat{v}_{\mf+1} \in \mathbb{R}^d$, $\widehat{\hf}^*_{\mf+1,\mf}$}
		$\widehat{v}_1=b/\lVert b \rVert_2$\\
		\For{ $j=1,\ldots,\mf$ }{
			$\widehat{q}_j=g(y,\tau \widehat{v}_j;\delta)$, \\
			$\widehat{w}_j=\widehat{q}_j$ \\
			\For{ $i=1,\ldots,j$}{
				$\widehat{\hf}^*_{ij}=\scal{\widehat{q}_j}{\widehat{v}_i}$\\
				$\widehat{w}_j=\widehat{w}_j - \widehat{\hf}^*_{ij}\widehat{v}_i$ \\
			}
			$\widehat{\hf}^*_{j+1,j}=\lVert \widehat{w}_j \rVert_2$\\
			$\widehat{v}_{j+1}=\widehat{w}_j/\widehat{\hf}^*_{j+1,j}$
		}
	\end{algorithm}
}

 \subsection{Cotas para aproximaciones Krylov-Padé libres de jacobiano}
  En está sección se enunciará un teorema para acotar el error de la aproximación~(\ref{def:gen_kp_aprox_fj}) en función de tamaño de paso $h$, la dimensión del subespacio de Krylov $\mf$, los órdenes de Padé $\pf$ y $\qf$, el orden $\eta$ de la aproximación $g(.,\delta)$. Para poder enunciar dicho teorema se necesita otro teorema que establece una relación entre el subespacio generado por el algoritmo de Arnoldi~(\ref{alg:Arnoldi}) y el subespacio generado por el algoritmo de Arnoldi sin matriz~(\ref{alg:iArnoldi}).

\begin{theorem} \label{theorem:krilobapproxequality} \cite{naranjo2023jacobian}~Sea $f_x$ la matriz jacobiana del campo vectorial $f$, $b$ y $y$ de la misma dimensión, y $\tau>0$. Entonces, los $\mf$-ésimos subespacios de Krylov $\mathcal{K}_\mf$ y $\widehat{\mathcal{K}}_\mf$ generados por los algoritmos \ref{alg:Arnoldi} y \ref{alg:iArnoldi} respectivamente, satisfacen 
	\[ \widehat{\mathcal{K}}_\mf(\tau f_x(y),b;\delta) = \mathcal{K}_\mf(\tau f_x(y)+R_\mf,b), \]
	donde $R_\mf =  \varepsilon^\mf\widehat{V}^T_\mf$,  
	$\varepsilon^\mf = [\varepsilon_1,\dots,\varepsilon_\mf] \in \mathbb{R}^{d\times \mf}$, $\varepsilon_j = g(y,\tau\widehat{v}_j,\delta)- \tau f_x(y)\widehat{v_j}$, y $\widehat{V}_{\mf}=[\widehat{v_1}\,\cdots \,\widehat{v}_\mf]$ son resultados del algoritmo \ref{alg:iArnoldi}. Además, 
	\[ \nnnorm{R_\mf}_2\leq \sqrt{\mf}\mathfrak{L} \tau^{\eta+1}\delta^{\eta}, \]
	donde $\eta$ es el orden de la aproximación $g(.;\delta)$ en el algoritmo \ref{alg:iArnoldi}, y $\mathfrak{L}$ es la constante positiva de (\ref{eq:g_bound}) correspondiente a $g(.;\delta)$.
\end{theorem}
\textbf{Demostración}. La primera afirmación es un resultado directo del Teorema 3.2 in \cite{brown1987local}.

Utilizando  (\ref{eq:g_bound}) se tiene
\[ \nnnorm{\varepsilon_j}_2=\nnnorm{g(y,\tau\widehat{v}_j,\delta)-\tau f_x(y)\widehat{v}_j}_2 \leq  \mathfrak{L} \tau^{\eta+1}\delta^{\eta}\;\;\;\text{for}\;j=1,\dots,\mf, \]
donde $L$ es una constante positiva. Entonces, 
\begin{eqnarray*}
	\nnnorm{R_\mf}_2 & = & \nnnorm{\varepsilon^\mf V_\mf^T}_2\leq\nnnorm{\varepsilon^\mf}_2\leq \nnnorm{\varepsilon^\mf}_F \\
	& \leq & \left( \nnnorm{\varepsilon_1}_2^2 + \cdots + \nnnorm{\varepsilon_\mf}_2^2 \right)^{1/2} \\
	& \leq & \sqrt{\mf}\mathfrak{L} \tau^{\eta+1}\delta^{\eta},
\end{eqnarray*}
con lo que concluye la demostración. $\blacksquare$\\

\begin{theorem}\label{theorem:Krylov-fj-bound}
\cite{naranjo2023jacobian}~Tomando $A=f_x(y)$, sea $\widehat{K}_{\mf,k}^{\pf,\qf}\left(\tau, A , b ; \eta, \delta, \beta \right)$ la aproximación \\
	$(\mf , \pf ,\qf , k)$-Krylov-Padé libre de jacobiano (\ref{eq:gen_kp_aprox_fj}) of $\tau \varphi_1(\tau A)b$. Entonces,
	\begin{equation}
		\left\lvert\left\lvert  \tau \varphi_1(\tau A)b -
		\widehat{K}_{\mf,k}^{\pf,\qf}\left( \tau, A , b ; \eta, \delta, \beta \right)\right\rvert\right\rvert_2%\nonumber\\
		\leq \mathfrak{c}_0 \tau^{\min\{\mf+2,\pf+\qf+1\}} + \mathfrak{c}_1 \tau^{\beta\eta+2}\delta^{\eta},
	\end{equation}
	con $\tau,\delta \in [0,1]$, donde $\mathfrak{c}_0$ y $\mathfrak{c}_1$ son constantes positivas que dependen de $\mf,\pf,\qf,k$, $\nnnorm{A}_2$, $\nnnorm{b}_2$ y $\mathfrak{L}$, siendo $\mathfrak{L}$ la constante~(\ref{eq:g_bound}) correspondiente a la aproximación $g(.;\delta)$ en el algoritmo de Arnoldi sin matriz~\ref{alg:iArnoldi}.
\end{theorem}
\textbf{Demostración} Del \ref{theorem:krilobapproxequality}, se tiene
\[ \widehat{\mathcal{K}}_\mf(\tau^\beta A,b;\delta)=\mathcal{K}_\mf(\tau^\beta A+R_\mf,b) \]
y
\begin{equation}
	\nnnorm{R_m}_2\leq \sqrt{\mf} \mathfrak{L} \tau^{\beta(\eta+1)}\delta^{\eta},  \label{eq:fjbound3}
\end{equation}
donde la matriz $R_\mf$ y la constante positiva $L$ están definidas en el Teorema \ref{theorem:krilobapproxequality}. 

De la propiedad de invarianza ante escalado de la base ortonormal del subespacio de Krylov se tiene que
\[ \mathcal{K}_\mf(\tau^\beta A+R_\mf,b) = \mathcal{K}_\mf(A+\widehat{R}_\mf,b), \]
donde $\widehat{R}_\mf = R_\mf/\tau^\beta$. De esta forma, $\widehat{\mathcal{K}}_\mf(\tau^\beta A,b;\delta)=\mathcal{K}_\mf(A+\widehat{R}_\mf,b)$, y por tanto
\begin{equation}
	\widehat{K}_{\mf,k}^{\pf,\qf}\left( \tau, A , b; \eta, \delta, \beta \right) = K_{\mf,k}^{\pf,\qf}\left( \tau, A+\widehat{R}_\mf , b \right), \label{identidad}
\end{equation}
donde $K_{\mf,k}^{\pf,\qf}\left( \tau, A+\widehat{R}_\mf , b \right)$ denota la aproximación $(\mf , \pf ,\qf , k)$-Krylov-Padé (\ref{eq:gen_kp_aprox}) de $\tau \varphi_1(\tau (A+\widehat{R}_\mf))b$. 

Teniendo en cuanta $\varphi_1(z) \leq \varphi_0(z)$ se tiene
\begin{align}
	\nnnorm{\tau \varphi_1(\tau A)b -\widehat{K}_{\mf,k}^{\pf,\qf}\left( \tau, A , b; \eta, \delta, \beta \right)}_2 & = \nnnorm{\tau \varphi_1(\tau A)b -K_{\mf,k}^{\pf,\qf}\left( \tau, A+\widehat{R}_\mf , b \right)}_2 \nonumber \\
	& \leq \nnnorm{\tau \varphi_1(\tau A)b - \tau \varphi_1(\tau A + \tau \widehat{R}_\mf)b}_2 \nonumber \\
	& \enspace + \nnnorm{\tau \varphi_1(\tau A + \tau \widehat{R}_\mf)b-K_{\mf,k}^{\pf,\qf}\left( \tau, A+\widehat{R}_\mf , b \right)}_2 \nonumber\\
	& \leq \tau \nnnorm{\me{\tau A}b - \me{\tau A + \tau \widehat{R}_\mf}b}_2 \nonumber \\
	& \enspace + \nnnorm{\tau \varphi_1(\tau A + \tau \widehat{R}_\mf)b-K_{\mf,k}^{\pf,\qf}\left( \tau, A+\widehat{R}_\mf , b \right)}_2. \label{eq:fjbound}
\end{align}

De la Desigualdad de Incremento Finito y la cota~(\ref{eq:fjbound3}) se obtiene
\begin{align}
	\nnnorm{\me{\tau A}b - \me{\tau A + \tau \widehat{R}_\mf}b}_2&\leq \nnnorm{b}_2 \nnnorm{\tau \widehat{R}_\mf}_2\me{\tau\nnnorm{A}_2+\tau\nnnorm{\widehat{R}_\mf}_2} \\ &\leq \nnnorm{b}_2   \sqrt{\mf}\mathfrak{L} \tau^{\beta\eta+1}\delta^\eta\me{\tau(\nnnorm{A}_2+\sqrt{\mf}\mathfrak{L})} \nonumber\label{eq:fjbound1}
\end{align}
y, del Teorema 3.1 en \cite{naranjo2021locally}, 
\vspace{-0.25cm}
\begin{equation}
	\nnnorm{\tau \varphi_1(\tau A+\widehat{R}_\mf)b-K_{\mf,k}^{\pf,\qf}\left( \tau, A+\widehat{R}_\mf , b \right)}_2  \leq \nnnorm{b}_2C_{\mf,\kt}^{\pf,\qf}(\nnnorm{A}_2+\sqrt{\mf}\mathfrak{L})\tau^{\min\{\mf+2,\pf+\qf+1\}}, \label{eq:fjbound2}
\end{equation}
donde $C_{\mf,\kt}^{\pf,\qf}$ está definido en el Teorema 3.1 in \cite{naranjo2021locally}.

Utilizando las desigualdades (\ref{eq:fjbound})-(\ref{eq:fjbound2})
\vspace{-0.25cm}
\begin{multline*}
	\nnnorm{\tau \varphi_1(\tau A)b -\widehat{K}_{\mf,k}^{\pf,\qf}\left( \tau, A , b; \eta, \delta, \beta \right)}_2 \leq 
	\nnnorm{b}_2 \sqrt{\mf}\mathfrak{L} \me{\tau(\nnnorm{A}_2+\sqrt{\mf}\mathfrak{L})} \tau^{\beta\eta+2}\delta^\eta \\ + \nnnorm{b}_2C_{\mf,\kt}^{\pf,\qf}(\nnnorm{A}_2+\sqrt{\mf}\mathfrak{L})\tau^{\min\{\mf+2,\pf+\qf+1\}} ,
\end{multline*}
con lo cual se concluye la demostración. $\Box$\\

\section{Aproximaciones de ecuaciones no autónomas}
En el marco de los integradores exponenciales para PVI de la forma
\begin{equation*}
	\frac{dx}{dt}=f(x),  \;\;\;\; x(t_0)=x_0 \in \mathbb{R}^{d},   \;\;\;\; t\in[t_0,T],
\end{equation*}
hay varios esquemas (ejemplo \cite{tokman2006efficient,delaCruz07,hochbruck2011exponential}) que requieren del cálculo de una combinación lineal de productos de múltiples integrales y vectores de la forma
\begin{equation*}
	\sum\limits_{i=1}^{l}\phi _{i}(\mathbf{A},t)\mathbf{a}_{i},   \;\;\;\mathrm{con} \;\; \phi _{i}(\mathbf{A},t)=\int_{0}^{t}e^{\mathbf{A}(t-s)}s^{i-1}ds,
\end{equation*}
esto es equivalente a resolver la ecuación no autónoma
\begin{equation*}
	\frac{dx(t)}{dt} = \mathbf{A}x(t)+\sum\limits_{i=0}^{l}\mathbf{a}_i\frac{t^i}{i!}
\end{equation*}
donde $\mathbf{A}$ es la matriz jacobiana $f_x$ de $f$, $\mathbf{a}_1,\ldots,\mathbf{a}_l$ vectores y $l$ un número natural. Además, existe una relación directa entre $\phi_i$ y $\varphi_i$
\begin{equation*}
	\phi_i(A,t)=(i-1)!t^i\varphi_i(At)\;.
\end{equation*}
Aplicando directamente el Teorema 1 en \cite{carbonell2008computing} se obtiene que\cite{carbonell2008computing,jimenez2006local}
\begin{equation}\label{sum2_phiXv}
\sum\limits_{i=1}^{l}\phi _{i}(\mathbf{A},t)\mathbf{a}_{i} = \mathbf{L} e^{t \mathbf{M}}\mathbf{r},
\end{equation}
con $\mathbf{L}=[\mathbf{I}_{d\times d}$ $\mathbf{0}_{d\times l}]$, $\mathbf{r}=[\mathbf{0}_{1\times (d+(l-1))}$ $1]^{\intercal }$, y
\begin{equation*}
\mathbf{M}=\left[
\begin{array}{ccccc}
\mathbf{A} & \overline{\mathbf{a}}_{l} & \overline{\mathbf{a}}_{l-1} & \cdots & \overline{\mathbf{a}}_{1} \\ 
0 & 0 & 1 & \cdots & 0 \\
0 & 0 & 0 & \ddots & 0 \\
\vdots & \vdots & \vdots & \ddots & 1 \\
0 & 0 & 0 & \cdots & 0
\end{array}%
\right],  \label{matrixH2}
\end{equation*}%
donde $\overline{\mathbf{a}}_{i}=\mathbf{a}_{i}(i-1)!$, para todo $i=1,\ldots ,l$.

Como se puede notar esta combinación lineal de múltiples integrales se puede reescribir como el producto de una exponencial matricial por un vector. Para aproximar esta exponencial utilizando las aproximaciones Krylov-Padé construidas en las Secciones \ref{section:krylov-pade-approx} y \ref{section:fj-krylov-pade-approx} se propone el siguiente resultado

\begin{theorem} \label{theorem:Krylov-and-Krylov-fj-bounds}
	\cite{naranjo2023computing}~Sea $\mathbf{M}$ matriz definida en \ref{sum2_phiXv}, $\mathbf{r}$ un vector $d$-dimensional y $t\geq 0$. Entonces
	\[
	e^{\mathbf{M}t}\mathbf{r} = \mathbf{r} + t\varphi (\mathbf{M}t)\mathbf{Mr},
	\]
	donde $\varphi$ esta definida en (\ref{DEF-PHI}). Además,
	\begin{equation}
      \left\Vert\sum\limits_{i=1}^{l}\phi _{i}(A,h)\mathbf{a}_{i}-	L K_{\mf,k}^{\pf,\qf}\left(t, \mathbf{M}, \mathbf{Mr} \right)
      \right\Vert_2 \leq C_{\mf,k}^{\pf,\qf}\left(\lvert\lvert \mathbf{M} \rvert\rvert_2\right) \;
	  \nnorm{\nnorm{\mathbf{Mr}}}_2 \; t^{\mathrm{min}\left\{ \mf+2,\pf+\qf+1 \right\}}, \label{phi_approx}
	\end{equation}
	\begin{equation}
		\left\Vert \sum\limits_{i=1}^{l}\phi _{i}(A,h)\mathbf{a}_{i} - L \widehat{K}_{\mf,k}^{\pf,\qf}\left(t, \mathbf{M} , \mathbf{Mr} ; \eta, \delta, \beta \right)
		\right\Vert_2 \leq \mathfrak{c}_0 h^{\min\{\mf+2,\pf+\qf+1\}} + \mathfrak{c}_1 t^{\beta\eta+2}\delta^{\eta}, \label{phi_approx_fj}
	\end{equation}
	donde $K_{\mf,k}^{\pf,\qf}\left(t, \mathbf{M}, \mathbf{Mr} \right)$ es la aproximación $(\mf , \pf ,\qf , k)$-Krylov-Padé de $t\varphi (\mathbf{M}t)\mathbf{Mr}$,  $C_{\mf,k}^{\pf,\qf}(\varLambda)$ definida en Teorema \ref{theorem:Krylov-bound}, $\widehat{K}_{\mf,k}^{\pf,\qf}\left(t, \mathbf{M} , \mathbf{Mr} ; \eta, \delta, \beta \right)$  es la aproximación $(\mf , \pf ,\qf , k)$-Krylov-Padé libre de Jacobiano de $t\varphi (\mathbf{M}t)\mathbf{Mr}$, $\mathfrak{c}_0$ y $\mathfrak{c}_1$ son constantes positivas que dependen de $\mf,\pf,\qf,k$, $\nnnorm{\mathbf{M}}_2$, $\nnnorm{\mathbf{r}}_2$, $\mathfrak{L}$(constante~\ref{eq:g_bound}) y $t \in [0,1]$.
\end{theorem}
\textbf{Demostración} La primera afirmación se demuestra de forma directa del hecho de que $e^{\mathbf{M}t}\mathbf{r}$ y $\mathbf{r} + t\varphi (\mathbf{M}t)\mathbf{Mr}$ representan la solución única del la ecuación differential ordinaria lineal
\[ d\mathbf{x}/dt= \mathbf{M}\mathbf{x},  \;\;\;\;\; \mathbf{x}(0)=\mathbf{r}, \]
para $t\geq 0$. La segunda y tercera afirmación son resultado directo de aplicar el Teorema \ref{theorem:Krylov-bound} y Teorema \ref{theorem:Krylov-fj-bound} respectivamente.
$\square$

\subsection{Simulaciones numéricas}
El objetivo de esta sección es realizar un conjunto de simulaciones numéricas para medir el desempeño de las aproximación propuestas y compararlas con aproximaciones existentes.

La aproximación dada en (\ref{phi_approx}) en términos de la acción de la función phi sobre un vector tiene dos ventajas principales sobre las propuestas \cite{hochbruck1997krylov,sidje1998expokit,jimenez2012convergence}
en términos de la acción de la exponencial matricial sobre un vector. Primero, la aproximación (\ref{phi_approx}) es, con respecto a la dimensión de Krylov $\mf$ como exponente de $t$, dos órdenes mayor que las otras existentes (véase el enunciado del Teorema \ref{theorem:Krylov-and-Krylov-fj-bounds} respecto a los enunciados en los Teoremas 4.3 en \cite{Saad92} y 5 en \cite{hochbruck1997krylov}, y el Lema 4.2 en \cite{jimenez2012convergence}) mientras que se realiza un número similar de operaciones matemáticas. Este es un resultado notable en el marco de los integradores exponenciales para ecuaciones diferenciales ya que implica un mayor orden de convergencia y eficiencia computacional \cite{naranjo2021locally}. En segundo lugar, mientras que en \cite{hochbruck1997krylov,sidje1998expokit,jimenez2012convergence} no hay instrucciones para determinar la dimensión de Krylov $\mf$ adecuada, $\mf$ en (\ref{phi_approx}) se puede estimar efectivamente como en \cite{naranjo2021locally,naranjo2023jacobian} en términos del error relativo
\begin{equation*}
	\varepsilon(\mf) = \left(\frac{1}{d}\sum\limits_{i=1}^{d} \left(\frac{
		\hf_{\mf+1,\mf} \mathbf{e}_{\mf}^T
		[\widetilde{\mathbf{E}}_\tau]_{14} \rho^{[i]} \nnorm{\nnorm{\mathbf{Mr}}}_2}{ATol+ RTol\cdotp
		\rvert \mathbf{r}^{[i]}\rvert}\right)^{2}\right)^{1/2},
\end{equation*}
donde $ATol$ and $RTol$ son las tolerancias absoluta y relative deseadas y $\mathbf{\rho} = \mathbf{M} \mathbf{v}_{\mf+1}$. 

Este error relativo está basado en lo que Y. Saad llamó ``\textit{error a posteriori}''~\cite{Saad92}. Esta estimación del error se basa en tomar como medida del error el siguiente término donde fue truncada la serie (\ref{PHI-EXPANSION}). Como se ha mencionado anteriormente, cuando los dos primeros términos de (\ref{PHI-EXPANSION}) se usan para aproximar $\tau\phi_1(\tau \mathbf{M})\mathbf{Mr}$, el tercer término $\left\lvert\left\lvert \mathbf{Mr} \right\rvert\right\rvert_2 \hf_{\mf+1,\mf} \tau^{3} e_\mf^{T}\phi_3( \tau H_\mf)e_1 \mathbf{M} v_{\mf+1} $ de (\ref{PHI-EXPANSION}) puede usarse como una medida práctica de error. Más precisamente, reemplazando $\tau^3\phi_3( \tau H_\mf)e_1$ por $[\widetilde{E}_{\tau}]_{14}$, la expresión $\left\lvert \left\lvert \mathbf{Mr} \right\rvert\right\rvert_2 \left\lvert\left\lvert \hf_{\mf+1,\mf} e_\mf^{T}[\widetilde{E}_{\tau}]_{14} \mathbf{M} v_{\mf+1} \right\rvert\right\rvert_2$ puede tomarse como el error de la aproximación (\ref{eq:kp_aprox}) a $\tau\phi_1(\tau \mathbf{M})\mathbf{Mr}$. Es importante notar que este error es válido tanto para la aproximación dada en (\ref{phi_approx}) como para (\ref{phi_approx_fj}).

\begin{sloppypar}
Comenzando con un valor factible de $\mf$, se construye el $\mf$-ésimo subespacio de Krylov $\mathcal{K}_{\mf}(\mathbf{M},\mathbf{Mr})$ y el error relativo $\varepsilon(\mf)$ es calculado. En el caso de que $\varepsilon(\mf)/\gamma \geq 1$, se estima un nuevo valor para la dimensión de Krylov mediante la expresión
\end{sloppypar}
\begin{equation*}
	\mf_{new} = \left\lceil \mf + \minn{\fac_{max} , \maxx{\fac_{min},
	\fac\cdot\Delta\mf} } \right\rceil,
\end{equation*}
donde $\Delta\mf=\log(\varepsilon(\mf)/\gamma) \label{delta_m}$, $\gamma=0.005$ es un factor de seguridad, $\fac_{max}= \maxx{ 1, \frac{\mf}{3} }$, $\fac_{min}= 1 $, $\fac=\frac{1}{\log(2)}$, y el símbolo $\left\lceil \cdot \right\rceil$ denota la función de techo (que devuelve el menor entero mayor o igual que un número real dado). Luego, con $\mf=\mf_{new}$, se repite el procedimiento explicado hasta que se satisface $\varepsilon(\mf)/\gamma< 1$.

Como se ha señalado en varios artículos (ver \cite{jimenez2009rate,jimenez2012convergence, Jimenez14AMC, jimenez2015convergence}), la selección de un valor apropiado del par $(\pf,\qf)$ en la aproximación del Padé es crucial para mejorar la eficiencia computacional de los integradores exponenciales. Para este propósito, la Tabla 1 en \cite{moler2003nineteen} se usa regularmente para configurar automáticamente los valores óptimos de $\pf$, con $\qf=\pf$, en función de la norma de la matriz, la tolerancia especificada y la orden de convergencia del integrador exponencial. La Tabla \ref{table:padep}, presenta un subconjunto de la Tabla 1 en \cite{moler2003nineteen} correspondiente a $\pf \ge 3$.

\begin{table}[htb]
	\caption{Valores óptimos de $\pf$ pra las aproximaciones  (\ref{phi_approx},\ref{phi_approx_fj}), con $\qf=\pf$, como función de $\lvert\lvert \tau\overline{H} \rvert\rvert_\infty$ y la tolerancia deseada $\mathrm{RTol}$.}
	\begin{center}
		\begin{tabular}{cccc}
			\hline
			$\lvert\lvert \tau\overline{H} \rvert\rvert_\infty \setminus RTol$ & $10^{-9}$ & $10^{-12}$ & $10^{-15}$ \\
			\hline
			$<1$ & 3 & 4 & 4 \\
			$\geq 1$ & 4 & 5 & 6 \\
			\hline
		\end{tabular}
		\label{table:padep}
	\end{center}
\end{table}

\subsubsection{Simulaciones numéricas para Aproximaciones Krylov-Padé}
La Figura \ref{fig:SumPhi} presenta los diagramas de tiempo-precisión para los métodos de \cite{hochbruck1997krylov}, \cite{sidje1998expokit}, \cite{niesen2012algorithm} y (\ref{phi_approx}), como función de la dimensión de Krylov $\mf$, en el cálculo de (\ref{sum2_phiXv}) para dos valores de $l$. La matriz $\mathbf{A}$ fue tomada como $10000 \times 10000$ \textit{Tridiag} de la galería de funciones de Matlab18a, $\mathbf{a}_1=\mathbf{a}_2=\mathbf{a}_3=\mathbf{1}$, y $t=0.1$. La norma euclidiana se usa para medir el error entre el valor ``exacto'' de $\mathbf{L} e^{t \mathbf{M}}\mathbf{r}$ y sus aproximaciones. Con fines comparativos, el valor ``exacto'' de $e^{t \mathbf{M}}$ y la matriz exponencial $\me{t\overline{\mathbf{H}}}$ en (\ref{phi_approx}) se calculan con la misma función de Matlab \textit{expm} que la utilizada en \cite{sidje1998expokit} y \cite{niesen2012algorithm} para calcular la exponencial de la matriz de Hessenberg. Como principal diferencia con la aproximación (\ref{phi_approx}) y con la de \cite{hochbruck1997krylov}, la precisión de los métodos de \cite{sidje1998expokit} y \cite{niesen2012algorithm} se mejora con una estrategia de cambio de tamaño de paso regulada por estimaciones de los errores de la aproximación. De esta forma, como se observa en la Figura \ref{fig:SumPhi}, para el mismo valor de $\mf$ los métodos de \cite{sidje1998expokit} y \cite{niesen2012algorithm} alcanzan o superan la precisión del método (\ref{phi_approx}) pero a expensas de un mayor costo computacional. Con respecto al método en \cite{hochbruck1997krylov} que calcula exp(M*t) directamente a través del método del subespacio de Krylov, la aproximación en (\ref{phi_approx}) incluye un segundo término con el producto escalar de dos vectores de baja dimensión, pero ambas aproximaciones comparten algoritmos más complejos y lentos: el algoritmo de Arnoldi para los mismos subespacios de Kylov y la evaluación de una exponencial matricial. Dado que la aproximación en (\ref{phi_approx}) también es dos órdenes superior, se espera que la aproximación (\ref{phi_approx}) sea más precisa que la de \cite{hochbruck1997krylov} con un costo computacional similar al que se muestra en Figura \ref{fig:SumPhi}.

\begin{figure}[ht]
	% \centering
	\includegraphics[scale=0.55]{Graphics/phil2l3.jpg}
	\caption{Diagrama tiempo-precisión para los métodos of \cite{hochbruck1997krylov}, \cite{sidje1998expokit}, \cite{niesen2012algorithm} and (\ref{phi_approx}), denotados por \textit{Exp}, \textit{Expv}, \textit{Phip} and \textit{Phi} respectivamente, como función de la dimensión de Krylov $\mf$, en el cálculo de  (\ref{sum2_phiXv}) para dos valores de $l$. Izquierda $l=2$, Derecha $l=3$. De arriba hacia debajo, $\mf=4,5,6,7,8$ para cada método, siendo $\mf=6$ el valor óptimo estimado como se especificó anteriormente con tolerancias $ATol=10^{-9}$ y $RTol=10^{-6}$.}
	\label{fig:SumPhi}
\end{figure}

\subsubsection{Simulaciones numéricas para Aproximaciones Krylov-Padé libre de Jacobiano}
Consideraremos el cálculo de (\ref{sum2_phiXv}) con matrix jacobiana $f_X$ y campo vectorial $f$ del PVI resultante de discretizar una ecuación diferencial parcial. La siguiente matriz jacobiana y la ecuación discretizada fue tomada de~\cite{tokman2006efficient}.

\begin{example}
	\label{ej:ej1-hpfj} Matriz jacobiana de $2N\times2N$
	\begin{equation*}
	f_{x}(x)=\left[ 
	\begin{array}{cc}
	diag(2u\cdot v-4) & diag(u\cdot u) \\ 
	diag(3-2u\cdot v) & -diag(u\cdot u)%
	\end{array}%
	\right] +\frac{\alpha }{(\Delta z)^{2}}\left[ 
	\begin{array}{cc}
	K & 0 \\ 
	0 & K%
	\end{array}%
	\right] ,\text{ \ \ \ con \ \ \ \ }x=\left[ 
	\begin{array}{c}
	u \\ 
	v%
	\end{array}%
	\right] ,
	\end{equation*}
	\[
	K=\left[ 
	\begin{array}{ccccc}
	-2 & 1 &  &  & \\
	1 & -2 & 1 &  &  \\
	& \ddots  & \ddots  & \ddots  &  \\
	&  & 1 & -2 & 1 \\
	&  &  & 1 & -2%
	\end{array}%
	\right]_{N\times N}
	\]
	de la  ecuación $2N$-dimensional Brusselator discretizada
	\begin{eqnarray*}
		\frac{du_{i}}{dt} &=&1+u_{i}^{2}v_{i}-4u_{i}+\frac{\alpha }{(\Delta z)^{2}}%
		(u_{i-1}-2u_{i}+u_{i+1}) \\
		\frac{dv_{i}}{dt} &=&3u_{i}-u_{i}^{2}v_{i}+\frac{\alpha }{(\Delta z)^{2}}%
		(v_{i-1}-2v_{i}+v_{i+1})
	\end{eqnarray*}
	con $\alpha =1/50$, $u_{i}(0)=1+\sin (2\pi z_{i})$, $v_{i}(0)=3$, $z_{i}=i/(N+1)$, $\Delta z =1/(N+1)$, $i=1,\ldots,N$, y $N=800$.
\end{example}

Los códigos de Matlab \textit{JF1-Phi} y \textit{JF2-Phi} implementan la aproximación Krylov-Padé libre de Jacobiano (\ref{phi_approx_fj}) con $\beta=0$ y las diferencias finitas de primer y segundo orden respectivamente
\begin{equation}\label{finite-differences}
	g(x,u;\delta)=\frac{f(x+\delta u)-f(x)}{\delta}  \;\;\; \text{and} \;\;\; g(x,u;\delta)=\frac{f(x+\delta u)-f(x-\delta u)}{2\delta}
\end{equation}
donde $\delta= \frac{\sqrt{(1+||x||_2)\epsilon_{mach}}}{\epsilon_{mach}+||u||_2}$ como se sugiere en~\cite{knoll2004jacobian}, siendo $\epsilon_{mach}$ el épsilon de la máquina. El código de Matlab \textit{Phi} implementa la aproximación Krylov-Padé (\ref{phi_approx}) con la matrix exacta.

En el ejemplo, la matriz y los vectores en (\ref{sum2_phiXv}) se definen como $A=f_x(x(0))$, $a_1=a_2=f(x(0))$, $a_3=2a_1 $ y $a_4=6a_1$ de acuerdo con el número de términos $l$ en cada ejemplo. La norma euclidiana se usa para medir el error entre el valor ``exacto'' de $L e^{h M}r$ en (\ref{sum2_phiXv}) y sus aproximaciones. Para fines comparativos, el valor ``exacto'' de $e^{h M}$ y la matriz exponencial $\me{\tau\overline{H}}$ en los códigos \textit{JF1-Phi}, \textit{JF2 -Phi} y \textit{Phi} se calculan con la misma función de Matlab \textit{expm}. La fila superior de la Figura \ref{fig:SumPhiBrusselator} presenta, para cada ejemplo, las gráficas logarítmicas de tolerancia relativa (\textit{rtol}) contra error (\textit{error}) en el cálculo de (\ref{sum2_phiXv}) mediante las aproximaciones \textit{JF1-Phi}, \textit{JF2-Phi} y \textit{Phi} con tolerancias relativas y absolutas $rtol=10^{-j}$ y $atol=0.1 rtol$, con $j=1,\ldots,6$. La fila inferior de esta figura presenta las gráficas de la dimensión de Krylov $\mf$ cont $log(rtol)$ correspondientes a las aproximaciones en la fila superior de las figuras. Los valores de $\mf$ son determinados automáticamente por cada código para cada una de las tolerancias especificadas $rtol$ y $atol$, tal y como se explicó anteriormente.


\begin{figure}[htb]
	\includegraphics[scale=0.57]{Graphics/kpfj-brusselator-em.jpg}
	\caption{Superior: Gráficos Log-log de tolerancia relativa (\textit{rtol}) contra error (\textit{error}) en el cálculo de $\phi _{1}(f_x,h)a_{1}+\phi _{2}(f_x,h)a_{3}+\phi _{3}(f_x,h)a_{3}$ para la ecuación de ejemplo \ref{ej:ej1-hpfj} mediante las aproximaciones \textit{JF1-Phi}, \textit{JF2-Phi} y \textit{Phi} con $rtol=10^{-j}$ y $j=1,\ldots,6$. De izquierda a derecha, con $h=0.01,0.001,0.0001$. Inferior: Gráficos de tolerancia relativa (\textit{rtol}) contra dimensión de Krylov $\mf$ correspondiente a las aproximaciones de la figura superior corresponding.}
	\label{fig:SumPhiBrusselator}
\end{figure}

Observe que una diferencia importante entre las aproximaciones libre de Jacobiano y las de matriz exacta es el umbral para los errores de las primeras a medida que disminuye la tolerancia. Como es de esperar, cuando la tolerancia $rtol$ disminuye, la dimensión de Krylov $\mf$ para las tres aproximaciones aumenta y, en consecuencia, sus errores también disminuyen. Sin embargo, para valores crecientes de $\mf$, el valor fijo del segundo término de la derecha en (\ref{phi_approx_fj}) domina los valores decrecientes del primer término, lo que explica los umbrales para los errores de las aproximaciones libres de jacobiano \textit{JF1-Phi} y \textit{JF2-Phi} en la Figura \ref{fig:SumPhiBrusselator}. Como se predice en (\ref{phi_approx}), el error de la aproximación con Jacobiano exacto \textit{Phi} en esta figura siempre disminuye cuando $\mf$ aumenta. Nótese también que, en correspondencia con el segundo término de la cota (\ref{phi_approx_fj}), la precisión de la aproximación con diferencia finita de segundo orden \textit{JF2-Phi} es ligeramente superior a la de la aproximación con diferencia finita de primer orden \textit{JF1-Phi} solo para los valores grandes de $h$ (gráfico superior izquierdo en la figura).

Las simulaciones numéricas corroboraron las principales implicaciones del análisis de errores para dicha aproximación libre de Jacobiano, es decir; el umbral para los errores disminuye cuando aumenta la dimensión del subespacio de Krylov; es menor el error de la aproximación con diferencia finita de segundo orden y  las aproximaciones libres de Jacobiano son menos precisas que las que utilizan matriz exacta.
\chapter{Método Runge-Kutta de Dormand y Prince Localmente Linealizado para EDOs de
dimensiones no pequeñas}\label{chapter:lldp}
En el Capítulo \ref{chapter:exp-int-and-ll-methods} se presentó la Aproximación Lineal Local de Orden Superior. Esta aproximación consiste en descomponer la solución de la ecuación diferencial original como la suma de las soluciones de dos ecuaciones diferenciales auxiliares, una lineal y la no lineal. En el Capítulo \ref{chapter:solve-non-smal-lineal-eq} se propusieron y utilizaron las aproximaciones Krylov-Padé para resolver la ecuación diferencial lineal auxiliar. En este capítulo se utilizará el método Runge-Kutta de Dormand y Prince para aproximar la solución de la ecuación diferencial no lineal auxiliar. Combinando estas aproximaciones se obtienen Método Runge-Kutta de Dormand y Prince Localmente Linealizado para EDOs de dimensiones no pequeñas.

Sin pérdida de generalidad consideremos el PVI autónomo $d$-dimensional
\begin{equation}\label{syst}
\frac{dx}{dt}=f(x), \; t\in[t_0,T],\end{equation}
\begin{equation}\label{systcond}
x(t_0)=x_0,
\end{equation}donde $f$ es una función diferenciable en una vecindad
$\mathfrak{D}$ del conjunto $\{x(t):t\in [t_0,T]\}$ de $\mathbb{R}^{d}$. Se asumen condiciones de Lipschitz y suavidad en la función $f$ para asegurar una solución única de esta ecuación en $\mathfrak{D}$.

\section{Fórmulas embebidas}

Sea $\left( t\right) _{h}=\left\{ t_{n}:n=0,1,\ldots ,N\right\}$ una discretización temporal con un tamaño de paso máximo $h$ definido como una secuencia de tiempos que satisfacen las condiciones $t_{0}<t_{1}<\cdots <t_{N}=T$, donde $h_{n}=t_{n+1}-t_{n}\leq h$ para $n=0,\ldots,N-1$.

Consideremos las fórmulas de Runge-Kutta localmente linealizadas de Dormand y Prince
\begin{equation} \label{lldis}
    z_{n+1}\,=\,z_n+u_s+h_n \sum_{j=1}^{s}b_j \kt_j \,\,\, \text{and} \,\,\, \
    \widehat{z}_{n+1}\,=\, z_n+u_s+h_n \sum_{j=1}^{s}\widehat{b}_j \kt_j
\end{equation}
introducidas en \cite{Jimenez14AMC} para aproximar la solución $x$ de (\ref{syst})-(\ref{systcond}) en $t_{n+1}$, para $n=0,\ldots,N -1$, donde s = 7 es el número de etapas,
\begin{equation*}
u_j=L\me{c_j M_n h_n}r,
\end{equation*}
\begin{equation*}
\kt_j = f\left( z_n+u_j+h_n \sum_{i=1}^{j-1}a_{j,i}\kt_i \right) - f( z_n) - f_x(z_n)u_j,
\end{equation*}
con $\kt_1 \equiv 0$, siendo $f_x$ la matriz jacobiana de $f$ y $a_{j,i}$, $b_j$, $\widehat{b}_j$ los coeficientes de Runge-Kutta de Dormand y Príncipe definido en la Tabla \ref{ButcherTabla}. Aquí,
\begin{equation*}
    M_{n}=\left[
    \begin{array}{cc}
        f_{x}(z_{n}) & f(z_{n}) \\
        0_{1\times d} & 0
    \end{array}
    \right] \in \mathbb{R}^{(d+1)\times (d+1)},
\end{equation*}
$ L=[I_d \;\; 0_{d\times 1}] $ and $r=[0_{1\times d}\;\; 1]^T$. Las fórmulas embebidas (\ref{lldis}) son instancias particulares de una clase más general de métodos de linealización local de alto orden propuestos en \cite{Jimenez13}.

\begin{table}[h]
	\caption{Tabla de coeficientes para las fórmulas embebidas de Dormand y Prince} \label{ButcherTabla}
	\begin{center}
		\begin{tabular}{ l@{\vrule height 5pt depth 10pt width 0pt}|lllllll}
			$0$ & \\
			$\frac{1}{5}\quad$ & $\frac{1}{5}$ \\
			$\frac{3}{10}\quad$ & $\frac{3}{40}$ & $\frac{9}{40}$ \\
			$\frac{4}{5}\quad$ & $\frac{44}{45}$ & $-\frac{56}{15}$ & $\frac{32}{9}$ \\
			$\frac{8}{9}\quad$ & $\frac{19372}{6561}$ & $-\frac{25360}{2187}$ & $\frac{64448}{6561}$ & $-\frac{212}{729}$ \\
			$1\quad$ & $\frac{9017}{3168}$ & $-\frac{355}{33}$ & $\frac{46732}{5247}$ & $\frac{49}{176}$
			& $-\frac{5103}{18656}$ \\
			$1\quad$ & $\frac{35}{384}$ & $0$ & $\frac{500}{1113}$ & $\frac{125}{192}$
			& $-\frac{2187}{6784}$ & $\frac{11}{84}$ \\
			\hline
			$\widehat{y}$ & $\frac{5179}{57600}$ & $0$ & $\frac{7571}{16695}$ & $\frac{393}{640}$
			& $-\frac{92097}{339200}$ & $\frac{187}{2100}$ & $\frac{1}{40}$
			\rule[-0.3cm]{0cm}{0.8cm} \\
			$y$ & $\frac{35}{384}$ & $0$ & $\frac{500}{1113}$ & $\frac{125}{192}$
			& $-\frac{2187}{6784}$ & $\frac{11}{84}$ & $0$
		\end{tabular}
	\end{center}
\end{table}

Las aproximaciones Krylov-Padé propuestas en el capítulo anterior serán utilizadas para aproximar los términos $u_j$ que aparecen en (\ref{lldis}). Además, para aproximar eficientemente los cinco términos $u_j$ correspondientes a los cinco únicos $c_j$ distintos de cero que según la Tabla \ref{ButcherTabla} aparecen en las fórmulas (\ref{lldis}), se utilizará combinación conveniente de la invarianza ante escalado de la base ortonormal de los subespacios de Krylov, la propiedad de flujo del operador exponencial y el Teorema 1 en~\cite{sidje1998expokit}. De hecho, con la matriz de Hessenberg superior $H^*_\mf$ resultante del algoritmo \ref{alg:Arnoldi} para $\mathcal{K}_\mf(h_nf_x,f)$, se obtiene la matriz de Hessenberg $H_\mf=H^*_\mf/h_n$ correspondiente a $\mathcal{K}_\mf(f_x,f)$, por lo que la exponencial de la matriz particionada
\begin{equation}
    \overline{H} = \left[\begin{array}{cccc}
    H_\mf & e_1 & 0_{\mf\times 1} & 0_{\mf\times 1}\\
    0_{1\times\mf} & 0 & 1 & 0\\
    0_{1\times\mf} & 0 & 0 & 1\\
    0_{1\times\mf} & 0 & 0 & 0
    \end{array}\right] \label{hhat}
\end{equation}
es calculada obteniéndose
\begin{equation}
    \me{\tau\overline{H}} = \left[\begin{array}{cccc}
    \me{\tau H_m} & \tau\phi_1(\tau H_m)e_1 & \tau^{2}\phi_2(\tau H_m)e_1 &
    \tau^{3}\phi_3(\tau H_m)e_1 \\
    & 1 & \tau & \frac{\tau^{2}}{2}\\
    &  & 1 & \tau \\
    &   &   & 1 \\
    \end{array}\right]\;. \label{phi_hhat}
\end{equation}
En particular, con $\gamma=\frac{1}{90}$, es calculada la matriz particionada $E_\gamma=\me{\gamma h_n\overline{H}}$, donde $\frac{1}{ 90}$ es el máximo común divisor de los coeficientes de Runge-Kutta $\frac{1}{5},\frac{3}{10},\frac{4}{5},\frac{8}{9} ,1$ de la Tabla \ref{ButcherTabla}. Usando la propiedad de flujo del operador exponencial se obtiene

\begin{align}
    E_{2/90}& =E_{1/90}E_{1/90} & E_{4/90}& =E_{2/90}E_{2/90}  \notag \\
    E_{8/90}& =E_{4/90}E_{4/90} & E_{16/90}& =E_{8/90}E_{8/90}  \notag \\
    E_{32/90}& =E_{16/90}E_{16/90} & E_{80/90}& =E_{32/90}E_{16/90}E_{32/90}
    \label{flow} \\
    E_{1/10}& =E_{8/90}E_{1/90} & E_{1/5}& =E_{1/10}E_{1/10}  \notag \\
    E_{2/5}& =E_{1/5}E_{1/5} & E_{4/5}& =E_{2/5}E_{2/5}  \notag \\
    E_{3/10}& =E_{1/10}E_{1/5} & E_{1}& =E_{4/5}E_{1/5}\;,  \notag
\end{align}
con lo cual las cinco matrices requeridas $E_{c_j}$ son obtenidas. Como la exponencial no puede ser calculada de forma exacta ene caso general se aproximará utilizando Padé como se explica en el Capítulo \ref{chapter:solve-non-smal-lineal-eq} obteniéndose $\widetilde{E}_{c_j}$.

Por lo tanto, utilizando la aproximación Krylov-Padé (\ref{eq:kp_aprox}) y (\ref{flow}) obtenemos las aproximaciones para cada $u_j$
\begin{eqnarray}
    \frac{1}{5}h_n\varphi_1\left(\frac{1}{5}h_n f_x\right)f & \approx & \beta   V_{\mf}\; [\widetilde{E}_{\frac{1}{5}}]_{12}\;  + \beta \hf_{\mf+1,\mf}e_\mf^T\; [\widetilde{E}_{\frac{1}{5}}]_{13}\;  v_{\mf+1} \notag \\
    \frac{3}{10}h_n\varphi_1\left(\frac{3}{10}h_n f_x\right)f & \approx & \beta   V_{\mf}\; [\widetilde{E}_{\frac{3}{10}}]_{12}\;  + \beta \hf_{\mf+1,\mf}e_\mf^T\; [\widetilde{E}_{\frac{3}{10}}]_{13}\;  v_{\mf+1} \notag \\
    \frac{4}{5}h_n\varphi_1\left(\frac{4}{5}h_n f_x\right)f & \approx & \beta   V_{\mf}\; [\widetilde{E}_{\frac{4}{5}}]_{12}\;  + \beta \hf_{\mf+1,\mf}e_\mf^T\; [\widetilde{E}_{\frac{4}{5}}]_{13}\;  v_{\mf+1} \label{phi_appox} \\
    \frac{8}{9}h_n\varphi_1\left(\frac{8}{9}h_n f_x\right)f & \approx & \beta   V_{\mf}\; [\widetilde{E}_{\frac{8}{9}}]_{12}\;  + \beta \hf_{\mf+1,\mf}e_\mf^T\; [\widetilde{E}_{\frac{8}{9}}]_{13}\;  v_{\mf+1} \notag \\
    h_n\varphi_1(h_n f_x)f & \approx &  \beta V_{\mf}\; [\widetilde{E}_1]_{12}\;  + \beta\hf_{\mf+1,\mf}e_\mf^T\; [\widetilde{E}_1]_{13}\;  v_{\mf+1}, \notag
\end{eqnarray}
donde $[\widetilde{E}_{c_j}]_{ik}$ denota la submatriz $i,k$ de la matriz particionada $\widetilde{E}_{c_j}$, $V_m$ es la matriz con base ortonormal de $\mathcal{K}_ \mf(f_x,f)$ ya calculado por el Algoritmo \ref{alg:Arnoldi} para $\mathcal{K}_\mf(h_nf_x,f)$, y $\beta=\nnorm{\nnorm{f} }_2$.

Utilizando la aproximación $(\mf , \pf ,\qf , k)$-Krylov-Padé (\ref{eq:kp_aprox}) para $u_j$ en las fórmulas embebidas (\ref{lldis}), obtenemos la Fórmulas de Runge-Kutta linealizadas localmente
\begin{equation}  \label{LLDPK scheme}
    y_{n+1}\,=\,y_n+\widetilde{u}_s+h_n \sum_{j=1}^{s}b_j \widetilde{\kt}_j \,\,\, \text{and} \,\,\, \
    \widehat{y}_{n+1}\,=\, y_n+\widetilde{u}_s+h_n \sum_{j=1}^{s}\widehat{b}_j \widetilde{\kt}_j,
\end{equation}
para integrar PVI de grandes dimensiones, donde
\begin{equation*}
    \widetilde{\kt}_j = f\left( y_n+\widetilde{u}_j+h_n \sum_{i=1}^{j-1}a_{j,i}\widetilde{\kt}_i \right) - f( y_n) - f_x(y_n)\widetilde{u}_j.
\end{equation*}
y $\widetilde{\kt}_1=0$, son $a_{j,i}$, $b_j$, $\widehat{b}_j$ los coeficientes de Runge-Kutta de  Dormand y Prince definidos en la Tabla \ref{ButcherTabla}.

Es importante destacar que, a diferencia de otros integradores exponenciales de alto orden, las fórmulas embebidas (\ref{LLDPK scheme}) involucran la aproximación de un solo vector de tiempo de función phi. En efecto, mientras que los integradores exponenciales en general requieren de aproximar varios términos de la forma $\tau^k \varphi_k(\tau A_k)b_k$ con más de un valor de $k$, las fórmulas embebidas (\ref{LLDPK scheme}) solo requieren la aproximación de la acción de una sola función phi sobre un único vector $\tau \varphi_1(\tau A_1)b_1$. Como se ha explicado anteriormente, las cinco aproximaciones $\widetilde{u}_j$ a los términos distintos de cero $c_jh_n\varphi_1(c_jh_nf_x)f$ que aparecen en (\ref{LLDPK scheme}) se calculan eficientemente en cada paso de integración por medio de solo una aproximación por un subespacio de Krylov que se construye mediante el algoritmo \ref{alg:Arnoldi} y solo una exponencial matricial mediante el método de Padé.

\section{Esquemas con tamaño de paso fijo y dimensión de Krylov variable}

\subsection{Selección de la dimensión de Krylov y orden del Padé}


\subsection{Sketch (renombrar esta sección)}

\subsection{Experimentos numéricos}


\section{Esquemas adaptativos con tamaño de paso variable}

\subsection{Estimación del tamaño de paso}

\subsection{Control del tamaño de paso}

\subsection{Reutilización del Jacobiano}

\subsection{Control del \textit{Breakdown}}

\subsection{Aproximación Krylov-Padé adaptativa}

\subsection{Sketch (renombrar esta sección)}

\subsection{Experimentos numéricos}
\chapter{Método de Linealización Local de Orden Superior Libre de Jacobiano}\label{chapter:llrk-fj}

En el Capítulo \ref{chapter:exp-int-and-ll-methods} se presentó la aproximación Lineal Local de Orden Superior. Por construcción, la discretización Lineal Local~(\ref{definition LLS}), contiene productos de la matriz jacobiana por un vector  $f_x(y)b$. Al aproximar dichos productos se construirá la aproximación Lineal Local de Orden Superior. Por tanto el objetivo de este capítulo es construir la familia de métodos de Linealización Local de Orden Superior Libre de Jacobiano, estimar las condiciones de orden esta familia. Procediendo similar al Capítulo \ref{chapter:lldp} se utilizará la aproximación Krylov-Padé libre de Jacobiano presentada en la Sección \ref{section:fj-krylov-pade-approx} para aproximar la nueva ecuación diferencial lineal auxiliar.

\section{Discretización y esquemas numéricos}

Sin perdida de generalidad, similar a Sección \ref{section:fj-krylov-pade-approx}, se asumirá que existe una función $(\eta+1)$ continuamente diferenciable $g: \mathbb{R}^{d}\times \mathbb{R}^{d} \times \mathbb{R}_+ \to \mathbb{R}^{d}$ que aproxima al producto $f_x(y)b$ con orden $\eta$ para la cual la cota
\begin{equation} \label{eq:g_bound2}
	\nnnorm{g(y,b;\delta)-f_x(y)b} \leq \mathfrak{L}\nnnorm{b}^{\eta+1}\delta^{\eta}
\end{equation}
se cumple, donde $f_x(y)$ es la matriz jacobiana del campo vectorial $f$ en el punto $y$, $y,b$ son vectores $d$-dimensionales y $\mathfrak{L}$ es una constante positiva que depende solamente de la norma de las derivadas de $f_x$.

Además, supondremos que la función (\ref{eq:g_bound2}) se utiliza para aproximar el producto de la matriz jacobiana por el vector en los PVI auxiliares (\ref{ODE r}), y que existe una aproximación libre de jacobiano $\widehat{z}(\cdotp ;\widetilde{y}_n)$ a la solución del IVP lineal
\begin{equation}
    \frac{dz(t)}{dt} = a(\widetilde{y}_n;z(t)) \,,\;\;\; z(t_n)=\widetilde{y}_n \,,\;\;\; t\in[t_n,t_{n+1}]\label{fjsyst0}
\end{equation}
donde $\widetilde{y}_n\approx x(t_n)$. Denotando por $v$ la solución del PVI no lineal
\begin{equation}
    \frac{dv(t)}{dt} = \widehat{q}(\widetilde{y}_n;t,v(t)) \,,\;\;\; v(t_n)=0 \,,\;\;\; t\in[t_n,t_{n+1}], \label{fjsyst}
\end{equation}
donde $\widehat{q}(\widetilde{y}_n;s,\xi)=f(\widehat{z}(s;\widetilde{y}_n)+\xi)-g(\widetilde{y}_n,\widehat{z}(s;\widetilde{y}_n)-\widetilde{y}_n;\delta)-f(\widetilde{y}_n)$ es una aproximación libre de Jacobiano al campo vectorial $q$ en~(\ref{ODE r})~y $g(\widetilde{y}_n,\widehat{z}(s;\widetilde{y}_n)-\widetilde{y}_n,\delta)$ una aproximación del tipo~(\ref{eq:g_bound2})~al producto $f_x(\widetilde{y}_n)(\widehat{z}(s;\widetilde{y}_n)-\widetilde{y}_n)$.

\begin{definition}\label{definition:holl-fj}
Sea $\hat{z}(\cdot;\widetilde{y}_n)$ una aproximación libre de Jacobiano de orden superior a la solución del PVI lineal (\ref{fjsyst0}), y $\widehat{v}_ {n+1}=\widehat{v}_n+h_n\Lambda^{\widetilde{y}_n}(t_n,\widehat{v}_n;h_n)$ un integrador de orden superior de paso simple para el PVI no lineal (\ref{fjsyst}) en $t_{n+1}$.Entonces toda recursividad de la forma
\begin{equation*}
    \widetilde{y}_{n+1}= \hat{z}(t_n+h_n;\widetilde{y}_n)+h_n\Lambda^{\widetilde{y}_n}(t_n,0;h_n)\;,
\end{equation*}
define un esquema de Linealización Local de Orden Superior Libre de Jacobiano para el PVI (\ref{syst}), para todo $n=0,\ldots,N-1$, con $\widetilde{y}_0=x(t_0)$.
\end{definition}

El siguiente presenta los órdenes de convergencia para los integradores libres de Jacobiano.

\begin{theorem} \label{theorem:fj-llrk-convergence}
Sea $\mathfrak{D}$ una vecindad de $\{x(t):t\in [t_0,T]\} \subseteq \mathbb{R}^{d}$ y $x$ la solución del PVI (\ref{syst}) con campo vectorial $f\in \mathcal{C}^{r+1}(\mathfrak{D})$ y $r \in \mathbb{N}$. Teniendo $t_n,t_{n+1}\in (t)_h$, sea $\hat{z}(t_n+h_n;\widetilde{y}_n)$ una aproximación libre de Jacobiano a la solución del PVI lineal (\ref{fjsyst0}) en $t_{n+1}$ y  $\widehat{v}_{n+1}=\widehat{v}_n+h_n\Lambda^{\widetilde{y}_n}(t_n,\widehat{v}_n;h_n)$ un integrador de paso simple para el PVI no lineal (\ref{fjsyst}) en $t_{n+1}$. Supongamos que
\begin{align}
	\nnnorm{z(t_n+h_n;\widetilde{y}_n)-\widehat{z}(t_n+h_n;\widetilde{y}_n)} \leq c_1 h_n^{p+1} \\
	\nnnorm{v(t_n+h_n)-h_n\Lambda^{\widetilde{y}_n}(t_n,0;h_n)}\leq c_2 h_n^{r+1}  \label{ineq:bound32}
\end{align}
para todo $\widetilde{y}_n \in \mathfrak{D}$, con  $p \in \mathbb{N}$ y constantes positivas  $c_1$ and $c_2$. Entonces para $h$ suficientemente pequeña, los esquemas libres de Jacobiano
\begin{equation*}
    \widetilde{y}_{n+1}= \hat{z}(t_n+h_n;\widetilde{y}_n)+h_n\Lambda^{\widetilde{y}_n}(t_n,0;h_n)\;
\end{equation*}
tienen error de truncamiento local
\begin{equation*}
    \nnnorm{x(t_{n+1})-\hat{z}(t_n+h_n;x(t_n))-h_n\Lambda^{x(t_n)}(t_n,0;h_n)}\leq \mathfrak{c}_1 h_n^{\min\{r,p\}+1} + \mathfrak{c}_2 h_n^{\eta+2}\delta^{\eta},
\end{equation*}
donde $\eta$ es el orden de convergencia de la aproximación $g(.;\delta)$ en el PVI (\ref{fjsyst}) y $\mathfrak{c}_1,\mathfrak{c}_2$ son constantes positivas. Además, con $\delta\propto h^{\alpha}$ y  $\alpha \geq 0$ el error global satisface
\begin{equation*}
    \nnnorm{x(t_{n+1})-\widetilde{y}_{n+1}}\leq Ch^{\min\{r,p,\alpha\eta+\eta+1\}}
\end{equation*}
para todo  $n=0,\ldots,N-1$, donde $C$ es una constante positiva.
\end{theorem}

\textbf{Demostración}
Sea $\mathcal{X}=\{ x(t): t\in [t_0,T] \}$ compacto. Como $\mathcal{X}$ es un conjunto compacto contenido en el conjunto abierto $\mathfrak{D}\subseteq \mathbb{R}^d$ entonces existe $\varepsilon>0$ tal que el conjunto compacto
\begin{equation*}
    \mathcal{A}_{\varepsilon}=\left\{ \zeta \in\mathbb{R}^d: \min\limits_{x(t)\in \mathcal{X}}\nnnorm{\zeta -x(t)}\leq \varepsilon \right\}
\end{equation*}
está contenido en $\mathfrak{D}$.

Primero tomaremos $\widetilde{y}_n = x(t_n)$ en las ecuaciones (\ref{fjsyst0}) y (\ref{fjsyst}). De la Definición \ref{definition:holl-fj} se obtiene
\begin{multline}
    \nnnorm{ x(t_{n+1}) - \hat{z}(t_{n+1};x(t_n))-h_n\Lambda^{x(t_n)}(t_n,0;h_n)} \\
    \leq \nnnorm{u(t_{n+1};x(t_n))-\hat{z}(t_{n+1},x(t_n))}
    + \nnnorm{r(t_{n+1};x(t_n))-v(t_{n+1};x(t_n))}  \\+ \nnnorm{v(t_{n+1};x(t_n))-h_n\Lambda^{x(t_n)}(t_n,0;h_n)}
    \label{ineq:main}
\end{multline}
donde  $u(t_{n+1};x(t_n))=x(t_n)+\phi(t_n,x(t_n);h_n)$ es la solución del PVI lineal (\ref{ODE-SYST-LINEAL-1}) en  $t_{n+1}$, $r(t_{n+1};x(t_n))$ es la solución del PVI (\ref{ODE r}) en  $t = t_{n+1}$.

Como $r$ y $v$ son soluciones de PVIs, del ``Lema fundamental'' (ver Teorema 10.2 en \cite{hairer1993solving}) se obtiene
\begin{equation}
    \nnnorm{r(t;x(t_n))-v(t;x(t_n))} \leq \frac{\epsilon}{\emph{L}}(\me{\emph{L}(t-t_n)}-1)\leq \epsilon h_n \me{\emph{L}h_n} \label{ineq:bound1}
\end{equation}
para $t\in[t_n,t_{n+1}]$, donde
\begin{align*}
    \epsilon & =  \sup\limits_{t\in[t_n,t_{n+1}]}\nnnorm{q(x(t_n);t,r(t))-\widehat{q}(x(t_n);t,r(t))}\\
    &\leq \sup\limits_{t\in[t_n,t_{n+1}]} \nnnorm{f(u(t)+r(t))-f(\widehat{z}(t)+r(t))}\\ 
    & + \sup\limits_{t\in[t_n,t_{n+1}]} \nnnorm{f_x(x(t_n))(u(t)-x(t_n))-g(x(t_n),\widehat{z}(t)-x(t_n);\delta)},
\end{align*}
y $\emph{L}$ es la constantes de Lipschitz de la función $q(x(t_n);\cdotp)$. Para el primer y segundo término del miembro derecho de la desigualdad anterior  se tiene
\begin{align}
    \nnnorm{f(u(t)+r(t))-f(\widehat{z}(t)+r(t))} \leq  \emph{L}_1\nnnorm{u(t)-\widehat{z}(t)} \label{termino1}
\end{align}
\begin{multline}
    \nnnorm{f_x(x(t_n))(u(t)-x(t_n))-g(x(t_n),\widehat{z}(t)-x(t_n);\delta)} \\
    \leq \nnnorm{f_x(x(t_n))\phi(t_n,x(t_n);t-t_n)-g(x(t_n),\phi(t_n,x(t_n);t-t_n);\delta) } \\
    \enspace+ \nnnorm{g(x(t_n),u(t)-x(t_n);\delta)-g(x(t_n),\widehat{z}(t)-x(t_n);\delta)},
    \label{termino2}
\end{multline}
y $\emph{L}_1$ es la constantes de Lipschitz de la función $f$. Utilizando  $\phi(t_n,x(t_n);h_n)=h_n\varphi_1(h_n f_x(x(t_n)))f(x(t_n))$ con el operador $\varphi_1(z)=(e^z-1)/z$ y la desigualdad (\ref{eq:g_bound2}), para el primer término del miembro derecho de la última desigualdad se obtiene
\begin{eqnarray}
    \nnnorm{ f_x(x(t_n))\phi(t_n,x(t_n);h_n) -g(x(t_n),\phi(t_n,x(t_n);h_n);\delta) } &  \leq & \mathfrak{c}_1 h_n^{\eta+1}\delta^{\eta}, \label{termino3}
\end{eqnarray}
donde  $\mathfrak{c}_1$ es una constante positiva que depende de la norma de las derivadas de $f_x$ en  $\mathcal{A}_\varepsilon$ y de $\sup\limits_{\xi\in \mathcal{A}_\varepsilon} \nnnorm{\varphi_1(h f_x(\xi))f(\xi)}$.

Para el segundo término de miembro derecho de (\ref{termino2}), se tiene
\begin{equation}
    \nnnorm{g(x(t_n),u(t)-x(t_n);\delta)-g(x(t_n),\widehat{z}(t)-x(t_n);\delta)} \leq  \emph{L}_2\nnnorm{u(t)-\widehat{z}(t)},
    \label{termino4}
\end{equation}
donde $\emph{L}_1$ es la constantes de Lipschitz de la función $g(x(t_n),.;\delta)$. Entonces de las desigualdades (\ref{termino1})-(\ref{termino4}) para $\epsilon$ en (\ref{ineq:bound1}) se obtiene
\begin{eqnarray*}
	\epsilon & \leq & \mathfrak{c}_1 h_n^{\eta+1}\delta^{\eta} +  (\emph{L}_1+\emph{L}_2) \sup\limits_{t\in[t_n,t_{n+1}]} \nnnorm{u(t;x(t_n))-\hat{z}(t,x(t_n))}.
\end{eqnarray*}

De las desigualdades (\ref{ineq:bound32})-(\ref{ineq:bound1}) se obtiene el error de truncamiento local
\begin{equation*}
    \nnnorm{x(t_{n+1})-\hat{z}(t_n+h_n;x(t_n))-h_n\Lambda^{x(t_n)}(t_n,0;h_n)}\leq \mathfrak{c}_1 h_n^{\eta+2}\delta^{\eta} + \mathfrak{c}_2 h_n^{\min\{r,p\}+1},
\end{equation*}
donde $\mathfrak{c}_2$ es una constante positiva. Con $\delta\propto h^{\alpha}$ y el Teorema 3.6 en \cite{hairer1993solving} se obtiene el error global
\begin{equation*}
    \nnnorm{x(t_{n+1})-\widetilde{y}_{n+1}}\leq Ch^\gamma \;,
\end{equation*}
con orden de convergencia  $\gamma = \min\{r,p,\alpha\eta+\eta+1\}$, donde $C$ es una constante positiva. Finalmente, para garantizar que
$\mathbf{y}_{n+1}\in \mathcal{A}_{\varepsilon }$
para todo $n=0,...,N-1$, es suficiente que  $0<h<\Delta $, donde  $\Delta$ es seleccionado de forma que  $C\Delta ^{\gamma
}\leq \varepsilon $. $\Box$

De acuerdo al Teorema \ref{theorem:fj-llrk-convergence}, los esquemas libres de Jacobiano de orden superior para PVI como (\ref{syst}) preservan el orden $r$ del esquema utilizado para integrar el sistema no lineal auxiliar (\ref{fjsyst}) si la condición
\begin{equation*}
    \min\{p,\alpha\eta+\eta+1\} \geq r
\end{equation*}
se cumple para los parámetros libres $p,\alpha,\eta$, lo cual proporciona una guía simple sobre cómo elegir la aproximación $\widehat{z}$ para resolver el PVI lineal (\ref{fjsyst0}) y la aproximación $g(.;h^{\alpha}_n)$ en el PVI no lineal (\ref{fjsyst}).

Es importante destacar que el teorema \ref{theorem:fj-llrk-convergence} extiende el resultado de convergencia del Teorema 15 en \cite{de2013local} a la familia de esquemas libres de jacobiano Localmente Linealizados de Orden Superior. Cuando la matriz jacobiana se toma exacta, los dichos esquemas convierten en esquemas ordinarios Localmente Linealizados de Orden Superior, y el resultado del Teorema \ref{theorem:fj-llrk-convergence} se reduce al del Teorema 15 en \cite{de2013local}.

\section{Esquemas basados en Krylov-Padé libre de Jacobiano}
En esta sección, se utilizará la aproximación Krylov-Padé libre de Jacobiano para diseñar esquemas Localmente Linealizado de Orden Superior. Para estos esquemas, tenemos el siguiente resultado.

\begin{theorem}\label{theorem:kp-fj-llrk-convergence}
	Sea $x$ la solución del PVI (\ref{syst}) con campo vectorial $f\in \mathcal{C}^{r+1}(\mathfrak{D}, \mathbb{R}^d)$ y $r \in \mathbb{N}$.
	Con $t_n,t_{n+1}\in (t)_h$ y $\widetilde{y}_n \in \mathfrak{D}$, denotaremos por $\widetilde{\phi}(t_n,\widetilde{y}_n;h_n)$ a la aproximación ($\mf,\pf,\qf,\kt$)-Krylov-Padé Libre de Jacobiano (\ref{eq:gen_kp_aprox_fj}) a $h_n\varphi_1(h_nf_x(\widetilde{y}_n))f(\widetilde{y}_n)$, 
	y por $\widehat{v}_{n+1}=\widehat{v}_n+h_n\Lambda^{\widetilde{y}_n}(t_n,\widehat{v}_n;h_n)$ al integrador de paso simple para el PVI (\ref{fjsyst}) con orden de convergencia $r$. Entonces, para $h$ suficientemente pequeña, es esquema de Linealización Local de Orden Superior
	\begin{equation}
	\widetilde{y}_{n+1}= \widetilde{y}_n+\widetilde{\phi}(t_n,\widetilde{y}_n;h_n)+h_n\Lambda^{\widetilde{y}_n}(t_n,0;h_n) \label{JFKPHOLL}
	\end{equation}
	posee error de truncamiento local
	\begin{align}
	\nnnorm{x(t_{n+1})-x(t_n)-\widetilde{\phi}(t_n,x(t_n);h_n)+h_n\Lambda^{x(t_n)}(t_n,0;h_n)}_2 \\ \leq \mathfrak{c}_0h_n^{\min\{\mf+1,\pf+\qf,r \}+1} + \mathfrak{c}_1h_n^{\beta\eta_1+2}\delta_1^{\eta_1} + \mathfrak{c}_2h_n^{\eta_2+2}\delta_2^{\eta_2}, \nonumber
	\end{align}
	donde $\eta_1$ es el orden de la aproximación $g_1(.;\delta_1)$ en el algoritmo de Arnoldi libre Jacobiano \ref{alg:iArnoldi} para el $\mf$-ésimo subespacio de Krylov $\widehat{\mathcal{K}}_\mf(h^\beta f_x(x(t_n)),f(x(t_n));\delta_1)$, $\eta_2$  es el orden de la aproximación $g_2(.;\delta_2)$ en el PVI (\ref{fjsyst}), y $\mathfrak{c}_0,\mathfrak{c}_1,\mathfrak{c}_2$ son constante positivas. 
	Además, con $\delta_1\propto h^{\alpha_1}$, $\delta_2\propto h^{\alpha_2}$, $\alpha_1,\alpha_2 \geq 0$, es error global satisface
	\[ \nnnorm{x(t_{n+1})-\widetilde{y}_{n+1}}_2\leq Ch^{\min\{\mf+1,\pf+\qf,r,(\beta+\alpha_1)\eta_1+1,(1+\alpha_2)\eta_2+1\}} \]
	para todo $n=0,\ldots,N-1$, donde $C$ es una constante positiva.
\end{theorem}
\textbf{Demostración} Sea $\widehat{z}(t_n+h_n;\widetilde{y}_n)=\widetilde{y}_n+\widehat{K}_{\mf,k}^{\pf,\qf}\left(h_n,f_x(\widetilde{y}_n), f(\widetilde{y}_n); \eta_1, \delta_1, \beta \right)$ la aproximación de la solución $z(t_n+h_n;\widetilde{y}_n)=\widetilde{y}_n+h_n\varphi_1(h_nf_x(\widetilde{y}_n))f(\widetilde{y}_n)$ de PVI lineal (\ref{fjsyst0}) data por la aproximación ($\mf,\pf,\qf,\kt$)-Krylov-Padé libre de Jacobiano  (\ref{eq:gen_kp_aprox_fj}). Del Teorema~\ref{theorem:Krylov-fj-bound} se tiene
\begin{equation*}
    \nnnorm{z(t_n+h_n;\widetilde{y}_n)-\widehat{z}(t_n+h_n;\widetilde{y}_n)} \leq \mathfrak{c}_0 h_n^{\min\{\mf+1,\pf+\qf \}+1} + \mathfrak{c}_1h_n^{\beta\eta_1+2}\delta_1^{\eta_1},
\end{equation*}
donde $\mathfrak{c}_0,\mathfrak{c}_1$ son constantes positivas. De la desigualdad anterior y el Teorema~\ref{theorem:fj-llrk-convergence} los errores local y global de los esquemas libre de Jacobiano (\ref{JFKPHOLL}) se obtienen de forma directa. $\Box$

En el caso de que se utilize un esquema Runge-Kutta de orden $r$ con $s$ etapas en (\ref{JFKPHOLL}) para aproximar el PVI no lineal (\ref{fjsyst}), se obtiene el esquema Runge-Kutta Localmente Linealizado(LLRK) libre de Jacobiano
\begin{equation}  \label{JFLLDPKa scheme}
    \widetilde{y}_{n+1}\,=\,\widetilde{y}_n+\widetilde{\phi}(t_n,\widetilde{y}_n;h_n)+h_n \sum_{j=1}^{s}b_j \widetilde{\kt}_j
\end{equation}
para integrar grandes PVIs (\ref{syst}), donde
\begin{equation} \label{JFLLDPKb scheme}
    \widetilde{\kt}_j = f\left( \widetilde{y}_n+\widetilde{\phi}(t_n,\widetilde{y}_n;c_jh_n)+h_n \sum_{i=1}^{j-1}a_{j,i}\widetilde{\kt}_i \right)  - g_2(\widetilde{y}_n,\widetilde{\phi}(t_n,\widetilde{y}_n;c_jh_n);h^{\alpha_2}_n) - f( \widetilde{y}_n)
\end{equation}
\begin{sloppypar}
y $\widetilde{\kt}_1=0$, siendo $a_{j,i}$, $b_j$, $c_j$ los coeficientes de Runge-Kutta, y $\widetilde{\phi}(t_n,\widetilde{ y}_n;c_jh_n)$ la aproximación de Krylov-Padé libre de Jacobiano $\widehat{K}_{\mf,k}^{\pf,\qf}\left(c_jh_n, f_x(\widetilde{y}_n) , f(\widetilde{y}_n) ; \eta_1, h_n^{\alpha_1}, \beta \right)$, definida en (\ref{eq:gen_kp_aprox_fj}), a $c_jh_n\varphi_1(c_jh_nf_x(\widetilde {y}_n))f(\widetilde{y}_n)$. De acuerdo con el Teorema \ref{theorem:kp-fj-llrk-convergence}, un esquema LLRK libre de Jacobiano (\ref{JFLLDPKa scheme}) para el PVI (\ref{syst}) conserva el orden $r$ del esquema de Runge-Kutta utilizado para integrar el PVI (\ref{fjsyst}) si la condición del orden
\end{sloppypar}
\begin{equation}\label{order condition}
    \min\{\mf+1,\pf+\qf,(\beta+\alpha_1)\eta_1+1,(1+\alpha_2)\eta_2+1\} \geq r
\end{equation}
\begin{sloppypar}
se cumple, la cual proporciona una guía sobre cómo elegir las aproximaciones $g_2(.;h^{\alpha_2}_n)$ en (\ref{JFLLDPKb scheme}) y $g_1(.;h^{\alpha_1}_n)$ en el Algoritmo de Arnoldi libre de Jacobiano \ref{alg:iArnoldi} para el subespacio de Krylov
$\widehat{\mathcal{K}}_\mf(h^\beta_n f_x(\widetilde{y}_n),f(\widetilde{y}_n))$. Cabe destacar que, desde el punto de vista de la implementación, el esquema de orden superior (\ref{JFLLDPKa scheme}) requiere solo una llamada al algoritmo de Arnoldi libre de Jacobiano \ref{alg:iArnoldi} en cada paso de integración para calcular las $s$ aproximaciones $ \widetilde{\phi}(t_n,\widetilde{y}_n;c_jh_n)$.
\end{sloppypar}

Como casos particulares, podemos utilizar los esquemas clásicos de Runge-Kutta de orden 3 y 4 para integrar el PVI (\ref{fjsyst}) y así obtener el esquemas LLRK libre de jacobiano de tercer orden
\begin{equation}
    \widetilde{y}_{n+1}=\widetilde{y}_n+\widetilde{\phi}(t_n,\widetilde{y}_n;h_n) + h_n(\frac{2}{3}\widetilde{k_2}+\frac{1}{6}\widetilde{k}_3) \label{JFLLRK3}
\end{equation}
con
\begin{equation*}
    \widetilde{k_j}=f(\widetilde{y}_n+\widetilde{\phi}(t_n,\widetilde{y}_n;c_jh_n)+2c_jh_n\widetilde{k}_{j-1})-g_2(\widetilde{y}_n,\widetilde{\phi}(t_n,\widetilde{y}_n;c_jh_n);h^{\alpha_2}_n)-f(\widetilde{y}_n),
\end{equation*}
$\widetilde{k}_1\equiv 0$, $c=[0,1/2,1]$, y condición (\ref{order condition}) with $r=3$; y el esquema LLRK libre de jacobiano de cuarto orden
\begin{equation}
    \widetilde{y}_{n+1}=\widetilde{y}_n+\widetilde{\phi}(t_n,\widetilde{y}_n;h_n) + \frac{h_n}{6}(2\widetilde{k_2}+2\widetilde{k}_3+\widetilde{k}_4) \label{JFLLRK4}
\end{equation}
con
\begin{equation*}
    \widetilde{k_j}=f(\widetilde{y}_n+\widetilde{\phi}(t_n,\widetilde{y}_n;c_jh_n)+c_jh_n\widetilde{k}_{j-1})-g_2(\widetilde{y}_n,\widetilde{\phi}(t_n,\widetilde{y}_n;c_jh_n);h^{\alpha_2}_n)-f(\widetilde{y}_n),
\end{equation*}
$\widetilde{k}_1\equiv 0$, $c=[0,1/2,1/2,1]$ y condición (\ref{order condition}) con $r=4$.

De acuerdo con las definiciones anteriores, se puede obtener un esquema de tercer orden a partir de (\ref{JFLLRK3}) utilizando la diferencia finita hacia adelante de primer orden como aproximaciones $g_1$ y $g_2$, y estableciendo $\beta=\alpha_1 =\alpha_2=1$. De manera similar, se puede obtener un esquema de cuarto orden a partir de (\ref{JFLLRK4}) utilizando la diferencia finita hacia adelante de primer y segundo orden como aproximaciones $g_1$ y $g_2$, respectivamente, y estableciendo $\beta= \alpha_1=1,5$ y $\alpha_2=0,5$. Los valores de $\mf,\pf,\qf$ en los esquemas mencionados se pueden estimar adaptativamente en cada paso de integración como se indica en la Sección \ref{section:num-sim-kp} y cumpliendo la condición de orden correspondiente.

\subsection{Simulaciones numéricas}

Se implementó la expresión LLRK libre de Jacobiano (\ref{JFLLRK4}) en un código Matlab flexible \textit{JF-LLRK} con parámetros variables. Con este código, se presentarán experimentos numéricos preliminares que ilustran los resultados de convergencia discutidos anteriormente. También se utilizó la expresión (\ref{JFLLRK4}) para implementar el código de Matlab \textit{JF-LLRK4} con el esquema LLRK libre de Jacobiano de cuarto orden mencionado en la sección anterior, con diferencias finitas de primer y segundo orden como aproximaciones $g_1$ y $g_2$ respectivamente, $\beta=\alpha_1=1.5$ y $\alpha_2=0.5$. Para ilustrar el potencial de la nueva clase de integradores, en un segundo conjunto de simulaciones, se comparará el rendimiento del código \textit{JF-LLRK4} con el de los códigos Matlab \textit{BDF4} y \textit{JF-Exp4}, que son implementaciones libres de Jacobiano de la fórmula diferencial hacia atrás de cuarto orden \cite{hairer1993solving} y el integrador exponencial de cuarto orden (5.8) de \cite{hochbruck1998exponential}. Esta comparación también incluye la implementación libre de Jacobiano \textit{JF-EPIRK4} del método de cuatro orden de paso constante \textit{EPIRK4} \cite{rainwater2016new} considerado en \cite{einkemmer2017performance}, que usa el código Matlab \textit {phipm} de \cite{niesen2012algorithm} para calcular la combinación lineal de productos  la función phi por vector y la diferencia finita hacia adelante para aproximar los productos del Jacobiano por un vector. Para una comparación justa, el valor de $\delta$ para calcular la diferencia finita en los códigos \textit{BDF4}, \textit{JF-Exp4} y \textit{JF-EPIRK4} se establece como en el código \textit {JF-LLRK4}, es decir, $\delta=h^{1.5}$. Es importante recalcar que en principio, desde un punto de vista práctico, la matriz jacobiana exacta o su producto con un vector en cualquier integrador numérico puede ser reemplazada por una aproximación pero, desde un punto de vista teórico, esa aproximación conduce a muchos cambios importantes en las propiedades de integrador original (tales como orden de convergencia, condición de orden y estabilidad) que distorsionan notablemente el desempeño del integrador original en la práctica \cite{hairer1993solving,hochbruck1998exponential, tranquilli2014rosenbrock}.


\subsubsection{Simulaciones preliminares}
En este primer conjunto de simulaciones, con el código \textit{JF-LLRK}, se evaluará el orden de convergencia del esquema (\ref{JFLLRK4}) en función de la dimensión de Krylov $\mf$, el orden de Padé $\pf$, los órdenes $\eta_1$ y $\eta_2$ de la diferencia finita que aproxima los productos de la matriz jacobiana por el vector, y los parámetros $\beta,\alpha_1,\alpha_2$ de estas dos aproximaciones relacionadas con el paso tamaño $h$. Para estas simulaciones, se utilizará la ecuación de prueba  Brusselator~(\ref{ex:Brus}).

\begin{figure}[t]
	\centering
	\subfigure{\includegraphics[width=0.95\textwidth]{Graphics/lldp-fj/mp_new.png}}
	\caption{Gráfico Log-log de error ${e_i=\max_{t_n\in(t)_{h_i}}\nnnorm{y_n-x(t_n)}_\infty}$ contra $h_i$ integrando la ecuación de Ejemplo \ref{ex:Brus} con el esquema (\ref{JFLLRK4}), fijados $\eta_1=\eta_2=2$, $\alpha_1=\alpha_2=0.5$, $\beta=1$ y $h_i=2^{-i}$, $i=7,8,9,10,11$, y : Izquierda, $\mf=1,2,3$, $\pf=2$; y Derecha, $\pf=1,2,3$, $\mf=3$.} \label{Fig1}
\end{figure}

\begin{table}[h]
	\centering
	\caption{
		Orden de convergencia $r$ del esquema (\ref{JFLLRK4}) y las estimaciones $\widetilde{r}$ para diferentes valores de $\mf$ y $\pf$, el $90\%$ límite de confianza $\Delta$ de $\widetilde{r}$, y el coeficiente de determinación $R^2$ de la recta ajustada en Figura \ref{Fig1}. Lo valores $\eta_1=\eta_2=2$, $\alpha_1=\alpha_2=0.5$ y $\beta=1$ se mantienen fijos.}
	\begin{adjustbox}{width=0.8\columnwidth,center}
		\begin{tabular}{cccccllccccc}
			\cline{1-12}
			&  & $\pf=2$ &  &  &  &  &  &  & $\mf=3$ &  &  \\ \cline{2-5}\cline{9-12}
			$\mf$ & $r$ & $\widetilde{r}$ & $\pm \varDelta$ & $R^{2}$ &  &  & $\pf+\pf$
			& $r$ & $\widetilde{r}$ & $\pm \varDelta$ & $R^{2}$ \\ 
			\cline{1-5}\cline{8-12}
			1 & 2 & 1.999 & 0.001 & 0.98 &  &  & 2 & 2 & 1.996 & 0.003 & 0.98 \\ 
			2 & 3 & 3.145 & 0.149 & 0.98 &  &  & 4 & 4 & 4.148 & 0.476 & 0.98 \\ 
			3 & 4 & 4.148 & 0.476 & 0.98 &  &  & 6 & 4 & 4.141 & 0.633 & 0.98 \\ 
			\cline{1-12}
		\end{tabular}
	\end{adjustbox}
	\label{tab:mporders}
\end{table}


Los errores $e_i=\max_{t_n\in(t)_{h_i}}\nnnorm{y_n-x(t_n)}_\infty$ del código \textit{JF-LLRK} en la integración de la ecuación de Brusselator fueron calculados para diferentes discretizaciones de tiempo $(t)_{h_i}$ con un tamaño de paso fijo $h_i$, donde la \textquotedblleft solución exacta\textquotedblright ~$x$ se estima mediante el código Matlab \textit{ode15s} con tolerancias $RTol= 10^{-12}$ y $ATol=10^{-14}$. La Figura \ref{Fig1}-izquierda muestra cuatro de estos errores para el código \textit{JF-LLRK} con diferentes valores de $\mf$, y fijo $\pf=2$, $\eta_1=\eta_2=2$ , $\alpha_1=\alpha_2=0.5$ y $\beta=1$, así como la recta ajustada a los puntos $(\log_2(h_i),\log_2(e_i))$ con $i=1,. ..,4$. La tabla \ref{tab:mporders}-izquierda presenta el valor de la pendiente $\widetilde{r}$ de la línea recta ajustada para cada valor de $\mf$, lo que proporciona una estimación del orden de convergencia del esquema. La tabla también presenta los $90\%$ límites de confianza de $\widetilde{r}$, el coeficiente de determinación como indicador de la bondad de la línea ajustada, y el orden de convergencia de convergencia esperado $r$ que - de acuerdo con el Teorema \ref {theorem:kp-fj-llrk-convergence} - el esquema (\ref{JFLLRK4}) tiene para los diferentes valores de $\mf$. La figura \ref{Fig1}-derecha y la tabla \ref{tab:mporders}-derecha presentan salidas similares pero para el código \textit{JF-LLRK} con varios valores de $\pf$ y $\mf=3$ fijados $\eta_1=\eta_2=2$, $\alpha_1=\alpha_2=0.5$ y $\beta=1$. Observe que el orden de convergencia de convergencia estimado $\widetilde{r}$ proporcionada en la Tabla \ref{tab:mporders} está de acuerdo con el orden de convergencia del esquema (\ref{JFLLRK4}) establecido por el Teorema \ref{theorem:kp-fj-llrk-convergence} para los valores considerados de $\mf$ y $\pf$.

Análogamente, las Figuras ref{Fig2}, \ref{Fig3} y las Tablas \ref{tab:out}, \ref{tab:in} presentan los resultados de las simulaciones correspondientes al orden de de convergencia del esquema (\ref{JFLLRK4}) en función de los parámetros de sus dos aproximaciones libres de jacobiano. Nuevamente el orden de convergencia de convergencia estimado $\widetilde{r}$ proporcionada en las tablas \ref{tab:in} y \ref{tab:out} está de acuerdo con el orden de convergencia de convergencia del esquema (\ref{JFLLRK4}) establecido por Teorema \ref{kp-fj-llrk-convergencia} para los valores considerados de $\eta_1$, $\eta_2$, $\alpha_1$, $\alpha_2$ y $\beta$.


\begin{figure}[h]
	\centering
	\includegraphics[width=0.95\textwidth]{Graphics/lldp-fj/out_new.png}
	\caption{Gráfico Log-log de error $e_i=\max_{t_n\in(t)_{h_i}}\nnnorm{y_n-x(t_n)}_\infty$ contra $h_i$ integrando la ecuación de Ejemplo \ref{ex:Brus} con el esquema (\ref{JFLLRK4}), fijados $\mf=14$, $\pf=4$, $\eta_1=2$, $\alpha_1=0.5$, $\beta=1$, y: Izquierda, $\eta_2=1$, $\alpha_2=0,0.5,1,1.5$, $h_i=2^{-i}$, $i=7,8,9,10$ y Derecha, $\eta_2=2$, $\alpha_2=0,0.125,0.25,0.5$, $h_i=2^{-i}$, $i=5,6,7,8,9$.}
	\label{Fig2}
\end{figure}

\begin{table}[h]
	\centering
	\caption{
		Orden de convergencia $r$ del esquema (\ref{JFLLRK4}) y las estimaciones $\widetilde{r}$ para diferentes valores de $\eta_2$ y $\alpha_2$, el $90\%$ límite de confianza $\Delta$ de $\widetilde{r}$, el coeficiente de determinación $R^2$ de la recta ajustada en Figura \ref{Fig2}. Los valores de $\mf=14$, $\pf=4$, $\eta_1=2$, $\alpha_1=0.5$ y $\beta=1$ se mantienen fijos. }
	\begin{adjustbox}{width=0.8\columnwidth,center}
		\begin{tabular}{ c  c c c c  c  c c c c c}
			\hline
			& \multicolumn{4}{c}{$\eta_2=1$} & & & \multicolumn{4}{c}{$\eta_2=2$} \\
			\cline{2-5} \cline{8-11}
			$\alpha_2$ & $r$ & $\widetilde{r}$ & $\pm\varDelta$ & $R^2$ & & $\alpha_2$ & $r$ & $\widetilde{r}$ & $\pm\varDelta$ & $R^2$ \\
			\hline
			0 & 2 & 1.999 & 0.001 & 0.98 & & 0 & 3 & 3.000 & 0.002 & 0.97 \\
			0.5 & 2.5 & 2.496 & 0.003 & 0.98 & & 0.125 & 3.25 & 3.247 & 0.003 & 0.97 \\
			1 & 3 & 2.992 & 0.005 & 0.98 & & 0.25 & 3.5 & 3.487 & 0.013 & 0.97 \\
			1.5 & 3.5 & 3.376 & 0.025 & 0.98 & & 0.5 & 4 & 3.986 & 0.029 & 0.97 \\
			\hline
		\end{tabular}
	\end{adjustbox}
	\label{tab:out}
\end{table}


\begin{figure}[h]
	\centering
	\includegraphics[width=0.95\textwidth]{Graphics/lldp-fj/in_new.png}
	\caption{Gráfico Log-log de error $e_i=\max_{t_n\in(t)_{h_i}}\nnnorm{y_n-x(t_n)}_\infty$ contra $h_i$ integrando la ecuación de Ejemplo \ref{ex:Brus} con el esquema (\ref{JFLLRK4}), fijados $\mf=14$, $\pf=4$, $\eta_2=2$, $\alpha_2=1$, $h_i=2^{-i}$, $i=5,6,7,8,9$, y: Izquierda, $\alpha_1=0,1,2$ for $\eta_1=1,\beta=1$; Derecha, $\alpha_1=0.5,1,1.5$ for $\eta_1=2,\beta=0$.}
	\label{Fig3}
\end{figure}


\begin{table}[h]
	\centering
	\caption{
        Orden de convergencia $r$ del esquema (\ref{JFLLRK4}) y las estimaciones $\widetilde{r}$ para diferentes valores de  $\eta_1$, $\alpha_1$ y $\beta$, el $90\%$ límite de confianza $\Delta$ de $\widetilde{r}$, el coeficiente de determinación $R^2$ de la recta ajustada en Figura \ref{Fig3}. Los valores de $\mf=14$, $\pf=4$, $\eta_2=2$ y$\alpha_2=1$ se mantienen fijos.}
	\begin{adjustbox}{width=0.8\columnwidth,center}
		\begin{tabular}{ccccccccccccc}
			\hline
			&  & \multicolumn{4}{c}{$\eta _{1}=1$} &  &  &  & \multicolumn{4}{c}{$\eta
				_{1}=2$} \\ \cline{3-6}\cline{10-13}
			$\beta $ & $\alpha _{1}$ & $r$ & $\widetilde{r}$ & $\pm \varDelta$ & $R^{2}$
			&  & $\beta $ & $\alpha _{1}$ & $r$ & $\widetilde{r}$ & $\pm \varDelta$ & $%
			R^{2}$ \\ \hline
			1 & 0 & 2 & 1.997 & 0.004 & 0.97 &  & 0 & 0.5 & 2 & 2.014 & 0.017 & 0.97 \\ 
			1 & 1 & 3 & 3.020 & 0.010 & 0.97 &  & 0 & 1 & 3 & 3.300 & 0.116 & 0.97 \\ 
			1 & 2 & 4 & 3.863 & 0.421 & 0.97 &  & 0 & 1.5 & 4 & 4.031 & 0.039 & 0.97 \\ 
			\hline
		\end{tabular}
	\end{adjustbox}
	\label{tab:in}
\end{table}

\subsubsection{Simulaciones comparativas}\label{sc:comparison}

En este conjunto de simulaciones, se utilizaran tres ecuaciones diferenciales parciales empleadas con frecuencia en la literatura como ecuaciones de prueba. Estas ecuaciones son: Brusselator 2D (\ref{ex:Brus}), Brusselator 2D (\ref{ex:Brus2D}), Burger's (\ref{ex:Burger}) y Gray-Scott 2D~(\ref{ex:GS2D}).

Para cada ecuación de prueba, los resultados de la integración numérica se sintetizan en los diagramas de precisión contra tiempo de la Figura \ref{work-precision diagram}. En estos diagramas, la precisión de los códigos \textit{JF-LLRK4}, \textit{JF-Exp4}, \textit{JF-EPIRK4} y \textit{BDF4} se mide por el error $e_i=\max\limits_ {t_n\in(t)_{h_i}}\nnnorm{y_n-x(t_n)}_\infty$ entre la \textquotedblleft solución exacta\textquotedblright~$x$ de la ecuación de prueba y la solución aproximada $y_n$ de cada código calculado en cinco particiones de tiempo con un tamaño de paso fijo $h_i$, donde la \textquotedblleft solución exacta\textquotedblright ~$x $ se estima nuevamente como en las simulaciones preliminares. La figura \ref{work-precision diagram} muestra que, para estas ecuaciones, el esquema LLRK libre de jacobiano (\ref{JFLLRK4}) exhibe una precisión similar o mayor que los otros tres integradores libres de jacobiano, pero con un costo computacional menor o similar. Esta diferencia en el tiempo computacional se explica por los resultados de las Tablas \ref{tab:br}-\ref{tab:gs2d}.

\begin{figure}[h]
	\centering
	\includegraphics[width=1\textwidth]{Graphics/lldp-fj/Diagram_new.jpg}
	\caption{Diagramas comparativos de precisión contra tiempo en escala log-log para cada uno de los cuatro códigos en la integración delas cuatro ecuaciones de prueba for the four codes in the integration of the four test equations.} \label{work-precision diagram}
\end{figure}

Para cada Código, las Tablas \ref{tab:br}-\ref{tab:gs2d} presentan el tamaño del paso \textit{h}, el número de pasos \textit{Pasos}, el número de evaluaciones del campo vectorial \textit{f-Eval}, el número de aproximaciones del subespacio Krylov \textit{K-subspace} a los productos de la función phi por un vector, el número de sistemas lineales en el código \textit{BDF4} resuelto por el método General Minimal Residue \textit {GMRES}, y el número de aproximaciones de Padé \textit{Padé} requeridas por los tres integradores exponenciales. Además, la dimensión mínima $\mf_{min}$, máxima $\mf_{max}$ y total $\mf_{total}$ de los subespacios de Krylov requerida por los códigos \textit{JF-LLRK4}, \textit{JF -Exp4} y \textit{JF-EPIRK4} para integrar las ecuaciones de prueba en todo el intervalo de integración también se especifica. Para el código \emph{BDF4}, $\mf_{min}$, $\mf_{max}$ y $\mf_{total}$ representan el número mínimo, máximo y total de iteraciones realizadas por el método GMRES sobre el intervalo de integración completo.

En cada paso de integración, al igual que para el esquema (5.8) de \cite{hochbruck1998exponential}, el código \textit{JF-Exp4} realiza tres descomposiciones en subespacios de Krylov y, al menos, tres aproximaciones (6,6)-Padé a la función $\varphi_1$ de las matrices de Hessenberg resultantes del Algoritmo de Arnoldi libre de Jacobiano \ref{alg:iArnoldi}. \textit{JF-Exp4} utiliza la diferencia hacia adelante directa de primer orden como una aproximación del producto de la matriz jacobiana por un vector en el algoritmo \ref{alg:iArnoldi}, lo cual reduce a tres el orden de convergencia del esquema (5.8) de \cite{hochbruck1998exponential} (ver Teorema 5.1 en \cite{hochbruck1998exponential}). Para estimar la dimensión de Krylov $\mf$, el código \textit{JF-Exp4} usa el algoritmo adaptativo de \cite{hochbruck1998exponential} y no hay restricción al valor mínimo para $\mf$.

El código \textit{JF-EPIRK4} requiere de las mismas descomposiciones de Krylov y un número similar de aproximaciones de Padé que el código \textit{JF-Exp4} en cada paso de integración, pero la dimensión de Krylov $\mf$ se estima automáticamente mediante el estrategia adaptativa de \cite{niesen2012algorithm} implementada en el código de Matlab \textit{phipm}.

Para resolver los sistemas algebraicos no lineales, en cada paso de integración, el código \textit{BDF4} emplea el método clásico de Newton junto con el método General Minimal Residue (GMRES) (función de Matlab \textit{gmres} con tolerancia $RTol=10^{-6}$) y la diferencia finita hacia adelante de primer orden como una aproximación al producto de la matriz jacobiana pro un vector. En la función de Matlab \textit{gmres}, se eliminaron las comprobaciones computacionalmente costosas de las funciones de Matlab \textit{iterchk} y \textit{iterapp}.

\begin{table}[h!]
	\caption{Desempeño de los códigos en la integración de la ecuación Brusselator con $M=100$, $d=200$.}
	\centering
	\begin{adjustbox}{width=0.9\columnwidth,center}
		\begin{tabular}{cccccccccc}
			\hline
			\textit{h} & Código & Pasos & f-Eval & K-subspace & GMRES & Padé & $\mf_{total}$ & $\mf%
			_{min}$ & $\mf_{max}$ \\ \hline
			\multicolumn{1}{l}{0.0250} & \multicolumn{1}{l}{JF-LLRK4} & 40 & 852 & 40 &
			0 & 40 & 492 & 4 & 16 \\
			\multicolumn{1}{l}{} & \multicolumn{1}{l}{JF-Exp4} & 40 & 3324 & 120 & 0 &
			1168 & 3124 & 4 & 48 \\
			\multicolumn{1}{l}{} & \multicolumn{1}{l}{BDF4} & 40 & 7318 & 0 & 400 & 0 &
			3305 & 3 & 24 \\
			\multicolumn{1}{l}{} & \multicolumn{1}{l}{JF-EPIRK4} & 40 & 1238 & 120 & 0 &
			625 & 718 & 4 & 11 \\
			\multicolumn{1}{l}{0.0200} & \multicolumn{1}{l}{JF-LLRK4} & 50 & 967 & 50 &
			0 & 50 & 517 & 4 & 13 \\
			\multicolumn{1}{l}{} & \multicolumn{1}{l}{JF-Exp4} & 50 & 2855 & 150 & 0 &
			1147 & 2605 & 4 & 48 \\
			\multicolumn{1}{l}{} & \multicolumn{1}{l}{BDF4} & 50 & 7156 & 0 & 448 & 0 &
			2780 & 2 & 23 \\
			\multicolumn{1}{l}{} & \multicolumn{1}{l}{JF-EPIRK4} & 50 & 1425 & 150 & 0 &
			716 & 775 & 3 & 10 \\
			\multicolumn{1}{l}{0.0100} & \multicolumn{1}{l}{JF-LLRK4} & 100 & 1477 & 100
			& 0 & 100 & 577 & 4 & 10 \\
			\multicolumn{1}{l}{} & \multicolumn{1}{l}{JF-Exp4} & 100 & 1773 & 300 & 0 &
			970 & 1273 & 2 & 36 \\
			\multicolumn{1}{l}{} & \multicolumn{1}{l}{BDF4} & 100 & 6574 & 0 & 476 & 0 &
			2269 & 2 & 13 \\
			\multicolumn{1}{l}{} & \multicolumn{1}{l}{JF-EPIRK4} & 100 & 2278 & 300 & 0 &
			976 & 978 & 2 & 7 \\
			\multicolumn{1}{l}{0.0050} & \multicolumn{1}{l}{JF-LLRK4} & 200 & 2609 & 200
			& 0 & 200 & 809 & 4 & 6 \\
			\multicolumn{1}{l}{} & \multicolumn{1}{l}{JF-Exp4} & 200 & 2256 & 600 & 0 &
			1242 & 1256 & 1 & 15 \\
			\multicolumn{1}{l}{} & \multicolumn{1}{l}{BDF4} & 200 & 7055 & 0 & 588 & 0 &
			2678 & 1 & 12 \\
			\multicolumn{1}{l}{} & \multicolumn{1}{l}{JF-EPIRK4} & 200 & 3894 & 600 & 0 &
			1294 & 1294 & 1 & 5 \\
			\multicolumn{1}{l}{0.0025} & \multicolumn{1}{l}{JF-LLRK4} & 400 & 5201 & 400
			& 0 & 400 & 1601 & 4 & 5 \\
			\multicolumn{1}{l}{} & \multicolumn{1}{l}{JF-Exp4} & 400 & 3945 & 1200 & 0 &
			1945 & 1945 & 1 & 4 \\
			\multicolumn{1}{l}{} & \multicolumn{1}{l}{BDF4} & 400 & 9530 & 0 & 919 & 0 &
			3556 & 1 & 10 \\
			\multicolumn{1}{l}{} & \multicolumn{1}{l}{JF-EPIRK4} & 400 & 7287 & 1200 & 0 &
			2087 & 2087 & 1 & 3 \\
			\hline
		\end{tabular}
	\end{adjustbox}
	\label{tab:br}
\end{table}



\begin{table}[h!]
	\caption{Desempeño de los códigos en la integración de la ecuación Brusselator 2D con $M=40$, $d=3200$.}
	\centering
	\begin{adjustbox}{width=0.9\columnwidth,center}
		\begin{tabular}{cccccccccc}
			\hline
			\textit{h} & Código & Pasos & f-Eval & K-subspace & GMRES & Padé & $\mf_{total}$ & $\mf%
			_{min}$ & $\mf_{max}$ \\ \hline
			\multicolumn{1}{l}{0.01000} & \multicolumn{1}{l}{JF-LLRK4} & 10 & 144 & 10
			& 0 & 10 & 54 & 4 & 8 \\
			\multicolumn{1}{l}{} & \multicolumn{1}{l}{JF-Exp4} & 10 & 251 & 30 & 0 & 139
			& 201 & 2 & 20 \\
			\multicolumn{1}{l}{} & \multicolumn{1}{l}{BDF4} & 10 & 1243 & 0 & 86 & 0 &
			283 & 2 & 8 \\
			\multicolumn{1}{l}{} & \multicolumn{1}{l}{JF-EPIRK4} & 10 & 263 & 30 & 0 &
			129 & 133 & 3 & 6 \\
			\multicolumn{1}{l}{0.00625} & \multicolumn{1}{l}{JF-LLRK4} & 16 & 218 & 16
			& 0 & 16 & 74 & 4 & 6 \\
			\multicolumn{1}{l}{} & \multicolumn{1}{l}{JF-Exp4} & 16 & 284 & 48 & 0 & 167
			& 204 & 2 & 15 \\
			\multicolumn{1}{l}{} & \multicolumn{1}{l}{BDF4} & 16 & 1618 & 0 & 118 & 0 &
			334 & 1 & 7 \\
			\multicolumn{1}{l}{} & \multicolumn{1}{l}{JF-EPIRK4} & 16 & 383 & 48 & 0 &
			174 & 175 & 2 & 6 \\
			\multicolumn{1}{l}{0.00500} & \multicolumn{1}{l}{JF-LLRK4} & 20 & 268 & 20 &
			0 & 20 & 88 & 4 & 6 \\
			\multicolumn{1}{l}{} & \multicolumn{1}{l}{JF-Exp4} & 20 & 307 & 60 & 0 & 184
			& 207 & 2 & 15 \\
			\multicolumn{1}{l}{} & \multicolumn{1}{l}{BDF4} & 20 & 1449 & 0 & 109 & 0 &
			313 & 1 & 6 \\
			\multicolumn{1}{l}{} & \multicolumn{1}{l}{JF-EPIRK4} & 20 & 464 & 60 & 0 &
			204 & 204 & 2 & 5 \\
			\multicolumn{1}{l}{0.00250} & \multicolumn{1}{l}{JF-LLRK4} & 40 & 521 & 40 &
			0 & 40 & 161 & 4 & 5 \\
			\multicolumn{1}{l}{} & \multicolumn{1}{l}{JF-Exp4} & 40 & 455 & 120 & 0 & 251
			& 255 & 1 & 8 \\
			\multicolumn{1}{l}{} & \multicolumn{1}{l}{BDF4} & 40 & 1452 & 0 & 120 & 0 &
			384 & 1 & 5 \\
			\multicolumn{1}{l}{} & \multicolumn{1}{l}{JF-EPIRK4} & 40 & 846 & 120 & 0 &
			326 & 326 & 1 & 4 \\
			\multicolumn{1}{l}{0.00200} & \multicolumn{1}{l}{JF-LLRK4} & 50 & 650 & 50
			& 0 & 50 & 200 & 4 & 4 \\
			\multicolumn{1}{l}{} & \multicolumn{1}{l}{JF-Exp4} & 50 & 541 & 150 & 0 & 289
			& 291 & 1 & 8 \\
			\multicolumn{1}{l}{} & \multicolumn{1}{l}{BDF4} & 50 & 1785 & 0 & 150 & 0 &
			457 & 1 & 5 \\
			\multicolumn{1}{l}{} & \multicolumn{1}{l}{JF-EPIRK4} & 50 & 1027 & 150 & 0 &
			377 & 377 & 1 & 4 \\
			\hline
		\end{tabular}
	\end{adjustbox}
	\label{tab:br2d}
\end{table}

\begin{table}[h!]
	\caption{Desempeño de los códigos en la integración de la ecuación Burger's con $M=400$, $d=400$.}
	\centering
	\begin{adjustbox}{width=0.9\columnwidth,center}
		\begin{tabular}{cccccccccc}
			\hline
			\textit{h} & Código & Pasos & f-Eval & K-subspace & GMRES & Padé & $\mf_{total}$ & $\mf%
			_{min}$ & $\mf_{max}$ \\ \hline
			\multicolumn{1}{l}{0.0050000} & \multicolumn{1}{l}{JF-LLRK4} & 100 & 1614 &
			100 & 0 & 100 & 714 & 4 & 10 \\
			\multicolumn{1}{l}{} & \multicolumn{1}{l}{JF-Exp4} & 100 & 6104 & 300 & 0 &
			2507 & 5604 & 3 & 27 \\
			\multicolumn{1}{l}{} & \multicolumn{1}{l}{BDF4} & 100 & 17304 & 0 & 910 & 0
			& 6469 & 2 & 12 \\
			\multicolumn{1}{l}{} & \multicolumn{1}{l}{JF-EPIRK4} & 100 & 2874 & 300 & 0 &
			1435 & 1574 & 3 & 7 \\
			\multicolumn{1}{l}{0.0025000} & \multicolumn{1}{l}{JF-LLRK4} & 200 & 2922 &
			200 & 0 & 200 & 1122 & 4 & 8 \\
			\multicolumn{1}{l}{} & \multicolumn{1}{l}{JF-Exp4} & 200 & 7287 & 600 & 0 &
			3784 & 6287 & 2 & 20 \\
			\multicolumn{1}{l}{} & \multicolumn{1}{l}{BDF4} & 200 & 27333 & 0 & 1665 & 0
			& 7783 & 2 & 10 \\
			\multicolumn{1}{l}{} & \multicolumn{1}{l}{JF-EPIRK4} & 200 & 4988 & 600 & 0 &
			2388 & 2388 & 2 & 5 \\
			\multicolumn{1}{l}{0.0012500} & \multicolumn{1}{l}{JF-LLRK4} & 400 & 5361 &
			400 & 0 & 400 & 1761 & 4 & 5 \\
			\multicolumn{1}{l}{} & \multicolumn{1}{l}{JF-Exp4} & 400 & 8123 & 1200 & 0 &
			4975 & 6123 & 1 & 11 \\
			\multicolumn{1}{l}{} & \multicolumn{1}{l}{BDF4} & 400 & 37402 & 0 & 2454 & 0
			& 9416 & 2 & 10 \\
			\multicolumn{1}{l}{} & \multicolumn{1}{l}{JF-EPIRK4} & 400 & 9111 & 1200 & 0 &
			3911 & 3911 & 1 & 4 \\
			\multicolumn{1}{l}{0.0006250} & \multicolumn{1}{l}{JF-LLRK4} & 800 & 10400 &
			800 & 0 & 800 & 3200 & 4 & 4 \\
			\multicolumn{1}{l}{} & \multicolumn{1}{l}{JF-Exp4} & 800 & 11223 & 2400 & 0
			& 7039 & 7223 & 1 & 8 \\
			\multicolumn{1}{l}{} & \multicolumn{1}{l}{BDF4} & 800 & 32269 & 0 & 2251 & 0
			& 8838 & 2 & 5 \\
			\multicolumn{1}{l}{} & \multicolumn{1}{l}{JF-EPIRK4} & 800 & 16766 & 2400 & 0 &
			6366 & 6366 & 1 & 4 \\
			\multicolumn{1}{l}{0.0003125} & \multicolumn{1}{l}{JF-LLRK4} & 1600 & 20800
			& 1600 & 0 & 1600 & 6400 & 4 & 4 \\
			\multicolumn{1}{l}{} & \multicolumn{1}{l}{JF-Exp4} & 1600 & 19555 & 4800 & 0
			& 11555 & 11555 & 1 & 4 \\
			\multicolumn{1}{l}{} & \multicolumn{1}{l}{BDF4} & 1600 & 58643 & 0 & 4360 & 0
			& 17910 & 2 & 8 \\
			\multicolumn{1}{l}{} & \multicolumn{1}{l}{JF-EPIRK4} & 1600 & 31872 & 4800 & 0 &
			11072 & 11072 & 1 & 3 \\
			\hline
		\end{tabular}
	\end{adjustbox}
	\label{tab:bg}
\end{table}


\begin{table}[h!]
	\caption{Desempeño de los códigos en la integración de la ecuación Gray-Scott 2D con $M=20$, $d=800$.}
	\centering
	\begin{adjustbox}{width=0.9\columnwidth,center}
		\begin{tabular}{cccccccccc}
			\hline
			\textit{h} & Código & Pasos & f-Eval & K-subspace & GMRES & Padé & $\mf_{total}$ & $\mf%
			_{min}$ & $\mf_{max}$ \\ \hline
			0.010000 & \multicolumn{1}{l}{JF-LLRK4} & 10 & 161 & 10 & 0 & 10 & 71 & 4 &
			10 \\
			& \multicolumn{1}{l}{JF-Exp4} & 10 & 656 & 30 & 0 & 258 & 606 & 4 & 36 \\
			& \multicolumn{1}{l}{BDF4} & 10 & 923 & 0 & 52 & 0 & 332 & 2 & 14 \\
			& \multicolumn{1}{l}{JF-EPIRK4} & 10 & 296 & 30 & 0 & 148 & 166 & 4 & 9 \\
			0.005000 & \multicolumn{1}{l}{JF-LLRK4} & 20 & 296 & 20 & 0 & 20 & 116 & 4 &
			8 \\
			& \multicolumn{1}{l}{JF-Exp4} & 20 & 643 & 60 & 0 & 327 & 543 & 2 & 27 \\
			& \multicolumn{1}{l}{BDF4} & 20 & 1064 & 0 & 72 & 0 & 371 & 1 & 10 \\
			& \multicolumn{1}{l}{JF-EPIRK4} & 20 & 494 & 60 & 0 & 228 & 234 & 2 & 7 \\
			0.002500 & \multicolumn{1}{l}{JF-LLRK4} & 40 & 547 & 40 & 0 & 40 & 187 & 4 &
			8 \\
			& \multicolumn{1}{l}{JF-Exp4} & 40 & 787 & 120 & 0 & 439 & 587 & 1 & 20 \\
			& \multicolumn{1}{l}{BDF4} & 40 & 1308 & 0 & 101 & 0 & 468 & 2 & 7 \\
			& \multicolumn{1}{l}{JF-EPIRK4} & 40 & 889 & 120 & 0 & 369 & 369 & 1 & 5 \\
			0.002000 & \multicolumn{1}{l}{JF-LLRK4} & 50 & 669 & 50 & 0 & 50 & 219 & 4 &
			6 \\
			& \multicolumn{1}{l}{JF-Exp4} & 50 & 876 & 150 & 0 & 488 & 626 & 1 & 15 \\
			& \multicolumn{1}{l}{BDF4} & 50 & 1472 & 0 & 118 & 0 & 528 & 2 & 7 \\
			& \multicolumn{1}{l}{JF-EPIRK4} & 50 & 1079 & 150 & 0 & 429 & 429 & 1 & 5 \\
			0.001250 & \multicolumn{1}{l}{JF-LLRK4} & 80 & 1050 & 80 & 0 & 80 & 330 & 4
			& 6 \\
			& \multicolumn{1}{l}{JF-Exp4} & 80 & 1118 & 240 & 0 & 619 & 718 & 1 & 15 \\
			& \multicolumn{1}{l}{BDF4} & 80 & 2151 & 0 & 184 & 0 & 715 & 1 & 6 \\
			& \multicolumn{1}{l}{JF-EPIRK4} & 80 & 1626 & 240 & 0 & 586 & 586 & 1 & 4 \\
			\hline
		\end{tabular}
	\end{adjustbox}
	\label{tab:gs2d}
\end{table}


Se puede observar de las Tablas \ref{tab:br}-\ref{tab:gs2d}, que el código \textit{BDF4} requiere de un número mucho mayor de evaluaciones del campo vectorial que los otros tres códigos, lo que explica su mayor costo computacional en los diagramas de precisión contra tiempo de la Figura \ref{work-precision diagram}. Por otro lado, para los tamaños de paso más grandes, el número de evaluaciones del campo vectorial de los códigos \textit{JF-Exp4} y \textit{JF-EPIRK4} es mayor que el del código \textit{JF-LLRK4}, por lo que su costo computacional es mucho mayor que el del código \textit{JF-LLRK4}. Para los tamaños de paso más pequeños, el número de evaluaciones del campo vectorial del código \textit{JF-Exp4} es ligeramente inferior o similar al del código \textit{JF-LLRK4}, lo que explica el costo computacional similar de estos dos códigos. En la parte inferior de los diagramas de precisión contra tiempo de la Figura \ref{work-precision diagram} correspondientes a las ecuaciones Brusselator y Brusselator 2D.

Además, las Tablas \ref{tab:br}-\ref{tab:gs2d} muestran la efectividad de la estrategia del código \textit{JF-LLRK4} para la selección de la dimensión Krylov $\mf$ en cada paso de integración, con mínima variación entre los valores de $\mf_{min}$ y $\mf_{max}$, y por tanto, con un valor de $\mf_{total}$ mucho menor que los demás códigos.

En resumen, las simulaciones han demostrado que, con un costo computacional similar, el nuevo integrador libre de jacobiano presenta una precisión mucho mayor que los otros tres integradores libres de jacobiano; mientras que, con una precisión similar, el primero es mucho más rápido que los segundos.

Para concluir esta sección, recordemos que la selección de $\delta$ y $h$ en esquemas prácticos libres de jacobiano se realiza bajo diferentes criterios que resultan en valores óptimos para $\delta$ y $h$ independientes entre sí \cite{knoll2004jacobian}. Con este conocimiento en mente, la teoría desarrollada hasta este punto y los experimentos numéricos realizados con $\delta$ vinculados a una potencia de $h$ pretenden sentar las bases para diseñar esquemas prácticos Localmente Linealizados de Orden Superior Libres Jacobiano con valores óptimos de $h$ y $\delta$.
% \chapter{Método Runge-Kutta de Dormand and Prince Localmente Linealizado Libre de Jacobiano}\label{chapter:lldp-fj}

\section{Fórmulas embebidas}

\section{Aproximación Krylov-Padé para \texorpdfstring{$\{u_1;\ldots;u_s\}$}{u\_1,...,u\_s} con orden de Padé y dimensión de Krylov variable}

\section{Estrategia de selección de tamaño de paso}

\section{Aproximaciones de orden variable del producto del Jacobiano por un vector}

\section{Simulaciones numéricas}

\backmatter

\begin{conclusions}

%    En este trabajo se desarrollaron códigos adaptativos para la integración de problemas de valor
%    inicial de dimensiones no pequeñas basados en esquemas embebidos de linealización local de orden
%    superior con y sin la evaluación del jacobiano.
    
    En ésta Tesis se desarrollaron métodos de Linealización Local de Orden Superior para integrar problemas de valor
    inicial de dimensiones no pequeñas. Especificamante: 

    1)- Se construyeron aproximaciones Krylov-Padé con y sin la evaluación de Jacobiano para la solución de ecuaciones diferenciales lineales de dimensiones no pequeñas y se acotaron sus errores. Para cada aproximación, se propusieron estrategia efectivas para la estimación práctica de la dimensión de Krylov, el orden de Padé y el error de aproximacion. Se comprobó la eficacia de ambas aproximaciones para calcular la acción de funciones phi sobre vectores y, en se aspecto, se mostró las ventajas del uso de la aproximación Krylov-Padé con evaluación de Jacobiano en relación con otras aproximaciones similares existentes. 
    
%    Estas aproximaciones presentan un buen desempeño al poseer balance entre precisión y tiempo de cómputo a la hora aproximar la acción de las funciones phi sobre vectores. La mayor diferencia entre la aproxima que evalúa el Jacobiano y la que no es que la que utiliza el Jacobiano exacta disminuye su error a medida aumenta la dimensión de Krylov; sin embargo, la libre de Jacobiano posee un segundo término de error que depende del parámetro $\delta$, por tanto su error a partir de ciento valor de la dimensión de Krylov va a estar dominado por este término dependiente de $\delta$. Por tanto las aproximaciones libres de Jacobiano son por lo general menos precisas que las que utilizan en Jacobiano exacto.
    
    2)- Se construyeron nuevas fórmulas embebidas de Dormand y Prince localmente linealizadas para problemas de valor inicial de dimensiones no pequeñas utilizando las aproximaciones Krylov-Padé con evaluación de Jacobiano en el cálculo de los productos de funcion phi por vector. Se derivaron cotas para los errores y condiciones de orden simples. Se  desarrollaron estrategias adaptativa para la reutilización del Jacobino, la selección del tamaño de paso de integración, la dimensión de Krylov y el orden de Padé con las que se implementaron esquemas con tamaño de paso fijo y dimensión de Krylov variable, y esquemas con tamaño de paso y dimensión de Krylov variable. Se comprobó la eficacia de esos nuevos esquemas en la integración de diferentes ecuaciones de prueba y se constató que presentan igual o mejor precisión que los esquemas con los que fueron comparados requiriendo, en la mayoría de los casos, un número menor de pasos de integración.
    
%    Además, el desempeño de este método fue probado en la integración de diferentes ecuaciones de prueba mostrando un excelente rendimiento. En general estos esquemas Runge-Kutta de Dormand y Prince Localmente Linealizados presentan igual o mejor precisión que los métodos con los que fue comparado requiriendo en la mayoría de los casos menos pasos de integración.

    3)- Se extendió la concepción de los métodos Localmente Linealizados de Orden Superior a problemas en los que no es posible evaluar la matriz Jacobiana obteniéndose la nueva familia de métodos de  Linealización Local de Orden Superior Libres de Jacobiano. Para ésta nueva familia se obtuvieron cotas de errores y condiciones de orden simples. Utilizando las aproximaciones de Krylov-Padé libre de Jacobiano, se introdujo la subclase particular de esquemas Runge-Kutta Localmente Linealizados Libres de Jacobiano y, en paticular, con las fórmulas Runge-Kutta embebidas de Dormand y  Prince Localmente Linealizadas, se implementó un esquema adaptativo de orden variable libre de Jacobiano. Se comprobó la eficacia de los nuevos esquemas libres de Jacobiano en la integración de ecuaciones de prueba y se constató que presentan igual o mejor precisión que los esquemas con los que fueron comparados, pero con un costo computacional menor en la mayoría de los casos. 
    
%    Al combinar las aproximaciones Runge-Kutta y las aproximaciones Krylov-Padé libres de Jacobiano, anteriormente propuestas, con esta nueva familia se obtiene un nuevo grupo de métodos  Runge-Kutta Localmente Linealizados Libres de Jacobiano. Una estrategia adaptativa para la selección del tamaño, la dimensión de Krylov, el orden de Padé y el orden de la aproximación del producto del Jacobiano por un vector, fue desarrollada. Se realizaron un conjunto de experimentos numéricos para corroborar las condiciones de orden obtenidas. Además, otro conjunto de simulaciones fuero realizadas para medir el desempeño de estos nuevos métodos construidos en la integración de ecuaciones de prueba. En general estos nuevos métodos muestran un gran desempeño ya que son igual o mas precisos que los esquemas con los que fueron comparados, además en la mayoría de los casos con un costo computacionalmente menor.



\end{conclusions}

\begin{recomendations}

Como recomendaciones para futuros trabajos se propone:
\begin{itemize}
	\item Desarrollar una estrategia adaptativa para la selección de la dimensión Krylov en los método LL libres de Jacobiano que tenga en cuenta el efecto de saturación en la precisión que tiene el error de la aproximación del producto del Jacobiano por un vector
	\item Investigar formas de optimizar la cantidad de aproximaciones del producto del Jacobiano por un vector necesarias en los esquemas embebidos libres de Jacobiano
	\item Investigar vias de extender los LL de alto orden a otros tipos de ecuaciones como las algebro-diferenciales o las ecuaciones diferenciales aleatorias
\end{itemize}

\end{recomendations}


\include{BackMatter/Bibliography}

\end{document}