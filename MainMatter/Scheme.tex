\chapter{Esquemas de Linealizaci\'on Local para sistemas de ecuaciones no pequeños}\label{chapter:SCHEME}

En la Secci\'on~\ref{section:ESQ-LLRK} se present\'o el esquema Runge Kutta de Dormand y Prince localmente linealizado
para problemas de valor inicial de dimensiones peque\~nas. En este capítulo se introducen dos esquemas an\'alogos, pero
para problemas de dimensiones moderadas o grandes. Uno de ellos est\'a basado en el m\'etodo de Krylov para
exponeciales matriciales y el otro  en el llamado m\'etodo de Krylov libre de Jacobiano~(\emph{Jacobian free Krylov Method}). Se propone adem\'as una estrategia adaptativa para seleccionar autom\'aticamente el tama\~no de paso de los esquemas, la dimensi\'on de los subespacios de Krylov y el orden de la  aproximaci\'on de Pad\'e. También se propone un criterio para decidir cuando reutilizar o no el Jacobiano del campo vectorial evaluado en un paso de integración anterior al actual. 

Sin perdida de generalidad, consideraremos problemas de valor inicial autónomos.
\section{Esquemas embebidos LLDP\label{KRLOV-PADE}}

En la secci\'on~\ref{section:ESQ-LLRK} se present\'o los esquemas embebidos  Runge Kutta de Dormand y Prince localmente linealizados
para problemas de valor inicial de  dimensiones peque\~nas. Para extender este esquema a problemas de dimensiones no pequeñas se sustituye la aproximaci\'on de Pad\'e de la exponencial matricial por la aproximaci\'on con subespacios
de Krylov. Con este objetivo, utilizando el 
Teorema \ref{exp-phi}, la función $\varphi_j=L\;\me{c_{j}h_{n}M_{n}}\;r$ definida en Sección \ref{section:ESQ-LLRK} puede reescribirse como
\[\varphi_j=c_jh_n\phi_1(c_jh_nf_x(t_n,y_n))f(t_n,y_n),\]
%por Teorema \ref{exp-phi} se tiene
%\begin{equation*}
%\me{M_nc_jh_n}=\left[ 
%\begin{array}{cc}
%\me{c_jh_nf_x(t_n,y_n)} & c_jh_n\phi_1(c_jh_nf_x(t_n,y_n))f(t_n,y_n)\\ 
%0 & 1%
%\end{array}%
%\right]
%\end{equation*}%
donde  $\phi_1(z)=\frac{\me{z}-I}{z}$. 

Una primera aproximación $\widetilde{\varphi}_j$ para $\varphi_j$ puede obtenerse utilizando la aproximaci\'on básica especificada en (\ref{MFUNC-APROX}) para el subespacio $\mathcal{K}_\mf(h_nf_x,f)$. Esto es, 
%En lugar de aproximar $\varphi$ por Pad\'e, se har\'a con un subespacio de Krylov de orden
% $\mf$, por lo que la aproximaci\'on~(\ref{MFUNC-APROX})~\cite{Saad92} ser\'a
\begin{equation*}
\widetilde{\varphi}_j = \beta c_jV_{\mf}\phi_1\left( c_j H_{\mf} \right) e_1, %\label{phi1-aprox}
\end{equation*}
 donde $\beta=\vert\vert h_nf(t_n,y_n) \vert\vert_2$.
 
 Como se explic\'o en la Secci\'on \ref{section:Krylov-sp}, para aproximar $\varphi_j$ con el  $\mf$-\'esimo subespacio de Krylov es necesario disponer de una base ortonormal de dicho subespacio. Esta base se construye utilizando el Algoritmo~\ref{alg:Arnoldi} y, mediante el cual, también se obtienen un vector adicional de la base $v_{\mf+1}$ y un elemento adicional de la matriz de Heissenber $\hf_{\mf+1,\mf}$. De esta forma, utilizando el segundo termino de la serie (\ref{PHI-EXPANSION}), se puede
 agregar, sin costo computacional extra, una correcci\'on a la anterior aproximaci\'on $\widetilde{\varphi}_j$ obteni\'endose la nueva aproximación
\begin{equation}\label{kaprox}
K_{\mf}\left(c_j,h_nf_x , f \right)=\beta c_jV_{\mf}\phi_1\left( c_j H_{\mf} \right) e_1 + \beta c_j^{2}\hf_{\mf+1,\mf}e_\mf^T\phi_2\left(c_j H_\mf\right) e_1.
\end{equation}

En el esquema~(\ref{LLDP scheme}) tambi\'en es necesario calcular los términos
$h_n\phi_1(h_n f_x)f$, $\frac{1}{5}h_n\phi_1(\frac{1}{5}h_nf_x)f$, $\frac{3}{10}h_n\phi_1(\frac{3}{10}h_n f_x)f$, 
$\frac{4}{5}h_n\phi_1(\frac{4}{5} h_nf_x)f$, $\frac{8}{9}h_n\phi_1(\frac{8}{9}h_n f_x)f$
por lo que, en principio, habr\'ia que construir un subespacio de Krylov por cada uno de ellos. Sin embargo, utilizando la propiedad de invarianza 
ante escalado del algoritmo de Arnoldi, la relaci\'on entre las funciones $\phi_k$ y la exponencial matricial dada por Teorema \ref{exp-phi}, y la propiedad de flujo de la exponencial solamente es necesario construir un solo subespacio de Krylov. Para ello se comienza construyendo la base ortonormal del subespacio $\mathcal{K}_\mf(h_nf_x,f)$ utilizando el Algoritmo~\ref{alg:Arnoldi} y con la calculada matriz de Heissenber $H_\mf$ se crea la matriz 
\begin{equation}
    \overline{H} = \left[\begin{array}{cccc}
        H_\mf & e_1 & 0 & 0\\
        0_{1\times\mf} & 0 & 1 & 0\\
        0_{1\times\mf} & 0 & 0 & 1\\
        0_{1\times\mf} & 0 & 0 & 0
    \end{array}\right] \label{hhat}
\end{equation} definida en Teorema~\ref{exp-phi}. De esta forma se obtiene que 
\begin{equation}
\left[\begin{array}{cccc}
\me{\tau H_m} & \tau\phi_1(\tau H_m)e_1 & \tau^{2}\phi_2(\tau H_m)e_1 &
\tau^{3}\phi_3(\tau H_m)e_1 \\
& 1 & \tau & \frac{\tau^{2}}{2}\\
&  & 1 & \tau \\
&   &   & 1 \\
\end{array}\right] = \me{\tau\overline{H}} \; \label{phi_hhat}
\end{equation}
para todo $\tau \in \mathbb{R}$. Como $\frac{1}{90}$ es el m\'inimo com\'un denominador de $\frac{1}{5},\frac{3}{10},\frac{4}{5},\frac{8}{9},1$, por la propiedad de invarianza 
ante escalado del algoritmo de Arnoldi podemos utilizar la matriz $\frac{1}{90}\overline{H}$ y calcular su exponencial. Denotemos por $E_{1/90}$ dicha exponencial matricial. Las restantes exponenciales matriciales se calculan entonces por la propiedad de flujo. Mas precisamente, por las expresiones
\begin{align*}
    E_{2/90} &= E_{1/90}E_{1/90} & E_{4/90} &= E_{2/90}E_{2/90}\\
    E_{8/90} &= E_{4/90}E_{4/90} & E_{16/90} &= E_{8/90}E_{8/90}\\
    E_{32/90} &= E_{16/90}E_{16/90} & E_{80/90} &= E_{32/90}E_{16/90}E_{32/90}\\
    E_{1/10} &= E_{8/90}E_{1/90} & E_{1/5} &= E_{1/10}E_{1/10}\\
    E_{2/5} &= E_{1/5}E_{1/5} & E_{4/5} &= E_{2/5}E_{2/5}\\
    E_{3/10} &= E_{1/10}E_{1/5} & E_1 &= E_{4/5}E_{1/5}.
\end{align*}
Con las calculadas exponenciales $E_{1}=\me{\overline{H}},E_{\frac{1}{5}}=\me{\frac{1}{5}\overline{H}},
E_{\frac{3}{10}}=\me{\frac{3}{10}\overline{H}},E_{\frac{4}{5}}=\me{\frac{4}{5}\overline{H}},
E_{\frac{8}{9}}=\me{\frac{8}{9}\overline{H}}$ y usando (\ref{phi_hhat}) en (\ref{kaprox}) finalmente se obtiene las siguientes aproximaciones 
\begin{eqnarray*}
    \frac{1}{5}h_n\phi_1\left(\frac{1}{5}h_n f_x\right)f & \approx & \beta  \frac{1}{5} V_{\mf}\; [E_{\frac{1}{5}}]_{12}\; e_1 + \beta \left(\frac{1}{5}\right)^{2}\hf_{\mf+1,\mf}e_\mf^T\; [E_{\frac{1}{5}}]_{13}\; e_1 v_{\mf+1} \\
    \frac{3}{10}h_n\phi_1\left(\frac{3}{10}h_n f_x\right)f & \approx & \beta  \frac{3}{10} V_{\mf}\; [E_{\frac{3}{10}}]_{12}\; e_1 + \beta \left(\frac{3}{10}\right)^{2}\hf_{\mf+1,\mf}e_\mf^T\; [E_{\frac{3}{10}}]_{13}\; e_1 v_{\mf+1} \\
    \frac{4}{5}h_n\phi_1\left(\frac{4}{5}h_n f_x\right)f & \approx & \beta  \frac{4}{5} V_{\mf}\; [E_{\frac{4}{5}}]_{12}\; e_1 + \beta \left(\frac{4}{5}\right)^{2}\hf_{\mf+1,\mf}e_\mf^T\; [E_{\frac{4}{5}}]_{13}\; e_1 v_{\mf+1} \\
    \frac{8}{9}h_n\phi_1\left(\frac{8}{9}h_n f_x\right)f & \approx & \beta  \frac{8}{9} V_{\mf}\; [E_{\frac{8}{9}}]_{12}\; e_1 + \beta \left(\frac{8}{9}\right)^{2}\hf_{\mf+1,\mf}e_\mf^T\; [E_{\frac{8}{9}}]_{13}\; e_1 v_{\mf+1}  \\
    h_n\phi_1(h_n f_x)f & \approx &  \beta V_{\mf}\; [E_1]_{12}\; e_1 + \beta\hf_{\mf+1,\mf}e_\mf^T\; [E_1]_{13}\; e_1 v_{\mf+1},
\end{eqnarray*}
donde $[E_{c_j}]_{ik}$ denota la submatriz $i,k$ de la matriz particionada $E_{c_j}$ definida por (\ref{phi_hhat}) con $\tau=c_j$.

Como estas aproximaciones para $\phi_1$ se obtienen de truncar la expansi\'on en serie (\ref{PHI-EXPANSION}) en el segundo t\'ermino, el tercer término de (\ref{PHI-EXPANSION}) puede tomarse como un estimado crudo del error absoluto $\varepsilon$ de esa aproximaciones, es decir,
\[ \varepsilon \approx \beta\left\lvert  c_j^{3}\hf_{\mf+1,\mf} e_\mf^{T}\phi_3( c_jH_\mf)e_1 \right\rvert \left\lvert\left\lvert h_nf_x v_{\mf+1} \right\rvert\right\rvert_2. \] 
Como $\phi_3$ es mon\'otona creciente, entonces
calculando el error de la aproximaci\'on con el mayor de los $c_j$ tendremos el m\'aximo de los errores. Así, el error relativo $\varepsilon_{K}$ de esas aproximaciones respecto a las tolerancias del esquema puede acotarse por 
\begin{equation}
\varepsilon_{K} \approx \left(\sum\limits_{i=1}^{d} \left(\frac{\beta\left\vert \mathfrak{c}^{3}\hf_{\mf+1,\mf}
   e_\mf^{T}\phi_3(\mathfrak{c}H_\mf)e_1 \right\vert \rho^{i}}{AbsTol + RelTol*y_{n}^{i}}\right)^{2}\right)^{1/2},\label{errrel}
\end{equation}
donde AbsTol y RelTol son las tolerancias absoluta y relativa, $\rho=h_nf_x v_{\mf+1}$ y $\mathfrak{c}=\maxx{c_j}$.

Con fines pr\'acticos la exponencial matricial $E_{\frac{1}{90}}$ debe ser calculada mediante alguna aproximaci\'on. Para ello se utiliza la aproximaci\'on de Padé con escalamiento y potenciación definida en (\ref{P-MA-2}). De esta forma tenemos la aproximación
\[\widetilde{E}_{\frac{1}{90}} = F_k^{\pf,\qf}\left(\frac{1}{90}\overline{H}\right) \]
para $E_{\frac{1}{90}}$ mediante la cual se define la siguiente aproximaci\'on $(\mf , \pf ,\qf , k)$-Krylov-Padé truncada de segundo orden
\begin{equation} \label{kp_aprox}
K_{\mf,k}^{\pf,\qf}\left(c_j,h_nf_x , f \right)=\beta c_jV_{\mf}\;[\widetilde{E}_{c_j}]_{12}\; e_1 + \beta c_j^{2}\hf_{\mf+1,\mf}e_\mf^T\;[\widetilde{E}_{c_j}]_{13}\; e_1
\end{equation}
para $\varphi_j$. Remarcamos nuevamente que ésta aproximación se obtiene como subproducto del algoritmo de Arnoldi con un costo computacional mínimo y que es obviamente más precisa que la aproximación $(\mf , \pf ,\qf , k)$-Krylov-Padé truncada de primer orden \cite{errorpade} definida solo por el primer término de (\ref{kp_aprox}). 

Finalmente, sustituyendo la aproximación $\widetilde{\varphi}_j$ de (\ref{LLDP scheme}) por $K_{\mf,k}^{\pf,\qf}\left(c_j,h_nf_x , f \right)$ se obtienen las siguientes fórmulas embebidas LLDP para problemas de valor inicial no pequeños
\begin{equation}
\widetilde{y}_{n+1}\,=\,\widetilde{y}_n+\widetilde{\varphi}_s+h_n \sum_{j=1}^{s}b_j \kt_j \,\,\, \text{and} \,\,\, \
\widehat{y}_{n+1}\,=\, \widetilde{y}_n+\widetilde{\varphi}_s+h_n \sum_{j=1}^{s}\widehat{b}_j \kt_j,
\label{LLDPK scheme}
\end{equation}
donde $s = 7$ es el número de estados, $ \widetilde{\varphi }_j =K_{\mf,k}^{\pf,\qf}\left(c_j,h_nf_x,f \right)$ y 
\[ \kt_j = f\left( t_n+c_jh_n\, , \, \widetilde{y}_n+\widetilde{\varphi}_j+h_n \sum_{i=1}^{j-1}a_{j,i}\kt_i \right) - f(t_n\, ,\, \widetilde{y}_n) - f_x(t_n\, , \, \widetilde{y}_n)\widetilde{\varphi}_j. \]

\section{Esquemas embebidos LLDP libre de Jacobiano\label{KRLOV-PADE-FJ}}

 Note que en el Algoritmo \ref{alg:Arnoldi} se requiere del producto de una matriz por un vector y nunca se utiliza la matriz completa por si sola.
Esto permite que al aplicar ese algoritmo para construir una base ortonormal del subespacio $\mathcal{K}_\mf(h_nf_x,v_{d\times 1})$ el producto $h_nf_x\cdot v_{d\times 1}$ pueda ser reemplazado su correspondiente derivada por cociente diferencial \cite{carr11,newtonkry}. Como el Jacobiano existe entonces la derivada de cada componente de la funci\'on vectorial existe. 
La derivada direccional es aproximada~\cite{numerical} con orden 1 por
\[f_x\cdot v = \lim\limits_{\delta\to 0}\frac{f(x + \delta v)-f(x)}{\delta} \approx \frac{f(x + \delta v)-f(x)}{\delta}\]
y con orden 2 por
\[  f_x\cdot v = \lim\limits_{\delta\to 0}\frac{f(x + \delta v)-f(x-\delta v)}{2\delta} \approx \frac{f(x + \delta v)-f(x-\delta v)}{2\delta},\]
donde $x,v\in\mathbb{R}^{d}$. Esta aproximaci\'on es buena para un $\delta$ suficientemente pequeño. Para obtener un $\delta$ balanceado 
entre el error de redondeo y $\epsilon_{mach}$ se utiliza la siguiente f\'ormula~\cite{freej}
\begin{equation}
	\delta = \frac{\sqrt{(1+ \vert\vert x \vert\vert_2)\epsilon_{mach}}}{\vert\vert v \vert\vert_2}. \label{deltavar}
\end{equation}
%\[\delta = \frac{\sqrt{(1+ \vert\vert x \vert\vert_2)\epsilon_{mach}}}{\vert\vert v \vert\vert_2}. \]

Si en las fórmulas embebidas (\ref{LLDPK scheme}) se utiliza la aproximación de segundo orden para $f_x(t_n\, , \, \widetilde{y}_n)\widetilde{\varphi}_j$ y en el proceso de construcci\'on de la base ortonormal de $\mathcal{K}_\mf(h_nf_x,v_{d\times 1})$ mediante el Algoritmo~\ref{alg:Arnoldi} se utiliza la aproximaci\'on de primer orden para el producto $h_nf_x\cdot v_{d\times 1}$, entonces se obtienen las siguientes fórmulas embebidas LLDP-FJ para problemas de valor inicial no pequeños
\begin{equation}
\widetilde{y}_{n+1}\,=\,\widetilde{y}_n+\widetilde{\varphi}_s+h_n \sum_{j=1}^{s}b_j k_j \,\,\, \text{and} \,\,\, \
\widehat{y}_{n+1}\,=\, \widetilde{y}_n+\widetilde{\varphi}_s+h_n \sum_{j=1}^{s}\widehat{b}_j k_j,
\label{LLDPKFJ scheme}
\end{equation}
donde $s = 7$ es el número de estados,
\[ \kt_j = f\left( t_n+c_jh_n\, , \, \widetilde{y}_n+\widetilde{\varphi}_j+h_n \sum_{i=1}^{j-1}a_{j,i}\kt_i \right) - f(t_n\, ,\, \widetilde{y}_n) - \frac{f(t_n\, ,\, \widetilde{y}_n+\delta\widetilde{\varphi}_j)-f(t_n\, ,\, \widetilde{y}_n-\delta\widetilde{\varphi}_j)}{2\delta} \]
y
\[ \widetilde{\varphi }_j =\widetilde{K}_{\mf,k}^{\pf,\qf}\left(c_j,h_nf_x,f \right). \] Aquí, 
\begin{equation}
\widetilde{K}_{\mf,k}^{\pf,\qf}\left(c_j,h_nf_x , f \right)=\beta c_jV^{FJ}_{\mf}\;[\widetilde{E}^{FJ}_{c_j}]_{12}\; e_1 + \beta c_j^{2}\hf^{FJ}_{\mf+1,\mf}(e^{FJ}_\mf)^T\;[\widetilde{E}^{FJ}_{c_j}]_{13}\; e_1. \label{kp_fj_aprox}
\end{equation}
es la aproximaci\'on $(\mf , \pf ,\qf , k)$-Krylov-Padé libre de Jacobiano de
$c_jh_n\phi_1(c_jh_nf_x(t_n,\widetilde{y}_n))f(t_n,\widetilde{y}_n)$, donde las matrices $\widetilde{E}^{FJ}_{c_j}$ son calculadas como las $\widetilde{E}_{c_j}$ en la sección anterior pero usando la matriz de Heissenber libre de Jacobiano $H^{FJ}_m$ en lugar de $H_m$ en (\ref{hhat}).

\section{Orden de convergencia de los esquemas embebidos LLDP}

Con el propósito de determinar el orden de convergencia de los esquemas embebidos LLDP, se hace necesario encontrar cotas más precisas que el estimado $\varepsilon_{abs}$ para el error de la aproximación (\ref{kp_aprox}) de  $\varphi_j$. Los próximos resultados están orientados a alcanzar ese objetivo.

%En el Teorema~\ref{exp-bound} se tiene una cota para la aproximaci\'on de la exponencial, pero es necesaria una cota m\'as general para la familia de funciones $\phi$.
\begin{lemma}
    \label{PHI-REST}
    Sea $s_{\mf-1}$ un polinomio de grado $\mf-1$ obtenido de la expansi\'on en serie de Taylor
    de $\phi_k$
    \[ s_{\mf-1}(x)= \sum_{j=0}^{\mf-1}\frac{x^{j}}{(j+k)!}\]
    y sea $t_\mf(z)=\phi_k(x)-s_{\mf-1}$ el resto. Entonces $\forall x\in\mathbb{R}_+$
    y $\forall \mf\in\mathbb{Z}_+$
    \[ t_\mf(x)\leq \frac{x^\mf\me{x}}{\mf(\mf+k-1)!} \]
\end{lemma}
\emph{Demostraci\'on} El resto integral de la serie de Taylor de $\phi_k$ es
\begin{equation*}
t_m(x)= \frac{x^\mf}{(\mf+k-1)!}\int_{0}^{1} \me{(1-\tau)x}\tau^{\mf-1} \,d\tau.
\label{PHI-INTEG-REST}
\end{equation*}
Utilizando la desigualdad $\me{(1-\tau)x}\leq\me{x}$ tenemos
\begin{eqnarray*}
    t_m(x) & \leq & \frac{x^\mf}{(\mf+k-1)!}\int_{0}^{1} \me{x}\tau^{\mf-1} \,d\tau\\
    %& \leq & \frac{x^\mf\me{x}}{(\mf+k-1)!} \left. \frac{ \tau^{\mf} }{ \mf } \right\vert_{0}^{1}\\
    & \leq & \frac{x^\mf\me{x}}{\mf(\mf+k-1)!},
\end{eqnarray*}
lo cual concluye la prueba. $\Box$
\newline

El siguiente teorema brinda una cota para el error en la aproximaci\'on de $\phi_k$ por un vector mediante la serie (\ref{PHI-EXPANSION}) truncada hasta el segundo termino.
\begin{theorem}\label{CORRECTED-ERROR}
    Sea $A\in\mathbb{C}^{N\times N}$ una matriz y \[{\widetilde{\phi}_k(A,b) = ||b||_2 V_\mf \phi_k(H_\mf)e_1}+||b||_2 \hf_{\mf+1,\mf}e_\mf^T\phi_2\left(H_\mf\right) e_1 v_{\mf+1} \] una aproximación para $\phi_k(A)b$ en términos del $\mf$-\'esimo subespacio de Krylov de $A$. 
     Entonces el error de aproximaci\'on esta acotado por  
    \begin{gather*}
     \left\lvert\left\lvert \phi_k(A)b - \widetilde{\phi}_k(A,b) \right\rvert\right\rvert_2%\nonumber\\
     \leq \frac{2\vert\lvert b \rvert\rvert_2 \left( \sigma(A) \right)^{\mf+1}\me{\sigma(A)}}{(\mf+1)(\mf+k)!},
    \end{gather*}
    donde $\sigma(A)$ es el radio espectral de $A$.
\end{theorem}
\emph{Demostraci\'on}

 De la demostraci\'on del Teorema 4.7 en~\cite{Saad92} y sustituyendo el Lema 4.2 en~\cite{Saad92} por el Lema~\ref{PHI-REST}
se demuestra lo propuesto. $\Box$\\

\begin{lemma}\label{H-bound}
Sea $H_\mf=hV^{T}_\mf f_xV_\mf$ y $\overline{H}$ la matriz definida en  (\ref{hhat}). Si $\lVert H_\mf \rVert_2\leq\lVert hf_x\rVert_2$, entonces
\[ \lVert\overline{H}\rVert_2 \leq \lVert hf_x\rVert_2. \]
\end{lemma}
\emph{Demostraci\'on}
De (\ref{hhat}) se tiene que
\begin{eqnarray*}
	\overline{H}&=&I_{(\mf+3)\times\mf}H_\mf I_{\mf\times(\mf+3)}+B
\end{eqnarray*}
con
\[B=\left[\begin{array}{ccc}
0_{\mf\times\mf} & e_1 & 0_{\times 2}\\
0_{2\times\mf} & 0 & I_2\\
0_{1\times\mf} & 0 & 0_{1\times 2}
\end{array}\right].\]
Luego, 
\begin{eqnarray*}
    \lVert\overline{H}\rVert_2 &\leq& \lVert I_{(\mf+3)\times\mf} \rVert_2 \lVert H_\mf \rVert_2 \lVert I_{\mf\times(\mf+3)} \rVert_2+ \lVert B \rVert_2\\
    &\leq&\lVert H_\mf \rVert_2\\
    &\leq& \lVert hf_x\rVert_2
\end{eqnarray*}
dado que $\lVert B\rVert_2=0$ y $\lVert I_{(\mf+3)\times\mf}\rVert_2=
\lVert I_{\mf\times(\mf+3)}\rVert_2=1$.   $\Box$\\

\begin{lemma}\label{lemma-Krylov}
    Sea $K_{\mf,k}^{\pf,\qf}\left( h,  f_x , f \right)$ la aproximaci\'on $(\mf , \pf ,\qf , k)$-Krylov-Padé  truncada segundo orden de $h\phi_1(hf_x)f$ definida en (\ref{kp_aprox}). Entonces el error de la aproximaci\'on est\'a 
    acotado por
    \begin{equation}
    \left\lvert\left\lvert  h\phi_1(hf_x)f - 
    K_{\mf,k}^{\pf,\qf}\left( h,  f_x , f \right)\right\rvert\right\rvert_2%\nonumber\\
     \leq C_{\mf,k}^{\pf,\qf}\left(\lvert\lvert hf \rvert\rvert_2, \lvert\lvert hf_x \rvert\rvert_2\right)
    \left(\lvert\lvert hf_x \rvert\rvert_2\right)^{\mathrm{min}\left\{ \mf+1,\pf+\qf+1 \right\}}
    \end{equation}
    donde $C_{\mf,k}^{\pf,\qf}(\vartheta,\varLambda)=\frac{2\vartheta\me{\varLambda}}{(\mf+1)(\mf+1)!}+
    \alpha\vartheta(1+\hf_{\mf+1,\mf}) 2^{-k(\pf+\qf)+3}\me{(1+\alpha(\frac{1}{2})^{\pf+\qf-3})\varLambda }$, 
    con $\alpha=\frac{\pf!\qf!}{(\pf+\qf)!(\pf+\qf+1)!}$.
\end{lemma}
\emph{Demostraci\'on} \\
Por desigualdad triangular,
\begin{eqnarray}\label{importatdeq}
\left\lvert\left\lvert  h\phi_1(hf_x)f-  K_{\mf,k}^{\pf,\qf}\left( h,  f_x , f \right)\right\rvert\right\rvert_2
&\leq&\left\lvert\left\lvert  h\phi_1(hf_x)f -  K_{\mf}\left( h,f_x , f \right) \right\rvert\right\rvert_2 \\ 
&+&\left\lvert\left\lvert  K_{\mf}\left( h,f_x , f \right) - 
K_{\mf,k}^{\pf,\qf}\left( h,  f_x , f \right)\right\rvert\right\rvert_2\nonumber
\end{eqnarray}
donde $K_{\mf}\left( h,f_x , f \right)$ es la aproximaci\'on de Krylov de $h\phi_1(hf_x)f$ definida en~(\ref{kaprox}).

Sea $\beta=\lVert hf \rVert_2$, $\lVert V_\mf \rVert_2=1$ donde
 $\overline{H}$ est\'a definido en (\ref{hhat}), $F_k^{\pf,\qf}(A)$ definido en f\'ormula~(\ref{P-MA-2}),
  entonces de las expresiones (\ref{kaprox}) y (\ref{kp_aprox}), y por desigualdad triangular se tiene
 \begin{equation}\label{proffktheouneq} 
 %\begin{array}{ccc}
      \left\lvert\left\lvert  K_{\mf}\left( hf_x , hf \right) -  
 K_{\mf,k}^{\pf,\qf}\left( h,  f_x , f \right)\right\rvert\right\rvert_2 
 \leq \left\lvert\left\lvert T_0 - 
  \widehat{T}_0\right\rvert\right\rvert_2 \\
 +\left\lvert\left\lvert T_1 - \widehat{T}_1
  \right\rvert\right\rvert_2
 %\end{array}
 \end{equation}
 donde 
 \begin{eqnarray*}
T_0&=&\beta V_{\mf}I_{\mf\times (\mf+3)}\me{\overline{H}}e_{\mf+1}\\
\widehat{T}_0&=&\beta V_{\mf}I_{\mf\times (\mf+3)}F_k^{\pf,\qf}(\overline{H})e_{\mf+1}\\
T_1&=&\beta \hf_{\mf+1,\mf}e_\mf^TI_{\mf\times (\mf+3)}\me{\overline{H}}e_{\mf+2}v_{\mf+1}\\
\widehat{T}_1&=&\beta \hf_{\mf+1,\mf}e_\mf^TI_{\mf\times (\mf+3)}F_k^{\pf,\qf}(\overline{H})e_{\mf+2}v_{\mf+1}
 \end{eqnarray*}
 
Para el primer t\'ermino en el miembro derecho de la desigualdad~(\ref{proffktheouneq}), por Teorema~\ref{Conv. Pade} y Lema~\ref{H-bound} se tiene que
\begin{eqnarray}
\left\lvert\left\lvert T_0 - \widehat{T}_0\right\rvert\right\rvert_2  
&\leq& \beta \left\lvert\left\lvert  V_{\mf}I_{\mf\times (\mf+3)} \right\rvert\right\rvert_2 \left\lvert\left\lvert \me{\overline{H}}-F_k^{\pf,\qf}(\overline{H}) \right\rvert\right\rvert_2
\left\lvert\left\lvert  e_{\mf+1} \right\rvert\right\rvert_2 \nonumber\\
&\leq& \beta c_{\pf,\qf}(k,\lvert\lvert\overline{H}\rvert\rvert_2)
 \left(\lvert\lvert\overline{H}\rvert\rvert_2\right)^{\pf+\qf+1}\nonumber\\
&\leq& \beta c_{\pf,\qf}(k,\lvert\lvert hf_x \rvert\rvert_2) 
\left(\lvert\lvert hf_x \rvert\rvert_2\right)^{\pf+\qf+1}\label{err1},
\end{eqnarray}
donde $c_{\pf,\qf}$ est\'a definido in (\ref{errpade}).
Similarmente, para el segundo t\'ermino en el miembro derecho de la desigualdad~(\ref{proffktheouneq}) se tiene que 
\begin{eqnarray}
\left\lvert\left\lvert T_1 - \widehat{T}_1 \right\rvert\right\rvert_2 
&\leq& \beta \lvert \hf_{\mf+1,\mf}\rvert \left\lvert\left\lvert I_{\mf\times (\mf+3)}\right\rvert\right\rvert_2 \left\lvert\left\lvert \me{\overline{H}}-
F_k^{\pf,\qf}(\overline{H}) \right\rvert\right\rvert_2\left\lvert\left\lvert v_{\mf+1}\right\rvert\right\rvert_2 \nonumber\\
&\leq&\beta \lvert \hf_{\mf+1,\mf}\rvert c_{\pf,\qf}
(k,\lvert\lvert\overline{H}\rvert\rvert_2) 
\left(\lvert\lvert\overline{H}\rvert\rvert_2\right)^{\pf+\qf+1}\nonumber\\
&\leq&\beta \lvert \hf_{\mf+1,\mf}\rvert c_{\pf,\qf}(k,\lvert\lvert hf_x \rvert\rvert_2) 
\left(\lvert\lvert hf_x \rvert\rvert_2\right)^{\pf+\qf+1}\label{err2}.
\end{eqnarray}
De (\ref{err1}) y (\ref{err2}) se tiene que
%\begin{eqnarray}
\begin{equation}
\left\lvert\left\lvert  K_{\mf}\left( hf_x , hf \right) -  
K_{\mf,k}^{\pf,\qf}\left( h,  f_x , f \right)\right\rvert\right\rvert_2
%&\leq&\beta c_{\pf,\qf}(k,\lvert\lvert hf_x \rvert\rvert_2) 
%\left(\lvert\lvert hf_x \rvert\rvert_2\right)^{\pf+\qf+1} \nonumber\\
%&+& \beta \lvert \hf_{\mf+1,\mf}\rvert c_{\pf,\qf}(k,\lvert\lvert hf_x \rvert\rvert_2) 
%\left(\lvert\lvert hf_x \rvert\rvert_2\right)^{\pf+\qf+1}\nonumber\\
\leq \beta(1+\lvert\hf_{\mf+1,\mf}\rvert)c_{\pf,\qf}(k,\lvert\lvert hf_x \rvert\rvert_2) 
\left(\lvert\lvert hf_x \rvert\rvert_2\right)^{\pf+\qf+1}\label{errp1}
\end{equation}
%\end{eqnarray}
De las desigualdades~(\ref{importatdeq}) y (\ref{errp1}), y del  Teorema~\ref{CORRECTED-ERROR} se tiene que
\begin{eqnarray*}
\left\lvert\left\lvert  h\phi_1(hf_x)f-  K_{\mf,k}^{\pf,\qf}\left( h,  f_x , f \right)\right\rvert\right\rvert_2
&\leq& \frac{2\beta \left( \vert\lvert hf_x\rvert\rvert_2 \right)^{\mf+1}
    \me{\lvert\lvert hf_x\rvert\rvert_2}}{(\mf+1)(\mf+1)!}\\
&+&
\beta(1+\hf_{\mf+1,\mf})c_{\pf,\qf}(k,\lvert\lvert hf_x \rvert\rvert_2) 
\lvert\lvert hf_x \rvert\rvert_2^{\pf+\qf+1}\\
& \leq &  C_{\mf,k}^{\pf,\qf}\left(\lvert\lvert hf \rvert\rvert_2, \lvert\lvert hf_x \rvert\rvert_2\right)
\left(\lvert\lvert hf_x \rvert\rvert_2\right)^{\mathrm{min}\left\{ \mf+1,\pf+\qf+1 \right\}}
\end{eqnarray*}
donde  $C_{\mf,k}^{\pf,\qf}$ est\'a definida como en el enunciado
del Lema. $\Box$
\newline

Finalmente, el siguiente teorema establece el orden de convergencia de los esquemas embebidos LLDP.

\begin{theorem}\label{Teorema Convergencia}
    Sea $x$ la soluci\'on de la EDO (\ref{ODE-SYST}) con campo vectorial $f$ que satisface las condiciones
    \begin{equation*}
    f\in \mathcal{C}^{6}(\mathfrak{K}) ,\; \mathfrak{K}\subset \left([t_0,T]\times\mathfrak{D}\right),\; \mathfrak{K}\text{ conjunto compacto}
    \end{equation*}
    \begin{equation*}
    ||f_x(z)||\leq \mathcal{M},\; \forall z\in \mathfrak{K},
    \end{equation*}
    y reescribamos las fórmulas embebidas (\ref{LLDPK scheme}) como 
    \[
    y_{n+1}=y_{n}+\digamma (t_{n},y_{n},h_{n})\text{ \ \ \ \ y \ \ \ \ }\widehat{y}%
    _{n+1}=y_{n}+\widehat{\digamma }(t_{n},y_{n},h_{n}).
    \]
 Entonces las fómulas embebidas (\ref{LLDPK scheme}) tienen error de truncamiento local
    \[\lvert\lvert x(t_{n+1}) - x(t_n) - \digamma(t_n,x(t_n),h_n) \rvert\rvert_2 \leq Kh_n^{\mathrm{min}\left\{ \mf+1,\pf+\qf+1  \right\}}+C_1h_n^{6}  \]
    \[\lvert\lvert x(t_{n+1}) - x(t_n) - \widehat{\digamma }(t_n,x(t_n),h_n) \rvert\rvert_2 \leq Kh_n^{\mathrm{min}\left\{ \mf+1,\pf+\qf+1 \right\}}+C_2h_n^{5}  \]
    y error global
    \[ \lvert\lvert x(t_{n+1}) - y_{n+1} \rvert\rvert_2 \leq M_1h^{\mathrm{min}\left\{ \mf,\pf+\qf,5 \right\}} \]
    \[ \lvert\lvert x(t_{n+1}) - \widehat{y}_{n+1} \rvert\rvert_2 \leq M_2h^{\mathrm{min}\left\{ \mf,\pf+\qf,4 \right\}} \]  
    para todo $t_{n+1},t_n\in(t)_h$ y $h$ suficientemente pequeño, donde  $K,M_1,M_2$ son constantes positivas. 
\end{theorem}
\emph{Demostraci\'on} 

Los errores de truncamiento locales y los errores globales se obtiene fácilmente de la
cota en el Lema~(\ref{lemma-Krylov}) y de la demostración del Teorema 1 en~\cite{Jimenez14AMC}. $\Box$

\section{Estrategia adaptativa\label{ADAPT-STR}}

Para escribir un c\'odigo que, en cada paso de integración, seleccione automáticamente un tama\~no de paso y una dimensi\'on de Krylov adecuados para obtener un error local con la tolerancia
requerida, una estrategia adaptativa es necesaria. 
Dada la similitud entre las fórmulas embebidas LLRK (\ref{LLDP scheme}) con las fórmulas embebidas (\ref{LLDPK scheme}) y (\ref{LLDPKFJ scheme}) propuestas en esta tesis, la estrategia de selección y control del tamaño del paso para (\ref{LLDP scheme}) pude, en principio, funcionar para las nuevas fórmulas con ciertas modificaciones que tomen en cuenta las especificaciones relativas a las aproximaciones de Krylov. 

A continuación se presentan desglosados varios aspectos a tener en cuenta por construir una estrategia adaptativa apropiada para los nuevos esquemas embebidos (\ref{LLDPK scheme}) y (\ref{LLDPKFJ scheme}).

Denotemos por $ATol$ y $RTol$ las tolerancias absolutas y relativas del esquema numérico, por $h_{min}$ y $h_{max}$ los tamaños de paso mínimo y máximo permisibles. 
  
\subsection{Selecci\'on del tama\~no de paso inicial}\label{inicio}
El tamaño de paso inicial $h_0$ se estima mediante la misma expresión usada para las fórmulas embebidas (\ref{LLDP scheme}) en \cite{Jimenez14AMC}. Esta es
\begin{equation}
h_0 = \minn{ h_{max}, \maxx{ h_{min} ,\Delta} } \label{hcero}
\end{equation}
donde
\[ \Delta = \begin{cases}
\frac{1}{r_h} & \text{si } h_{max}\cdot r_h>1\\
h_{max} & \text{en otro caso}
\end{cases} \]	
con
\[ r_h = \frac{1\mathord{.}25}{RTol^{1/5}}\max_{i=1\ldots d}\left\{\frac{ \left\lvert f^{[i]}(t_0,y_0) \right\rvert }
{\maxx{\left\lvert y^{[i]}_0 \right\rvert ,tr}}\right\} \]
y $tr = \frac{ATol}{RTol}$.

\subsection{Control de la evaluaci\'on del Jacobiano}\label{jaccontrol}
 Evaluar el Jacobiano del campo vectorial es costoso, sobre todo el de las ecuaciones de dimensiones no pequeñas. Por eso
 es necesaria una estrategia para reutilizar el Jacobiano calculado en pasos anteriores siempre que sea posible. Para este fin se
 necesita cuantificar cuanto difiere el Jacobiano que se va a reutilizar del actual. Como las bases de los
 subespacios de Krylov se construyen a partir de potencias del Jacobiano por el campo vectorial de la ecuaci\'on
 diferencial, entonces, una forma de cuantificar esta diferencia en el punto $(t_n,\widetilde{y}_n)$ sería por  
 \begin{equation*}
 	\Delta J = J\cdot f(t_n,\widetilde{y}_n) - J_{old}\cdot f(t_n,\widetilde{y}_n),
 \end{equation*}
donde $J=f_x(t_n,\widetilde{y}_n)$, $J_{old}=f_x(t_i,\widetilde{y}_i),\, i<n,$ es el jacobiano evaluado en el \'ultimo
$(t_i,\widetilde{y}_i)$ donde se acept\'o evaluarlo y $f_{old} = f(t_i,\widetilde{y}_i)$. Sin embargo, la cuantificación de la diferencia relativa 
\begin{equation}\label{jerror}
	\varepsilon_J = \max_{i=1\ldots d}\frac{\nnorm{\Delta J^{[i]}}}
	{\maxx{\nnorm{ \left(J_{old}\cdot f_{old}\right)^{[i]} } ,\frac{ATol}{RTol}}},
\end{equation}
con respecto a las tolerancias del esquema numérico y con respecto a la \'ultima vez que se evalu\'o el Jacobiano es preferible.

Como en la expresi\'on~(\ref{jerror}) no se tiene $J$, porque el Jacobiano no se ha evaluado todav\'ia, entonces 
se va a aproximar $J\cdot f(t_n,\widetilde{y}_n)$, que es una derivada direccional, por el cociente diferencial hacia atr\'as
\begin{equation}\label{backapprox}
 J\cdot f(t_n,\widetilde{y}_n) \approx \frac{f(t_n,\widetilde{y}_n) -
 	f\left(t_n,\widetilde{y}_n-\delta f(t_n,\widetilde{y}_n)\right)}{\delta}
\end{equation}
para $\delta$ suficientemente peque\~no. Esta aproximaci\'on requiere evaluar el campo vectorial
en un punto intermedio, para evitar tener que evaluarlo se utilizara una aproximaci\'on convexa de este punto
\begin{equation}\label{aproxaprox}
  f\left(t_n,\widetilde{y}_n-\delta f(t_n,\widetilde{y}_n)\right) \approx 
  (1-\delta) f(t_n,\widetilde{y}_n)+ \delta \mathtt{K}
\end{equation}
donde $K = \kt_{6} + f(t_{n-1},\widetilde{y}_{n-1}) + f_x(t_{n-1},\widetilde{y}_{n-1})\widetilde{\varphi}_{6} $ y
\[ \kt_6 = f\left( t_{n-1}+h_{n-1}\, , \, \widetilde{y}_{n-1}+\widetilde{\varphi}_6+h_{n-1} \sum_{i=1}^{5}a_{6,i}\kt_i \right) - f(t_{n-1}\, ,\, \widetilde{y}_{n-1}) - f_x(t_{n-1}\, , \, \widetilde{y}_{n-1})\widetilde{\varphi}_6 \]
 es la etapa $6$ del esquema~(\ref{LLDPK scheme}) en el paso anterior. Es importante notar que como $\delta$
es peque\~no el mayor peso de la aproximaci\'on lo tiene el campo vectorial en el punto actual $(t_n,\widetilde{y}_n)$. Sustituyendo~(\ref{backapprox})-(\ref{aproxaprox}) en (\ref{jerror}) se tiene finalmente la medida de diferencia relativa 
\begin{equation}\label{jerroraprox}
\varepsilon_J\approx \max_{i=1\ldots d}\frac{\nnorm{f^{[i]}(t_n,\widetilde{y}_n) - \mathtt{K}^{[i]} }}
{\maxx{\nnorm{ \left(J_{old}\cdot f_{old}\right)^{[i]} } ,\frac{ATol}{RTol}}}.
\end{equation}
De esta forma, solo si $fac \cdot \varepsilon_J \geq 1$ el Jacobiano de $f$ se evalua en el punto $(t_n,y_n)$, donde $fac$ es un factor de seguridad. 

\subsection{Control del tama\~no de paso}\label{hcontrol}
Para obtener la tolerancia deseada al calcular la aproximación de Pad\'e (\ref{P-MA-2}) dentro del proceso de Krylov se recomienda
cambiar el tama\~no de paso $h$ si $\nnorm{\nnorm{ \overline{H} }}_\infty\leq 600$ \cite{Jimenez14AMC}. Como $\overline{H}$ es escalado por $\frac{1}{90}$
en el proceso, entonces es necesario que se cumpla $\nnorm{\nnorm{ \overline{H} }}_\infty\leq 54000$. Se tiene
 $|| \overline{H} ||_2\leq || hf_x ||_2$, $|| \overline{H} ||_\infty\leq \sqrt{\mf}||\overline{H}||_2$, 
 $||hf_x||_2\leq \sqrt{d} ||hf_x||_\infty$
 
\begin{eqnarray*}
	\left\lVert \overline{H} \right\rVert_\infty & \leq & \sqrt{\mf}\left\lVert \overline{H}\right\rVert_2 \\
	& \leq & \sqrt{\mf}\left\lVert hf_x \right\rVert_2 \\
	& \leq & \sqrt{\mf} \sqrt{d} \left\lVert hf_x \right\rVert_\infty \\    
\end{eqnarray*}
De la desigualdad anterior se tiene:
\begin{eqnarray*}
	\left\lVert hf_x \right\rVert_\infty & \leq & \frac{54000}{\sqrt{\mf} \sqrt{d}} \\
	& \Downarrow & \\
	\left\lVert \overline{H} \right\rVert_\infty & \leq 54000\\
	& \Downarrow & \\
	\left\lVert \frac{\overline{H}}{90} \right\rVert_\infty & \leq & 600    
\end{eqnarray*}
Si $\left\lVert hf_x \right\rVert_\infty  \geq  \frac{54000}{\sqrt{\mf} \sqrt{d}}$ el tama\~no de paso debe ser cambiado
$h=\frac{54000}{\sqrt{\mf} \sqrt{d}\left\lVert f_x \right\rVert_\infty}$. Durante la experimentaci\'on se obtuvo que con
una condici\'on m\'as débil se obten\'ian los mismos resultados. Dicha condici\'on es 
$\left\lVert hf_x \right\rVert_\infty \leq 54000$, si no se cumple entonces se cambia $h$ por 
$h=\frac{54000}{\left\lVert f_x \right\rVert_\infty}$.

\subsection{Selecci\'on del orden de la aproximaci\'on de Pad\'e} \label{pade-order}
 La precisión de la aproximación de Padé-$(\pf,\pf)$ con  escalamiento y potenciaci\'on aumenta con el orden $\pf$, pero también aumenta su costo computacional. Por esta razón es necesario controlar el orden $\pf$ de modo que preserve el orden de convergencia de los esquemas embebidos (dado en Teorema \ref{Teorema Convergencia}) y a la vez satisfaga la tolerancia deseada \cite{Jimenez14AMC}.
 Para tolerancias RTol menores que $10^{-9}$,  $\pf = 3$ es el valor óptimo. De lo contrario $\pf$ se escoge según la Tabla \ref{tab:padep}.
\begin{table}[H]
	\begin{center}
		\begin{tabular}{c|ccc} 
			$\lvert\lvert \frac{1}{90}h_n\overline{H} \rvert\rvert/\mathrm{RTol}$ & $10^{-9}$ & $10^{-12}$ & $10^{-15}$ \\ 
			\hline 
			$<1$ & 3 & 4 & 4 \\ 
			$\geq 1$ & 4 & 5 & 6 \\  
		\end{tabular} 
		\caption{Valores \'optimos de $\pf$ para la aproximaci\'on $(\pf,\pf)$-Pad\'e con escalamiento y potenciaci\'on}
		\label{tab:padep}
	\end{center}
\end{table}

\subsection{Selecci\'on  de la dimensi\'on de Krylov}\label{selkrydim}
Luego de construir la base ortogonal para el $\mf$-\'esimo subespacio de Krylov se tienen dos situaciones posibles: $\varepsilon_{K}/\gamma< 1$ o lo contrario, donde $\varepsilon_{K}$ es el error relativo (\ref{errrel}) y $\gamma$ es un factor de seguridad.

En caso de cumplirse que $\varepsilon_{K}/\gamma< 1$ entonces la aproximación (\ref{kp_aprox}) derivada del $\mf$-\'esimo subespacio de Krylov es aceptada y se calcula la dimensión del subespacio que se utilizará en el siguiente paso de integración mediante la fórmula
\begin{equation}
\mathfrak{m}_{new}= \maxx{\mf_{min}, \minn{\mf , \mf_{max} }}  \label{calcmnew}
\end{equation}
donde 	
\begin{equation*}
\mf = \left\lfloor \mf_{n} + \maxx{\fac_{max} , \minn{\fac_{min}, 
		\fac\cdot\Delta\mf} } \right\rfloor, 
\end{equation*}	
\begin{equation}
\Delta\mf=\log(\varepsilon_{K}/\gamma) \label{delta_m} 
\end{equation} 
$\fac_{max}= -\frac{\mf_n}{4}$, $\fac_{min}= \frac{\mf_n}{3}$ y $\fac=\frac{1}{\log(2)}$. En el otro caso, $\mf$ se calcula por    
\begin{equation*}
\mf = \left\lceil \mf_{n} + \minn{\fac_{max} , \maxx{\fac_{min}, 
		\fac\cdot\Delta\mf} } \right\rceil, 
\end{equation*}
donde $\fac_{max}= \maxx{ 1,\frac{\mf_{n}}{3} }$, $\fac_{min}= 1 $ y $\fac=\frac{1}{\log(2)}$.


\subsection{Control del \emph{BreakDown}}\label{brcontrol}
El \emph{BreakDown} en Algoritmo~\ref{alg:Arnoldi} (l\'inea $9$) ocurre cuando hay p\'erdida de ortogonalidad
en el algoritmo de Arnoldi por acumulaci\'on de errores num\'ericos. Por tanto el algoritmo termina antes de
alcanzar la iteración $\mf$ deseada y el \'ultimo vector de la base que se estaba construyendo es inservible. Cuando
esto ocurre pueden suceder dos desenlaces: 1) que $\varepsilon_{K}/\gamma< 1$, o 2) que $\varepsilon_{K}/\gamma \geq 1$, donde $\varepsilon_{K}$ es el error relativo (\ref{errrel}) y $\gamma$ es un factor de seguridad. Independientemente del desenlace, la información aportada por este
proceso fallido se va a reutilizar para los nuevos estimados.

 En el caso que $\varepsilon_{K}/\gamma\geq 1$ las aproximación de Krylov-Padé no cumplen con la tolerancia 
 requerida y son descartadas, por lo que es necesario repetir el proceso de ortogonalización con nuevos valores de $\mf$ y $h$. El nuevo valor de la dimensión del subespacio de Krylov se estima mediante la fórmula 
 \begin{equation}\label{kry-reject-m1}
 \mathfrak{m}_{new}= \maxx{\mf_{min}, \minn{\mf , \mf_{max} }}
 \end{equation}
 \begin{equation*}
 \mf = \left\lceil \mf_{cut} + \minn{\fac_{max} , \maxx{\fac_{min}, 
 		\fac\cdot\Delta\mf} } \right\rceil,
 \end{equation*}
donde $\Delta\mf$ está definida en (\ref{delta_m}), $\fac_{max}= \maxx{ 1,\frac{\mf_{cut}}{3} }$, $\fac_{min}= 1 $, $\fac=\frac{1}{\log(2)}$ 
y $\mf_{cut}$ es la dimensión donde ocurrió el \emph{BrakDown}. El nuevo valor para el tamaño de paso $h_{new}$ es recalculado por \cite{phi}
\begin{equation}
h_{new} = h_n\left(\frac{\gamma^{2}}{\varepsilon_{K}}\right)^{1/(\hat{q}+1)}\label{kry-reject-h1},
\end{equation}
donde $\hat{q}$ es una estimación heurística del orden del proceso de Krylov
si se tiene la información de dos intentos en el mismo paso, en otro caso $\hat{q}=\mf/4$
\begin{equation*}%\label{qestimkri}
\hat{q} = \frac{\log\left( h/h_{old} \right)}{\log\left(\varepsilon_{K}/\varepsilon_{old}  \right)}.
\end{equation*}

En el caso que $\varepsilon_{K}/\gamma< 1$ hay que verificar la dimensión donde se interrumpió
el algoritmo. Como se mencionó previamente el último vector de la base es inservible para ser usado en la aproximación (\ref{kp_aprox}). Por tanto para no violar los límites impuestos a la dimensión de Krylov se tiene que cumplir 
$\mf_{cut}> \mf_{min}+2$ para aceptar dichas aproximaciones. En otro caso se estaría aproximando
en una dimensión menor que la mínima fijada. Por esto si sucede que $\mf_{cut}\leq \mf_{min}+2$, 
entonces $h_n$ es rechazada, y nuevos valores para $h$ y $\mf$ son estimados  por las fórmulas
\begin{equation}
h_{new}=0\mathord{.}9\cdot h_n\label{kry-reject-h2}
\end{equation}
y
 \begin{equation}\label{kry-reject-m2}
\mathfrak{m}_{new}= \maxx{\mf_{min}, \minn{\mf , \mf_{max} }}
\end{equation}
donde 
\begin{equation*}
\mf = \left\lceil \mf_{cut} + \minn{\fac_{max} , \maxx{\fac_{min}, 
		\fac\cdot\Delta\mf} } \right\rceil,
\end{equation*}
$\fac_{max}= \maxx{ 1,\frac{\mf_{cut}}{3} }$, $\fac_{min}= 2 $ y $\fac=\frac{1}{\log(2)}$.


\subsection{Estimaci\'on del nuevo tama\~no de paso}
El nuevo tama\~no de paso $h_{new}$ se calcula mediante la misma expresión usada para las formulas embebidas (\ref{LLDP scheme}) en \cite{Jimenez14AMC}. Esta es
\begin{equation}
	h_{new} = \minn{h_{max},\maxx{h_{min},\Delta}}, \label{hnewcalc}
\end{equation}
donde 
\[ \Delta = \begin{cases}
h_n/\rho & \text{si } \varepsilon_y\leq RTol \text{ y } \rho>0\mathord{.}2\\
5\cdot h_n & \text{si } \varepsilon_y\leq RTol \text{ y } \rho\leq 0\mathord{.}2\\
\maxx{0\mathord{.}1,1/\rho}\cdot h_n & \text{si } \varepsilon_y > RTol \text{ y } fail=0\\
0\mathord{.}5\cdot h_n & \text{si } \varepsilon_y > RTol \text{ y } fail=1
\end{cases} \]
siendo $\rho = 1\mathord{.}25 \left( \frac{\varepsilon_y}{RTol} \right)^{1/5}$, $\varepsilon_y$ una medida del error relativo entre los esquemas $y$ y $\hat y$, y $fail$ indica si un valor anterior de $h_{new}$ calculado a partir de $h_n$ fue rechazado.   

\subsection{Esbozo general de la estrategia adaptativa para esquemas embebidos con paso variable}\label{adaptstrat}

A continuación se presenta un esbozo general de una estrategia adaptativa con tamaño de paso y dimensión de Krylov variable para los esquemas embebidos (\ref{LLDPK scheme}), la cual tiene en cuenta todos los aspectos tratados en las siete subsecciones anteriores.   

\begin{enumerate}
    \item Inicio:
    \begin{itemize}
    	\item Inicializar $n=0$, $fail=0$.
    	\item Calcular $h_0$ por fórmula (\ref{hcero})
		\item Seleccionar $\mf_{min},\mf_{max}$, donde $\mf_{min}$ es la menor dimensión del subespacio de Krylov
		y $\mf_{max}$ es la dimensión máxima y hacer $\mf_0=\mf_{min}$.
		\item Evaluar $f_x(t_0,\widetilde{y}_{0})$. Inicializar $J_{old}=f_x(t_0,\widetilde{y}_{0})$ y $freshJ=1$
    \end{itemize} 

	\item Control de la evaluación del Jacobiano $f_x$ (subsección \ref{jaccontrol}):
	\begin{itemize}
	\item Si $fail=0$ y $n>0$, se calcula $\varepsilon_J$ por 
	f\'ormula~(\ref{jerroraprox}),
	\begin{itemize}
		\item Si $\fac\cdot\varepsilon_J \geq 1$, con $\fac = 0\mathord{.}25$, se eval\'ua $f_x$ en el punto actual ($t_n,\widetilde{y}_{n}$). Entonces,  $J_{old}=f_x(t_n,\widetilde{y}_{n})$ y $freshJ=1$,
		\item Si $\fac\cdot\varepsilon_J<1$ se reutiliza el Jacobiano $J_{old}$. Entonces, $f_x(t_n,\widetilde{y}_{n})=J_{old}$ y $freshJ=0$
	\end{itemize}

	\item Si $fail=1$ y $freshJ=0$, se eval\'ua $f_x$ en el punto actual ($t_n,\widetilde{y}_{n}$). Entonces, $J_{old}=f_x(t_n,\widetilde{y}_{n})$ y $freshJ=1$.

	\end{itemize} 

    \item Control del tama\~no de paso (subsecci\'on \ref{hcontrol}):

    Si $||f_xh_n||_\infty>54000$ entonces $h_n$ es recalculada $h_n=\frac{54000}{||f_x||_\infty}$
    
    \item Construir un subespacio de Krylov de dimensión $\mf_n$ mediante Algoritmo~\ref{alg:Arnoldi},
    seleccionar el orden $\pf$ de Pad\'e (subsección \ref{pade-order}) según las tolerancias y estimar el error relativo $\varepsilon_{K}$ por la f\'ormula~(\ref{errrel}).

    \item Control del error $\varepsilon_{K}$:
    \begin{itemize}
        \item si $breakdown$ es $false$
        \begin{itemize}
            \item Si $\varepsilon_{K}/\gamma< 1$, con $\gamma=0\mathord{.}1$, se acepta $\mf_n$.
            \item Si $\varepsilon_{K}/\gamma\geq 1$ y $\mf_n<\mf_{max}$,  entonces $\mf_{new}$ es calculada por la f\'ormula (\ref{calcmnew}), se extiende el actual subespacio de Krylov hasta la dimensión $\mf_{new}$ 
            mediante Algoritmo~\ref{alg:Arnoldi},
            se selecciona el orden $\pf$ de Pad\'e (subsección \ref{pade-order}) según las tolerancias, se reestima el error relativo $\varepsilon_{K}$ por la f\'ormula~(\ref{errrel}), y se retorna al paso 5 con $\mf_n=\mf_{new}$.
            \item Si $\varepsilon_{K}/\gamma\geq 1$ y $\mf_n=\mf_{max}$,  entonces $h_n$ es rechazada, $h_{new}$ es calculada por la fórmula (\ref{kry-reject-h1}), $\mf_{new}$ es calculada por la 
            fórmula (\ref{calcmnew}) y se regresa al paso 4 con $h_n=h_{new}$ y $\mf_n=\mf_{new}$.
        \end{itemize}
         \item si $breakdown$ es $true$ (subsección \ref{brcontrol}):
        \begin{itemize}
            \item Si $\varepsilon_{K}/\gamma< 1$ y $\mf_{cut}> \mf_{min}+2$,  se acepta $\mf_n$.
            
            \item Si $\varepsilon_{K}/\gamma< 1$ y $\mf_{cut}\leq \mf_{min}+2$, entonces $h_n$ es rechazada, $h_{new}$ es calculada por la fórmula (\ref{kry-reject-h2}), $\mf_{new}$ es calculada por la 
            fórmula (\ref{kry-reject-m2}) y se regresa al paso 4 con $h_n=h_{new}$ y $\mf_n=\mf_{new}$.
            
            \item Si $\varepsilon_{K}/\gamma\geq 1$, entonces $\mf_{new}$ es calculada por la f\'ormula (\ref{kry-reject-m1}), $h_{new}$ es calculada por la fórmula (\ref{kry-reject-h1})
            y se regresa al paso 4 con $h_n=h_{new}$ y $\mf_n=\mf_{new}$.
        \end{itemize}
    \end{itemize}


    \item Estimar la  nueva dimensión de Krylov $\mf_{new}$ por la fórmula (\ref{calcmnew})

    \item Calcular $c_jh_n\phi_1(c_jh_nf_x)f$ por la f\'ormula (\ref{kp_aprox})

    \item Evaluar los esquemas embebidos (\ref{LLDPK scheme})

    \item Estimaci\'on del error relativo
    \[ \varepsilon_y =  \max_{i=1\ldots d}\left\{ \frac{\left\lvert \widetilde{y}^{[i]}_{n+1}-\hat{y}^{[i]}_{n+1} \right\rvert}
    {\maxx{\left\lvert \widetilde{y}^{[i]}_{n+1}  \right\rvert,\left\lvert\widetilde{y}^{[i]}_{n}\right\rvert,tr}} \right\} \]

    \item Estimar el nuevo tama\~no de paso $h_{new}$ por fórmula (\ref{hnewcalc})   

    \item Validaci\'on  de $\widetilde{y}_{n+1}$:\\
    Si $\varepsilon_y<RTol$ se acepta $\widetilde{y}_{n+1}$ como aproximaci\'on de $x$ en $t_{n+1}=t_n+h_n$. En caso contrario, se retorna al paso 2 con $h_n=h_{new}$, $\mf_n = \mf_{new}$ y $fail = 1$.

    \item Control del paso final:\\
    Si $t_n+h_n=T$, se detiene. Si $t_n+h_n+h_{new}>T$ entonces se redefine $h_{new}=T-(t_n+h_n)$.

    \item Regresar al paso 2 con $n=n+1$, $h_n=h_{new}$, $\mf_n = \mf_{new}$ y $fail=0$.
    
\end{enumerate}
Para los esquemas embebidos (\ref{LLDPKFJ scheme}) libres de Jacobiano, la estrategia adaptativa es similar a la anterior pero con los siguientes cambios: \
\begin{itemize}
	\item Se eliminan los pasos $2$ y $3$, y la última inicialización del paso 1.
	\item El paso $7$ se sustituye por: Calcular $c_jh_n\phi_1(c_jh_nf_x)f$ por f\'ormula (\ref{kp_fj_aprox}).
	\item El paso $8$ se sustituye por: Evaluar la f\'ormula~(\ref{LLDPKFJ scheme})
\end{itemize}

\subsection{Esbozo general de la estrategia adaptativa para esquemas con paso fijo y dimensi\'on de Krylov variable}\label{adaptstratfix}

Se propone la siguiente estrategia adaptativa para el esquema de paso fijo $h$ y dimensi\'on de Krylov variable construido con la f\'ormula~(\ref{LLDPK scheme}) de orden $r(=4,5)$. 

\begin{enumerate}
    \item Inicio:
    \begin{itemize}
        \item Inicializar $n=0$.
        \item Seleccionar $\mf_{min},\mf_{max}$, donde $\mf_{min}$ es la menor dimensión del subespacio de Krylov
        y $\mf_{max}$ es la dimensión máxima y hacer $\mf_0=\mf_{min}$.
    \end{itemize} 
    
    \item Construir un subespacio de Krylov de dimensión $\mf_n$ mediante Algoritmo~\ref{alg:Arnoldi},
    seleccionar el orden $\pf$ de Pad\'e (subsección \ref{pade-order}) según las tolerancias y estimar el error relativo $\varepsilon_{K}$ por la f\'ormula~(\ref{errrel}).
    
    \item Control del error $\varepsilon_{K}$:
    \begin{itemize}

            \item Si $\varepsilon_{K}/\gamma< 1$, con $\gamma=0\mathord{.}1$, se acepta $\mf_n$.
            \item Si $\varepsilon_{K}/\gamma\geq 1$ y $\mf_n<\mf_{max}$ entonces $\mf_{new}$ es calculada por la f\'ormula (\ref{kry-reject-m1}), se extiende
             el actual subespacio de Krylov hasta la dimensión $\mf_{new}$ mediante Algoritmo~\ref{alg:Arnoldi},
            se selecciona el orden $\pf$ de Pad\'e (subsección \ref{pade-order}) según las tolerancias, se reestima el error relativo $\varepsilon_{K}$ por la f\'ormula~(\ref{errrel}), y se retorna al paso 3 con $\mf_n=\mf_{new}$.
            \item Si $\varepsilon_{K}/\gamma\geq 1$ y $\mf_n=\mf_{max}$, se acepta $\mf_n$.
    \end{itemize}
    
    
    \item Estimar la  nueva dimensión de Krylov $\mf_{new}$ por la fórmula (\ref{calcmnew})
    
    \item Calcular $c_jh_n\phi_1(c_jh_nf_x)f$ por la f\'ormula (\ref{kp_aprox})
    
    \item Evaluar la f\'ormula~(\ref{LLDPK scheme}) de orden $r$
    
    \item Control del paso final:\\
    Si $t_n+h_n=T$, se detiene. Si $t_n+h_n+h_{new}>T$ entonces se redefine $h_{new}=T-(t_n+h_n)$.
    
    \item Regresar al paso 2 con $n=n+1$, $\mf_n = \mf_{new}$.
    
\end{enumerate}
Para el esquema de paso fijo y dimensi\'on de Krylov variable construido con la
f\'ormula~(\ref{LLDPKFJ scheme}) de orden $r$, la estrategia adaptativa es 
similar a la anterior pero con los siguientes cambios: \
\begin{itemize}
    \item El paso $5$ se sustituye por: Calcular $c_jh_n\phi_1(c_jh_nf_x)f$ por f\'ormula (\ref{kp_fj_aprox}).
    \item El paso $6$ se sustituye por: Evaluar la f\'ormula~(\ref{LLDPKFJ scheme}) de orden $r$.
\end{itemize}

