\chapter{C\'{a}lculo de la exponencial de una  matriz}\label{chapter:EXPM-Krylov}

En este capitulo se presentan los resultados básicos sobre el c\'{a}lculo de la exponencial matricial que se requieren para esta tesis.

\begin{definition}
    \label{EXPM}\cite{Golub96} Sea $A$ una matriz de $n\times n$ real o compleja. La exponencial de $A$ denotada por 
    $ \me{A} $ o $\mathrm{exp}(A)$ es una matriz de $n\times n$ dada por la serie de potencias
    \[\me{A}=\sum_{k=0}^{\infty}\frac{1}{k!}A^{k}.\]
\end{definition}
En muchas aplicaciones se utiliza $\me{tA}$ donde $t$ es un escalar, por tanto 
\begin{equation}
\me{tA}=I+tA+\frac{t^{2}A^{2}}{2!}+\cdots\label{EXPM-SERIE}
\end{equation}
\begin{theorem}\cite{IntroMatrix}
    La serie de matrices definida en~(\ref{EXPM-SERIE}) existe para toda matriz $A$ para $t$ fijo y
    para todo $t$ para $A$ fija. La serie converge uniformemente en cualquier regi\'on de t del plano complejo.
\end{theorem}

En \cite{VanLoan03} se presenta una revisi\'on de los m\'etodos num\'ericos
para aproximar
la exponencial de matrices. Los m\'etodos m\'as utilizados son la aproximaci\'on de Pad\'e con escalamiento
y potenciaci\'on para matrices peque\~nas y densas, y el de los subespacios de Krylov para matices grandes y esparcidas. 

\section{Aproximaci\'on de Pad\'e}

\begin{definition}
    \cite{Golub96} La aproximaci\'{o}n racional de Pad\'{e} $P_{\pf,\qf}(z)$
    de $\me{z}$ est\'{a} dada por 
    \begin{equation*}
    P_{\pf,\qf}(z)=\dfrac{R_{\pf,\qf}(z)}{Q_{\pf,\qf}(z)},  \label{FORM PADE}
    \end{equation*}%
    donde 
    \[
    R_{\pf,\qf}(z)=1+\dfrac{\pf}{\pf+\qf}z+\dfrac{\pf(\pf-1)}{(\pf+\qf)(\pf+\qf-1)}\dfrac{z^{2}}{2!}%
    +\ldots +\frac{\pf(\pf-1)\ldots 1}{(\pf+\qf)(\pf+\qf-1)\ldots (\qf+1)}\dfrac{z^{\pf}}{\pf!}
    \]%
    y 
    \[
    Q_{\pf,\qf}(z)=R_{\qf,\pf}(-z).
    \]
\end{definition}

Cuando la norma de $A$ es grande, el costo computacional  y los errores de redondeo aumentan, por lo que
es necesaria la alternativa de escalamiento y potenciaci\'on.

\begin{definition}\cite{Golub96}~
    Sea $A$ una matriz cuadrada, y sea $P_{\pf,\qf}(2^{-k}A%
    )$ la aproximaci\'{o}n de Pad\'{e} de orden $(\pf,\qf)\ $de $\me{2^{-k}A%
    }$, donde $k$ es el menor número natural tal que $\left\vert 2^{-k}A%
    \right\vert \leq \frac{1}{2}$. \ Se define la aproximación de  Pad\'{e}-$(\pf,\qf)$ con escalamiento y potenciación como 
    \begin{equation}
    F_{k}^{\pf,\qf}(A)=(P_{\pf,\qf}(2^{-k}A))^{2^{k}}.
    \label{P-MA-2}
    \end{equation}
\end{definition}

Resultados sobre el error y la estabilidad de las aproximaciones de Padé se presentan a continuación.

\begin{theorem}
    \label{Conv. Pade}\cite{errorpade} El error
    relativo y absoluto de la aproximaci\'on de Pad\'{e} con escalamiento y
    potenciaci\'on (\ref{P-MA-2}) de $\me{A}$ est\'{a} dado por: 
    \[
    \frac{\left\vert e^{A}-F_{k}^{\pf,\qf}(A)\right\vert 
    }{\left\vert e^{A}\right\vert }\leq \epsilon _{p,q}\left\vert 
    A\right\vert e^{\epsilon _{p,q}\left\vert A\right\vert }
    \]%
    y 
    \begin{equation}
    \left\vert \me{A}-F_{k}^{\pf,\qf}(A)\right\vert \leq
    c_{p,q}(k,\left\vert A\right\vert )\left\vert A\right\vert
    ^{p+q+1}, \label{errpade}
    \end{equation}
    donde $\epsilon _{p,q}=\alpha (\frac{1}{2})^{p+q-3}$, $%
    c_{p,q}(k,\left\vert A\right\vert )=\alpha
    2^{-k(p+q)+3}\me{(1+\epsilon _{p,q})\left\vert A\right\vert }$ y $%
    \alpha =\frac{p!q!}{(p+q)!(p+q+1)!}$.
\end{theorem}

\begin{theorem}\label{Stab. Pade}[Teorema 4.12 en \cite{Hairer-Wanner96}] 
    La aproximaci\'on de Pad\'e $P_{\pf,\qf}(z)$  es \emph{A-estable} si y solo si $\pf\leq \qf\leq \pf+2$. 
    Todos los ceros y todos los polos son simples.
\end{theorem}

\section{Aproximación de Krylov\label{section:Krylov-sp}}
Los estudios muestran que la aproximaci\'on de Pad\'e con escalamiento y potenciaci\'on
es efectiva para calcular exponenciales de matrices relativamente peque\~nas y densas; pero no as\'i para matrices de dimensiones moderadas o grandes.

Es conocido del \'Algebra Lineal Num\'erica que para problemas de
grandes dimensiones los m\'etodos iterativos son preferidos por sobre los m\'etodos directos.
En particular,
los m\'etodos iterativos de los subespacios de Krylov han cobrado auge en la resoluci\'on de
problemas de \'Algebra Lineal que involucran el producto de una funci\'on matricial y un vector,
tales como la resoluci\'on de grandes sistemas lineales y en el c\'alculo de valores
propios~\cite{Golub96}.

Es importante destacar que en la mayor\'ia de los integradores exponenciales
 no se necesita el c\'alculo de la matriz $\me{A}$ completa, sino
el producto $\me{A}b$ donde $b$ es un vector determinado. Por ejemplo, la soluci\'on del problema de valor inicial homog\'eneo $\dot{x}(t)=Ax(t),\;x(0)=x_0$, es $x(t)=\me{At}x_0$ en forma de producto exponencial de matriz y vector. Esta es la raz\'on por la que
los m\'etodos de los subespacios de Krylov han sido ampliamente utilizados para evaluar productos de esta forma.

\begin{definition}
        \cite{hoch97} Para $\mf\geq 1$, el $\mf$-\'esimo subespacio de Krylov de la matriz $A\in\mathbb{C}^{N\times N}$
    y el vector $b\in\mathbb{C}^{N}$ se define como
    \[ \mathcal{K}_\mf(A,b)=\mathrm{span}\left\{ b,Ab,A^{2},b,\ldots,A^{\mf-1}b \right\}. \]
\end{definition}

%\textcolor{red}{?donde se usa?: Este subespacio es invariante ante traslaci\'on $\mathcal{K}_m(A,b)=\mathcal{K}_m(A-\gamma I,b),\gamma\in\mathbb{C}$. Mas adelante se usa: invarianza ante escalado del algoritmo de Arnoldi}

%\begin{theorem}
%   Sea $p_{A,b}^{\mathrm{min}}$ polinomio de grado m\'inimo de $A$ respecto a $b$ y sea $\mu=deg\left( p_{A,b}^{\mathrm{min}} \right)$ el
% \'indice de invarianza. Para $\mf\geq \mu$ los subespacios de Krylov son invariantes por A y $\mathcal{K}_\mf(A,b)=\mathcal{K}_\mu(A,b)$
%   \[ \mathcal{K}_1(A,b) \subsetneq \mathcal{K}_2(A,b) \subsetneq  \cdots \subsetneq \mathcal{K}_\mu(A,b)=\mathcal{K}_{\mu+1}(A,b)=\cdots \]
%\end{theorem}
%El \'indice de invarianza suele ser grande, por lo que se elige una aproximaci\'on en un subespacio de orden $m\leq \mu$ utilizando la
%compresi\'on de $A$ en $\mathcal{K}_\mf(A,b)$. 

Antes de definir la aproximaci\'on por subespacios de Krylov del producto de una funci\'on matricial y un vector es 
necesario construir una base ortonormal. Para construir dicha base se utiliza el algoritmo de
Arnoldi~\ref{alg:Arnoldi} para matrices generales y el algoritmo de 
Lanczos para matrices hermíticas~\cite{arnoldi,saad2003iterative}.

\begin{algorithm}
    \caption{Algoritmo de Arnoldi para construir una base ortonormal del subespacio de Krylov $\mathcal{K}_\mf(\tau A,b)$}
    \label{alg:Arnoldi}
    \KwIn{Matriz $A\in \mathbb{R}^{d\times d}$, vector $b\in \mathbb{R}^{d}$ y constante $\tau$}
    \KwOut{Base ortonormal $\{v_1,v_2,\ldots,v_\mf,v_{\mf+1}\}$ de $K_\mf(\tau A,b)$ y matriz de Hessenberg $H_\mf=V_\mf^{\intercal}\tau A V_\mf $,
        donde $V_{\mf+1}=[v_1\,v_2\,\cdots \,v_\mf\,v_{\mf+1}]\in \mathbb{R}^{d\times \mf+1}$, $breakdown$ }
    $breakdown=false$\\
    $v_1=b/\lVert b \rVert_2$\\
    \For{ $j=1,2,\ldots,\mf$ }{
        $w_j=\tau A v_j$\\
        \For{ $i=1,\ldots,j$}{ 
            $\hf_{ij}=\langle w_j,v_i \rangle$\\
            $w_j=w_j-\hf_{ij}v_i$       
        }
        $\hf_{j+1,i}=\lVert w_j \rVert_2$\\
        \eIf(\tcp*[h] \emph{Break Down}\label{alg:breakdown}){$\hf_{j+1,i}<2\epsilon_{mach}$}{
            $\mf_{cut}=j$\\
             $breakdown=true$\\
            Stop
        }{
            $v_{j+1}=w_j/\hf_{j+1,i}$
        }
    }
\end{algorithm}


\begin{definition}
    \cite{hoch97} Sea $V_\mf$ una base ortonormal de $\mathcal{K}_\mf(A,b)$ y sea $f$ una funci\'on matricial definida
    en el espectro de $A$. La aproximaci\'on de $f(A)b$ por el  $\mf$-\'esimo subespacio de Krylov se define
    \begin{equation}
        f(A)b\approx ||b||_2 V_\mf f(H_\mf)e_1, \label{MFUNC-APROX}
    \end{equation}
    donde $H_\mf=V_\mf^{\intercal}AV_\mf$ y $e_1$ es el primer vector canónico de $\mathbb{R}^\mf$.
    
\end{definition}

Cuando la funci\'on $f$ en~(\ref{MFUNC-APROX}) es la exponencial matricial se obtiene el siguiente resultado.
\begin{theorem}\label{exp-bound}
	\cite{Saad92} Sea $A\in\mathbb{C}^{N\times N}$ una matriz. El error de aproximaci\'on de $\me{A}b$ por el $\mf$-\'esimo subespacio
	de Krylov es
	\begin{equation*}
	\left\lvert\left\lvert \me{A}b - \lvert\lvert b \rvert\rvert_2 V_\mf\me{H_\mf}e_1 \right\rvert\right\rvert_2 
	\leq \frac{2\vert\lvert b \rvert\rvert_2 \left( \sigma(A) \right)^{\mf}\me{\sigma(A)}}{\mf!},
	\end{equation*}
	donde $\sigma(A)$ es el radio espectral de $A$.
\end{theorem}

Cuando la funci\'on $f$ en~(\ref{MFUNC-APROX}) es la función $\phi_k$ definida en (\ref{DEF-PHI}) se obtiene el siguiente resultado.
\begin{theorem}
    \cite{Saad92,Sidje98}~Sea $A\in\mathbb{C}^{N\times N}$ una matriz. El error de aproximaci\'on de $\phi_k(A)b$ por el $\mf$-\'esimo subespacio de Krylov es
    \begin{equation}
        ||\phi_k(A)b - ||b||_2 V_\mf \phi_k(H_\mf)e_1||_2  \leq ||b||_2\left\lvert\hf_{\mf+1,\mf} e_\mf^T\phi_{k+1}(H_\mf)e_1\right\rvert\left\lvert\left\lvert v_{\mf+1}\right\rvert\right\rvert_2.
    \end{equation}
\end{theorem}

En general, para la familia de funciones $\phi_k$ se establece la expansión en serie \cite{Saad92,Sidje98}
\begin{equation}
\phi_k(\tau A)b=||b||_2\tau^{k} V_\mf \phi_k(\tau H_\mf)e_1 + ||b||_2\hf_{\mf+1,\mf}
\sum_{j=k+1}^{\infty}\tau^{j} e_\mf^T\phi_j(\tau H_\mf)e_1A^{j-k-1}v_{m+1} \label{PHI-EXPANSION}
\end{equation}
donde $\tau$ es un número positivo y $e_\mf$ es el  $\mf$-ésimo vector canónico de $\mathbb{R}^\mf$.


