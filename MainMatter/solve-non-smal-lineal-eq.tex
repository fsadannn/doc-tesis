\chapter{Solución de ecuaciones lineales de dimensiones no pequeñas}\label{chapter:solve-non-smal-lineal-eq}

En el capítulo anterior, se presentó la discretización Lineal Local~(\ref{definition LLS}) , así como su implementación basada en Padé~(\ref{LL-scheme}) para pequeños PVI. Para PVI de medianas y grandes dimensiones la aproximación de Padé(\ref{section:pade-approx}) para $\varphi$ ya no es computacionalmente eficiente y necesita ser reemplazada por alguna aproximación basada en subspacios de Krylov o alguna técnica proyectiva similar. Con este propósito, utilizando el Teorema \ref{theorem:exp-phi}, la función $\varphi_j=L\me{c_jh_nM_n}r$ definida en la sección \ref{section:ll-methods} puede reescribirse como
\begin{equation*}
    \phi_j=c_jh_n\varphi_1(c_jh_nf_x(t_n,y_n))f(t_n,y_n)
\end{equation*}
donde $\varphi_1$ está definida en (\ref{phi-definition}).

\todo[inline]{mejorar oración}
En este capítulo se introducen las aproximaciones Krylov-Padé para exponeciales matriciales, así como dichas aproximaciones sin evaluar el Jacobiano. También se derivan cotas para las aproximaciones propuestas.

\section{Aproximaciones Krylov-Padé}

 La familia de operadors $\phi$ tiene un papel fundamental en las aproximaciones Localmente Linealizadas, por esto es necesario poder evaluar estos operadores de forma rápida y precisa. Además, como estos operadores aparecen actuando sobre un vector, entonces se aproxima la acción sobre el vector. La expansión en serie~(\ref{PHI-EXPANSION}) permite relacionar la acción de un operador sobre un vector por un subspacio de Krylov. Truncando la serie~(\ref{PHI-EXPANSION}) diferentes aproximaciones son obtenidas. Típicamente~\cite{niesen2009krylov,sidje1998expokit,tokman2006efficient} el primer término de la serie es tomado y el segundo término es utilizado como medida del error; pero, en este trabajo, similar a~\cite{Saad92} se tomarán los dos primeros términos de la serie (\ref{PHI-EXPANSION})
 \begin{equation}\label{exact-two-terms-km}
    K_\mf(\tau,A,b) = \norme{b}\tau V_\mf\phi_1(\tau H_\mf)e_1 + \norme{b}\tau^2\hf_{\mf+1,\mf}e^T_\mf\phi_2(\tau H_\mf)e_1v_{\mf+1}
 \end{equation}
para aproximar $\tau\phi_1(\tau A)b$ y el tercer término como medida del error.
-
Al aplicar el algoritmo de Arnoldi (\ref{alg:Arnoldi}), la propiedad de invarianza ante escalado de la base ortonormal del subspacion y el Teorema \ref{theorem:exp-phi} podemos calcular las $\phi_j$ necesarias realizando una exponencial; del Algoritmo (\ref{alg:Arnoldi}) se obtiene la matrix de Hessenberg $H^*\mf$ para $\mathcal{K}_\mf(h_n f_x,f)$. Escalando esta matriz se obtiene $H_\mf=H^*\mf/h_n$ para $\mathcal{K}_\mf(f_x,f)$ y al realizar la exponencial de la matriz particionada
\begin{equation*}
    \overline{H} = \left[\begin{array}{cccc}
    H_\mf & e_1 & 0_{\mf\times 1} & 0_{\mf\times 1}\\
    0_{1\times\mf} & 0 & 1 & 0\\
    0_{1\times\mf} & 0 & 0 & 1\\
    0_{1\times\mf} & 0 & 0 & 0
    \end{array}\right] \label{hhat_hessenberg_matrix}
    \end{equation*}
se obtiene de aplicar el Teorema \ref{theorem:exp-phi}
\begin{equation}
    \me{\tau\overline{H}} = \left[\begin{array}{cccc}
    \me{\tau H_m} & \tau\phi_1(\tau H_m)e_1 & \tau^{2}\phi_2(\tau H_m)e_1 &
    \tau^{3}\phi_3(\tau H_m)e_1 \\
    & 1 & \tau & \frac{\tau^{2}}{2}\\
    &  & 1 & \tau \\
    &   &   & 1 \\
    \end{array}\right] \,. \label{phi_hhat_exponential}
\end{equation}

Se denotará por $E_\tau = \me{\tau \overline{H} }$ y $[E_\tau]_{ij}$ a las particiones de la matriz $E_\tau$. Al aplicar (\ref{phi_hhat_exponential}) en (\ref{exact-two-terms-km}) se obtiene
\begin{equation*}
    K_\mf(\tau,A,b) = \norme{b}\tau V_\mf [E_\tau]_{12} + \norme{b}\tau^2\hf_{\mf+1,\mf}e^T_\mf[E_\tau]_{13}v_{\mf+1} \, .
 \end{equation*}

La exponencial (\ref{phi_hhat_exponential}) no puede ser calculada de forma exacta en el caso general, por lo que se utilizará los aproximantes de Padé para aproximarla. La exponencial matricial es aproximada por
\begin{equation*}
    \widetilde{E}_{\tau} = F_k^{\pf,\qf}\left(\tau\overline{H}\right),
\end{equation*}
donde $F_k^{\pf,\qf}$~(\ref{P-MA-2}) denota la aproximación de Padé con estrategia escalamiento y potenciación de la función exponencial. Consecuentemente la aproximación (\ref{exact-two-terms-km}) quedaría
\begin{equation} \label{eq:kp_aprox}
    K_{\mf,k}^{\pf,\qf}(\tau,A,b) = \norme{b}\tau V_\mf [\widetilde{E}_\tau]_{12} + \norme{b}\tau^2\hf_{\mf+1,\mf}e^T_\mf[\widetilde{E}_\tau]_{13}v_{\mf+1} \, .
 \end{equation}

 \subsection{Cotas para aproximaciones Krylov-Padé}
 En está sección se enunciará un teorema para acotar el error de la (\ref{eq:kp_aprox}) en función de tamaño de paso $h$, la dimensión del subespacio de Krylov $\mf$ y los órdenes de Padé $\pf$ y $\qf$. Para poder enunciar dicho teorema dos lemas adicionales son necesarios. El primer lema extiende para $\tau \phi_1(\tau A)b$ los resultados del Teorema 4.7 en~\cite{Saad92} para aproximaciones de $\phi_0(A)b$ mediante subespacios de Krylov. El segundo lema ofrece una cota para la matrix $\overline{H}$.

 \begin{lemma}\cite{naranjo2021locally}\label{lemma:CORRECTED-ERROR}
	Sea $A\in\mathbb{C}^{d\times d}$ una matrix, $b\in\mathbb{C}^{d}$ un vector, $\tau$ un número positivo, y \[{K_{\mf}\left(\tau,A , b \right) = \tau ||b||_2 V_\mf \phi_1(\tau H_\mf)e_1}+\tau^2 ||b||_2 \hf_{\mf+1,\mf}e_\mf^T\phi_2\left(\tau H_\mf\right) e_1 v_{\mf+1} \] la aproximación a $\tau \phi_1(\tau A)b$ tomando los dos primeros términos de la serie (\ref{PHI-EXPANSION}). Entonces,
	\begin{gather*}
	\left\lvert\left\lvert \tau\phi_1(\tau A)b - K_{\mf}\left(\tau,A , b \right) \right\rvert\right\rvert_2%\nonumber\\
	\leq \frac{2\vert\lvert b \rvert\rvert_2\tau^{\mf+2}\rho^{\mf+1}\me{\tau \rho}}{(\mf+2)!},
	\end{gather*}
	donde $\rho=\nnorm{\nnorm{A}}_2$.
\end{lemma}

\emph{Demostración}
Sigiendo la definión (\ref{phi-definition}), $\tau \phi_1(\tau A)b$ puede ser reecrito como
\begin{eqnarray*}
	\tau \phi_1(\tau A)b &=& \tau b+\tau^{2}A\phi_{2}(\tau A)b \\
	&=& \tau b+\tau^{2}A \,\ (\,\ \nnorm{\nnorm{b}}_2V_\mf\phi_{2}(\tau H_\mf)e_1 +s_\mf(\tau A) \,\ ), %\label{eq:tauaphi1tauab}
\end{eqnarray*}
donde \[ s_\mf (\tau A)= \phi_{2}(\tau A)b -\nnorm{\nnorm{b}}_2V_\mf\phi_{2}(\tau H_\mf)e_1 .\]
De la reescritura de $\phi$, utilizando $AV_\mf=V_\mf H_\mf + \hf_{\mf+1,\mf}v_{\mf+1}e^{T}_\mf$ y  $\nnorm{\nnorm{b}}_2V_\mf e_1=b$ se obtiene
\begin{equation}
\tau\phi_1(\tau A)b = \tau\nnorm{\nnorm{b}}_2V_\mf\phi_1(\tau H_\mf)e_1 + \tau^{2}\nnorm{\nnorm{b}}_2\hf_{\mf+1,\mf}e^T_\mf\phi_{2}(\tau H_\mf)e_1v_{\mf+1}+\tau^{2}As_\mf (\tau A), %\label{eq:tauaphi1tauabapprox}
\end{equation}
y se tiene que
\begin{equation}
\tau\phi_1(\tau A)b - K_{\mf}\left(\tau,A , b \right) = \tau^{2}As_\mf(\tau A), \label{eq:phidiffasm}
\end{equation}
donde \[{K_{\mf}\left(\tau,A , b \right) = \tau ||b||_2 V_\mf \phi_1(\tau H_\mf)e_1}+\tau^2 ||b||_2 \hf_{\mf+1,\mf}e_\mf^T\phi_2\left(\tau H_\mf\right) e_1 v_{\mf+1} \] es una aproximación de $\tau \phi_1(\tau A)b$.

Utilizando el Lema 4.1 en \cite{Saad92} se tiene
\begin{equation}
s_\mf(\tau A)=\nnorm{\nnorm{b}}_2 ( r_\mf(\tau A)v_1-V_\mf r_\mf(\tau H_\mf)e_1) ,\label{eq:smeq}
\end{equation}
donde $r_\mf(z) = \phi_{2}(z)-p_{2,\mf-1}(z)$, and $p_{2,\mf-1}$ es un polinomio de grado $\mf-1$. Tomando
\begin{equation*}
p_{2,\mf-1}(z)\equiv \frac{p_{1,\mf}(z)-1}{z},
\end{equation*}
donde $p_{1,\mf}$ es la expansion de Taylor de $\phi_1$ hasta el orden  $\mf$
definida por
\[ p_{1,\mf}(z)= \frac{p_{0,\mf+1}(z)-1}{z}, \]
donde $p_{0,\mf+1}(z)$  es la expansion de Taylor de $\me{z}$ hasta el orden $\mf+1$ dada por
\[ p_{0,m+1}(z)=\sum\limits^{\mf+1}_{j=0}\frac{z^j}{j!}. \]

Con estos polinomios y (\ref{phi-definition}) se tiene
\begin{equation*}
r_\mf(z) = \frac{\phi_1(z)-1}{z}-\frac{p_{1,\mf}(z)-1}{z}=\frac{\me{z}-1}{z^2}-\frac{p_{0,m+1}(z)-1}{z^2}
=\frac{\me{z}-p_{0,m+1}(z)}{z^2}.
\end{equation*}

El Lema 4.2 en \cite{Saad92} enuncia que
\[ \nnorm{\me{z}-p_{0,m+1}(z)} \leq \frac{z^{\mf+2}\me{z}}{(\mf+2)!}\;\;, \]
de esto se obtiene
\begin{equation}
\nnorm{\nnorm{r_\mf(\tau A)v_1}}_2\leq \frac{\rho^{\mf}\me{\rho}}{(\mf+2)!}\;\;\;\;\; and \;\;\;\;\; \nnorm{\nnorm{r_\mf(\tau H_\mf)v_1}}_2\leq \frac{\widehat{\rho}^{\mf}\me{\widehat{\rho}}}{(\mf+2)!}, \label{polinom}
\end{equation}
donde $\rho=\nnorm{\nnorm{\tau A}}_2 $ y  $\widehat{\rho}=\nnorm{\nnorm{\tau H_\mf}}_2$. Ya que $\nnorm{\nnorm{H_\mf}}\leq\nnorm{\nnorm{A}}$, de (\ref{eq:smeq}) y (\ref{polinom}) se obtiene
\begin{equation*}
\nnorm{\nnorm{\tau As_\mf(\tau A)}}_2  \leq \frac{2 \nnorm{\nnorm{b}}_2\rho^{\mf+1}\me{\rho}}{(\mf+2)!}.
\end{equation*}
Finalmente, de estas desigualdades and (\ref{eq:phidiffasm}) se obtiene
\begin{equation*}
\nnorm{\nnorm{\tau\phi_1(\tau A)b - K_{\mf}\left(\tau,A , b \right)}}_2  \leq
\frac{2 \nnorm{\nnorm{b}}_2\tau^{\mf+2}\nnorm{\nnorm{A}}_2^{\mf+1}\me{\tau \nnorm{\nnorm{A}}_2}}{(\mf+2)!},
\end{equation*}
lo cual completa la prueba. $\Box$

\begin{lemma}\cite{naranjo2021locally}\label{H-bound}
	Sea $\overline{H}$ la matrix definida en in~(\ref{hhat_hessenberg_matrix}), sea  $H_\mf=V^{T}_\mf AV_\mf$ la matrix de Hessenberg asociada al subespacio de Krylov $\mathcal{K}_\mf(A,b)$ con base ortonormal $V_m$. Entonces
	\[ \lVert\overline{H}\rVert_2 \leq 1 +  \lVert A\rVert_2. \]
\end{lemma}
\emph{Demostración}
De (\ref{hhat_hessenberg_matrix}) se tiene
\begin{eqnarray*}
	\overline{H}&=&L H_\mf L^T + B
\end{eqnarray*}
donde $ L^T=[I_\mf \;\; 0_{\mf\times 3}] $ y
\[B=\left[\begin{array}{ccc}
0_{\mf\times\mf} & e_1 & 0_{\mf \times 2}\\
0_{2\times\mf} & 0_{2 \times 1} & I_2\\
0_{1\times\mf} & 0 & 0_{1\times 2}
\end{array}\right].\]
Ya que $\lVert B\rVert_2 = \lVert L \rVert_2 = 1$, y $\lVert H_m\rVert_2 \le \lVert A \rVert_2$, se obtiene que
\begin{eqnarray*}
	\lVert\overline{H}\rVert_2 &\leq& \lVert L \rVert_2 \lVert H_\mf \rVert_2 \lVert L^T \rVert_2 + \lVert B \rVert_2\\
	&\leq& 1 + \lVert A\rVert_2
\end{eqnarray*}
$\Box$\\ \\


\begin{theorem}\cite{naranjo2021locally}\label{theorem:Krylov-bound}
	Sea 
	\begin{equation} \label{eq:gen_kp_aprox}
	K_{\mf,k}^{\pf,\qf}\left(\tau, A , b \right)=\nnorm{\nnorm{b}}_2 V_{\mf}\;[\widetilde{E}_{\tau}]_{12} + \nnorm{\nnorm{b}}_2 \hf_{\mf+1,\mf}e_\mf^T\;[\widetilde{E}_{\tau}]_{13} v_{\mf+1}
	\end{equation}
	la aproximación $(\mf , \pf ,\qf , k)$-Krylov-Padé de $\tau \phi_1(\tau A)b$, donde las matrices $V_{\mf}\in\mathbb{R}^{d\times \mf}$ y $H_{\mf}\in\mathbb{R}^{{\mf} \times {\mf}}$, el vector $v_{\mf+1}$, y el número $\hf_{\mf+1,\mf}$ son obtenidos del algoritmo de Arnoldi para el $\mf$-ésimo subespacio de Krylov $\mathcal{K}_\mf(A,b)$;  $\widetilde{E}_{\tau}=F_k^{\pf,\qf}\left(\tau\overline{H}\right)$ es la aproximación $(\pf,\qf)$-Padé con escalamiento $k$ para la exponencial matricial (\ref{phi_hhat_exponential}), y $e_m$ el $m$-ésimo vector canónico de $\mathbb{R}^\mf$.
	Entonces,
	\begin{equation}
	\left\lvert\left\lvert  \tau \phi_1(\tau A)b -
	K_{\mf,k}^{\pf,\qf}\left( \tau, A , b \right)\right\rvert\right\rvert_2%\nonumber\\
	\leq C_{\mf,k}^{\pf,\qf}\left(\lvert\lvert A \rvert\rvert_2\right) \;
	\nnorm{\nnorm{b}}_2 \; \tau^{\mathrm{min}\left\{ \mf+2,\pf+\qf+1 \right\}}
	\end{equation}
	donde $C_{\mf,k}^{\pf,\qf}(\varLambda)= \frac{2 \varLambda^{m+1} \me{\varLambda}}{(\mf+2)!}+
	\alpha (1+\hf_{\mf+1,\mf}) (1+\varLambda^{p+q+1}) 2^{-k(\pf+\qf)+3}\me{(1+\alpha(\frac{1}{2})^{\pf+\qf-3})(1+\varLambda)} $,
	con $\alpha=\frac{\pf!\qf!}{(\pf+\qf)!(\pf+\qf+1)!}$ and $\tau \in [0,1]$.
\end{theorem}
\textbf{Demostración}. Por desigualdad triangular
\begin{eqnarray} \label{importatdeq}
\left\lvert\left\lvert  \tau \phi_1(\tau A)b -
K_{\mf,k}^{\pf,\qf}\left( \tau , A , b \right)\right\rvert\right\rvert_2
& \leq & \left\lvert\left\lvert \tau \phi_1(\tau A)b -  K_{\mf}\left( \tau , A , b \right) \right\rvert\right\rvert_2 \\
& & + \left\lvert\left\lvert  K_{\mf}\left( \tau , A , b \right) -
K_{\mf,k}^{\pf,\qf}\left( \tau, A , b \right)\right\rvert\right\rvert_2, \nonumber
\end{eqnarray}
donde
\begin{equation*}
K_{\mf}\left(\tau,A , b \right)=\nnorm{\nnorm{b}}_2 \tau V_{\mf}\phi_1\left( \tau H_{\mf} \right) e_1 + \nnorm{\nnorm{b}}_2 \tau^{2}\hf_{\mf+1,\mf}e_\mf^T\phi_2\left(\tau H_\mf\right) e_1 v_{\mf+1}
\end{equation*}
es la aproximación de Krylov de $\tau \phi_1(\tau A)b$ definida en~(\ref{eq:kp_aprox}), y
\begin{equation*}
K_{\mf,k}^{\pf,\qf}\left(\tau, A , b \right)=\nnorm{\nnorm{b}}_2 V_{\mf}\;[\widetilde{E}_{\tau}]_{12} + \nnorm{\nnorm{b}}_2 \hf_{\mf+1,\mf}e_\mf^T\;[\widetilde{E}_{\tau}]_{13} v_{\mf+1}
\end{equation*}
la aproximación Krylov-Padé de $\tau \phi_1(\tau A)b$ definida en~(\ref{eq:gen_kp_aprox}).

De (\ref{phi_hhat_exponential}) se tiene que
$\tau^j \phi_j\left( \tau H_{\mf} \right) e_1 = L \me{\tau \overline{H}} r_j$
y
$[\widetilde{E}_{\tau}]_{1,j+1} = L F_k^{\pf,\qf}(\tau \overline{H}) r_j$,
con $j=1,2$, donde $F_k^{\pf,\qf}(\tau \overline{H})$ es la aproximación $(\pf,\qf)$-Padé de $\me{\tau \overline{H}}$ con estrategia escalamiento y potenciación,
$ L=[I_\mf \;\; 0_{\mf\times 3}] $, $r_1=[0_{1\times \mf}\;\; 1 \;\;0 \;\;0]^T$ y $r_2=[0_{1\times \mf}\;\; 0 \;\;1 \;\;0]^{T}$. De la desigualdad triangular
\begin{equation}\label{proffktheouneq}
%\begin{array}{ccc}
\left\lvert\left\lvert   K_{\mf}\left( \tau , A , b \right) -
K_{\mf,k}^{\pf,\qf}\left( \tau, A , b \right)\right\rvert\right\rvert_2
\leq \left\lvert\left\lvert T_1 -
\widetilde{T}_1\right\rvert\right\rvert_2 \\
+\left\lvert\left\lvert T_2 - \widetilde{T}_2
\right\rvert\right\rvert_2,
%\end{array}
\end{equation}
donde
\begin{eqnarray*}
	T_1&=&\nnorm{\nnorm{b}}_2 V_{\mf} L \me{\tau \overline{H}} r_1\\
	\widetilde{T}_1&=&\nnorm{\nnorm{b}}_2 V_{\mf} L F_k^{\pf,\qf}(\tau \overline{H}) r_1\\
	T_2&=&\nnorm{\nnorm{b}}_2 \hf_{\mf+1,\mf}e_\mf^T\;L \me{\tau \overline{H}} r_2 v_{\mf+1}\\
	\widetilde{T}_2&=&\nnorm{\nnorm{b}}_2 \hf_{\mf+1,\mf}e_\mf^T\;L F_k^{\pf,\qf}(\tau \overline{H}) r_2 v_{\mf+1}.
\end{eqnarray*}

Del Lema 4.1 en \cite{jimenez2012convergence} y el Lema~\ref{H-bound}, se tiene que
\begin{eqnarray}
\left\lvert\left\lvert T_1 - \widetilde{T}_1\right\rvert\right\rvert_2
&\leq& \nnorm{\nnorm{b}}_2 \left\lvert\left\lvert  V_{\mf} \right\rvert\right\rvert_2 \left\lvert\left\lvert  L \right\rvert\right\rvert_2 \left\lvert\left\lvert \me{\tau  \overline{H}}-F_k^{\pf,\qf}(\tau \overline{H}) \right\rvert\right\rvert_2
\left\lvert\left\lvert  r_1 \right\rvert\right\rvert_2 \nonumber\\
&\leq& \nnorm{\nnorm{b}}_2 \;  c_{\pf,\qf}(k,\lvert\lvert\tau\overline{H}\rvert\rvert_2) \;
\lvert\lvert\tau \overline{H}\rvert\rvert_2^{\pf+\qf+1}\nonumber\\
&\leq& \nnorm{\nnorm{b}}_2 \; c_{\pf,\qf}(k,\tau (1+\lvert\lvert A \rvert\rvert_2))
\; (1 + \lvert\lvert A \rvert\rvert_2^{\pf+\qf+1}) \; \tau^{\pf+\qf+1}
\label{err1}
\end{eqnarray}
y
\begin{eqnarray}
\left\lvert\left\lvert T_2 - \widetilde{T}_2 \right\rvert\right\rvert_2
&\leq& \nnorm{\nnorm{b}}_2 \lvert \hf_{\mf+1,\mf}\rvert \left\lvert\left\lvert  e_\mf^T \right\rvert\right\rvert_2 \left\lvert\left\lvert  L \right\rvert\right\rvert_2 \left\lvert\left\lvert \me{\tau \overline{H}}-
F_k^{\pf,\qf}(\tau \overline{H}) \right\rvert\right\rvert_2  \left\lvert\left\lvert  r_2 \right\rvert\right\rvert_2 \left\lvert\left\lvert v_{\mf+1}\right\rvert\right\rvert_2 \nonumber\\
&\leq&\nnorm{\nnorm{b}}_2 \lvert \hf_{\mf+1,\mf}\rvert \; c_{\pf,\qf}
(k,\lvert\lvert\tau \overline{H}\rvert\rvert_2) \;
\lvert\lvert\tau \overline{H}\rvert\rvert_2^{\pf+\qf+1}\nonumber\\
&\leq& \nnorm{\nnorm{b}}_2 \lvert \hf_{\mf+1,\mf}\rvert \; c_{\pf,\qf}(k,\tau (1+\lvert\lvert A \rvert\rvert_2))
\; (1 + \lvert\lvert A \rvert\rvert_2^{\pf+\qf+1}) \; \tau^{\pf+\qf+1}\label{err2},
\end{eqnarray}
donde $c_{\pf,\qf}(k,\vartheta )=\alpha
2^{-k(\pf+\qf)+3}\me{(1+\alpha(\frac{1}{2})^{\pf+\qf-3})\vartheta }$ y $%
\alpha =\frac{\pf!\qf!}{(\pf+\qf)!(\pf+\qf+1)!}$. sustituyendo (\ref{err1}) y (\ref{err2}) en (\ref{proffktheouneq}), se obtiene que
%\begin{eqnarray}
\begin{equation}
\left\lvert\left\lvert   K_{\mf}\left( \tau , A , b \right) -
K_{\mf,k}^{\pf,\qf}\left( \tau, A , b \right) \right\rvert\right\rvert_2
\leq \nnorm{\nnorm{b}}_2 (1+\lvert\hf_{\mf+1,\mf}\rvert) \; c_{\pf,\qf}(k,\tau (1+\lvert\lvert A \rvert\rvert_2))
\; (1 + \lvert\lvert A \rvert\rvert_2^{\pf+\qf+1}) \; \tau^{\pf+\qf+1}. \label{errp1}
\end{equation}
%\end{eqnarray}

Finalmente, de las desigualdades~(\ref{importatdeq}) y (\ref{errp1}), el Lema \ref{lemma:CORRECTED-ERROR}, y la condición $\tau \in [0,1]$ se obtiene
\begin{eqnarray*}
	\left\lvert\left\lvert  \tau \phi_1(\tau A)b-  K_{\mf,k}^{\pf,\qf}\left( \tau,  A , b \right)\right\rvert\right\rvert_2
	&\leq& \frac{2 \nnorm{\nnorm{b}}_2 \tau^{\mf+2}  \vert\lvert A\rvert\rvert^{\mf+1}_2
		\me{\tau \lvert\lvert A\rvert\rvert_2}}{(\mf+2)!}\\
	&+&
	\nnorm{\nnorm{b}}_2 (1+\lvert\hf_{\mf+1,\mf}\rvert) \; c_{\pf,\qf}(k,\tau (1+\lvert\lvert A \rvert\rvert_2))
	\; (1 + \lvert\lvert A \rvert\rvert_2^{\pf+\qf+1}) \; \tau^{\pf+\qf+1}\\
	& \leq & \nnorm{\nnorm{b}}_2 C_{\mf,k}^{\pf,\qf}\left(\lvert\lvert A \rvert\rvert_2\right)\tau^{\mathrm{min}\left\{ \mf+2,\pf+\qf+1 \right\}},
\end{eqnarray*}
lo cual concluye la prueba.
$\Box$

\section{Aproximaciones Krylov-Padé libre de Jacobiano}

Por construcción, la discretización discretización Lineal Local~(\ref{definition LLS}), contiene productos de la matriz jacobiana por un vector. Tradicionalmente, en los casos que no es posible evaluar or almacenar la matriz jacobiana o es muy difícil obtenerla, los products de dicha matriz por un vector son reemplazados por aproximaciones~(ver~\cite{steihaug1979attempt,schmitt1995matrix,weiner1997rowmap,hochbruck1998exponential,hosseini1999matrix,tranquilli2014rosenbrock}), preferiblemente diferencias finitas. Aunque estas aproximaciones son útiles en la práctica, estudios teóricos y simulaciones muestran que los esquemas resultantes sufren de pérdida de precisión y dismibuciónb del orden de convergencia (ver~\cite{wanner1996solving,hochbruck1998exponential,tranquilli2014rosenbrock})


By construction, the mentioned numerical schemes involve the computation of products of Jacobian matrix times vector and, in some of them, the whole or partial evaluation of Jacobian matrices to form preconditioner matrices when are required. Routinely, in the case that evaluating and storing Jacobian matrices is unfeasible or when the exact Jacobian matrix is difficult to obtain - due to the insufficient capability of the available computational resources or to the use of complex spatial discretizations - the mentioned products of Jacobian matrix times vector are replaced by approximations (see, e.g., \cite{Steihaug79,Schmitt95,Weiner97,Hochbruck98,Hosseini99,Tranquilli14_SIAM}), preferably, by finite differences. Although this procedure is useful in practice, theoretical and simulation studies have shown that the resulting Jacobian-free schemes may suffer from a loss of precision and order of convergence (see, e.g., \cite{Hairer96,hochbruck1998exponential,tranquilli2014rosenbrock}). 

In the previous section, a new family of Jacobian-free methods was introduced for integrating both, small and large systems of IVPs. With the purpose of design Jacobian-free schemes for large systems of IVPs, in this section, we will adapt the Krylov-Padé approximation to the phi-function times vector $\tau \varphi_1(\tau A)b$ introduced in \cite{Naranjo-Noda21}, for the operator $\varphi_1(z)=(e^z-z)/z$, squared matrix $A$, vector $b$ and $\tau>0$. With this aim, we will use an important result from \cite{brown1987local} that relates the classic Arnoldi algorithm specified in Algorithm \ref{alg:Arnoldi} with the Matrix-free Arnoldi algorithm listed in Algorithm \ref{alg:iArnoldi}.



\section{Aproximaciones de ecuaciones no autónomas}
