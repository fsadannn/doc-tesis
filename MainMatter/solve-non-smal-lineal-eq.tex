\chapter{Solución de ecuaciones lineales de dimensiones no pequeñas}\label{chapter:solve-non-smal-lineal-eq}

En el capítulo anterior, se presentó la discretización Lineal Local~(\ref{definition LLS}) , así como su implementación basada en Padé~(\ref{LL-scheme}) para pequeños PVI. Para PVI de medianas y grandes dimensiones la aproximación de Padé(\ref{section:pade-approx}) para $\varphi$ ya no es computacionalmente eficiente y necesita ser reemplazada por alguna aproximación basada en subspacios de Krylov o alguna técnica proyectiva similar. Con este propósito, utilizando el Teorema \ref{theorem:exp-phi}, la función $\varphi_j=L\me{c_jh_nM_n}r$ definida en la sección \ref{section:ll-methods} puede reescribirse como
\begin{equation*}
    \phi_j=c_jh_n\varphi_1(c_jh_nf_x(t_n,y_n))f(t_n,y_n)
\end{equation*}
donde $\varphi_1$ está definida en (\ref{phi-definition}).

% \todo[inline]{mejorar oración}
En este capítulo se introducen las aproximaciones Krylov-Padé para exponeciales matriciales, así como dichas aproximaciones sin evaluar el Jacobiano. También se derivan cotas para las aproximaciones propuestas.

\section{Aproximaciones Krylov-Padé}\label{section:krylov-pade-approx}

 La familia de operadors $\phi$ tiene un papel fundamental en las aproximaciones Localmente Linealizadas, por esto es necesario poder evaluar estos operadores de forma rápida y precisa. Además, como estos operadores aparecen actuando sobre un vector, entonces se aproxima la acción sobre el vector. La expansión en serie~(\ref{PHI-EXPANSION}) permite relacionar la acción de un operador sobre un vector por un subspacio de Krylov. Truncando la serie~(\ref{PHI-EXPANSION}) diferentes aproximaciones son obtenidas. Típicamente~\cite{niesen2009krylov,sidje1998expokit,tokman2006efficient} el primer término de la serie es tomado y el segundo término es utilizado como medida del error; pero, en este trabajo, similar a~\cite{Saad92} se tomarán los dos primeros términos de la serie (\ref{PHI-EXPANSION})
 \begin{equation}\label{exact-two-terms-km}
    K_\mf(\tau,A,b) = \norme{b}\tau V_\mf\phi_1(\tau H_\mf)e_1 + \norme{b}\tau^2\hf_{\mf+1,\mf}e^T_\mf\phi_2(\tau H_\mf)e_1v_{\mf+1}
 \end{equation}
para aproximar $\tau\phi_1(\tau A)b$ y el tercer término como medida del error.
-
Al aplicar el algoritmo de Arnoldi (\ref{alg:Arnoldi}), la propiedad de invarianza ante escalado de la base ortonormal del subspacion y el Teorema \ref{theorem:exp-phi} podemos calcular las $\phi_j$ necesarias realizando una exponencial; del Algoritmo (\ref{alg:Arnoldi}) se obtiene la matrix de Hessenberg $H^*\mf$ para $\mathcal{K}_\mf(h_n f_x,f)$. Escalando esta matriz se obtiene $H_\mf=H^*\mf/h_n$ para $\mathcal{K}_\mf(f_x,f)$ y al realizar la exponencial de la matriz particionada
\begin{equation*}
    \overline{H} = \left[\begin{array}{cccc}
    H_\mf & e_1 & 0_{\mf\times 1} & 0_{\mf\times 1}\\
    0_{1\times\mf} & 0 & 1 & 0\\
    0_{1\times\mf} & 0 & 0 & 1\\
    0_{1\times\mf} & 0 & 0 & 0
    \end{array}\right] \label{hhat_hessenberg_matrix}
    \end{equation*}
se obtiene de aplicar el Teorema \ref{theorem:exp-phi}
\begin{equation}
    \me{\tau\overline{H}} = \left[\begin{array}{cccc}
    \me{\tau H_m} & \tau\phi_1(\tau H_m)e_1 & \tau^{2}\phi_2(\tau H_m)e_1 &
    \tau^{3}\phi_3(\tau H_m)e_1 \\
    & 1 & \tau & \frac{\tau^{2}}{2}\\
    &  & 1 & \tau \\
    &   &   & 1 \\
    \end{array}\right] \,. \label{phi_hhat_exponential}
\end{equation}

Se denotará por $E_\tau = \me{\tau \overline{H} }$ y $[E_\tau]_{ij}$ a las particiones de la matriz $E_\tau$. Al aplicar (\ref{phi_hhat_exponential}) en (\ref{exact-two-terms-km}) se obtiene
\begin{equation*}
    K_\mf(\tau,A,b) = \norme{b}\tau V_\mf [E_\tau]_{12} + \norme{b}\tau^2\hf_{\mf+1,\mf}e^T_\mf[E_\tau]_{13}v_{\mf+1} \, .
 \end{equation*}

La exponencial (\ref{phi_hhat_exponential}) no puede ser calculada de forma exacta en el caso general, por lo que se utilizará los aproximantes de Padé para aproximarla. La exponencial matricial es aproximada por
\begin{equation*}
    \widetilde{E}_{\tau} = F_k^{\pf,\qf}\left(\tau\overline{H}\right),
\end{equation*}
donde $F_k^{\pf,\qf}$~(\ref{P-MA-2}) denota la aproximación de Padé con estrategia escalamiento y potenciación de la función exponencial. Consecuentemente la aproximación (\ref{exact-two-terms-km}) quedaría
\begin{equation} \label{eq:kp_aprox}
    K_{\mf,k}^{\pf,\qf}(\tau,A,b) = \norme{b}\tau V_\mf [\widetilde{E}_\tau]_{12} + \norme{b}\tau^2\hf_{\mf+1,\mf}e^T_\mf[\widetilde{E}_\tau]_{13}v_{\mf+1} \, .
 \end{equation}

 \subsection{Cotas para aproximaciones Krylov-Padé}
 En está sección se enunciará un teorema para acotar el error de la (\ref{eq:kp_aprox}) en función de tamaño de paso $h$, la dimensión del subespacio de Krylov $\mf$ y los órdenes de Padé $\pf$ y $\qf$. Para poder enunciar dicho teorema dos lemas adicionales son necesarios. El primer lema extiende para $\tau \phi_1(\tau A)b$ los resultados del Teorema 4.7 en~\cite{Saad92} para aproximaciones de $\phi_0(A)b$ mediante subespacios de Krylov. El segundo lema ofrece una cota para la matrix $\overline{H}$.

 \begin{lemma}\cite{naranjo2021locally}\label{lemma:CORRECTED-ERROR}
	Sea $A\in\mathbb{C}^{d\times d}$ una matrix, $b\in\mathbb{C}^{d}$ un vector, $\tau$ un número positivo, y \[{K_{\mf}\left(\tau,A , b \right) = \tau ||b||_2 V_\mf \phi_1(\tau H_\mf)e_1}+\tau^2 ||b||_2 \hf_{\mf+1,\mf}e_\mf^T\phi_2\left(\tau H_\mf\right) e_1 v_{\mf+1} \] la aproximación a $\tau \phi_1(\tau A)b$ tomando los dos primeros términos de la serie (\ref{PHI-EXPANSION}). Entonces,
	\begin{gather*}
	\left\lvert\left\lvert \tau\phi_1(\tau A)b - K_{\mf}\left(\tau,A , b \right) \right\rvert\right\rvert_2%\nonumber\\
	\leq \frac{2\vert\lvert b \rvert\rvert_2\tau^{\mf+2}\rho^{\mf+1}\me{\tau \rho}}{(\mf+2)!},
	\end{gather*}
	donde $\rho=\nnorm{\nnorm{A}}_2$.
\end{lemma}

\emph{Demostración}
Sigiendo la definión (\ref{phi-definition}), $\tau \phi_1(\tau A)b$ puede ser reecrito como
\begin{eqnarray*}
	\tau \phi_1(\tau A)b &=& \tau b+\tau^{2}A\phi_{2}(\tau A)b \\
	&=& \tau b+\tau^{2}A \,\ (\,\ \nnorm{\nnorm{b}}_2V_\mf\phi_{2}(\tau H_\mf)e_1 +s_\mf(\tau A) \,\ ), %\label{eq:tauaphi1tauab}
\end{eqnarray*}
donde \[ s_\mf (\tau A)= \phi_{2}(\tau A)b -\nnorm{\nnorm{b}}_2V_\mf\phi_{2}(\tau H_\mf)e_1 .\]
De la reescritura de $\phi$, utilizando $AV_\mf=V_\mf H_\mf + \hf_{\mf+1,\mf}v_{\mf+1}e^{T}_\mf$ y  $\nnorm{\nnorm{b}}_2V_\mf e_1=b$ se obtiene
\begin{equation}
\tau\phi_1(\tau A)b = \tau\nnorm{\nnorm{b}}_2V_\mf\phi_1(\tau H_\mf)e_1 + \tau^{2}\nnorm{\nnorm{b}}_2\hf_{\mf+1,\mf}e^T_\mf\phi_{2}(\tau H_\mf)e_1v_{\mf+1}+\tau^{2}As_\mf (\tau A), %\label{eq:tauaphi1tauabapprox}
\end{equation}
y se tiene que
\begin{equation}
\tau\phi_1(\tau A)b - K_{\mf}\left(\tau,A , b \right) = \tau^{2}As_\mf(\tau A), \label{eq:phidiffasm}
\end{equation}
donde \[{K_{\mf}\left(\tau,A , b \right) = \tau ||b||_2 V_\mf \phi_1(\tau H_\mf)e_1}+\tau^2 ||b||_2 \hf_{\mf+1,\mf}e_\mf^T\phi_2\left(\tau H_\mf\right) e_1 v_{\mf+1} \] es una aproximación de $\tau \phi_1(\tau A)b$.

Utilizando el Lema 4.1 en \cite{Saad92} se tiene
\begin{equation}
s_\mf(\tau A)=\nnorm{\nnorm{b}}_2 ( r_\mf(\tau A)v_1-V_\mf r_\mf(\tau H_\mf)e_1) ,\label{eq:smeq}
\end{equation}
donde $r_\mf(z) = \phi_{2}(z)-p_{2,\mf-1}(z)$, and $p_{2,\mf-1}$ es un polinomio de grado $\mf-1$. Tomando
\begin{equation*}
p_{2,\mf-1}(z)\equiv \frac{p_{1,\mf}(z)-1}{z},
\end{equation*}
donde $p_{1,\mf}$ es la expansion de Taylor de $\phi_1$ hasta el orden  $\mf$
definida por
\[ p_{1,\mf}(z)= \frac{p_{0,\mf+1}(z)-1}{z}, \]
donde $p_{0,\mf+1}(z)$  es la expansion de Taylor de $\me{z}$ hasta el orden $\mf+1$ dada por
\[ p_{0,m+1}(z)=\sum\limits^{\mf+1}_{j=0}\frac{z^j}{j!}. \]

Con estos polinomios y (\ref{phi-definition}) se tiene
\begin{equation*}
r_\mf(z) = \frac{\phi_1(z)-1}{z}-\frac{p_{1,\mf}(z)-1}{z}=\frac{\me{z}-1}{z^2}-\frac{p_{0,m+1}(z)-1}{z^2}
=\frac{\me{z}-p_{0,m+1}(z)}{z^2}.
\end{equation*}

El Lema 4.2 en \cite{Saad92} enuncia que
\[ \nnorm{\me{z}-p_{0,m+1}(z)} \leq \frac{z^{\mf+2}\me{z}}{(\mf+2)!}\;\;, \]
de esto se obtiene
\begin{equation}
\nnorm{\nnorm{r_\mf(\tau A)v_1}}_2\leq \frac{\rho^{\mf}\me{\rho}}{(\mf+2)!}\;\;\;\;\; and \;\;\;\;\; \nnorm{\nnorm{r_\mf(\tau H_\mf)v_1}}_2\leq \frac{\widehat{\rho}^{\mf}\me{\widehat{\rho}}}{(\mf+2)!}, \label{polinom}
\end{equation}
donde $\rho=\nnorm{\nnorm{\tau A}}_2 $ y  $\widehat{\rho}=\nnorm{\nnorm{\tau H_\mf}}_2$. Ya que $\nnorm{\nnorm{H_\mf}}\leq\nnorm{\nnorm{A}}$, de (\ref{eq:smeq}) y (\ref{polinom}) se obtiene
\begin{equation*}
\nnorm{\nnorm{\tau As_\mf(\tau A)}}_2  \leq \frac{2 \nnorm{\nnorm{b}}_2\rho^{\mf+1}\me{\rho}}{(\mf+2)!}.
\end{equation*}
Finalmente, de estas desigualdades and (\ref{eq:phidiffasm}) se obtiene
\begin{equation*}
\nnorm{\nnorm{\tau\phi_1(\tau A)b - K_{\mf}\left(\tau,A , b \right)}}_2  \leq
\frac{2 \nnorm{\nnorm{b}}_2\tau^{\mf+2}\nnorm{\nnorm{A}}_2^{\mf+1}\me{\tau \nnorm{\nnorm{A}}_2}}{(\mf+2)!},
\end{equation*}
lo cual completa la prueba. $\Box$

\begin{lemma}\cite{naranjo2021locally}\label{H-bound}
	Sea $\overline{H}$ la matrix definida en in~(\ref{hhat_hessenberg_matrix}), sea  $H_\mf=V^{T}_\mf AV_\mf$ la matrix de Hessenberg asociada al subespacio de Krylov $\mathcal{K}_\mf(A,b)$ con base ortonormal $V_m$. Entonces
	\[ \lVert\overline{H}\rVert_2 \leq 1 +  \lVert A\rVert_2. \]
\end{lemma}
\emph{Demostración}
De (\ref{hhat_hessenberg_matrix}) se tiene
\begin{eqnarray*}
	\overline{H}&=&L H_\mf L^T + B
\end{eqnarray*}
donde $ L^T=[I_\mf \;\; 0_{\mf\times 3}] $ y
\[B=\left[\begin{array}{ccc}
0_{\mf\times\mf} & e_1 & 0_{\mf \times 2}\\
0_{2\times\mf} & 0_{2 \times 1} & I_2\\
0_{1\times\mf} & 0 & 0_{1\times 2}
\end{array}\right].\]
Ya que $\lVert B\rVert_2 = \lVert L \rVert_2 = 1$, y $\lVert H_m\rVert_2 \le \lVert A \rVert_2$, se obtiene que
\begin{eqnarray*}
	\lVert\overline{H}\rVert_2 &\leq& \lVert L \rVert_2 \lVert H_\mf \rVert_2 \lVert L^T \rVert_2 + \lVert B \rVert_2\\
	&\leq& 1 + \lVert A\rVert_2
\end{eqnarray*}
$\Box$\\ \\


\begin{theorem}\cite{naranjo2021locally}\label{theorem:Krylov-bound}
	Sea 
	\begin{equation} \label{eq:gen_kp_aprox}
	K_{\mf,k}^{\pf,\qf}\left(\tau, A , b \right)=\nnorm{\nnorm{b}}_2 V_{\mf}\;[\widetilde{E}_{\tau}]_{12} + \nnorm{\nnorm{b}}_2 \hf_{\mf+1,\mf}e_\mf^T\;[\widetilde{E}_{\tau}]_{13} v_{\mf+1}
	\end{equation}
	la aproximación $(\mf , \pf ,\qf , k)$-Krylov-Padé de $\tau \phi_1(\tau A)b$, donde las matrices $V_{\mf}\in\mathbb{R}^{d\times \mf}$ y $H_{\mf}\in\mathbb{R}^{{\mf} \times {\mf}}$, el vector $v_{\mf+1}$, y el número $\hf_{\mf+1,\mf}$ son obtenidos del algoritmo de Arnoldi para el $\mf$-ésimo subespacio de Krylov $\mathcal{K}_\mf(A,b)$;  $\widetilde{E}_{\tau}=F_k^{\pf,\qf}\left(\tau\overline{H}\right)$ es la aproximación $(\pf,\qf)$-Padé con escalamiento $k$ para la exponencial matricial (\ref{phi_hhat_exponential}), y $e_m$ el $m$-ésimo vector canónico de $\mathbb{R}^\mf$.
	Entonces,
	\begin{equation}
	\left\lvert\left\lvert  \tau \phi_1(\tau A)b -
	K_{\mf,k}^{\pf,\qf}\left( \tau, A , b \right)\right\rvert\right\rvert_2%\nonumber\\
	\leq C_{\mf,k}^{\pf,\qf}\left(\lvert\lvert A \rvert\rvert_2\right) \;
	\nnorm{\nnorm{b}}_2 \; \tau^{\mathrm{min}\left\{ \mf+2,\pf+\qf+1 \right\}}
	\end{equation}
	donde $C_{\mf,k}^{\pf,\qf}(\varLambda)= \frac{2 \varLambda^{m+1} \me{\varLambda}}{(\mf+2)!}+
	\alpha (1+\hf_{\mf+1,\mf}) (1+\varLambda^{p+q+1}) 2^{-k(\pf+\qf)+3}\me{(1+\alpha(\frac{1}{2})^{\pf+\qf-3})(1+\varLambda)} $,
	con $\alpha=\frac{\pf!\qf!}{(\pf+\qf)!(\pf+\qf+1)!}$ and $\tau \in [0,1]$.
\end{theorem}
\textbf{Demostración}. Por desigualdad triangular
\begin{eqnarray} \label{importatdeq}
\left\lvert\left\lvert  \tau \phi_1(\tau A)b -
K_{\mf,k}^{\pf,\qf}\left( \tau , A , b \right)\right\rvert\right\rvert_2
& \leq & \left\lvert\left\lvert \tau \phi_1(\tau A)b -  K_{\mf}\left( \tau , A , b \right) \right\rvert\right\rvert_2 \\
& & + \left\lvert\left\lvert  K_{\mf}\left( \tau , A , b \right) -
K_{\mf,k}^{\pf,\qf}\left( \tau, A , b \right)\right\rvert\right\rvert_2, \nonumber
\end{eqnarray}
donde
\begin{equation*}
K_{\mf}\left(\tau,A , b \right)=\nnorm{\nnorm{b}}_2 \tau V_{\mf}\phi_1\left( \tau H_{\mf} \right) e_1 + \nnorm{\nnorm{b}}_2 \tau^{2}\hf_{\mf+1,\mf}e_\mf^T\phi_2\left(\tau H_\mf\right) e_1 v_{\mf+1}
\end{equation*}
es la aproximación de Krylov de $\tau \phi_1(\tau A)b$ definida en~(\ref{eq:kp_aprox}), y
\begin{equation*}
K_{\mf,k}^{\pf,\qf}\left(\tau, A , b \right)=\nnorm{\nnorm{b}}_2 V_{\mf}\;[\widetilde{E}_{\tau}]_{12} + \nnorm{\nnorm{b}}_2 \hf_{\mf+1,\mf}e_\mf^T\;[\widetilde{E}_{\tau}]_{13} v_{\mf+1}
\end{equation*}
la aproximación Krylov-Padé de $\tau \phi_1(\tau A)b$ definida en~(\ref{eq:gen_kp_aprox}).

De (\ref{phi_hhat_exponential}) se tiene que
$\tau^j \phi_j\left( \tau H_{\mf} \right) e_1 = L \me{\tau \overline{H}} r_j$
y
$[\widetilde{E}_{\tau}]_{1,j+1} = L F_k^{\pf,\qf}(\tau \overline{H}) r_j$,
con $j=1,2$, donde $F_k^{\pf,\qf}(\tau \overline{H})$ es la aproximación $(\pf,\qf)$-Padé de $\me{\tau \overline{H}}$ con estrategia escalamiento y potenciación,
$ L=[I_\mf \;\; 0_{\mf\times 3}] $, $r_1=[0_{1\times \mf}\;\; 1 \;\;0 \;\;0]^T$ y $r_2=[0_{1\times \mf}\;\; 0 \;\;1 \;\;0]^{T}$. De la desigualdad triangular
\begin{equation}\label{proffktheouneq}
%\begin{array}{ccc}
\left\lvert\left\lvert   K_{\mf}\left( \tau , A , b \right) -
K_{\mf,k}^{\pf,\qf}\left( \tau, A , b \right)\right\rvert\right\rvert_2
\leq \left\lvert\left\lvert T_1 -
\widetilde{T}_1\right\rvert\right\rvert_2 \\
+\left\lvert\left\lvert T_2 - \widetilde{T}_2
\right\rvert\right\rvert_2,
%\end{array}
\end{equation}
donde
\begin{eqnarray*}
	T_1&=&\nnorm{\nnorm{b}}_2 V_{\mf} L \me{\tau \overline{H}} r_1\\
	\widetilde{T}_1&=&\nnorm{\nnorm{b}}_2 V_{\mf} L F_k^{\pf,\qf}(\tau \overline{H}) r_1\\
	T_2&=&\nnorm{\nnorm{b}}_2 \hf_{\mf+1,\mf}e_\mf^T\;L \me{\tau \overline{H}} r_2 v_{\mf+1}\\
	\widetilde{T}_2&=&\nnorm{\nnorm{b}}_2 \hf_{\mf+1,\mf}e_\mf^T\;L F_k^{\pf,\qf}(\tau \overline{H}) r_2 v_{\mf+1}.
\end{eqnarray*}

Del Lema 4.1 en \cite{jimenez2012convergence} y el Lema~\ref{H-bound}, se tiene que
\begin{eqnarray}
\left\lvert\left\lvert T_1 - \widetilde{T}_1\right\rvert\right\rvert_2
&\leq& \nnorm{\nnorm{b}}_2 \left\lvert\left\lvert  V_{\mf} \right\rvert\right\rvert_2 \left\lvert\left\lvert  L \right\rvert\right\rvert_2 \left\lvert\left\lvert \me{\tau  \overline{H}}-F_k^{\pf,\qf}(\tau \overline{H}) \right\rvert\right\rvert_2
\left\lvert\left\lvert  r_1 \right\rvert\right\rvert_2 \nonumber\\
&\leq& \nnorm{\nnorm{b}}_2 \;  c_{\pf,\qf}(k,\lvert\lvert\tau\overline{H}\rvert\rvert_2) \;
\lvert\lvert\tau \overline{H}\rvert\rvert_2^{\pf+\qf+1}\nonumber\\
&\leq& \nnorm{\nnorm{b}}_2 \; c_{\pf,\qf}(k,\tau (1+\lvert\lvert A \rvert\rvert_2))
\; (1 + \lvert\lvert A \rvert\rvert_2^{\pf+\qf+1}) \; \tau^{\pf+\qf+1}
\label{err1}
\end{eqnarray}
y
\begin{eqnarray}
\left\lvert\left\lvert T_2 - \widetilde{T}_2 \right\rvert\right\rvert_2
&\leq& \nnorm{\nnorm{b}}_2 \lvert \hf_{\mf+1,\mf}\rvert \left\lvert\left\lvert  e_\mf^T \right\rvert\right\rvert_2 \left\lvert\left\lvert  L \right\rvert\right\rvert_2 \left\lvert\left\lvert \me{\tau \overline{H}}-
F_k^{\pf,\qf}(\tau \overline{H}) \right\rvert\right\rvert_2  \left\lvert\left\lvert  r_2 \right\rvert\right\rvert_2 \left\lvert\left\lvert v_{\mf+1}\right\rvert\right\rvert_2 \nonumber\\
&\leq&\nnorm{\nnorm{b}}_2 \lvert \hf_{\mf+1,\mf}\rvert \; c_{\pf,\qf}
(k,\lvert\lvert\tau \overline{H}\rvert\rvert_2) \;
\lvert\lvert\tau \overline{H}\rvert\rvert_2^{\pf+\qf+1}\nonumber\\
&\leq& \nnorm{\nnorm{b}}_2 \lvert \hf_{\mf+1,\mf}\rvert \; c_{\pf,\qf}(k,\tau (1+\lvert\lvert A \rvert\rvert_2))
\; (1 + \lvert\lvert A \rvert\rvert_2^{\pf+\qf+1}) \; \tau^{\pf+\qf+1}\label{err2},
\end{eqnarray}
donde $c_{\pf,\qf}(k,\vartheta )=\alpha
2^{-k(\pf+\qf)+3}\me{(1+\alpha(\frac{1}{2})^{\pf+\qf-3})\vartheta }$ y $%
\alpha =\frac{\pf!\qf!}{(\pf+\qf)!(\pf+\qf+1)!}$. sustituyendo (\ref{err1}) y (\ref{err2}) en (\ref{proffktheouneq}), se obtiene que
%\begin{eqnarray}
\begin{equation}
\left\lvert\left\lvert   K_{\mf}\left( \tau , A , b \right) -
K_{\mf,k}^{\pf,\qf}\left( \tau, A , b \right) \right\rvert\right\rvert_2
\leq \nnorm{\nnorm{b}}_2 (1+\lvert\hf_{\mf+1,\mf}\rvert) \; c_{\pf,\qf}(k,\tau (1+\lvert\lvert A \rvert\rvert_2))
\; (1 + \lvert\lvert A \rvert\rvert_2^{\pf+\qf+1}) \; \tau^{\pf+\qf+1}. \label{errp1}
\end{equation}
%\end{eqnarray}

Finalmente, de las desigualdades~(\ref{importatdeq}) y (\ref{errp1}), el Lema \ref{lemma:CORRECTED-ERROR}, y la condición $\tau \in [0,1]$ se obtiene
\begin{eqnarray*}
	\left\lvert\left\lvert  \tau \phi_1(\tau A)b-  K_{\mf,k}^{\pf,\qf}\left( \tau,  A , b \right)\right\rvert\right\rvert_2
	&\leq& \frac{2 \nnorm{\nnorm{b}}_2 \tau^{\mf+2}  \vert\lvert A\rvert\rvert^{\mf+1}_2
		\me{\tau \lvert\lvert A\rvert\rvert_2}}{(\mf+2)!}\\
	&+&
	\nnorm{\nnorm{b}}_2 (1+\lvert\hf_{\mf+1,\mf}\rvert) \; c_{\pf,\qf}(k,\tau (1+\lvert\lvert A \rvert\rvert_2))
	\; (1 + \lvert\lvert A \rvert\rvert_2^{\pf+\qf+1}) \; \tau^{\pf+\qf+1}\\
	& \leq & \nnorm{\nnorm{b}}_2 C_{\mf,k}^{\pf,\qf}\left(\lvert\lvert A \rvert\rvert_2\right)\tau^{\mathrm{min}\left\{ \mf+2,\pf+\qf+1 \right\}},
\end{eqnarray*}
lo cual concluye la prueba.
$\Box$

\section{Aproximaciones Krylov-Padé libre de Jacobiano}
En general para PVI de grandes dimensiones puede no ser posible la evaluación de la matriz jacobiana $f_x$ del campo vectorial $f$, ya sea por no ser eficiente en términos de memoria u operaciones flotantes. En estos casos algunas aproximaciones libres de Jacobiano~\cite{al2009complex,knoll2004jacobian} son necesarias. Por construcción, la discretización Lineal Local~(\ref{definition LLS}), contiene productos de la matriz jacobiana por un vector  $f_x(y)b$ que pueden ser aproximadas por diferencias finitas de orden 1 or 2. A modo ilustrativo se tomará la diferencia finita hacia adelante
\begin{equation*}
	\frac{F(u+\delta b)-F(u)}{\delta} =  \left(\begin{array}{c}
		\frac{F_1(u_1+\delta b_1,u_2+\delta b_2)-F_1(u_1,u_2)}{\delta}\\
		\frac{F_2(u_1+\delta b_1,u_2+\delta b_2)-F_2(u_1,u_2)}{\delta}\\
	\end{array}\right)
\end{equation*}
para aproximar el producto del jacobiano por un vector \begin{equation*}
	J(u)b = \left[\begin{array}{c}
		b_1\frac{\partial F_1(u_1,u_2)}{\partial u_1} + b_2\frac{\partial F_1(u_1,u_2)}{\partial u_2}\\
		b_1\frac{\partial F_2(u_1,u_2)}{\partial u_1} + b_2\frac{\partial F_2(u_1,u_2)}{\partial u_2}\\
	\end{array}\right],
\end{equation*}
donde $J(u)$ es la matriz jacobiana de la función vectorial suave 2-dimensional $F(u_1,u_2)$, y $b=[b_1,b_2]$ es un vector 2-dimensional. Aproximando $F(u+\delta b)$ con la expansión en seria de Taylor de primer orden $u$, se tiene
\begin{equation*}
	\frac{F(u+\delta b)-F(u)}{\delta} =  \left(\begin{array}{c}
		\frac{F_1(u_1,u_2) + \delta b_1 \frac{\partial F_1}{\partial u_1} + \delta b_2 \frac{\partial F_1}{\partial u_2} -F_1(u_1,u_2)}{\delta}\\
		\frac{F_2(u_1,u_2) + \delta b_1 \frac{\partial F_2}{\partial u_1} + \delta b_2 \frac{\partial F_2}{\partial u_2} -F_2(u_1,u_2)}{\delta}\\
	\end{array}\right)  + O(\delta\nnnorm{b}^2),
\end{equation*}
y por tanto
\begin{equation*}
	\frac{F(u+\delta b)-F(u)}{\delta} =  J(u)b + O(\delta\nnnorm{b}^2).
\end{equation*}

En lo que resta del documento se asumirá q existe una función $(\eta+1)$ continuamente diferenciable $g: \mathbb{R}^{d}\times \mathbb{R}^{d} \times \mathbb{R}_+ \to \mathbb{R}^{d}$ que aproxima al producto $f_x(y)b$ con orden $\eta$ para la cual la cota
\begin{equation} \label{eq:g_bound}
	\nnnorm{g(y,b;\delta)-f_x(y)b} \leq L\nnnorm{b}^{\eta+1}\delta^{\eta}
\end{equation}
se cumple, donde $f_x(y)$ es la matriz jacobiana del campo vectorial $f$ en el punto $y$, $y,b$ son vectores $d$-dimensionales y $L$ es una constante positiva que depende solamente de la norma de las derivadas de $f_x$.

%~(ver~\cite{steihaug1979attempt,schmitt1995matrix,weiner1997rowmap,hochbruck1998exponential,hosseini1999matrix,tranquilli2014rosenbrock}), preferiblemente diferencias finitas. Aunque estas aproximaciones son útiles en la práctica, estudios teóricos y simulaciones muestran que los esquemas resultantes sufren pérdida de precisión y disminución del orden de convergencia (ver~\cite{wanner1996solving,hochbruck1998exponential,tranquilli2014rosenbrock}).


Al utilizar el algoritmo de Arnoldi~(\ref{alg:Arnoldi}), la aproximación $K_{\mf,k}^{\pf,\qf}(\tau,A,b)$~(\ref{eq:gen_kp_aprox}) requiere de la evaluación explícita de la matriz jacobiana. Al reemplazar el algoritmo de Arnoldi~(\ref{alg:Arnoldi}) utilizado en~(\ref{eq:gen_kp_aprox}) por el algoritmo de Arnoldi sin matriz~(\ref{alg:iArnoldi}) se obtiene la siguiente aproximación libre de jacobiano.

\begin{definition}\label{def:gen_kp_aprox_fj}
	Sean las matrices $\widehat{V}_{\mf}\in\mathbb{R}^{d\times \mf}$ y $\widehat{H}^*_{\mf}\in\mathbb{R}^{{\mf} \times {\mf}}$, el vector $\widehat{v}_{\mf+1}\in\mathbb{R}^d$, y el número  $ \widehat{\hf}^*_{\mf+1,\mf}$ salidas del algoritmo de Arnoldi sin matriz~(\ref{alg:iArnoldi}) para el $\mf$-ésimo subespacio de Krylov $\widehat{\mathcal{K}}_\mf(\tau^\beta f_x(y),b;\delta)$, donde $f_x$ es la matriz jacobiana del campo vectorial $f$, $b,y\in\mathbb{R}^d$ vectores, $\tau,\delta>0$ y $\beta \ge 0$. Sea $\eta$ el orden de la aproximación $g(.;\delta)$ en el algoritmo~\ref{alg:iArnoldi}. Con $A=f_x(y)$, la aproximación $(\mf , \pf ,\qf , k)$-Krylov-Padé libre de jacobiano de $\tau \varphi_1(\tau A)b$ se define como 
	\begin{equation} \label{eq:gen_kp_aprox_fj}
		\widehat{K}_{\mf,k}^{\pf,\qf}\left(\tau, A , b ; \eta, \delta, \beta \right)=\nnorm{\nnorm{b}}_2 \widehat{V}_{\mf}\;[\widetilde{P}_{\tau}]_{12} + \nnorm{\nnorm{b}}_2 \widehat{\hf}_{\mf+1,\mf}e_\mf^T\;[\widetilde{P}_{\tau}]_{13} \widehat{v}_{\mf+1},
	\end{equation}
	donde $\widetilde{P}_{\tau}$ denota la aproximación $(\pf,\qf)$-Padé con escalamineto y potenciación $k$ para la exponencial matricial $\me{\tau\overline{H}}$, 
	\begin{equation}
		\overline{H} = \left[\begin{array}{cccc}
			\widehat{H}_\mf & e_1 & 0_{\mf\times 1} & 0_{\mf\times 1}\\
			0_{1\times\mf} & 0 & 1 & 0\\
			0_{1\times\mf} & 0 & 0 & 1\\
			0_{1\times\mf} & 0 & 0 & 0
		\end{array}\right] \label{hhat},
	\end{equation}
	$\widehat{H}_\mf=\widehat{H}^*_\mf/\tau^\beta$, $\widehat{\hf}_{\mf+1,\mf}=\widehat{\hf}^*_{\mf+1,\mf}/\tau^\beta$, y $e_i$ el $i$-ésimo vector canónico de $\mathbb{R}^\mf$.
\end{definition}


{\SetAlgoNoLine
	\begin{algorithm}
		\caption{\cite{brown1987local} Algoritmo de Arnoldi sin matriz para construir una base ortonormal $\{\widehat{v}_1,\ldots,\widehat{v}_\mf \}$ 
			del $\mf$-ésimo subespacio de Krylov $\widehat{\mathcal{K}}_\mf(\tau f_x(y),b;\delta)$}
		\label{alg:iArnoldi}
		\KwIn{función $g: \mathbb{R}^{d}\times \mathbb{R}^{d}\times \mathbb{R}_+ \to \mathbb{R}^{d}$ definida en (\ref{eq:g_bound}), $y,b \in \mathbb{R}^{d}$, $\tau,\delta>0$, y $\mf$ la dimensión del subespacio de Krylov}
		\KwOut{$\widehat{V}_{\mf}=[\widehat{v_1}\,\cdots \,\widehat{v}_\mf]\in \mathbb{R}^{d\times \mf}$,
			upper Hessenberg matrix $\widehat{H}^*_\mf=(\widehat{\hf}^*_{ij})\in \mathbb{R}^{\mf\times \mf} $,
			$\widehat{v}_{\mf+1} \in \mathbb{R}^d$, $\widehat{\hf}^*_{\mf+1,\mf}$}
		$\widehat{v}_1=b/\lVert b \rVert_2$\\
		\For{ $j=1,\ldots,\mf$ }{
			$\widehat{q}_j=g(y,\tau \widehat{v}_j;\delta)$, \\
			$\widehat{w}_j=\widehat{q}_j$ \\
			\For{ $i=1,\ldots,j$}{
				$\widehat{\hf}^*_{ij}=\scal{\widehat{q}_j}{\widehat{v}_i}$\\
				$\widehat{w}_j=\widehat{w}_j - \widehat{\hf}^*_{ij}\widehat{v}_i$ \\
			}
			$\widehat{\hf}^*_{j+1,j}=\lVert \widehat{w}_j \rVert_2$\\
			$\widehat{v}_{j+1}=\widehat{w}_j/\widehat{\hf}^*_{j+1,j}$
		}
	\end{algorithm}
}

 \subsection{Cotas para aproximaciones Krylov-Padé libres de jacobiano}
  En está sección se enunciará un teorema para acotar el error de la aproximación~(\ref{def:gen_kp_aprox_fj}) en función de tamaño de paso $h$, la dimensión del subespacio de Krylov $\mf$, los órdenes de Padé $\pf$ y $\qf$, el orden $\eta$ de la aproximación $g(.,\delta)$. Para poder enunciar dicho teorema se necesita otro teorema que establece una relación entre el subespacio generado por el algoritmo de Arnoldi~(\ref{alg:Arnoldi}) y el subespacio generado por el algoritmo de Arnoldi sin matriz~(\ref{alg:iArnoldi}).

\begin{theorem} \label{theorem:krilobapproxequality} \cite{naranjo2021locally}~Sea $f_x$ la matriz jacobiana del campo vectorial $f$, $b$ y $y$ de la misma dimensión, y $\tau>0$. Entonces, los $\mf$-ésimos subespacios de Krylov $\mathcal{K}_\mf$ y $\widehat{\mathcal{K}}_\mf$ generados por los algoritmos \ref{alg:Arnoldi} y \ref{alg:iArnoldi} respectivamente, satisfacen 
	\[ \widehat{\mathcal{K}}_\mf(\tau f_x(y),b;\delta) = \mathcal{K}_\mf(\tau f_x(y)+R_\mf,b), \]
	donde $R_\mf =  \varepsilon^\mf\widehat{V}^T_\mf$,  
	$\varepsilon^\mf = [\varepsilon_1,\dots,\varepsilon_\mf] \in \mathbb{R}^{d\times \mf}$, $\varepsilon_j = g(y,\tau\widehat{v}_j,\delta)- \tau f_x(y)\widehat{v_j}$, y $\widehat{V}_{\mf}=[\widehat{v_1}\,\cdots \,\widehat{v}_\mf]$ son resultados del algoritmo \ref{alg:iArnoldi}. Además, 
	\[ \nnnorm{R_\mf}_2\leq \sqrt{\mf}L \tau^{\eta+1}\delta^{\eta}, \]
	donde $\eta$ es el orden de la aproximación $g(.;\delta)$ en el algoritmo \ref{alg:iArnoldi}, y $L$ es la constante positiva de (\ref{eq:g_bound}) correspondiente a $g(.;\delta)$.
\end{theorem}
\textbf{Demostración}. La primera afirmación es un resultado directo del Teorema 3.2 in \cite{brown1987local}.

Utilizando  (\ref{eq:g_bound}) se tiene
\[ \nnnorm{\varepsilon_j}_2=\nnnorm{g(y,\tau\widehat{v}_j,\delta)-\tau f_x(y)\widehat{v}_j}_2 \leq  L \tau^{\eta+1}\delta^{\eta}\;\;\;\text{for}\;j=1,\dots,\mf, \]
donde $L$ es una constante positiva. Entonces, 
\begin{eqnarray*}
	\nnnorm{R_\mf}_2 & = & \nnnorm{\varepsilon^\mf V_\mf^T}_2\leq\nnnorm{\varepsilon^\mf}_2\leq \nnnorm{\varepsilon^\mf}_F \\
	& \leq & \left( \nnnorm{\varepsilon_1}_2^2 + \cdots + \nnnorm{\varepsilon_\mf}_2^2 \right)^{1/2} \\
	& \leq & \sqrt{\mf}L \tau^{\eta+1}\delta^{\eta},
\end{eqnarray*}
con lo que concluye la demostración. $\blacksquare$\\

\begin{theorem}\label{theorem:Krylov-fj-bound}
\cite{naranjo2021locally}~Tomando $A=f_x(y)$, sea $\widehat{K}_{\mf,k}^{\pf,\qf}\left(\tau, A , b ; \eta, \delta, \beta \right)$ la aproximación \\
	$(\mf , \pf ,\qf , k)$-Krylov-Padé libre de jacobiano (\ref{eq:gen_kp_aprox_fj}) of $\tau \varphi_1(\tau A)b$. Entonces,
	\begin{equation}
		\left\lvert\left\lvert  \tau \varphi_1(\tau A)b -
		\widehat{K}_{\mf,k}^{\pf,\qf}\left( \tau, A , b ; \eta, \delta, \beta \right)\right\rvert\right\rvert_2%\nonumber\\
		\leq \mathfrak{c}_0 \tau^{\min\{\mf+2,\pf+\qf+1\}} + \mathfrak{c}_1 \tau^{\beta\eta+2}\delta^{\eta},
	\end{equation}
	con $\tau,\delta \in [0,1]$, donde $\mathfrak{c}_0$ y $\mathfrak{c}_1$ son constantes positivas q dependen de $\mf,\pf,\qf,k$, $\nnnorm{A}_2$, $\nnnorm{b}_2$ y $L$, siendo $L$ la constante~(\ref{eq:g_bound}) correspondiente a la aproximación $g(.;\delta)$ en el algoritmo de Arnoldi sin matriz~\ref{alg:iArnoldi}.
\end{theorem}
\textbf{Demostración} Del \ref{theorem:krilobapproxequality}, se tiene
\[ \widehat{\mathcal{K}}_\mf(\tau^\beta A,b;\delta)=\mathcal{K}_\mf(\tau^\beta A+R_\mf,b) \]
y
\begin{equation}
	\nnnorm{R_m}_2\leq \sqrt{\mf} L \tau^{\beta(\eta+1)}\delta^{\eta},  \label{eq:fjbound3}
\end{equation}
donde la matriz $R_\mf$ y la constante positiva $L$ están definidas en el Theorem \ref{theorem:krilobapproxequality}. 

De la propiedad de invarianza ante escalado de la base ortonormal del subespacio de Krylov se tiene que
\[ \mathcal{K}_\mf(\tau^\beta A+R_\mf,b) = \mathcal{K}_\mf(A+\widehat{R}_\mf,b), \]
donde $\widehat{R}_\mf = R_\mf/\tau^\beta$. De esta forma, $\widehat{\mathcal{K}}_\mf(\tau^\beta A,b;\delta)=\mathcal{K}_\mf(A+\widehat{R}_\mf,b)$, y por tanto
\begin{equation}
	\widehat{K}_{\mf,k}^{\pf,\qf}\left( \tau, A , b; \eta, \delta, \beta \right) = K_{\mf,k}^{\pf,\qf}\left( \tau, A+\widehat{R}_\mf , b \right), \label{identidad}
\end{equation}
donde $K_{\mf,k}^{\pf,\qf}\left( \tau, A+\widehat{R}_\mf , b \right)$ denota la aproximación $(\mf , \pf ,\qf , k)$-Krylov-Padé (\ref{eq:gen_kp_aprox}) de $\tau \varphi_1(\tau (A+\widehat{R}_\mf))b$. 

Teniendo en cuanta $\varphi_1(z) \leq \varphi_0(z)$ se tiene
\begin{align}
	\nnnorm{\tau \varphi_1(\tau A)b -\widehat{K}_{\mf,k}^{\pf,\qf}\left( \tau, A , b; \eta, \delta, \beta \right)}_2 & = \nnnorm{\tau \varphi_1(\tau A)b -K_{\mf,k}^{\pf,\qf}\left( \tau, A+\widehat{R}_\mf , b \right)}_2 \nonumber \\
	& \leq \nnnorm{\tau \varphi_1(\tau A)b - \tau \varphi_1(\tau A + \tau \widehat{R}_\mf)b}_2 \nonumber \\
	& \enspace + \nnnorm{\tau \varphi_1(\tau A + \tau \widehat{R}_\mf)b-K_{\mf,k}^{\pf,\qf}\left( \tau, A+\widehat{R}_\mf , b \right)}_2 \nonumber\\
	& \leq \tau \nnnorm{\me{\tau A}b - \me{\tau A + \tau \widehat{R}_\mf}b}_2 \nonumber \\
	& \enspace + \nnnorm{\tau \varphi_1(\tau A + \tau \widehat{R}_\mf)b-K_{\mf,k}^{\pf,\qf}\left( \tau, A+\widehat{R}_\mf , b \right)}_2. \label{eq:fjbound}
\end{align}

De la Desigualdad de Increment Finito y la cota~(\ref{eq:fjbound3}) se obtiene
\begin{align}
	\nnnorm{\me{\tau A}b - \me{\tau A + \tau \widehat{R}_\mf}b}_2&\leq \nnnorm{b}_2 \nnnorm{\tau \widehat{R}_\mf}_2\me{\tau\nnnorm{A}_2+\tau\nnnorm{\widehat{R}_\mf}_2} \\ &\leq \nnnorm{b}_2   \sqrt{\mf}L \tau^{\beta\eta+1}\delta^\eta\me{\tau(\nnnorm{A}_2+\sqrt{\mf}L)} \nonumber\label{eq:fjbound1}
\end{align}
y, del Teorema 3.1 en \cite{naranjo2021locally}, 
\vspace{-0.25cm}
\begin{equation}
	\nnnorm{\tau \varphi_1(\tau A+\widehat{R}_\mf)b-K_{\mf,k}^{\pf,\qf}\left( \tau, A+\widehat{R}_\mf , b \right)}_2  \leq \nnnorm{b}_2C_{\mf,\kt}^{\pf,\qf}(\nnnorm{A}_2+\sqrt{\mf}L)\tau^{\min\{\mf+2,\pf+\qf+1\}}, \label{eq:fjbound2}
\end{equation}
donde $C_{\mf,\kt}^{\pf,\qf}$ está definido en el Teorema 3.1 in \cite{naranjo2021locally}.

Utilizando las desigualdades (\ref{eq:fjbound})-(\ref{eq:fjbound2})
\vspace{-0.25cm}
\begin{multline*}
	\nnnorm{\tau \varphi_1(\tau A)b -\widehat{K}_{\mf,k}^{\pf,\qf}\left( \tau, A , b; \eta, \delta, \beta \right)}_2 \leq 
	\nnnorm{b}_2 \sqrt{\mf}L \me{\tau(\nnnorm{A}_2+\sqrt{\mf}L)} \tau^{\beta\eta+2}\delta^\eta \\ + \nnnorm{b}_2C_{\mf,\kt}^{\pf,\qf}(\nnnorm{A}_2+\sqrt{\mf}L)\tau^{\min\{\mf+2,\pf+\qf+1\}} ,
\end{multline*}
con lo cual se concluye la demostración. $\Box$\\


\section{Aproximaciones de ecuaciones no autónomas}
