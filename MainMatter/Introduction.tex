%===================================================================================
% Chapter: Introduction
%===================================================================================
\chapter*{Introducción}\label{chapter:introduction}
\addcontentsline{toc}{chapter}{Introducción}
%===================================================================================

Las Ecuaciones en Derivadas Parciales (EDP) son frecuentemente utilizadas para describir la evolución temporal de complejos fenómenos físicos, biofísicos y físico-químicos (ver, por ejemplo, \cite{tikhonov2013,gray1990,cronin1987}), por lo que resulta de gran interés práctico obtener sus soluciones. Sin embargo, las soluciones analíticas explícitas solo son conocidas para unas pocas EDP lineales definidas sobre geometrías simples (ver, por ejemplo, \cite{tikhonov2013}). Por esta razón, típicamente, las soluciones de las EDP son aproximadas mediante algún método numérico. En general, existen varias familias de métodos numéricos entre los que se encuentran los métodos variacionales, los de elementos finitos, los de diferencia finita y los de líneas (ver, por ejemplo, \cite{prenter2008,samarskii2008,schiesser2012}). La gran mayoría de esos integradores numéricos convencionales tienen un inconveniente en común, ellos requieren de la solución de grandes sistemas de ecuaciones algebraicas mal condicionadas y, por lo tanto, necesitan del uso de matrices precondicionadoras específicas para cada tipo de EDP. Como excepción se distinguen los llamados integradores exponenciales para las ecuaciones diferenciales ordinarias (EDO) que resultan de la discretización espacial de las EDP mediante el método de las líneas. En lugar de las mencionadas ecuaciones algebraicas, estos integradores requieren del cálculo de exponenciales matriciales y/o de productos de las llamadas funciones $\varphi$ por vectores que involucran la evaluación de matrices Jacobianas asociadas a la EDO.

Inicialmente, desde principio de los años 60 del siglo pasado, los primeros  integradores exponenciales se derivaron de diferentes aproximaciones a la integral de la fórmula de variación de constantes para la solución de las EDO semilineales resultantes de la discretización espacial de las EDP. Integradores derivados de este enfoque son, por ejemplo, el método de Lawson y Lawson generalizado, la diferenciación temporal exponencial y los métodos exponenciales de Runge-Kutta (ver \cite{Berland07} para una revisión y Capítulo 1 para detalles), y se agrupan dentro de la categoría de integradores exponenciales Globalmente Linealizados~(GL). En general, estos métodos implican el cálculo de exponenciales matriciales y/o de productos de funciones phi por vector solamente en el paso de integración inicial y funcionan bien cuando el término no lineal en las EDO semilineales mencionadas es pequeño o está acotado en términos del término lineal. Sin embargo, cuando el término no lineal es relevante para la dinámica de las EDO semilineales, la efectividad de estos métodos se pierde \cite{hochbruck2009exponential}.


Más recientemente, para solventar la mencionada deficiencia de los integradores GL, se propuso el uso de los llamados integradores de Linealización Local~(LL). Estos métodos se derivan en dos pasos (ver Capítulo 1 para detalles): 1) la linealización local de las mencionadas EDO en cada paso de integración, y 2) una aproximación a la integral de la fórmula de variación de constantes para la solución de la ecuación linealizada del paso 1). A diferencia de los GL, los integradores LL conllevan el cálculo de exponenciales matriciales y/o de productos de funciones phi por vector en cada paso de integración. Integradores de éste tipo son, por ejemplo, el método de propagación iterativa exponencial \cite{tokman2006efficient}, el método exponencial-Rosenbrock \cite{hochbruck2009exponential} y el método exponencial-Adams \cite{hochbruck2011exponential}. Estudios de simulaciones han mostrado que éstos métodos son mucho más eficiente que los integradores tradicionales utilizados con el método de las líneas para resolver EDP. No obstante, el alto orden de convergencia de estos integradores exponenciales conlleva al cálculo de varios productos de funciones $\varphi$ por vector en cada paso de integración lo que conforma la mayor parte  del costo computacional de estos métodos \cite{naranjo2021locally}. La ausencia de una estrategia adaptativa para el tamaño de paso de integración y la necesidad de evaluar la matriz Jacobiana exacta del campo vectorial de la EDO son otras de las limitaciones prácticas de esos métodos.

Por otra parte, con anterioridad a los trabajos mencionados en el párrafo anterior y en un contexto diferente al de la solución de EDP, los integradores de Linealización Local habían sido exitosamente desarrollados para resolver varias clases de ecuaciones diferenciales de dimensiones pequeñas \cite{jimenez2020}.
\redmark{Entiéndase por dimensión pequeña aquella de una matriz cuadrada para la cual la aproximación racional de Padé de la exponencial matricial es más eficiente que las aproximaciones iterativas en término del costo computacional.} En el caso específico de EDO, resultados teóricos y de simulaciones muestran que el método LL clásico de orden 2  \cite{pope1963exponential} posee una variedad de propiedades deseables entre las que se encuentran la A-estabilidad, la ausencia de puntos de equilibrio espurios bajo condiciones bastante generales y la preservación del comportamiento dinámico de la solución alrededor de puntos de equilibrio hiperbólicos y órbitas periódicas \cite{Jimenez02AMC}. Si embargo, su bajo orden de convergencia constituye un inconveniente importante. Para solucionar esta limitación \cite{delaCruz06,delaCruz07,Jimenez13}, se propusieron los métodos de Linealización Local de Orden Superior (LLOS) que se obtienen de adicionar al esquema LL clásico de orden 2 una aproximación a la representación integral o a la diferencial de su residuo (ver Capítulo 1 para detalles). Por construcción, los métodos LLOS derivados de la forma diferencial contienen un solo  producto de función phi por vector en cada paso de integración, característica que los distingue muy positivamente de los derivados de la representación integral. Entre los primeros se encuentran los métodos Runge-Kuta Localmente Linealizados de orden 3 y 4, y las fórmulas embebidas Runge-Kutta de Dormand y Prince Localmente Linealizadas \cite{delaCruz06,Jimenez13,Jimenez14AMC}.

Motivado por el hecho de que las fórmulas embebidas Runge-Kutta de Dormand y Prince Localmente Linealizadas comparten un solo producto de función phi por vector en cada paso de integración y, también, por el éxito de su implementación adaptativa para integrar EDO de dimensiones pequeñas, en ésta tesis se valora la posibilidad de construir una implementación adaptativa de dichas fórmulas para integrar EDO de dimensiones no pequeñas. Con ese propósito, en las mencionadas fórmulas embebidas, se requiere sustituir la aproximación de Padé para la acción de la función phi sobre un vector por la de algún método iterativo. Basado en el método de los subespacios de Krylov, en trabajos previos se han propuesto varias de esas aproximaciones con diferentes estrategias para controlar sus errores, y varios enfoques para estimar la dimensión de Krylov \cite{hochbruck1998exponential,sidje1998expokit,niesen2012algorithm,gaudreault2018kiops}. Alternativamente, en ésta tesis, se propone una aproximación de orden mayor, una nueva medida del error y una nueva forma de estimar la dimensión óptima de Krylov.

En muchas situaciones prácticas, debido a los insuficientes recursos computacionales disponibles o al uso de discretizaciones espaciales complejas, no es viable evaluar y almacenar la matriz Jacobiana de la EDO u obtener su expresión analítica exacta. Para esos casos, en esta tesis se propone y estudia la nueva familia de métodos LLOS Libres de Jacobiano. Con ese propósito, para el cálculo de los productos de funciones phi por vector se introduce una aproximación de Krylov libre de Jacobiano, una medida de su error y un criterio para estimar la dimensión de Krylov. Como caso particular, se presenta la clase de esquemas de Runge-Kutta Localmente Linealizados Libres de Jacobiano y se construyen explícitamente esquemas de tercer a quinto orden. Modificando las fórmulas embebidas  Runge-Kutta de Dormand y Prince Localmente Linealizadas se implementa un esquema adaptativo de orden variable y libre de Jacobiano.

El documento de tesis está organizado de la siguiente forma. En el primer capítulo se presenta una breve revisión de los integradores exponenciales y los métodos de Linealización Local necesarios para la comprensión de este trabajo. En el segundo capítulo, se construyen las aproximaciones Krylov-Padé con y sin la evaluación del Jacobiano para calcular los productos de funciones phi por vector que representan la solución de EDO lineales, se determinan cotas para dichas aproximaciones y se prueba su efectividad en un conjunto de ecuaciones de prueba. En el Capítulo 3, se derivan las nuevas fórmulas embebidas Runge-Kutta de Dormand y Prince Localmente Linealizadas para EDO de dimensiones no pequeñas y se establece su velocidad de convergencia. Se presentan también estrategias adaptativas para diseñar tanto esquemas con tamaño de paso fijo y dimensión de Krylov variable, como esquemas con tamaño de paso y dimensión de Krylov variable. En el último capítulo, se introduce una clase de métodos LLOS Libres de Jacobiano y se obtiene un resultado general sobre la velocidad de convergencia de la nueva familia de integradores. Se introduce, además, la clase particular de esquemas Runge-Kutta Localmente Linealizados Libres de Jacobiano y se modifican las fórmulas embebidas Runge-Kutta  de Dormand y Prince Localmente Linealizadas para implementar un esquema adaptativo de orden variable y libre de Jacobiano.
En los Capítulos 3 y 4, se evalúa el desempeño de los nuevos esquemas en la integración de ecuaciones de prueba conocidas y se compara con otros integradores exponenciales.

Los resultados recopilados en los Capítulos 2, 3 y 4 de la Tesis fueron tomados de artículos \cite{naranjo2021locally,naranjo2023jacobian,naranjo2023computing} publicados por el autor en tres revistas \textit{peer review} con clasificación JCR-Q1 y en los Reportes de Investigación del ICIMAF \cite{naranjo2022RT,naranjo2023RT} con aval de oponentes.
