%===================================================================================
% Chapter: Introduction
%===================================================================================
\chapter*{Introducción}\label{chapter:introduction}
\addcontentsline{toc}{chapter}{Introducción}
%===================================================================================

Las Ecuaciones en Derivadas Parciales(EDP) se utilizan para modelar de manera eficiente la evolución temporal de complejos fenómenos físicos, biofísicos y físico-químicos. En general no es posible resolver dichas ecuaciones de manera analítica, por lo que se recurre a métodos numéricos para aproximar sus soluciones. Una vía de solución es utilizar una discretización espacio-temporal para transformar la EDO en un sistema algebraico, el cual es mal condicionado y se necesita del uso de precondicionadores para cada problema en específico. Otra vía de solución es utilizar el método de línea para convertir la EDP en un sistema EDO que suele ser de tipo \emph{stiff}.

Debido a las características de las EDO tipo \emph{stiff} es necesario utilizar integradores implícitos. Estos integradores implícitos también necesitan resolver in sistema algebraico en cada paso de integración. Para resolver de manera eficiente estas EDOs se diseñaron los integradores de tipo exponencial. El desarrollo de esta clase de métodos ha sido estimulado por su capacidad para preservar numerosas propiedades dinámicas de las EDOs con tamaños de pasos mucho mayores que los integradores explícitos y con menor costo computacional que los implícitos.

Inicialmente, los integradores exponenciales se centraron principalmente en la integración de sistemas de Ecuaciones Diferenciales Ordinarias(EDO) semilineales resultantes de la discretización espacial de las mencionadas EDP. Los integradores derivados de este enfoque llamados Integradores Exponenciales Globalmente Linealizados~(IEGL), son, por ejemplo, el método de Lawson y Lawson generalizado, la diferenciación temporal exponencial y los métodos exponenciales de Runge-Kutta(ver \cite{Berland07} para una revisión). En general, estos métodos implican el cálculo de las llamadas funciones phi en el paso de integración inicial y funcionan bien cuando el término no lineal en las EDO semilineales mencionadas es pequeño o está acotado en términos del término lineal. Sin embargo, cuando el término no lineal es relevante para la dinámica de las EDO semilineales, la efectividad de estos métodos se pierde.

Para solventar la deficiencia que tenían IEGL y extenderlos a EDOs en general se introdujo un nuevo tipo de integradores exponenciales. Estos nuevos integradores, conocidos como esquemas de Linealización Local~(LL), se derivan de un principio común: la linealización local de la EDO en cada paso de integración. Resultados teóricos y de simulaciones muestran que los métodos de linealización local poseen un gran número de propiedades deseables entre las que se encuentran la A-estabilidad, ausencia de puntos de equilibrio espurios bajo condiciones bastante generales y la preservación del comportamiento dinámico de la solución alrededor de puntos de equilibrio hiperbólicos y órbitas periódicas ~\cite{Jimenez02AMC,delaCruz06,delaCruz07,Jimenez13}. La estabilidad y las propiedades dinámicas antes mencionadas son de vital importancia para la integración de EDOs, no obstante su bajo orden de convergencia constituía una limitante en algunas circunstancias. Para solucionar el bajo orden de convergencia de los LL se desarrollaron los métodos localmente linealizados de orden superior los cuales retienen las propiedades antes mencionadas, pero con orden de convergencia mayor. Estos nuevos métodos se obtienen de agregar a los LL un término adicional de orden superior que resulta de aproximar la parte no lineal de la EDO linealizada. Los integradores localmente linealizados son, por ejemplo, el método de propagación iterativa exponencial~\cite{tokman2006efficient}, el método exponencial-Rosenbrock~\cite{tranquilli2014rosenbrock} y el método exponencial-Adams~\cite{hochbruck2011exponential}, los esquemas de Linealización Local de orden 2~\cite{pope1963exponential,hochbruck1997krylov,hochbruck1998exponential} y de Linealización Local de Orden Superior~~(LLOS)\cite{delaCruz06,delaCruz07,Jimenez13,Jimenez14AMC}.

Los integradores exponenciales localmente linealizados requieren el cálculo de productos de diferentes funciones phi por vectores en cada paso de integración, los cual tiene el mayor costo computacional. Por lo tanto, un punto crítico en la eficiencia computacional de estos integradores exponenciales es la aproximación efectiva de las funciones phi en cada paso de integración. Para EDOs de dimensiones pequeñas estas funciones phi son aproximadas mediante expansiones de Taylor, Padé o Chebychev. Estas expansiones son efectivas para calcular exponenciales de matrices relativamente pequeñas y densas; pero no así para matrices
de dimensiones moderadas o grandes. Con el propósito de extender el cálculo de exponenciales matriciales a matrices grandes, se han propuesto varias aproximaciones utilizando subespacios de Krylov con diferentes estrategias para controlar sus errores, y varios enfoques para estimar las dimensiones óptimas de Krylov \cite{hochbruck1998exponential,sidje1998expokit,niesen2012algorithm,gaudreault2018kiops}. Algunos integradores exponenciales han sido extendidos a grades dimensiones al combinarlos dichas aproximaciones de Krylov, obteniéndose así esquemas numéricos con dimensión variable de Krylov pero con tamaño de paso fijo \cite{tokman2006efficient,tranquilli2014rosenbrock,hochbruck2011exponential,kloeden2011exponential}.

La adaptabilidad de los integradores exponenciales localmente linealizados, mediante fórmulas embebidos de alto orden y estrategias adaptativas para la selección de tamaño de paso variable, es clave en la mejorar su desempeño en la integración de EDOs. La gran mayoría de los integradores exponenciales son de paso fijo con excepciones notables como los esquemas de tamaño de paso variable propuestos en \cite{hochbruck1998exponential,caliari2009implementation,tokman2012new,luan2014exponential} con estrategias adaptativas basadas en fórmulas embebidas. Sin embargo, las fórmulas integradas de alto orden de estos esquemas requieren el cálculo de varias funciones phi por vectores en el mismo paso de integración, lo que implica un alto costo computacional. Otro aspecto importante a tener en cuenta es el caso en que no es posible calcular o almacenar el Jacobiano del EDO que se está resolviendo. Para estos casos son necesarios métodos libres de Jacobiano. Libre de Jacobiano quiere decir métodos que nunca calculan o almacenan el Jacobiano parcial o completamente.

En general los integradores exponenciales representan una alternativa a los métodos implícitos para resolver ecuaciones diferenciales \emph{stiff} o altamente oscilatorias. Sin embargo, hasta el momento, estos integradores o son de paso fijo o poseen una estrategia adaptativa pero tienen un alto costo computacional ya que requieren del cálculo de varias funciones phi por diferentes vectores. Además existen pocos casos en que se estudia como afectan las aproximaciones libres de Jacobiano al orden de convergencia. Estas razones justifican la construcción de códigos de Linealización Local eficientes que permitan la integración de las EDOs que usualmente aparecen en situaciones prácticas. Trabajos en esta dirección han sido \cite{Jimenez13,Jimenez14AMC} donde se han desarrollado esquemas adaptativos Runge-Kutta Localmente Linealizados, pero para pequeñas dimensiones. Por esto el objetivo de este trabajo es analizar la factibilidad de aplicar los métodos de Linealización Local de Orden Superior para resolver problemas de valor inicial de grandes dimensiones. Para ello se desarrollaran las aproximaciones Krylov-Padé con y sin la evaluación del Jacobiano. Estas aproximaciones se utilizarán para la construcción de métodos de Linealización Local de Orden Superior para resolver problemas de valor inicial de grandes dimensiones y se estudiará su desempeño en problemas de prueba.

El documento está organizado de la siguiente forma. En el primer capítulo se presenta una breve revisión de los integradores exponenciales y los métodos de Linealización Local necesarios para la comprensión de este trabajo. En el segundo capítulo se construirán las aproximaciones Krylov-Padé con y sin la evaluación del Jacobiano. Estas aproximaciones se utilizarán en la integración de EDOs lineales. Además, se hallarán las cotas de dichas aproximaciones y se medirá su desempeño en un conjunto de ecuaciones de prueba. En el tercer capítulo, los métodos Runge-Kutta de Dormand y Prince Localmente Linealizado, se construirán. En el cuarto capítulo se extenderán los resultad de los métodos de linealización local de orden superior a métodos de linealización local de orden superior libres de Jacobiano; se construirán esquemas basados en estos nuevos métodos libres de jacobiano y se medirá el desempeño de estos nuevos esquemas en ecuaciones de prueba.


