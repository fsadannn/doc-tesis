%===================================================================================
% Chapter: Introduction
%===================================================================================
\chapter*{Introducción}\label{chapter:introduction}
\addcontentsline{toc}{chapter}{Introducción}
%===================================================================================

Las Ecuaciones en Derivadas Parciales (EDP) son frecuentemente utilizadas para describir la evolución temporal de complejos fenómenos físicos, biofísicos y físico-químicos (ver, por ejemplo, \cite{tikhonov2013,gray1990,cronin1987}), por lo que resulta de gran interés práctico obtener sus soluciones. Sin embargo, las soluciones analíticas explicitas solo son conocidas para unas pocas EDP lineales definidas sobre geometrías simples (ver, por ejemplo, \cite{tikhonov2013}). Por ésta razón, típicamente, las soluciones de las EDP son aproximadas mediante algún método numérico. En general, existen varias familias de métodos numéricos entre los que se encuentran los métodos variacionales, los de elementos finitos, los de diferencia finita y los de líneas (ver, por ejemplo, \cite{prenter2008,samarskii2008,schiesser2012}). La gran mayoría de esos integradores numéricos convencionales tienen un inconveniente en común, ellos requieren de la solución de grandes sistemas de ecuaciones algebraicas mal condicionadas y, por lo tanto, necesitan del uso de matrices pre-condicionaras especificas para cada tipo de EDP. Como excepción se distinguen los llamados integradores exponenciales para las ecuaciones diferenciales ordinarias (EDO) que resultan de la discretización espacial de las EDP mediante el método de las lineas. En lugar de las mencionadas ecuaciones algebraicas, éstos integradores requieren del cálculo de exponenciales matriciales y/o de productos de las llamadas funciones phi por vectores que involucran la evaluación de matrices Jacobianas asociadas a la EDO.

%Una vía de solución es utilizar una discretización espacio-temporal para transformar la EDO en un sistema algebraico, el cual es mal condicionado y se necesita del uso de precondicionadores para cada problema en específico. Otra vía de solución es utilizar el método de línea para convertir la EDP en un sistema EDO que suele ser de tipo \emph{stiff}. 
%
%Debido a las características de las EDO tipo \emph{stiff} es necesario utilizar integradores implícitos. Estos integradores implícitos también necesitan resolver in sistema algebraico en cada paso de integración. Para resolver de manera eficiente estas EDOs se diseñaron los integradores de tipo exponencial. El desarrollo de esta clase de métodos ha sido estimulado por su capacidad para preservar numerosas propiedades dinámicas de las EDOs con tamaños de pasos mucho mayores que los integradores explícitos y con menor costo computacional que los implícitos.

Inicialmente, desde principio de los años 60 del siglo pasado, los primeros  integradores exponenciales se derivaron de diferentes aproximaciones a la integral de la fórmula de variación de las constantes para la solución de las EDO semilineales resultantes de la discretización espacial de las EDP. Integradores derivados de este enfoque son, por ejemplo, el método de Lawson y Lawson generalizado, la diferenciación temporal exponencial y los métodos exponenciales de Runge-Kutta (ver \cite{Berland07} para una revisión y Capitulo 1 para detalles), y se agrupan dentro de la categoría de integradores exponenciales Globalmente Linealizados~(GL). En general, estos métodos implican el cálculo de exponenciales matriciales y/o de productos de funciones phi por vector solamente en el paso de integración inicial y funcionan bien cuando el término no lineal en las EDO semilineales mencionadas es pequeño o está acotado en términos del término lineal. Sin embargo, cuando el término no lineal es relevante para la dinámica de las EDO semilineales, la efectividad de estos métodos se pierde \cite{hochbruck2009exponential}.


Más recientemente, para solventar la mencionada deficiencia de los integradores GL, se propuso el uso de los llamados integradores de Linealización Local~(LL). Estos métodos se derivan en dos pasos (ver Capitulo 1 para detalles): 1) la linealización local de las mencionadas EDO en cada paso de integración, y 2) una aproximación a la integral de la fórmula de variación de las constantes para la solución de la ecuación linealizada del paso 1). A diferencia de los GL, los integradores LL conllevan el cálculo de exponenciales matriciales y/o de productos de funciones phi por vector en cada paso de integración. Integradores de éste tipo son, por ejemplo, el método de propagación iterativa exponencial \cite{tokman2006efficient}, el método exponencial-Rosenbrock \cite{hochbruck2009exponential} y el método exponencial-Adams \cite{hochbruck2011exponential}. Estudios de simulaciones han mostrado que éstos métodos son mucho más eficiente que los integradores tradiciones utilizados con el método de las líneas para resolver EDP. No obstante, el alto orden de convergencia de estos integradores exponenciales conlleva al cálculo de varios productos de funciones phi por vector en cada paso de integración lo que conforma la mayor parte  del costo computacional de estos métodos \cite{naranjo2021locally}. La ausencia de una estrategia adaptativa para el tamaño de paso de integración y la necesidad de evaluar la matriz Jacobiana exacta del campo vectorial de la EDO son otras de las limitaciones prácticas de esos métodos.

Por otra parte, con anterioridad a los trabajos mencionados en el párrafo anterior y en un contexto diferente al de la solución de EDP, los integradores de Linealización Local habían sido exitosamente desarrollados para resolver varias clases de ecuaciones diferenciales de dimensiones pequeñas \cite{jimenez2020}, entendiéndose por dimensión pequeña aquella de una matriz cuadrada para la que es posible calcular eficientemente la exponencial matricial mediante la aproximación racional de Padé. En el caso específico de EDO, resultados teóricos y de simulaciones muestran que el método LL clásico de orden 2  \cite{pope1963exponential} posee una variedad de propiedades deseables entre las que se encuentran la A-estabilidad, la ausencia de puntos de equilibrio espurios bajo condiciones bastante generales y la preservación del comportamiento dinámico de la solución alrededor de puntos de equilibrio hiperbólicos y órbitas periódicas \cite{Jimenez02AMC}. Si embargo, su bajo orden de convergencia constituye un inconveniente importante. Para solucionar esta limitación \cite{delaCruz06,delaCruz07,Jimenez13}, se propusieron los métodos de Linealización Local de Orden Superior (LLOS) que se obtienen de adicionar al esquema LL clásico de orden 2 una aproximación a la representación integral o a la diferencial de su residuo (ver Capitulo 1 para detalles). Por construcción, los métodos LLOS derivados de la forma diferencial contienen un solo  producto de función phi por vector en cada paso de integración, característica que los distingue muy positivamente de los derivados de la representación integral. Entre los primeros se encuentran los métodos Runge-Kuta Localmente Linealizados de orden 3 y 4, y las formulas Runge-Kutta embebidas de Dormand y Prince Localmente Linealizadas \cite{delaCruz06,Jimenez13,Jimenez14AMC}.

Motivado por el hecho de que las formulas Runge-Kutta embebidas de Dormand y Prince Localmente Linealizadas comparten un solo producto de función phi por vector en cada paso de integración y, también, por el éxito de su implementación adaptativa para integrar EDO de dimensiones pequeñas, en ésta tesis se valora la posibilidad de construir una implementación adaptativa de dichas fórmulas para integrar EDO de dimensiones no pequeñas. Con ese propósito, en las mencionadas fórmulas embebidas, se require sustituir la aproximación de Padé para acción de la función phi sobre un vector por la de algún método iterativo. Basado en el método de los subespacios de Krylov, en trabajos previos se han propuesto varias de esas aproximaciones con diferentes estrategias para controlar sus errores, y varios enfoques para estimar la dimensión óptimas de Krylov \cite{hochbruck1998exponential,sidje1998expokit,niesen2012algorithm,gaudreault2018kiops}. Alternativamente, en ésta tesis, se propone una aproximación de orden mayor, una nueva medida del error y una nueva forma de estimar la dimensión óptima de Krylov.

En muchas situaciones prácticas, debido a los insuficientes recursos computacionales disponibles o al uso de discretizaciones espaciales complejas, no es viable evaluar y almacenar la matriz Jacobiana de la ODE u obtener su expresión analítica exacta. Para esos casos, en esta tesis se propone y estudia la nueva familia de métodos LLOS Libres de Jacobiano. Con ese propósito, para el cálculo de los productos de funciones phi por vector se introduce una aproximación de Krylov libre de Jacobiano, una medida de su error y un criterio para estimar la dimensión de Krylov. Como caso particular, se presenta la clase de esquemas de Runge-Kutta Localmente Linealizados Libres de Jacobiano y se construyen explícitamente esquemas de tercer a quinto orden. Utilizando las  formulas Runge-Kutta embebidas de Dormand y Prince Localmente Linealizadas se implementa un esquema adaptativo de orden variable y libre de Jacobiano.

%Para solventar la deficiencia que tenían IEGL y extenderlos a EDOs en general se introdujo un nuevo tipo de integradores exponenciales. Estos nuevos integradores, conocidos como esquemas de Linealización Local~(LL), se derivan de un principio común: la linealización local de la EDO en cada paso de integración. Resultados teóricos y de simulaciones muestran que los métodos de linealización local poseen un gran número de propiedades deseables entre las que se encuentran la A-estabilidad, ausencia de puntos de equilibrio espurios bajo condiciones bastante generales y la preservación del comportamiento dinámico de la solución alrededor de puntos de equilibrio hiperbólicos y órbitas periódicas ~\cite{Jimenez02AMC,delaCruz06,delaCruz07,Jimenez13}. La estabilidad y las propiedades dinámicas antes mencionadas son de vital importancia para la integración de EDOs, no obstante su bajo orden de convergencia constituía una limitante en algunas circunstancias. Para solucionar el bajo orden de convergencia de los LL se desarrollaron los métodos localmente linealizados de orden superior los cuales retienen las propiedades antes mencionadas, pero con orden de convergencia mayor. Estos nuevos métodos se obtienen de agregar a los LL un término adicional de orden superior que resulta de aproximar la parte no lineal de la EDO linealizada. Los integradores localmente linealizados son, por ejemplo, el método de propagación iterativa exponencial~\cite{tokman2006efficient}, el método exponencial-Rosenbrock~\cite{tranquilli2014rosenbrock} y el método exponencial-Adams~\cite{hochbruck2011exponential}, los esquemas de Linealización Local de orden 2~\cite{pope1963exponential,hochbruck1997krylov,hochbruck1998exponential} y de Linealización Local de Orden Superior~~(LLOS)\cite{delaCruz06,delaCruz07,Jimenez13,Jimenez14AMC}.

%Los integradores exponenciales localmente linealizados requieren el cálculo de productos de diferentes funciones phi por vectores en cada paso de integración, los cual tiene el mayor costo computacional. Por lo tanto, un punto crítico en la eficiencia computacional de estos integradores exponenciales es la aproximación efectiva de las funciones phi en cada paso de integración. Para EDOs de dimensiones pequeñas estas funciones phi son aproximadas mediante expansiones de Taylor, Padé o Chebychev. Estas expansiones son efectivas para calcular exponenciales de matrices relativamente pequeñas y densas; pero no así para matrices
%de dimensiones moderadas o grandes. Con el propósito de extender el cálculo de exponenciales matriciales a matrices grandes, se han propuesto varias aproximaciones utilizando subespacios de Krylov con diferentes estrategias para controlar sus errores, y varios enfoques para estimar las dimensiones óptimas de Krylov \cite{hochbruck1998exponential,sidje1998expokit,niesen2012algorithm,gaudreault2018kiops}. Algunos integradores exponenciales han sido extendidos a grades dimensiones al combinarlos dichas aproximaciones de Krylov, obteniéndose así esquemas numéricos con dimensión variable de Krylov pero con tamaño de paso fijo \cite{tokman2006efficient,tranquilli2014rosenbrock,hochbruck2011exponential,kloeden2011exponential}.

%La adaptabilidad de los integradores exponenciales localmente linealizados, mediante fórmulas embebidos de alto orden y estrategias adaptativas para la selección de tamaño de paso variable, es clave en la mejorar su desempeño en la integración de EDOs. La gran mayoría de los integradores exponenciales son de paso fijo con excepciones notables como los esquemas de tamaño de paso variable propuestos en \cite{hochbruck1998exponential,caliari2009implementation,tokman2012new,luan2014exponential} con estrategias adaptativas basadas en fórmulas embebidas. Sin embargo, las fórmulas integradas de alto orden de estos esquemas requieren el cálculo de varias funciones phi por vectores en el mismo paso de integración, lo que implica un alto costo computacional. Otro aspecto importante a tener en cuenta es el caso en que no es posible calcular o almacenar el Jacobiano del EDO que se está resolviendo. Para estos casos son necesarios métodos libres de Jacobiano. Libre de Jacobiano quiere decir métodos que nunca calculan o almacenan el Jacobiano parcial o completamente.

%En general los integradores exponenciales representan una alternativa a los métodos implícitos para resolver ecuaciones diferenciales \emph{stiff} o altamente oscilatorias. Sin embargo, hasta el momento, estos integradores o son de paso fijo o poseen una estrategia adaptativa pero tienen un alto costo computacional ya que requieren del cálculo de varias funciones phi por diferentes vectores. Además existen pocos casos en que se estudia como afectan las aproximaciones libres de Jacobiano al orden de convergencia. Estas razones justifican la construcción de códigos de Linealización Local eficientes que permitan la integración de las EDOs que usualmente aparecen en situaciones prácticas. Trabajos en esta dirección han sido \cite{Jimenez13,Jimenez14AMC} donde se han desarrollado esquemas adaptativos Runge-Kutta Localmente Linealizados, pero para pequeñas dimensiones. Por esto el objetivo de este trabajo es analizar la factibilidad de aplicar los métodos de Linealización Local de Orden Superior para resolver problemas de valor inicial de grandes dimensiones. Para ello se desarrollaran las aproximaciones Krylov-Padé con y sin la evaluación del Jacobiano. Estas aproximaciones se utilizarán para la construcción de métodos de Linealización Local de Orden Superior para resolver problemas de valor inicial de grandes dimensiones y se estudiará su desempeño en problemas de prueba.

El documento de tesis está organizado de la siguiente forma. En el primer capítulo se presenta una breve revisión de los integradores exponenciales y los métodos de Linealización Local necesarios para la comprensión de este trabajo. En el segundo capítulo, se construyen las aproximaciones Krylov-Padé con y sin la evaluación del Jacobiano para calcular los productos de funciones phi por vector que representan la solución de EDO lineales, se determinan cotas para dichas aproximaciones y se prueba su efectividad en un conjunto de ecuaciones de prueba. En el Capitulo 3, se derivan las nuevas fórmulas de Runge-Kutta Localmente Linealizadas de Dormand y Prince para EDOs de dimensiones no pequeñas y se establece su velocidad de convergencia. Se presentan también estrategias adaptativas para diseñar tanto esquemas con tamaño de paso fijo y dimensión de Krylov variable, como esquemas con tamaño de paso y dimensión de Krylov variable. En el último capitulo, se introduce una clase de métodos LLOS Libres de Jacobiano y se obtiene un resultado general sobre la velocidad de convergencia de la nueva familia de integradores. Se introduce la clase particular de esquemas Runge-Kutta Localmente Linealizados Libres de Jacobiano y se utilizan las formulas Runge-Kutta embebidas de Dormand y Prince Localmente Linealizadas para implementar un esquema adaptativo de orden variable y libre de Jacobiano.
En los Capítulos 3 y 4, se evalúa el desempeño de los nuevos esquemas en la integración de ecuaciones de prueba conocidas y se compara con otros integradores exponenciales.

Los resultados recopilados en los Capítulos 2, 3 y 4 de la Tesis fueron tomados de artículos \cite{naranjo2021locally,naranjo2023jacobian,naranjo2023computing} publicados por el autor en tres revistas \textit{peer review} con clasificación JCR-Q1 y en los Reportes de Investigación del ICIMAF \cite{naranjo2022RT,naranjo2023RT} con aval de oponentes.  


