\begin{conclusions}

%    En este trabajo se desarrollaron códigos adaptativos para la integración de problemas de valor
%    inicial de dimensiones no pequeñas basados en esquemas embebidos de linealización local de orden
%    superior con y sin la evaluación del jacobiano.
    
    En ésta Tesis se desarrollaron métodos de Linealización Local de Orden Superior para integrar problemas de valor
    inicial de dimensiones no pequeñas. Especificamante: 

    1)- Se construyeron aproximaciones Krylov-Padé con y sin la evaluación de Jacobiano para la solución de ecuaciones diferenciales lineales de dimensiones no pequeñas y se acotaron sus errores. Para cada aproximación, se propusieron estrategia efectivas para la estimación práctica de la dimensión de Krylov, el orden de Padé y el error de aproximacion. Se comprobó la eficacia de ambas aproximaciones para calcular la acción de funciones phi sobre vectores y, en se aspecto, se mostró las ventajas del uso de la aproximación Krylov-Padé con evaluación de Jacobiano en relación con otras aproximaciones similares existentes. 
    
%    Estas aproximaciones presentan un buen desempeño al poseer balance entre precisión y tiempo de cómputo a la hora aproximar la acción de las funciones phi sobre vectores. La mayor diferencia entre la aproxima que evalúa el Jacobiano y la que no es que la que utiliza el Jacobiano exacta disminuye su error a medida aumenta la dimensión de Krylov; sin embargo, la libre de Jacobiano posee un segundo término de error que depende del parámetro $\delta$, por tanto su error a partir de ciento valor de la dimensión de Krylov va a estar dominado por este término dependiente de $\delta$. Por tanto las aproximaciones libres de Jacobiano son por lo general menos precisas que las que utilizan en Jacobiano exacto.
    
    2)- Se construyeron nuevas fórmulas embebidas de Dormand y Prince localmente linealizadas para problemas de valor inicial de dimensiones no pequeñas utilizando las aproximaciones Krylov-Padé con evaluación de Jacobiano en el cálculo de los productos de funcion phi por vector. Se derivaron cotas para los errores y condiciones de orden simples. Se  desarrollaron estrategias adaptativa para la reutilización del Jacobino, la selección del tamaño de paso de integración, la dimensión de Krylov y el orden de Padé con las que se implementaron esquemas con tamaño de paso fijo y dimensión de Krylov variable, y esquemas con tamaño de paso y dimensión de Krylov variable. Se comprobó la eficacia de esos nuevos esquemas en la integración de diferentes ecuaciones de prueba y se constató que presentan igual o mejor precisión que los esquemas con los que fueron comparados requiriendo, en la mayoría de los casos, un número menor de pasos de integración.
    
%    Además, el desempeño de este método fue probado en la integración de diferentes ecuaciones de prueba mostrando un excelente rendimiento. En general estos esquemas Runge-Kutta de Dormand y Prince Localmente Linealizados presentan igual o mejor precisión que los métodos con los que fue comparado requiriendo en la mayoría de los casos menos pasos de integración.

    3)- Se extendió la concepción de los métodos Localmente Linealizados de Orden Superior a problemas en los que no es posible evaluar la matriz Jacobiana obteniéndose la nueva familia de métodos de  Linealización Local de Orden Superior Libres de Jacobiano. Para ésta nueva familia se obtuvieron cotas de errores y condiciones de orden simples. Utilizando las aproximaciones de Krylov-Padé libre de Jacobiano, se introdujo la subclase particular de esquemas Runge-Kutta Localmente Linealizados Libres de Jacobiano y, en paticular, con las fórmulas Runge-Kutta embebidas de Dormand y  Prince Localmente Linealizadas, se implementó un esquema adaptativo de orden variable libre de Jacobiano. Se comprobó la eficacia de los nuevos esquemas libres de Jacobiano en la integración de ecuaciones de prueba y se constató que presentan igual o mejor precisión que los esquemas con los que fueron comparados, pero con un costo computacional menor en la mayoría de los casos. 
    
%    Al combinar las aproximaciones Runge-Kutta y las aproximaciones Krylov-Padé libres de Jacobiano, anteriormente propuestas, con esta nueva familia se obtiene un nuevo grupo de métodos  Runge-Kutta Localmente Linealizados Libres de Jacobiano. Una estrategia adaptativa para la selección del tamaño, la dimensión de Krylov, el orden de Padé y el orden de la aproximación del producto del Jacobiano por un vector, fue desarrollada. Se realizaron un conjunto de experimentos numéricos para corroborar las condiciones de orden obtenidas. Además, otro conjunto de simulaciones fuero realizadas para medir el desempeño de estos nuevos métodos construidos en la integración de ecuaciones de prueba. En general estos nuevos métodos muestran un gran desempeño ya que son igual o mas precisos que los esquemas con los que fueron comparados, además en la mayoría de los casos con un costo computacionalmente menor.



\end{conclusions}
