\begin{conclusions}

    En esta Tesis se desarrollaron métodos de Linealización Local de Orden Superior para integrar problemas de valor
    inicial de dimensiones no pequeñas. Específicamente:


    1)- Se construyeron aproximaciones Krylov-Padé con y sin la evaluación de Jacobiano para la solución de ecuaciones diferenciales lineales de dimensiones no pequeñas y se acotaron sus errores. Para cada aproximación, se propusieron estrategias efectivas para la estimación práctica de la dimensión de espacio Krylov, el orden de Padé y el error de aproximación. Se comprobó la eficacia de ambas aproximaciones para calcular la acción de funciones phi sobre vectores y, en ese aspecto, se mostró las ventajas del uso de la aproximación Krylov-Padé con evaluación de Jacobiano en relación con otras aproximaciones similares existentes.


    2)- Se construyeron nuevas fórmulas embebidas de Dormand y Prince localmente linealizadas para problemas de valor inicial de dimensiones no pequeñas utilizando las aproximaciones Krylov-Padé con evaluación de Jacobiano en el cálculo de los productos de función phi por vector. Se derivaron cotas para los errores y condiciones de orden simples. Se  desarrollaron estrategias adaptativas para la reutilización del Jacobiano, la selección del tamaño de paso de integración, la dimensión de Krylov y el orden de Padé con las que se implementaron esquemas con tamaño de paso fijo y dimensión de Krylov variable, y esquemas con tamaño de paso y dimensión de Krylov variable. Se comprobó la eficacia de esos nuevos esquemas en la integración de diferentes ecuaciones y se constató que presentan igual o mejor precisión que los esquemas con los que fueron comparados requiriendo, en la mayoría de los casos, un número menor de pasos de integración.


    3)- Se extendió la concepción de los métodos Localmente Linealizados de Orden Superior a problemas en los que no es posible evaluar la matriz Jacobiana obteniéndose la nueva familia de métodos de  Linealización Local de Orden Superior Libres de Jacobiano. Para ésta nueva familia se obtuvieron cotas de errores y condiciones de orden simples. Utilizando las aproximaciones de Krylov-Padé libre de Jacobiano, se introdujo la subclase particular de esquemas Runge-Kutta Localmente Linealizados Libres de Jacobiano y, en particular, con las fórmulas Runge-Kutta embebidas de Dormand y  Prince Localmente Linealizadas, se implementó un esquema adaptativo de orden variable libre de Jacobiano. Se comprobó la eficacia de los nuevos esquemas libres de Jacobiano en la integración de ecuaciones de prueba y se constató que presentan igual o mejor precisión que los esquemas con los que fueron comparados, pero con un costo computacional menor en la mayoría de los casos. 




\end{conclusions}
