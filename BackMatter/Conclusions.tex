\begin{conclusions}
	
	En esta Tesis se evaluó la factibilidad de aplicar los métodos de Linealización Local de Orden Superior para resolver problemas de valor inicial de dimensiones no pequeñas. Se cumplió el objetivo general y los objetivos específicos propuestos. Con el desarrollo de métodos de Linealización Local de Orden Superior para integrar problemas de valor inicial de dimensiones no pequeñas y su subsiguiente evaluación en un conjunto de ecuaciones se valida la hipótesis. Específicamente podemos concluir:


    1)- Se construyeron aproximaciones Krylov-Padé con y sin la evaluación de Jacobiano para la solución de ecuaciones diferenciales lineales de dimensiones no pequeñas y se acotaron sus errores. Estas aproximaciones poseen un alto orden de convergencia. Para cada aproximación, se propusieron estrategias efectivas para la estimación práctica de la dimensión de subespacio de Krylov, el orden de Padé y el error de aproximación. Importante mencionar que la estrategia de selección de la dimensión del subespacio de Krylov se basa en un error relativo y permite estimar la dimensión de Krylov óptima. Se mostró las ventajas del uso de la aproximación Krylov-Padé con evaluación de Jacobiano en relación con otras aproximaciones similares existentes.


    2)- Se construyeron nuevas fórmulas embebidas de Dormand y Prince localmente linealizadas para problemas de valor inicial de dimensiones no pequeñas utilizando las aproximaciones Krylov-Padé con evaluación de Jacobiano en el cálculo de los productos de función phi por vector. Se derivaron cotas para los errores y condiciones de orden simples. Se  desarrollaron estrategias adaptativas para la reutilización del Jacobiano, la selección del tamaño de paso de integración, la dimensión de Krylov y el orden de Padé con las que se implementaron esquemas con tamaño de paso fijo y dimensión de Krylov variable, y esquemas con tamaño de paso y dimensión de Krylov variable. Se comprobó la eficacia de esos nuevos esquemas en la integración de diferentes ecuaciones y se constató que presentan igual o mejor precisión que los esquemas con los que fueron comparados requiriendo, en la mayoría de los casos, un número menor de pasos de integración.


    3)- Se extendió la concepción de los métodos Localmente Linealizados de Orden Superior a problemas en los que no es posible evaluar la matriz Jacobiana obteniéndose la nueva familia de métodos de  Linealización Local de Orden Superior Libres de Jacobiano. Para ésta nueva familia se obtuvieron cotas de errores y condiciones de orden simples. Esta nueva familia preserva el orden de convergencia mientras se satisfagan las cotas de errores, a diferencia de otros esquemas que sufren pérdida de orden al utilizar aproximaciones libres de Jacobiano.
    
    
    4)- Utilizando las aproximaciones de Krylov-Padé libre de Jacobiano junto a la la nueva familia de métodos de  Linealización Local de Orden Superior Libres de Jacobiano se introdujo la subclase particular de esquemas Runge-Kutta Localmente Linealizados Libres de Jacobiano. En particular, con las fórmulas Runge-Kutta embebidas de Dormand y  Prince Localmente Linealizadas, se implementó un esquema adaptativo de orden variable libre de Jacobiano. Se comprobó la eficacia de los nuevos esquemas libres de Jacobiano en la integración de ecuaciones de prueba y se constató que presentan igual o mejor precisión que los esquemas con los que fueron comparados, pero con un costo computacional menor en la mayoría de los casos. 


	5)- Se implementaron varios códigos de los esquemas propuestos
	\begin{itemize}
		\item LLDP4: esquema Runge-Kutta Localmente Linealizado de Dromand y Prince de orden 4 y paso fijo
		\item LLDP5: esquema Runge-Kutta Localmente Linealizado de Dromand y Prince de orden 5 y paso fijo
		\item LLDP45: formulas embebidas Runge-Kutta de Dormand y Prince localmente linealizadas
		\item JF-LLRK4: esquema Runge-Kutta Localmente Linealizado libre de Jacobiano de orden 4 y paso fijo
		\item LLDP1: formulas embebidas Runge-Kutta de Dormand y Prince localmente linealizadas libres de Jacobiano
		\item LLDP2: formulas embebidas Runge-Kutta de Dormand y Prince localmente linealizadas libres de Jacobiano con diferencia finita de orden 1 en la parte lineal
	\end{itemize}


\end{conclusions}
