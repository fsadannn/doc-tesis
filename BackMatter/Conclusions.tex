\begin{conclusions}

    En este trabajo se desarrollaron códigos adaptativos para la integración de problemas de valor
    inicial de dimensiones no pequeñas basados en esquemas embebidos de linealización local de orden
    superior con y sin la evaluación del jacobiano.

    Se construyeron las aproximaciones Krylov-Padé con y sin la evaluación del Jacobiano además de obtener las cotas de error de estás aproximaciones. Además una estrategia efectiva para la estimación práctica de la dimensión de Krylov y el error de aproximaciones se propuso. Estas aproximaciones presentan un buen desempeño al poseer balance entre precisión y tiempo de cómputo a la hora aproximar la acción de las funciones phi sobre vectores. La mayor diferencia entre la aproxima que evalúa el Jacobiano y la que no es que la que utiliza el Jacobiano exacta disminuye su error a medida aumenta la dimensión de Krylov; sin embargo, la libre de Jacobiano posee un segundo término de error que depende del parámetro $\delta$, por tanto su error a partir de ciento valor de la dimensión de Krylov va a estar dominado por este término dependiente de $\delta$. Por tanto las aproximaciones libres de Jacobiano son por lo general menos precisas que las que utilizan en Jacobiano exacto.

    Se construyó el método Runge-Kutta de Dormand y Prince Localmente Linealizado para EDOs de dimensiones no pequeñas utilizando las aproximaciones Krylov-Padé para calcular los productos de funciones phi por vectores. Las cotas de errores fueron estimadas para este método ademas de obtener condiciones de orden simples las cuales están en concordancia con las simulaciones numéricas realizada con el objetivo de estimar el orden. Fue desarrollada una estrategia adaptativa para reutilización del Jacobino y la sección del tamaño de paso, la dimensión de Krylov y el orden de Padé. Además, el desempeño de este método fue probado en la integración de diferentes ecuaciones de prueba mostrando un excelente rendimiento. En general estos esquemas Runge-Kutta de Dormand y Prince Localmente Linealizados presentan igual o mejor precisión que los métodos con los que fue comparado requiriendo en la mayoría de los casos menos pasos de integración.

    Se extendieron los resultados de los métodos Localmente Linealizados de Orden Superior a métodos libres de Jacobiano obteniéndose de esta forma una nueva familia de métodos Linealización Localmente de Orden Superior Libres de Jacobiano. Para esta nueva familia se obtuvieron las cotas de errores y las condiciones de orden, resultado en condiciones de orden simple. Al combinar las aproximaciones Runge-Kutta y las aproximaciones Krylov-Padé libres de Jacobiano, anteriormente propuestas, con esta nueva familia se obtiene un nuevo grupo de métodos  Runge-Kutta Localmente Linealizados Libres de Jacobiano. Una estrategia adaptativa para la selección del tamaño, la dimensión de Krylov, el orden de Padé y el orden de la aproximación del producto del Jacobiano por un vector, fue desarrollada. Se realizaron un conjunto de experimentos numéricos para corroborar las condiciones de orden obtenidas. Además, otro conjunto de simulaciones fuero realizadas para medir el desempeño de estos nuevos métodos construidos en la integración de ecuaciones de prueba. En general estos nuevos métodos muestran un gran desempeño ya que son igual o mas precisos que los esquemas con los que fueron comparados, además en la mayoría de los casos con un costo computacionalmente menor.



\end{conclusions}
