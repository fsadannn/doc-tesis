\begin{abstract}
	Las Ecuaciones en Derivadas Parciales (EDP) se utilizan para describir fenómenos físicos, biofísicos y físico-químicos, pero solo se conocen soluciones analíticas para unas pocas EDP lineales. Por lo tanto, se utilizan métodos numéricos, como elementos finitos y diferencias finitas, para aproximar las soluciones. Sin embargo, la mayoría de los integradores numéricos convencionales tienen que solucionar grandes sistemas de ecuaciones algebraicas mal condicionadas, lo que lleva al uso de matrices pre-condicionadoras específicas para cada tipo de EDP. Los integradores exponenciales para las ecuaciones diferenciales ordinarias (EDO), que se obtienen de utilizar el método de las lineas, son una excepción, ya que calculan exponenciales matriciales y/o funciones phi por vectores en lugar de resolver ecuaciones algebraicas. Los integradores exponenciales están disponibles en dos tipos: Globalmente Linealizados (GL) y Localmente Linealizados (LL). Los integradores GL funcionan bien para EDO con términos no lineales pequeños o acotados por la parte lineal, mientras que los integradores LL mejoran a los integradores GL y, en general, a los integradores tradiciones utilizados con el método de las líneas para resolver EDP. Sin embargo, estos integradores involucran el cálculo de varios productos de funciones phi por vector con un alto costo  computacional. La ausencia de una estrategia adaptativa para el tamaño de paso de integración y la necesidad de evaluar la matriz Jacobiana exacta del campo vectorial de la EDO son otras de las limitaciones prácticas de esos métodos. Para EDO de dimensiones pequeñas, los métodos LL de orden superior (LLOS) mejoran la eficacia del método LL clásico de orden 2 al adicionar una aproximación a la representación integral o a la diferencial de su residuo. Los métodos LLOS derivados de la forma diferencial contienen solo un producto de función phi por vector en cada paso de integración, lo que los distingue positivamente de los derivados de la representación integral y motivan la investigación de ésta tesis. 
	
	En la tesis, para EDO de dimensiones no pequeñas, se propone la implementación adaptativa de las fórmulas Runge-Kutta embebidas de Dormand y Prince Localmente Linealizadas, utilizando una aproximación de Krylov de orden superior, una nueva medida de error y una nueva forma de estimar la dimensión óptima de Krylov. También se propone una nueva familia de métodos LLOS Libres de Jacobiano para casos en los que no es viable evaluar y almacenar la matriz Jacobiana de la ODE. Estos métodos utilizan una aproximación de Krylov libre de Jacobiano, una medida de su error y un criterio para estimar la dimensión de Krylov. Se presenta la clase de esquemas de Runge-Kutta Localmente Linealizados Libres de Jacobiano y se construyen explícitamente esquemas de tercer a quinto orden. Además, se implementa un esquema adaptativo de orden variable y libre de Jacobiano utilizando las fórmulas Runge-Kutta embebidas de Dormand y Prince Localmente Linealizadas. Experimentos numéricos muestran la eficacia de nuevos esquemas en la integración de ecuaciones de prueba conocidas y se compara con la de otros integradores exponenciales.
\end{abstract}

\newpage

\begin{center}
	{\large \textbf{Hipótesis}}
\end{center}
¿Es factible aplicar los métodos de Linealización Local de Orden Superior para resolver problemas de valor inicial de dimensiones no pequeñas?

\qquad

\begin{center}
	{\large \textbf{Objetivos de la tesis}}
\end{center}
\textbf{Objetivo general}

Desarrollo de métodos de Linealización Local de Orden Superior para resolver problemas de valor inicial de dimensiones no pequeñas.

\qquad\\
\textbf{Objetivos específicos}

\begin{enumerate}
	\item Desarrollo de aproximaciones Krylov-Padé para la solución de ecuaciones lineales de dimensiones no pequeñas.
	\item Construcción de nuevas fórmulas embebidas de Dormand y Prince localmente linealizadas para problemas de valor inicial de dimensiones no pequeñas.
	\item Desarrollo de métodos de Linealización Local de Orden Superior Libres de Jacobiano
\end{enumerate}
